

\WyrdCapLine{T}{he} Wyrd Engine is designed for fast-paced, story-driven play, blending the narrative freedom of Fate with a more structured approach to character abilities. The system emphasises quick character creation and streamlined mechanics, making it an excellent choice for one-shots and episodic campaigns. Game Masters should be able to generate all player characters for a session in less than an hour, and players should be able to pick up a pre-made character and start playing within minutes, allowing for flexible, drop-in play that suits rotating groups or short, focused sessions.

With accessibility in mind, \wyrd is built to be intuitive for newcomers to tabletop roleplaying games. By reducing mechanical complexity and focusing on descriptive actions, it ensures that even those with no prior experience can easily engage with the game. The system provides a strong foundation for storytelling while avoiding cumbersome rules, making it ideal for groups that want to dive straight into adventure without an extended learning curve.


\section{Types of Play}

Roleplaying games can be structured in different ways, each offering a unique experience. \wyrd is primarily designed for \emph{one-shots} and \emph{episodic play}, but it can also support longer campaigns with some adjustments.

\subsection{One-Shots}
A one-shot is a self-contained session that tells a complete story in a single sitting. These are excellent for introducing new players, testing out new settings, or running short, focused narratives without long-term commitment.

\subsubsection{Pros:}
\begin{itemize}
    \item Easy to set up and play with minimal preparation.
    \item Great for newcomers and drop-in players.
    \item Allows for high-stakes storytelling without long-term consequences.
\end{itemize}

\subsubsection{Cons:}
\begin{itemize}
    \item Limited time for character development.
    \item Less room for complex, unfolding plots.
\end{itemize}

\subsection{Episodic Play}
Episodic games consist of multiple short adventures featuring recurring characters. Each session is largely self-contained, but there may be ongoing story threads that connect them.

\subsubsection{Pros:}
\begin{itemize}
    \item Balances flexibility with continuity.
    \item Easy to accommodate changing player rosters.
    \item Encourages character growth while keeping stories manageable.
\end{itemize}

\subsubsection{Cons:}
\begin{itemize}
    \item May lack the deep, overarching narrative of long campaigns.
    \item Requires careful pacing to make each session feel complete.
\end{itemize}

\subsection{Campaign Play}
A campaign is a long-running game with an ongoing story, often spanning multiple sessions with the same characters and overarching narrative.

\subsubsection{Pros:}
\begin{itemize}
    \item Allows for deep character development and long-term storytelling.
    \item Provides a sense of progression and investment.
\end{itemize}

\subsubsection{Cons:}
\begin{itemize}
    \item Requires long-term player commitment.
    \item Can be difficult to maintain momentum if players miss sessions.
\end{itemize}

\wyrd is optimised for one-shots and episodic games, ensuring quick character creation and fast-paced play. However, it can support campaigns with minor modifications, such as introducing progression mechanics or expanding character options over time.

\section{Philosophy and Design Goals}
The Wyrd Engine is built upon the following key design principles:

\subsection{Narrative-Driven Mechanics}
While many systems provide detailed simulationist mechanics, The Wyrd Engine prioritises narrative flow. Rules are designed to reinforce storytelling rather than constrain it, ensuring that mechanics facilitate player agency and character development rather than slow down the action.

\subsection{Modular and Setting-Agnostic}
The Wyrd Engine is intended to be adaptable to multiple settings, from Victorian steampunk mysteries to cosmic horror and high fantasy. Core mechanics remain consistent, while setting-specific options allow groups to tailor the experience to their preferred genre.

\subsection{Accessibility and Ease of Play}
Complexity often serves as a barrier to entry for new players. Two staples of roleplaying games—\emph{narrative play}, where players act out scenes, and \emph{detailed rule sets}, rooted in strategy games—can be stumbling blocks. These two elements are paradoxically at odds: if improvisation is difficult, rules help resolve interactions, but overly complex systems slow down play. \textbf{The Wyrd Engine} leans toward narrative play, with most outcomes determined through roleplaying and the Game Master's discretion. However, its simple skills and traits system provides a structured resolution method when needed.

%\subsection{Character Progression}
%%% TODO: If there comes a section on character progression, update this section
%\todo{There is currently no progression, so rewrite this}
%Characters in The Wyrd Engine develop through a flexible advancement system that ensures steady growth while maintaining balance. Skill caps and structured trait progression prevent power creep, allowing for a long-term campaign structure where characters evolve meaningfully without becoming overpowered.

\subsection{Collaborative Storytelling}
Roleplaying is a shared experience, and The Wyrd Engine encourages player collaboration. Mechanics are designed to give all players opportunities to contribute meaningfully to the story, ensuring that every character has a role to play in the unfolding narrative.

\section{What The Wyrd Engine Is Not}
While the system borrows elements from both narrative and tactical games, it is not intended to be a rigid simulation of reality. It does not use attributes, equipment-heavy mechanics, or detailed statistical modelling. Instead, it focuses on storytelling flexibility while maintaining just enough mechanical structure to create meaningful choices in gameplay.

By keeping these goals in mind, The Wyrd Engine offers a roleplaying experience that is both structured and freeing, supporting deep character development and immersive storytelling without unnecessary mechanical complexity.

\WyrdFooterImage{img/pageart/gears-bottom-right-cropped}
