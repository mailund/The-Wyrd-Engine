
\section{The Tools in the Box}

At the heart of \wyrd are three core mechanics: \textbf{Skills}, \textbf{Traits}, and \textbf{Stress}. These simple, flexible systems can be adapted in a variety of ways to support different genres, tones, or levels of complexity. Whether you're adjusting for a new setting or tailoring the rules to your group's preferences, these are the tools you'll be working with.

\subsection{Skills}

The skill list in \wyrd is intentionally broad and compact. Each skill is designed to cover a wide range of actions, making character creation quick and gameplay fast-paced. However, you may wish to expand or refine the skill list to better suit the tone of your game.

The setting will determine the general shape of the skill list. For example, a gritty noir game might include \textit{Investigate}, \textit{Deceive}, and \textit{Stealth}. A high fantasy setting might use \textit{Lore}, \textit{Magic}, and \textit{Survival}. A science fiction game might feature \textit{Technology}, \textit{Piloting}, and \textit{Engineering}. A horror game might include \textit{Survival}, \textit{Fear}, and \textit{Occult}. A social drama might introduce \textit{Politics} or \textit{Etiquette} as distinct skills.

The \textbf{granularity} of the skills is also important. The basic skill list is designed to be broad enough to cover most actions, but you can break skills down into more specific areas if you want to focus on particular elements of gameplay. For example, in a game with a strong combat focus, you might divide \textit{Combat} into \textit{Melee}, \textit{Ranged}, and \textit{Unarmed}. You could go even further, creating skills for specific weapon categories such as \textit{Swords}, \textit{Guns}, and \textit{Bows}, or even individual weapon types like \textit{Rapier}, \textit{Revolver}, and \textit{Longbow}.

The type and granularity of skills affect the game in two main ways: they help set the tone of the game, and they influence the \textbf{complexity} of character creation and gameplay. A more granular skill list allows for highly specialised characters, but it can slow down character creation and add overhead during play. A broader skill list is easier to manage but may result in characters that feel more generalist.

We return to ideas for using skills to adapt \wyrd to your needs on page~\pageref{toolbox:sec:adapting-skills}.

\begin{CommentBox}{Example Skill Lists}
    \begin{itemize}
        \item \textbf{Gritty Noir:} \textit{Investigate}, \textit{Deceive}, \textit{Stealth}, \textit{Combat}, \textit{Contacts}, \textit{Drive}.
        \item \textbf{High Fantasy:} \textit{Lore}, \textit{Magic}, \textit{Survival}, \textit{Combat}, \textit{Crafts}, \textit{Animal Handling}.
        \item \textbf{Science Fiction:} \textit{Technology}, \textit{Piloting}, \textit{Engineering}, \textit{Combat}, \textit{Negotiation}, \textit{Hacking}.
        \item \textbf{Horror:} \textit{Survival}, \textit{Fear}, \textit{Occult}, \textit{Combat}, \textit{Investigate}.
    \end{itemize}
\end{CommentBox}

\subsection{Traits}

Traits are the unique abilities, advantages, and edges that define what makes a character exceptional. They can represent training, supernatural powers, social status, species features, or personal quirks. In a science fiction setting, Traits might include \textit{Cybernetic Interface} or \textit{Zero-G Adaptation}. In a magical world, you might see \textit{Pyromancer}, \textit{Familiar Companion}, or \textit{Blessed by the Moon}. Traits are one of the most adaptable parts of the system and offer a powerful way to express the flavour of your setting.

The core rules give each player character three Traits that either provide a \textbf{+2 bonus} to a relevant skill check, allow the character to attempt actions that others cannot, or grant a \textbf{once per scene/session} special ability. This is a solid starting point, but you can modify it to suit your setting. You might allow different levels of bonuses, more flexible activation conditions, or variable Trait counts.

You can also experiment with negative or limiting Traits. For example, a Trait that imposes a \textbf{-2 penalty} under specific circumstances could be exchanged for a stronger or broader positive Trait elsewhere—such as a larger bonus, wider applicability, or more frequent use. This approach supports more diverse and flavourful character builds.

We explore adapting Traits further on page~\pageref{toolbox:sec:adapting-traits}.

\begin{CommentBox}{Example Trait Types}
    \begin{itemize}
        \item \textbf{Situational Bonus:} A Trait provides a \emph{+2 bonus} to any relevant skill check when it clearly applies.
        \item \textbf{Expanded Capabilities:} A Trait allows a character to attempt actions that others cannot, such as deciphering an ancient language or crafting advanced machinery.
        \item \textbf{Once per Scene/Session:} A Trait grants a powerful ability usable once per scene or session, like escaping a trap or declaring a helpful ally nearby.
    \end{itemize}
\end{CommentBox}

\subsection{Stress}

Stress in \wyrd is a simplified form of damage or pressure. It does not distinguish between physical, mental, or emotional harm. Instead, it represents the overall toll that events take on a character, supporting a more narrative-driven approach where the focus stays on the story.

That said, you can use the same mechanism to track different types of stress. For example, a horror game might benefit from tracking physical injury separately from psychological trauma. A political drama might introduce a stress track for \textit{Reputation} or \textit{Favour}. This modular design allows you to shape the stress system around your setting’s themes.

You can also use stress tracks to monitor other expendable resources, such as magical energy, divine favour, or battery power—making stress a universal mechanic for whatever matters in your game.

The number of Fatigue and Wound boxes a character has can be scaled up or down to fit the tone. A gritty, high-stakes game might give players fewer boxes, making every setback feel impactful. A more cinematic game might allow greater endurance, supporting fast-paced action and dramatic comebacks.

You can also adjust how stress is applied, what effects it imposes, and how characters recover. In horror, stress might linger or worsen over time. In pulp action, it might reset between scenes. These decisions shape not only the mechanics but the emotional pacing of the game.

We return to ideas for adapting stress on page~\pageref{toolbox:sec:adapting-stress}.
