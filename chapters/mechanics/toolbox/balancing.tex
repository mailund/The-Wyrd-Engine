\section{Balancing}\index{Balancing}

Before we dive into the techniques for adapting the core mechanics, it's worth taking a moment to discuss the role of \textbf{balance} in your game. Many traditional RPG systems place a heavy emphasis on mechanical balance—ensuring that all player characters are equally powerful, or that monsters and NPCs scale precisely with the players' capabilities. \wyrd takes a different approach.

\subsection{Balancing Player Characters}

In some systems, strict rules are used to ensure characters remain roughly equal in power. These systems often rely on classes, levels, or point-buy mechanics to produce balanced builds, and considerable effort goes into ensuring that a wizard and a warrior of the same level are comparable in combat effectiveness. In practice, however, these systems rarely achieve true balance—some builds will always be more effective in certain situations than others.

\wyrd does not assume that mechanical balance between characters is necessary or even desirable. The goal is not to make every character equally powerful in a mechanical sense, but to ensure that every player has a meaningful and enjoyable place in the story. 

Balance in \wyrd is achieved at the \textbf{narrative level}, not the mathematical level. What matters is that each character has opportunities to shine, and each player gets a fair share of the spotlight. A party made up of one massive ogre and three nimble goblins may be wildly imbalanced in raw power, but it can still be balanced in terms of narrative weight, character focus, and fun. In fact, such contrasts often make for the most memorable stories.

As a GM, your role is not to enforce mechanical parity, but to ensure that every character has a role to play in the unfolding events. A character who is physically weak might be the only one who can read the ancient runes. Another might lack combat ability, but serve as the party’s voice in diplomatic scenes. So long as each character is relevant to the story being told—and given space to contribute—they are “balanced” in the ways that matter most.

\subsection{Balancing Encounters}

Many traditional RPGs structure progression around steadily increasing power. As characters advance in levels or gain equipment, they become stronger—and to keep the game challenging, enemies and obstacles are scaled up accordingly.

\wyrd does not assume this kind of scaling is necessary. Characters may improve over time, but improvement is generally narrative or situational, not exponential. More importantly, the mechanics are scale-independent. Since skill modifiers are relative, any change to a player’s abilities can be matched by adjusting the difficulty of the task or the capabilities of the opposition.

There is no need to calculate experience levels, hit dice, or challenge ratings. You do not need to scale up monsters or NPCs in order to make them “fair” for the players. Simply assess how difficult you want a given encounter to be, and assign it a target difficulty or build an NPC with abilities that present a meaningful challenge. A +2 bonus for a player or a +2 bonus for a monster works the same way—both shift the odds, but the relative difference is what really matters.

This makes encounter design fast, flexible, and focused on the fiction. You can decide how hard a challenge should feel, and adjust accordingly without needing to worry about perfect symmetry.

\subsection{Balancing the Game}

In the suggestions for adapting the rules in the remainder of this chapter—and in the following chapters focused on specific rule variants—we often provide guidance for building characters using fixed numbers of “points” assigned to various aspects. This is not intended as a strict point-buy system, but rather as a helpful rule of thumb. Assigning the same number of points to each character can ensure that everyone has a similar range of abilities and options, which can be useful during character creation.

However, as discussed above, mechanical equality is not the same as meaningful balance. Giving every character the same number of points does not guarantee a balanced or enjoyable game. What truly matters is that each character feels relevant to the story, has opportunities to act meaningfully, and receives a fair share of the narrative spotlight.

If some characters are significantly more powerful than others on paper, that’s not necessarily a problem—as long as the story provides space for everyone to shine. A clever but frail investigator can be just as important as a combat powerhouse, depending on the situation. The key lies not in rigidly enforcing parity, but in designing scenarios that offer a variety of challenges and highlight the strengths of different characters.

When adapting the rules, always keep this broader view of balance in mind. Create \emph{interesting} characters with unique roles in the story, and shape your game around moments that let each of them take centre stage. That’s where true balance lies—not in the numbers, but in the shared spotlight.

\begin{CommentBox}{A Note to GMs: Spotlight over Symmetry}
    It can be tempting to obsess over keeping characters mathematically equal—but don't. Your goal isn’t to ensure everyone has the same numbers; it’s to ensure everyone has a reason to be at the table. One player might solve puzzles, another might command in battle, and a third might charm their way through a tense negotiation. As long as the story makes space for them all, you've achieved balance where it counts.
\end{CommentBox}
