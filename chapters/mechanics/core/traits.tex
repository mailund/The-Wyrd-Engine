\section{Traits}
\index{Traits}
\label{core:traits}

In \emph{The Wyrd Engine}, Traits represent unique abilities, specialised knowledge, or personal characteristics that distinguish characters and items from one another. Unlike skills, which define general competence, Traits provide a \emph{mechanical advantage} or \emph{narrative permission} in certain situations.

Each player character has exactly \textbf{three Traits}, carefully chosen to enhance their strengths or reflect their backstory. Non-player characters and monsters can have fewer or far more traits. Traits are broader than skills and allow a character to \emph{break} or \emph{bend} normal rules in ways that make them feel distinct.

Items can also have traits (but not skills). This is a way to add game-mechanic flavour to non-creatures and replaces weapon bonuses and similar mechanisms in other role-playing rule sets.

\subsection{How Traits Work}

Traits function in the following ways:

\begin{itemize}
    \item \textbf{Situational Bonus:} A Trait can provide a \emph{+2 bonus} to any relevant skill check if it clearly applies.
    \item \textbf{Expanded Capabilities:} A Trait may allow a character to attempt actions that others simply \emph{cannot}, such as deciphering an ancient language or crafting elaborate mechanical devices.
    \item \textbf{Once per Scene/Session Special Ability:} Some Traits grant a powerful ability that can be used once per scene or once per session, such as instantly escaping a locked room or declaring an old friend in the right place at the right time.
\end{itemize}

Traits \emph{do not stack}—if multiple Traits apply to a roll, the player must choose which one to use.

\subsection{Creating Effective Traits}

When designing Traits, they should:
\begin{itemize}
    \item Be \emph{broad} enough to be useful in multiple situations.
    \item Be \emph{specific} enough to define a unique aspect of the character.
    \item Provide a \emph{clear mechanical or narrative benefit}.
\end{itemize}

Traits can reflect personality, training, supernatural gifts, or anything else that defines a character’s abilities. Below are examples of well-crafted Traits:

\begin{WyrdExampleSidebar}[float=!t]{Example Traits}
    \begin{itemize}
        \item \textbf{Master Duelist} – Gain \emph{+2 to Fight} when using a rapier or fencing techniques.
        \item \textbf{Shadow Walker} – Can move silently even in well-lit areas, allowing \emph{Stealth rolls in places others couldn’t}.
        \item \textbf{Unshakable Will} – Once per session, completely ignore the effects of fear, mind control, or intimidation.
        \item \textbf{Underworld Connections} – Gain \emph{+2 to Contacts} when dealing with criminals, smugglers, or fences.
        \item \textbf{Inventive Genius} – Can craft \emph{unique gadgets} with Crafts that would be impossible for an ordinary engineer.
    \end{itemize}
\end{WyrdExampleSidebar}

\subsection{Using Traits in Play}

In the following examples we see how traits can be used in different situations to help our characters resolve a situation they find themselves in.



\begin{WyrdExample}[Example 1: Applying a +2 Bonus]
	\textbf{Situation:} Felix Cavendish, an eccentric inventor, is attempting to repair a damaged mechanical safe under a tight time limit. His player wants to use his Trait \emph{“Inventive Genius”}.

	\noindent\textbf{Difficulty Level:} The GM sets the repair difficulty at \Arduous, as the damage is severe.

    \noindent\textbf{Resolution:} Felix rolls \FudgeRes{+--0} and adds his \textbf{Crafts skill (+3)} for a total of \textbf{+2} which would normally be a failure. However, because his Trait \emph{Inventive Genius} applies, he adds an additional \textbf{+2}, bringing his final result to \textbf{+4} which is a success. The safe is repaired flawlessly and even runs more efficiently than before.
\end{WyrdExample}

\begin{WyrdNPC}[float=!t]{Felix Cavendish}
	\emph{A brilliant but erratic inventor-for-hire, Felix Cavendish is both a mechanical genius and a walking disaster. His creations are revolutionary—when they don’t explode. A rogue innovator who skirts the edges of legality, he thrives on the challenge of solving impossible problems with machines that push the limits of science.}

	\vspace{0.5\baselineskip}

	\begin{SkillsBox}
		\Expert & Crafts \\
		\Skilled & Investigate, Resources \\
		\Novice & Lore, Will, Contacts \\
		\Untrained & Notice, Stealth, Deceive, Athletics \\
	\end{SkillsBox}

	\begin{TraitsBox}
		\item[Master Tinkerer] — \emph{Gain +2 to Crafts when repairing or modifying machinery.}
		\item[Unstable Prototype] — \emph{Once per session, declare an experimental gadget with an unpredictable effect.}
		\item[A Calculated Risk] — \emph{Use Will instead of Athletics when escaping dangerous situations.}
	  \end{TraitsBox}
\end{WyrdNPC}


\begin{WyrdExample}[Example 2: Expanded Capabilities]
	\textbf{Situation:} Isadora Lovelace, a gifted spiritualist, wants to communicate with a recently deceased victim in order to uncover clues about a murder. Normally, the \textbf{Lore} skill wouldn’t allow this.

	\noindent\textbf{Trait:} \emph{“A Glimpse Beyond the Veil”} allows her to attempt supernatural interactions.

	\noindent\textbf{Resolution:} Since her Trait permits it, the GM allows a roll using \textbf{Lore}. The outcome determines how much information she can extract.
\end{WyrdExample}

\begin{WyrdNPC}[float=!b]{Isadora "Isa" Lovelace}
	\emph{A renowned spiritualist and occult investigator, Isa Lovelace walks the thin line between science and the supernatural. Some believe she is merely an expert in human nature, while others whisper that she truly communes with forces beyond the veil. With piercing intuition and an enigmatic presence, she seeks knowledge that others fear to uncover.}
  
	\vspace{0.5\baselineskip}

	\begin{SkillsBox}
		\Expert & Empathy \\
		\Skilled & Investigate, Lore \\
		\Novice & Rapport, Will, Notice \\
		\Untrained & Stealth, Deceive, Resources, Contacts \\
	\end{SkillsBox}
  
	\begin{TraitsBox}
	  \item[A Glimpse Beyond the Veil] — \emph{Gain +2 to Empathy when sensing the emotions of the deceased.}
	  \item[Foreboding Intuition] — \emph{Once per session, declare a warning based on an unseen force.}
	  \item[The Cards Never Lie] — \emph{Use Lore instead of Investigate when predicting an outcome.}
	\end{TraitsBox}
  \end{WyrdNPC}
  

\begin{WyrdExample}[Example 3: Once Per Session Ability]
	\textbf{Situation:} Cornelius Flint, a silver-tongued rogue, has been cornered in an alley by the city watch. Escape seems impossible.

	\noindent\textbf{Trait:} \emph{“Always an Escape Plan”} allows him, once per session, to declare he had an escape route planned all along.

	\noindent\textbf{Resolution:} Instead of rolling, the GM allows him to describe a secret hatch in the alley leading to the sewers, letting him escape cleanly.
\end{WyrdExample}
 
\begin{WyrdNPC}[float=!t]{Cornelius "Corny" Flint}
	\emph{A silver-tongued thief and a master of misdirection, Cornelius Flint moves between high society and the criminal underworld with effortless charm. He lives by one rule—if someone is foolish enough to leave their wealth unguarded, it deserves a new owner. While he prefers to talk his way out of danger, he always has an escape plan ready when words fail.}
  
	\vspace{0.5\baselineskip}

	\begin{SkillsBox}
		\Expert & Deceive \\
		\Skilled & Burglary, Rapport \\
		\Novice & Athletics, Stealth, Notice \\
		\Untrained & Contacts, Fight, Will, Resources \\
	\end{SkillsBox}
  
	\begin{TraitsBox}
	  \item[Master of Misdirection] — \emph{Gain +2 to Deceive when distracting someone in conversation.}
	  \item[Sleight of Hand] — \emph{Once per session, declare you have already lifted a small item unnoticed.}
	  \item[Always an Escape Plan] — \emph{Use Burglary instead of Athletics when escaping confinement.}
	\end{TraitsBox}
\end{WyrdNPC}
  

 
\subsection{Final Notes on Traits}

Traits are not just mechanical advantages; they define a character’s core competencies and role in the narrative. Players should use them creatively, and GMs should reward clever applications that fit the story.

\begin{DndComment}{Game Master Tip}
 	If a player wants to use a Trait in a way that isn’t obvious, ask them to describe \emph{how} it applies. Encourage creativity while keeping balance in mind.
\end{DndComment}

