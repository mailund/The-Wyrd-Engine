\section{Basic Combat in The Wyrd Engine}
\index{Combat}

The role of combat varies by setting, scenario, and playstyle. Some games favour \textbf{quick, brutal encounters}, where a single shot or swift blade ends a fight instantly, while others emphasise \textbf{heroic battles} against overwhelming foes. The \textbf{tone and pacing} should reflect the game's themes—whether it’s gritty realism, where injuries are severe, or cinematic action, where characters endure incredible feats. Combat may be \textbf{tactically complex}, rewarding careful planning, or more \textbf{freeform}, focusing on dramatic exchanges over strict mechanics. \wyrd offers a flexible system to suit different narratives. For combat mechanics and customisation, see \fullchapref{chap:combat}.

Basic combat in \wyrd, as described in this chapter, is designed to be \textbf{fast and cinematic}. Most combat encounters resolve within a few quick rounds of opposition rolls, keeping the action moving without bogging down in excessive mechanics. At the same time, characters are relatively hard to take out. In real life, a single blow, stab, or gunshot wound is enough to kill a person, but in the combat rules in this chapter, taking out a character will take a few rounds unless the opponent is super-human in capabilities.

The combat rules do not distinguish between physical and mental combat. If your setting involves psychic or magic attacks, the wounds characters can suffer might all be on the inside, but the game mechanics will be the same as physical combat.

\subsection{Initiative: Who Acts First?}
\index{Combat!Initiative} 

Combat follows a structured yet flexible turn order:

\begin{Example}[Determining Initiative]
	\begin{itemize}
    	\item \textbf{Surprise \& Readiness:} If one side is clearly ambushing the other, they act first.
	    \item \textbf{Tactical Positioning:} If no clear ambush is present, the GM determines turn order based on readiness.
	    \item \textbf{Rolling for Initiative:} If multiple characters are competing to act first, roll \textbf{4dF + Notice} (or another relevant skill). The highest roll acts first, with ties resolved narratively.
	\end{itemize}
\end{Example}

\subsection{Taking Actions in Combat}
\index{Combat actions}

On their turn, a character can do the following:
\begin{itemize}
    \item \textbf{One primary action} (Attack, defend, use an item, complex manoeuvre)
    \item \textbf{One minor action} (Draw a weapon, reposition, open a door, shout a command)
    \item \textbf{Free actions} (Speaking briefly, minor environmental interactions)
\end{itemize}

\subsection{Attacking and Defending}
\index{Combat!Attacking}\index{Combat!Defending}

Attacks are resolved using opposed rolls:
\begin{Example}[Attack Resolution]
	\begin{itemize}
    	\item The attacker rolls \textbf{\Attack = \FudgeRoll + skill + traits}
	    \item The defender rolls \textbf{\Defend = \FudgeRoll + skill + traits} 	    
	    \item If \textbf{Attack > Defend}, the attack lands and deals damage.
	\end{itemize}
\end{Example}

Relevant skills depend on the setting, but attack skills could be \textbf{Fight} for melee or \textbf{Shoot} for firearms, while defence skills could be \textbf{Athletics} for dodging or \textbf{Fight} for parrying. Traits are any relevant character or gear traits that match the combat situation.

If the defender has a higher score than or equal to the attacker, the attack is averted, and no damage is dealt. Ties are always in the defender's favour. If the attacker scores higher, the damage inflicted on the defender is the attacker's score minus the defender's.

\begin{Example}[Calculating Damage]
    \textbf{\Damage = \Attack - \Defend} when \textbf{\Attack > \Defend}.
\end{Example}

\Damage is determined by how much the \Attack exceeds \Defend. Effects such as weapons efficiency or armour thickness are considered through the gear's traits in the combat rolls. This has the same effect as adding weapon and defence bonuses within the existing conflict resolution system. We don't need extra combat rules if we don't want them.

\begin{Example}[Example Attack]
	Jonathan Blackwood swings a cane at an enemy thug. He rolls \textbf{4dF +2 (Fight)}, while the thug rolls \textbf{4dF +1 (Athletics)} to dodge. If Jonathan’s result is higher, the hit lands; otherwise, it is defended.
	
	Jonathan rolls \FudgeRes{+++-} = 2 and gets a score of \textbf{Attack = +4} when combined with his \textbf{Fight} skill. The thug then rolls \FudgeRes{+--0} = -1, giving him a score of \textbf{Defend = 0} when combined with his \textbf{Athletics}.
	
	Since Jonathan's score is higher, so he scores a hit, and the damage he inflicts is \textbf{\Damage = \Attack - \Defend = +4 - 0 = +4}. The thug takes \textbf{+4} in damage.
\end{Example}


\subsection{Stress: Fatigue  and Wounds}
\index{Damage}\index{Stress}\index{Fatigue}\index{Wounds}
\index{Damage!Stress}\index{Damage!Wounds}
\index{Stress!Soaking up damage}

\wyrd uses the term \textbf{Stress} for all types of damage a character can sustain. This includes physical, mental, and social damage. Stress is a measure of how much damage a character can take before they are incapacitated. The term \textbf{Stress} is used to represent the overall damage a character can take, while \textbf{Fatigue} and \textbf{Wounds} are used to represent different types of damage.

\wyrd uses \textbf{Fatigue}\index{Damage!Fatigue}\index{Fatigue} to represent minor injuries and \textbf{Wounds}\index{Damage!Wounds}\index{Wounds} for more serious, lasting harm. Neither kind of stress is necessarily physical; mental damage is lumped in with physical stress in the core rules. In games where reputation or social standing is important, social damage can be represented as stress or wounds. The GM and players should agree on how to represent these types of damage in the game.

\begin{Example}[Fatigue and Wounds]
	\begin{itemize}
    	\item \textbf{Fatigue:} Represents minor setbacks, fatigue, or temporary injuries. These are automatically cleared after a fight.
	    \item \textbf{Wounds} come in three levels of severity. They take longer to heal, and adds penalties for future actions.
	\end{itemize}
\end{Example}

Any damage inflicted must be soaked up by either \textbf{Fatigue} or \textbf{Wounds}. Each player has four \emph{Fatigue boxes}, \FatigueBoxes, and five \emph{Wounds boxes} where the wounds are split into three categories: three \textbf{Mild Wounds} (\MildWounds), two \textbf{Moderate Wounds} (\ModerateWounds), and one \textbf{Severe Wounds} (\SevereWounds). These boxes, combined, are where a character can soak up damage.

\DamageBox

When a character sustains \Damage, the damage dealt is converted one-to-one into these stress and wound boxes. Stress is soaked up by the boxes top-to-bottom; the fatigue boxes will soak up the first four points of stress. After that, the following three stress points are inflicted as mild wounds, the next two as moderate wounds, and finally, the character suffers a severe wound. If all stress boxes are ticked off, the character is \textbf{out of action} (see \textsc{Death and the End of Combat} on page~\pagereftext{core:death}).

%% Example of a character taking stress damage
\begin{NPC}{Captain Elias Mercer}
	\emph{A daring sky pirate and master pilot, Elias Mercer is a rogue smuggler with a reputation for getting the job done—no matter how dangerous. Once a decorated naval officer, he now flies under his own banner, evading bounty hunters, rival captains, and the law alike. He lives by one rule: a captain never abandons his crew.}
  
	\vspace{0.5\baselineskip}

	\begin{SkillsBox}
		\Expert  & Pilot \\
		\Skilled & Shoot, Deception \\
		\Novice  & Athletics, Awareness, Combat \\
	\end{SkillsBox}
  
	\begin{TraitsBox}
	  \item[Always One Step Ahead] — \emph{Gain a bonus when avoiding pursuit or laying traps.}
	  \item[A Captain Never Abandons His Crew] — \emph{Once per session, resist an effect that would separate him from his crew.}
	  \item[Knows Every Trick in the Book] — \emph{Can reroll a failed Deception test when lying or fast-talking.}
	\end{TraitsBox}
  
	\DamageBox
  \end{NPC}
  
\begin{Example}[Example: Fatigue Damage]
	As Captain Elias Mercer crouches in the engine room, setting the last charge to sabotage the enemy airship moored at the Tower of London, he is caught off guard by a patrolling crew member. A swift jab to the ribs and a pistol whip to the shoulder deal \textbf{+3} damage. With no previous injuries, the damage is absorbed entirely by his fatigue boxes. 
	
	\vspace{0.5\baselineskip}
	\DamageBox[fatigue=3]

	\noindent
	Gritting his teeth, he shoves the attacker aside and makes his escape—knowing the real danger will come when the explosives detonate.
\end{Example}


When you tick off fatigue boxes, the damage has no noticeable effect. Fatigue is not considered lasting damage but the exhaustion accumulating from combat (or the ``flesh wounds'' from 90s action movies). Once the damage goes into wounds, however, future skill rolls are affected.

\begin{DndTable}[header=]{ll}
    \textbf{Wound Type} & \textbf{Effect} \\
    \textbf{Mild Wound}     & -1 to relevant skill rolls \\
    \textbf{Moderate Wound} & -2 to relevant skill rolls \\
    \textbf{Severe Wound}   & -3 to all physical actions \\
\end{DndTable}

When taking a wound of any of the three kinds, the player and Game Master decide on which relevant skills or traits are affected by the wound. The \textbf{-2} and \textbf{-3} penalties for \textbf{Moderate} and \textbf{Severe} Wounds can be split among multiple skills as long as the total penalty remains the same. Any future rolls involving those skills or traits will have the penalty applied until the wound is healed (see \textsc{Healing and Recovery}~\pagereftext{core:healing}). Additional wounds of the same kind do not add additional penalties when using the core rules.

\begin{Example}[Example: Wound Damage]
	While fleeing the engine room, Captain Mercer, the enemy he knocked aside, recovers, reaches for his gun, and fires off a shot (\FudgeRes{++-0} + \textbf{Shoot (+1)} for an \Attack of \textbf{+2}). Mercer attempts to duck (\FudgeRes{+--0} + \textbf{Atheletics (+1)} for a \Defend of \textbf{0}). The difference is a \Damage of \textbf{+2}.
	
	Mercer only has one fatigue box left, so one of the damage points goes into a \textbf{Mild} wound, and the player and GM decide that the bullet graces Captain Mercer's shoulder, which would affect the \textbf{Athletics} skill.
	
	\vspace{0.5\baselineskip}
	\DamageBox[fatigue=4,mild=1,mildtext=\textbf{Athletics (-1)}]

	Taking the wound in his strides, he exits the room and continues his escape.
\end{Example}

As long as a character has any damage is in a wound category, the penalty applies. Additional stress to a wound category that is already marked does not add additional penalties. Penalties from different categories can stack, however.

\begin{Example}[Example: Wound Damage]
	Captain Mercer rushes to the railing of the airship to throw himself off before the explosive device he planted detonates. The interruption in the engine room, unfortunately, has delayed him too long. The second he jumps, the bomb detonates. The shockwave hits his back with a whooping \textbf{+3} of damage. He can absorb two with his remaining \textbf{Mild} wounds, but one will go into his \textbf{Moderate} wounds. The \textbf{Moderate} wounds give him a penalty of \textbf{-2}, which he and the GM decide to split between \textbf{Atheletics} and \textbf{Awareness} (reasoning that getting blown up is likely to affect Mercer both physically and mentally).
	
	\vspace{0.5\baselineskip}
	\DamageBox[%
		fatigue=4,%
		mild=3,%
		mildtext=\textbf{Athletics (-1)},%
		moderate=1,%
		moderatetext=\textbf{Athletics (-1), Awareness (-1)}%
	]
	
	\noindent
	The two penalties to \textbf{Athletics} stack, so any roll involving \textbf{Athletics} will have a \textbf{-2} penalty.
\end{Example}

If all stress boxes are filled, the character is out of action. What this means is up to the GM, but games are usually more fun if player characters live to fight another day. For one-shot games, it is okay to kill off characters towards the end of the session, but don't do it early in the game.


\subsection{Healing and Recovery}\label{core:healing}
\index{Healing}\index{Recovery}

\begin{itemize}
    \item \textbf{Fatigue} clears at the end of a scene.
    \item \textbf{Mild Wounds} require a short rest (a few hours) or first aid.
    \item \textbf{Moderate Wounds} require days of rest or professional medical care.
    \item \textbf{Severe Wounds} require weeks of rest, surgery, or supernatural healing (if applicable).
\end{itemize}

When healing wounds, \emph{all} marked wound boxes are cleared at the same time. They are healed in parallel, so a character with both \textbf{Mild} and \textbf{Moderate} wounds will have the mild wounds healed the following day (regardless of how many wounds are ticked) and the moderate wounds after a week (with no delay because the mild wounds were healing at the same time).

\subsection{Combat Maneuvers and Special Actions}
Instead of simply attacking, players can use tactical manoeuvres:

\begin{Example}[Combat Maneuvers]
	\begin{itemize}
    	\item \textbf{Disarm:} Use Fight to knock a weapon from an opponent’s hands.
	    \item \textbf{Grapple:} Use Fight vs. Athletics to restrain an enemy.
    	\item \textbf{Push:} Use Athletics to shove an opponent into hazards.
	    \item \textbf{Feint:} Use Deceive to trick an enemy into missing a defence.
    	\item \textbf{Suppressing Fire:} Use Shoot to force enemies into cover.
	    \item \textbf{Intimidate:} Use Provoke to demoralize foes.
	\end{itemize}
\end{Example}

\wyrd does not have rules for all the myriad ways actions can be used in combat. However, the GM should generally convert an action into either an unopposed or opposed obstacle and let the outcome affect bonuses and penalties for future dice rolls. Using manoeuvres gives players a way to use skills besides the obvious combat skills (e.g. \textbf{Fight} or \textbf{Shoot}) as part of a combat encounter. A character with poor combat skills, with little chance of effectively dealing damage, might use other skills to stack up bonuses until an effective attack is possible.

\subsection{Weapons and Gear in Combat}
Weapons do not deal numeric damage but affect combat through \textbf{Traits}. Weapon traits work the same way as any gear trait and can be used when attacking or defending.

\begin{Example}[Types of Weapon Traits]
	\begin{itemize}
    	\item \textbf{Weapons with Traits} grant \textbf{+2} in relevant situations (e.g., “Mastercrafted Rapier” gives +2 to Fight in duels).
	    \item \textbf{Firearms} can inflict instant Wounds if the shot is well-placed.
    	\item \textbf{Improvised Weapons} may impose a penalty unless the character is skilled in their use.
	\end{itemize}
\end{Example}

When a weapon's \textbf{Trait} adds to the attack of a character, it will indirectly affect the damage the attack is inflicting. More interesting uses of weapon traits give other advantages to their wielder.

\begin{Example}[Example Weapon Traits]
	\begin{itemize}
    	\item \textbf{Fine Dueling Sabre} – \textit{+2 to Fight when dueling.}
	    \item \textbf{Hidden Derringer} – \textit{Once per scene, draw a concealed firearm unnoticed.}
	    \item \textbf{Reinforced Cane} – \textit{Can be used as both a weapon and a defensive tool.}
	\end{itemize}
\end{Example}

\subsection{Death and the End of Combat}\label{core:death}
When a character suffers a \textbf{Severe Wound} and takes further damage, they are at risk of death. The simplest choice here is to equate all damage boxes ticked off and character death, but this is not always the best option. It might be fine for nameless mooks the players are fighting but for player characters or important (or just interesting) NPCs, it is often more interesting to consider such a character \textbf{defeated} rather than \textbf{dead}.

Instead of killing off characters, take them captured. Beat them up and leave them for death. Anything \emph{interesting} that can still count as a defeat. Of course, depending on their situation and the setting you are playing in. A zombie is unlikely to capture a character, so true to the zombie genre, you might want to kill off characters there. A vampire, on the other hand, could start monologing about vampiric superiority for long enough that the character could be rescued. 

If you do consider the last wound as essentially death, you might still allow:
\begin{itemize}
    \item A final desperate action before succumbing.
    \item A chance to survive if an ally intervenes.
    \item A dramatic consequence, such as permanent injury.
\end{itemize}

\begin{GmTips}
	If a player is at risk of death, consider narrative consequences rather than instant removal. A major wound or permanent injury can be more interesting than a sudden death.
\end{GmTips}


\subsection{Conceding the Fight}\index{Conceding}\label{core:conceding}

Taking damage until every \textbf{Stress} and \textbf{Wound} box is filled isn’t the only way to lose a fight. Aside from relentless automatons or mindless undead, few combatants fight to the bitter end if they can avoid it. Most will choose survival over certain death, whether that means surrendering, retreating, or negotiating terms. Even when capture is worse than death, most characters will attempt to escape rather than throw their lives away in a hopeless battle.

In \wyrd, conceding a fight is a structured choice, not a failure. When a character concedes, they avoid immediate defeat on their opponent’s terms but \textbf{must accept significant narrative consequences}. The victorious side determines the outcome, though the conceding player can influence how events unfold. A character might escape, but only after dropping their weapons and fleeing unarmed. They might surrender and be taken prisoner, leading to future complications. A successful concession may even allow a character to bargain their way out, leaving them battered but still in play.

By conceding, players trade mechanical defeat for \textbf{a more dramatic and survivable consequence}, shaping the story in ways a simple knockout never could. Game Masters should encourage this approach—fights that end in death leave no room for development, but those that end in setbacks, bargains, or rivalries fuel engaging future encounters.

Combat ends when one side is defeated, flees, or surrenders. Survivors must then deal with the consequences of their wounds, the choices they made, and the path ahead.

\begin{GmTips}
	Avoid fights to the death if you can avoid them. Sometimes, the real story begins when the players choose to live. If they decide to retreat, create a scenario where escape comes at a steep price: perhaps a vengeful enemy is set on their trail, or a priceless artefact slips into the hands of their foes. 
	
	The same goes for foes. You can let the players kill their enemies if that is what the story calls for, but having enemies escape---to inform their superiors about the players' plans or to return with backup later---can be far more interesting.
	
	Remember, death is merely the final chapter of a battle, but the consequences that haunt the survivors can turn a simple conflict into a rich, unfolding saga.
\end{GmTips}

