
\section{Skills}
\index{Skills}
\label{core:skills}

\begin{WyrdExampleSidebar}[float=!t]{Skills in the ``Grand Casebook'' Setting}
	\subsection*{Investigation \& Knowledge}  
	\begin{itemize}
    	\item \emph{Investigate}---Analysing crime scenes, following leads, searching for hidden clues.
	    \item \emph{Lore}---Understanding history, science, the occult, and the unnatural.
	    \item \emph{Notice}---Spotting details, sensing danger, and staying aware of surroundings.
	\end{itemize}

	\subsection*{Social \& Influence}  
	\begin{itemize}
    	\item \emph{Rapport}---Gaining trust, persuading, and negotiating.
	    \item \emph{Deceive}---Lying, creating convincing cover stories, and disguises.
    	\item \emph{Provoke}---Intimidation, interrogation, and getting a reaction from others.
	    \item \emph{Contacts}---Knowing the right people, gathering information through connections.
		\item \emph{Empathy}---Reading emotions, understanding motives, and connecting with others.
	\end{itemize}

	\subsection*{Physical \& Dexterity}  
	\begin{itemize}
    	\item \emph{Athletics}---Running, jumping, climbing, and escaping dangerous situations.
	    \item \emph{Stealth}---Moving unseen, tailing a suspect, sneaking into restricted areas.
	    \item \emph{Fight}---Engaging in hand-to-hand combat, fencing, or using melee weapons.
	    \item \emph{Shoot}---Firearms, throwing weapons, and ranged combat.
	\end{itemize}

	\subsection*{Resilience \& Mental Fortitude}  
	\begin{itemize}
    	\item \emph{Will}---Resisting fear, staying composed under pressure, enduring mental strain.
	    \item \emph{Physique}---Strength, endurance, and the ability to withstand injury or exhaustion.
	\end{itemize}

	\subsection*{Mechanical \& Practical Skills}  
	\begin{itemize}
    	\item \emph{Burglary}---Lockpicking, safecracking, and breaking into places unseen.
	    \item \emph{Resources}---Access to wealth, favours, or valuable possessions.
	    \item \emph{Crafts}---Repairing devices, modifying tools, or working with mechanical systems.
	\end{itemize}
\end{WyrdExampleSidebar}

In The Wyrd Engine, skills represent a character's proficiency in various actions, from keen observation and quick reflexes to mastery in combat or persuasion. Whenever a character attempts a significant action where success is uncertain, they roll \textbf{4dF} and add their relevant skill modifier. The total is then compared against a \textbf{difficulty rating (DR)} set by the Game Master (GM) or an opposed roll from another character.

For player characters and most non-player characters they encounter, skills are range from \Untrained to \Expert:

\begin{DndTable}[header=Skill Levels in \wyrd]{lX}
    \textbf{Skill Level} & \textbf{Description}\\
    \hline
    \Untrained & A character with no special training, relying on instinct or common sense. \\
    \Novice & Someone with basic knowledge or minimal hands-on experience in a skill. \\
    \Skilled & A well-trained individual who regularly practices and applies their ability. \\
    \Expert & A master in the field, capable of performing under extreme conditions. \\
\end{DndTable}

For extreme monsters, e.g., demons, dragons, or killer robots, skills might go higher. You will usually not go lower than \Untrained unless a character is impaired, e.g., drugged or recovering after severe physical or mental trauma, in which case you can.


Characters begin with a defined set of skill ranks, representing their strengths and weaknesses. Unlike systems with extensive skill lists, The Wyrd Engine keeps skills broad and flexible, allowing them to cover a wide range of related actions. For instance, a character with a high \textbf{Athletics} skill might use it to outrun pursuers, climb treacherous cliffs, or leap between rooftops. Similarly, depending on the character's background, Lore could represent expertise in ancient history, arcane knowledge, or scientific principles.

The list of skills a character can have will depend on the setting in which the game is taking place, and there is not a fixed list of skills for all Wyrd games. Generally, you should feel free to make up your own skills---remembering to keep them broad in scope---and decide between player and GM when a skill is applicable. If you like, though, you can make more detailed skill lists if that is more to your taste. In the sidebar, you can see an example of this from \emph{The Grand Casebook} setting, a Victorian/Steampunk/Gothic Horror setting.

When a character lacks a skill, they roll with a default modifier of 0, relying solely on luck and circumstance. This ensures that even untrained characters have a chance—however slim—of succeeding in tasks outside their expertise.

Let us throw the character \emph{Inspector Julian Hargrave} (see sidebar) into some difficult situations and see how he can use his skills to resolve them.

\begin{WyrdNPC}[float=!t]{Inspector Julian Hargrave}
	\emph{Determined and methodical, Inspector Julian Hargrave is a seasoned detective. His years of experience have made him an expert at uncovering the truth, though his rigid approach sometimes clashes with the unpredictable nature of crime-solving.}

	\vspace{0.5\baselineskip}
	\SkillsBox[%
		expert={Investigate},%
		skilled={Notice, Rapport},%
		novice={Will, Provoke, Athletics},%
		untrained={Stealth, Burglary, Shoot, Resources}%
	]
\end{WyrdNPC}

\subsection{Skills in Action}


\begin{WyrdExample}[Analysing a Crime Scene]
	\textbf{Situation:} A renowned socialite has been found dead in her study. The room appears to suggest suicide, but something about the scene seems off. Julian examines the area for inconsistencies.

	\noindent
	\textbf{Difficulty Rating:} The GM decides that the difficulty rating is \Formidable – The crime scene is staged well, but subtle clues remain for an expert to notice.

	\noindent
	\textbf{Resolution:} Julian rolls \FudgeRes{++-0} and adds \textbf{+3 (Investigate)} for a total of +4. Since he exceeds the DR, he notices an overturned chair that contradicts the suicide setup. A closer look reveals a footprint near the window, suggesting an intruder.
\end{WyrdExample}


\begin{WyrdExample}[Spotting an Ambush]
	\textbf{Situation:} Julian follows a suspect through the fog-laden streets when he hears an unusual shuffle behind him. Is someone trailing him?

	\noindent\textbf{Difficulty Rating:} The GM determines that the difficulty level is \Difficult – The follower is cautious but not an expert in stealth.

	\noindent\textbf{Resolution:} Julian rolls \FudgeRes{+--0} and adds  \textbf{+2 (Notice)}, for a total of +1, meeting the DR. He catches the reflection of a blade in a shop window just in time to evade an ambush.
\end{WyrdExample}


\begin{WyrdExample}[Gaining a Witness’ Trust]
	\textbf{Situation:} A frightened maid refuses to discuss her employer’s illicit dealings. Julian must convince her to cooperate.

	\noindent\textbf{Difficulty Rating:} The GM decides that the difficulty is \Challenging – She is hesitant but not impossible to persuade.

	\noindent\textbf{Resolution:} Julian rolls \FudgeRes{+---} and adds \textbf{+2 (Rapport)} for a total of +0. A tie is a failure, or is it? If he changes his tactics or offers protection to try again, it might turn into a partial success.
\end{WyrdExample}


\begin{WyrdExample}[Intimidating a Thief]
	\textbf{Situation:} A pickpocket is caught red-handed. Instead of arresting him, Julian wants to frighten him into revealing who he works for.

	\noindent\textbf{Difficulty Rating:} The GM judges that the difficulty is \Basic – The thief is young and inexperienced but used to trouble.

	\noindent\textbf{Resolution:} Julian rolls \FudgeRes{++00} and adds \textbf{+1 (Provoke)} for a total of +3. He exceeds the DR, causing the thief to stammer out the name of a notorious smuggler before running off.
\end{WyrdExample}


