\section{The Warden’s Path}

\begin{Example}
    The Warden’s Path winds through deep forests, high peaks, sunlit plains, and storm-wracked shores. It is not a road of cities or empires, but of roots, rivers, and stones warmed by ancient fire. Those who walk it do not command the elements—they listen to them, learn from them, and earn their trust.

    A Warden feels the tremor beneath the ground before it speaks. They know when the rain is mercy and when it is warning. They do not cast spells—they shape their will through discipline, ritual, and connection to the world around them.

    And when the balance is broken, they are the ones who rise to restore it.
\end{Example}

\textit{The Warden’s Path} is a \emph{medium-soft to medium-hard} magic system built on balance, focus, and elemental harmony. It is ideal for martial elementalists, ritual guardians, wandering monks, or nature-bound mystics. Power flows through alignment—not domination—and magic is shaped through action, breath, and will.

Where \textit{The Gift of Twilight} is narrative and mysterious, the Warden’s Path is equally narative in structure, but building up magical effects involves the core mechanics to a larger degree. It offers players a structured but flexible toolkit, grounded in consistent effects and cinematic pacing.

\begin{GmTips}
    The Warden’s Path is ideal for players who want magic that is rhythmic, grounded, and tactical. The build-up mechanic supports tension and dramatic releases, while elemental attunement reinforces character identity.
    \begin{itemize}
        \item \textbf{Pros:} Evocative, balanced, works well in both exploration and combat; supports big moments.
        \item \textbf{Cons:} Requires players to think ahead; less spontaneous than freeform systems.
        \item \textbf{Best For:} Elemental guardians, ritual casters, martial mystics, nature-based traditions.
        \item \textbf{Not For:} Chaotic or academic spellcasters (see \textit{Codex Infinitum} for those).
    \end{itemize}
\end{GmTips}

\subsection{The Mechanics of the Warden’s Path}

Magic in this system is built from two components: \textbf{Elemental Traits} and \textbf{Skills}.

\begin{itemize}
    \item \textbf{Elemental Traits} represent the character’s attunement to a specific element, such as fire, earth, water, or air.
    \item \textbf{Skills} represent how that element is directed—whether to attack, defend, reshape the environment, or endure hardship. These are the usual skills from the core rules.
\end{itemize}

\begin{Example}[Elemental Traits]
    \begin{itemize}
        \item \textit{Heart of the Flame} — attuned to fire and heat  
        \item \textit{Stonebound} — attuned to earth and endurance  
        \item \textit{Voice Like Thunder} — attuned to air and storm  
        \item \textit{Dancer of Tides} — attuned to water and flow  
    \end{itemize}
\end{Example}

Together, the elemental trait and skills allow the character to channel magic through action. There are two primary modes of use \textbf{Build-Up} and \textbf{Release}.

\subsubsection{Build-Up}
A \textbf{Build-Up} action lets the caster use an element combined with a skill to channel magical energy into a new or held spell. Each successful build-up adds a \textbf{+2 bonus}. These bonuses stack until the spell is released.


\subsubsection{Release}
A \textbf{Release} action combines an element + skill to perform an immediate magical effect (standard skill roll). The element only adds narrative flavour, not mechanical bonuses.


\subsection{Casting and Channeling}

% \begin{Example}[Build-Up]
%     \textit{“I want to channel energy into a fireball.”}    
%     \vspace{0.5em}
%     The player rolls \textbf{Elemental Control} and gains a +2 bonus for each successful build-up.
% \end{Example}

When a Warden uses elemental magic, they declare which element and skill they are using. The GM sets the difficulty as normal.

- \textbf{Immediate use} = roll and resolve as normal.
- \textbf{Channeling} = roll as a preparation action. On success, gain \textbf{+2} to your next elemental spell. This can be done repeatedly to build up energy over time.

\begin{Example}[Using and Building Magic]
    \begin{tabular}{@{}l p{0.8\linewidth}@{}}
        \textbf{Player:} & “I want to raise a stone wall between the villagers and the raiders.” \\
        \textbf{GM:} & “Using your \textit{Stonebound} trait and \texttt{Elemental Control}? You can cast now or spend a round channeling.” \\
        \textbf{Player:} & “I’ll channel—try to raise something strong.” \\
        \textbf{GM:} & “Great. Roll \texttt{Elemental Control}. Success gives you a +2 for when you release.” \\
        \textbf{Player:} & “I got a Great (+4)!” \\
        \textbf{GM:} & “The stones tremble at your call. You may build up further, or release next round with a +2.”
    \end{tabular}
\end{Example}

The player can now:
- Channel again for another +2 (if successful), or  
- Release the spell and apply all accumulated bonuses to a dramatic effect.

This creates meaningful tension—does the Warden act now, or build for a greater impact?

\subsection{Elemental Limits}

Wardens can only manipulate elements they are attuned to. A character with only \textit{Dancer of Tides} cannot raise fire or command earth—unless they acquire another Trait through training, trial, or mystical revelation.

This encourages thematic specialisation and makes each Warden feel distinct.

\subsection{Optional Mechanics}

\begin{itemize}
    \item \textbf{Burnout:} If a character fails a channeling attempt, they take 1 Fatigue. This adds risk to continued buildup.
    \item \textbf{Unstable Cast:} If a character channels for more than +4 and fails their release, the spell misfires.
    \item \textbf{Elemental Stress:} Channeling can cost stress instead of a roll, allowing automatic buildup—at a price.
\end{itemize}

\subsection{Magic Skills}

The core skill used for elemental manipulation is often called \texttt{Elemental Control}, but the GM may allow alternative skills like:

\begin{itemize}
    \item \texttt{Discipline} — to contain or direct powerful elemental forces  
    \item \texttt{Endurance} — to absorb an elemental effect  
    \item \texttt{Combat} — to weaponise the element  
    \item \texttt{Survival} — to sense or respond to natural conditions  
\end{itemize}

Players are encouraged to describe how their action fits their skill.

\subsection{In Summary}

\textit{The Warden’s Path} is a disciplined form of elemental magic, focused on harmony, patience, and power with purpose. Its unique build-up mechanic rewards foresight and character-driven expression. Players who enjoy cinematic timing, strategic tension, and the poetry of elemental balance will find themselves at home.
