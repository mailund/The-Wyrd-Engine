\section{Codex Infinitum}

\begin{Example}
    The Codex is not a book you hold. It is written in the bones of the world, encoded in starlight, hidden in the spaces between words. But for those who study, who sacrifice, who inscribe its logic upon their minds—it opens.

    In the Great Archive, beneath miles of stone, Arcinel traces the glyphs again. Not by rote, but by understanding. Geometry, breath, and balance. He speaks the final syllable and the construct animates, bronze limbs unfolding with a hiss of steam and spell.

    One mistake would have meant ruin. But knowledge—true knowledge—is power. And he has earned it.
\end{Example}

\textit{Codex Infinitum} is a \emph{hard magic} system—structured, tactical, and rule-bound. Magic is studied, codified, and practiced like an arcane science. Every spell has a defined effect, cost, and scope. It is ideal for high fantasy campaigns, magical academies, arcane duels, or settings where power comes through discipline and logic rather than intuition.

Unlike the symbolic ambiguity of \textit{The Gift of Twilight} or the intuitive flow of \textit{The Warden’s Path}, this system prioritises clarity, precision, and balance. It appeals to players who enjoy mechanical depth, reliable outcomes, and strategic spell use.

\begin{GmTips}
    Codex Infinitum is a structured spellcasting system for players who enjoy crunchy mechanics, defined options, and tactical play. It’s ideal for magical scholars, arcane tacticians, or campaigns with rigid magical laws.
    \begin{itemize}
        \item \textbf{Pros:} Clear rules, consistent resolution, easy to balance, satisfying for planner-type players.
        \item \textbf{Cons:} Less improvisational; may feel rigid in dreamlike or mythic settings.
        \item \textbf{Best For:} Magical universities, wizard duels, fantasy warfare, arcane espionage.
        \item \textbf{Not For:} Folkloric or mystery-focused campaigns (see \textit{The Gift of Twilight} for those).
    \end{itemize}
\end{GmTips}

\subsection{The Codex Mechanics}

The Codex system is built around \textbf{spell skills}.\index{Spell skills} While the GM may require a Trait (e.g., “Trained at the Obsidian Spire”) to justify access, it is not mechanically necessary—spells function like any other Skill.

To cast a spell, the player chooses one from their known list and resolves it like any action: the GM sets a difficulty, the player rolls, and the outcome determines the effect. Spells are either broad skills (e.g. \textit{Warding}) or specific entries from a spellbook-like list.

\subsubsection{Spells as Skills}

In the simplest approach, spells are purchased like any other Skill and function similarly. Each represents a broad domain of arcane expertise.

\begin{Example}[Spell Skills]
    \begin{itemize}
        \item \textit{Veilcraft} — manipulate light, sound, or sensation to obscure or deceive  
        \item \textit{Flameworking} — conjure, shape, or project fire  
        \item \textit{Warding} — create magical barriers, seals, or protections  
        \item \textit{Chronoshaping} — manipulate time in small, focused ways  
    \end{itemize}
\end{Example}

Because spells are purchased like Skills, they should have similarly broad scope. While magic invites narrative freedom, the goal is to keep effects consistent and balanced with the rest of the game.

\subsubsection{Grimoires and Spellbooks}

Alternatively, GMs may define a separate spell list distinct from normal skills. This allows for more granularity in effect and cost, and supports different advancement rules. You can require spells to be purchased from a separate budget, or at higher costs than regular skills.

\begin{Example}[Spells with Detailed Effects]
    \begin{itemize}
        \item \textit{Mirror Veil} — Appear as another humanoid. Lasts until damaged or disrupted. Costs 2 stress. Opposed by \textbf{Notice} or magical detection.
        \item \textit{Flame Tongue} — Imbue a weapon with fire. Grants \textbf{+2} damage and ignition. Lasts one scene. Costs 2 stress.
        \item \textit{Ward of Binding} — Seal an entryway. Lasts 1 hour or until dispelled. Costs 3 stress. Opposed by \textbf{Will}.
        \item \textit{Time Slip} — Take two actions this round or gain \textbf{+2} to initiative and evasion. Costs 3 stress. May not be used back-to-back.
    \end{itemize}
\end{Example}

\subsubsection{Magic Stress}

Stress is used to limit spell use and create tactical pressure. Each spell costs stress to cast—either from your main Fatigue/Wounds tracks or from a separate \textbf{Magic Stress} pool. If you want a more heroic or flexible tone, use Magic Stress as a distinct track. If you want a grittier tone, use the normal stress boxes.

\begin{Example}[Magic Fatigue]
    \begin{tabular}{@{}l p{0.8\linewidth}@{}}
        \textbf{Player:} & “Can I still cast \textit{Flame Tongue}? I’m nearly out of Fatigue.” \\
        \textbf{GM:} & “You can, but you’ll need to mark a Wound box instead. That fire has to come from somewhere.” \\
        \textbf{Player:} & “Let’s do it.”  
    \end{tabular}
\end{Example}

\subsubsection{Spell Ranks}

To reflect increasing power, spells can be ranked. A spellcaster may only cast a spell at a rank equal to or below their Skill level in that spell. Higher ranks have stronger effects and higher stress costs.

\begin{Example}[Spell Ranks]
    \begin{itemize}
        \item \textit{Mirror Veil} — Costs 1–3 stress
        \begin{itemize}
            \item Rank 1: Appear as a specific person you've studied
            \item Rank 2: Appear as any generic humanoid of chosen type
            \item Rank 3: Shift appearance at will; mimic voice or gait
        \end{itemize}
        
        \item \textit{Flame Tongue} — Costs 1–3 stress
        \begin{itemize}
            \item Rank 1: Weapon deals +2 damage
            \item Rank 2: Weapon deals +4, ignites flammables
            \item Rank 3: Weapon deals +6 and can emit a flame burst (area effect)
        \end{itemize}
    \end{itemize}
\end{Example}

\subsubsection{Magic Schools}

Schools offer another axis for character customisation. Each school defines a category of spells. Characters may specialise in a school to gain bonuses or unlock more advanced spells.

\begin{Example}[Magic Schools]
    \begin{itemize}
        \item \textit{Elementalism} — Fire, water, air, earth. Terrain shaping, blasts, weather magic.
        \item \textit{Necromancy} — Raise the dead, drain life, bind souls.
        \item \textit{Illusion} — Glamour, misdirection, invisibility.
        \item \textit{Divination} — Foresight, scrying, omen-reading.
    \end{itemize}
\end{Example}

Schools can be implemented as Traits or gating requirements. You may require a Trait like “Disciple of the Red Tower” to access spells of a given school—or allow open learning at reduced effect unless a school is mastered. Schools can also offer bonuses (e.g. +2 to fire-related spells) or unlock higher ranks.

\subsection{Guidelines for GMs}

\begin{itemize}
    \item Define spells with your players. Leave room to expand as the campaign grows.
    \item Consider spell rarity. Forbidden or legendary spells might be unlocked via quests or secrets.
    \item Mastery tiers—Apprentice (Rank 1), Adept (Rank 2), Master (Rank 3)—help track progress.
    \item Adjust stress recovery based on your setting. In high-magic worlds, it may reset each scene. In darker ones, it might take a ritual or rest.
\end{itemize}

\subsection{In Summary}

\textit{Codex Infinitum} offers a disciplined, tactical approach to magic. For players who enjoy clearly defined powers, mastery through learning, and meaningful resource trade-offs, it provides a rich and flexible system. Magic, in this vision, is not wonder—it is knowledge made dangerous.
