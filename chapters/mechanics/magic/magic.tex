\chapter{Magic}

\WyrdCapLine{M}{agic} can do anything that the story demands—at least when it’s in the hands of non-player characters, monsters, gods, or mysterious artefacts. In those cases, the Game Master can simply decide what magic does, how powerful it is, and what its limits are (if any). The power level is set not by fixed rules, but by what serves the narrative best.

However, as soon as player characters are expected to interact with magic in a consistent or mechanical way—\emph{especially} if they can wield it themselves—we need structure. We need rules that define what magic can do, how it works, and how it fits into the rest of the system. Without that, magic becomes either arbitrary or unfair.

Whether your setting treats magic as rare and mysterious or common and scientific, this chapter provides tools and examples for creating magic systems that are flexible, balanced, and narratively satisfying. You can use these as written, combine elements, or use them as a foundation for crafting your own.

\subimport{./}{design-goals}
\subimport{./}{building-blocks}

\subimport{./}{gift-of-twilight}
\subimport{./}{wardens-path}
\subimport{./}{codex-infinitum}
\subimport{./}{known-and-the-named}





