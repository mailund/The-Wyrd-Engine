\subsection{Act 2: The Investigation} 

Players must navigate the web of lies surrounding the Brass Orchid’s elite clientele and staff. Key locations include:
\begin{Example}[Key Locations]
	\begin{itemize}
		\item \textbf{The performers’ dressing rooms}, where whispers of illicit affairs and secret dealings emerge.
		\item \textbf{The club’s bar}, where a bartender, \textbf{Henry ``Rigs'' Rigby}, may know more than he lets on.
		\item \textbf{The back office}, where financial records hint at Mercer’s recent blackmail attempts.
	\end{itemize}
\end{Example}

\begin{GmTips}
	Encourage players to interact with the environment beyond skill rolls—describe how their characters examine the clues, interpret body language, and make logical leaps. If they become stuck, use an NPC to nudge them toward a promising line of inquiry rather than outright giving answers.
\end{GmTips}

\noindent
A chase scene or social confrontation may occur if a suspect attempts to flee or cover up crucial evidence. The club’s owner, \textbf{Madame Yvette Duval}, will insist on discretion, urging players to avoid drawing attention.

\subimport{npcs}{staff}
\subimport{npcs}{patrons}

\subsubsection{The Performers’ Dressing Rooms}
A backstage sanctuary for the Brass Orchid’s entertainers, the dressing rooms are filled with the scent of perfume, powder, and secrets. Between costume changes and whispered conversations, this space holds clues about hidden relationships, illicit affairs, and last-minute confrontations. If anyone saw Beatrice Langley before the murder, it would have been here.

Speaking with the club’s performers, the investigators learn that \textbf{Mercer and Beatrice} have been spending an unusual amount of time together lately. The prevailing gossip suggests an affair, though many find this unlikely—such a scandal would not go unnoticed, and \textbf{Madame Duval} would never tolerate it. Still, secrets have a way of slipping through even the most watchful eyes…


If the investigators take the time to search the dressing rooms carefully, they can uncover additional clues that paint a clearer picture of Beatrice’s state of mind before the murder:  

\begin{CommentBox}{Clues to Discover}
	\begin{itemize}
		\item \textbf{Beatrice’s Travel Bag:} A half-packed bag in her dressing room suggests she was preparing to leave in haste. Its hurried state implies she either abandoned the plan or ran out of time.
		\item \textbf{A Torn Letter:} A small stove used to heat the performers' dressing room contains scraps of partially burned paper that can be spotted with a \Challenging \textbf{Notice} roll.
		A \Difficult \textbf{Notice} or \textbf{Crafting} reveals that the paper matches the torn note found in Mercer’s hand. If pieced together, it may hint at the nature of their final confrontation.
		\item \textbf{Testimonies from Performers:} Some performers recall Beatrice arriving shaken before her performance, while others remember her slipping away after intermission. None, however, can say where she went.
	\end{itemize}
\end{CommentBox}


\subsubsection{The Club’s Bar}  
A bustling hub of conversation and vice, the club’s bar is where fortunes are won and lost, secrets change hands, and alliances are forged over a well-poured drink. The air is thick with the mingling scents of brandy, cigars, and ambition. At the centre of it all stands \textbf{Henry "Rigs" Rigby}, a bartender with an ear for whispers and a knack for knowing when to keep his mouth shut. He’s seen it all—but getting him to share what he knows will require a delicate touch or a not-so-subtle push.

As the investigators enter, they catch a glimpse of \textbf{Rigs hurriedly slipping something into his pocket}. Keen-eyed characters may notice a \textbf{hint of gold} flashing before it disappears (a \textbf{Notice} roll at \Difficult will confirm this). It’s \textbf{Mercer’s pocket watch}, and Rigs isn’t keen on explaining how he came by it. At first, he’ll feign ignorance, but a successful \textbf{Interrogate}, \textbf{Intimidate}, or \textbf{Rapport} roll at \Formidable will loosen his tongue—grudgingly.

\begin{CommentBox}{Clues to Discover}  
	\begin{itemize}  
		\item \textbf{Mercer’s Missing Pocket Watch:} Rigs found it in the servers’ area after the murder, where Beatrice likely dropped it in her rush to escape. He will only admit this if pressured.  
		\item \textbf{Unsettled Debts:} A bar ledger records Mercer’s outstanding tabs—far higher than usual. However, in the past few weeks, he had been paying off large amounts, suggesting a new source of income.  
		\item \textbf{Patron Gossip:} Some recall Mercer speaking privately with Beatrice earlier that night, while others overheard him boasting about a “big payday” that was going to change everything.  
		\item \textbf{The Pneumatic Tube Exit:} The bar’s pneumatic system, normally used to deliver drinks to private lounges, has a discreet access point beneath the counter. Investigators examining it will find \textbf{signs of forced entry}—a clear indication of tampering.  
	\end{itemize}  
\end{CommentBox}  

\begin{Example}[How Rigs Found the Watch]
	\textbf{Henry "Rigs" Rigby} insists he had nothing to do with Mercer’s murder—he just found himself in the wrong place at the wrong time. After the body was discovered and the club went into lockdown, he was doing his usual post-shift rounds behind the bar when something \textbf{caught his eye}. A glint of gold beneath the service counter near the \textbf{servers’ area}.  

	Curiosity got the better of him. He bent down, fished it out, and immediately wished he hadn’t. It was \textbf{Mercer’s pocket watch}. Rigs didn’t know how it had ended up there, but he did know one thing: he wanted no part of it. If word got out, he’d be dragged into the mess, and he had no interest in becoming a suspect. So, he \textbf{shoved it into his pocket} and carried on pouring drinks, hoping no one would notice.  

	Under pressure, Rigs will admit that he found the watch \textbf{near the entrance to the back hall}, close to where servers pick up drinks for the private lounges. This strongly suggests that \textbf{someone—likely staff or someone who knew their way around—moved through that area after Mercer’s death}. More importantly, the way it was found indicates that it had been \textbf{dropped in a hurry}, likely by someone escaping through the \textbf{pneumatic tube system}.  

	If investigators push him further, Rigs will recall something odd: \textbf{he heard a soft thud from the back hall minutes before he found the watch}. At the time, he thought nothing of it, assuming it was a staff member shifting crates. But now, in hindsight, it might have been the sound of someone landing after climbing out of the pneumatic system.  
\end{Example}  

\begin{Example}[The Pneumatic System]
	Beneath the bar, tucked behind a row of gleaming brass pipes and aged mahogany panelling, lies a \textbf{discreet access point} to the club’s \textbf{pneumatic tube system}. Normally, these tubes are used to send drink orders, notes, and discreet payments between the private lounges and the bar, but this particular panel has been \textbf{forcibly pried open}. The latch usually kept flush with the wall, is now bent slightly out of shape as if someone had wrenched it open in haste.  

	Upon closer inspection (\Challenging \textbf{Investigate}), \textbf{faint scratches} on the brass lining suggest that something—or someone—was pulled through recently. A \textbf{thin layer of dust} clings to the inner rim of the tube, disturbed in streaks where fingers or fabric may have brushed against it. Investigators with a \textbf{keen eye} may notice a \textbf{small shard of glass} caught between the tubing's metal framework, its edges glistening under the low bar light. If examined, it matches the \textbf{broken vial} found at the exit point, the lingering scent of bitter almonds confirming its deadly purpose.  

	The tube itself is narrow, \textbf{just large enough for a slender person to squeeze through}. A metal \textbf{service ladder} is affixed to the interior, meant for maintenance workers to access the system when needed. However, one of the lower rungs has been bent, possibly from the weight of someone climbing through in a hurry. Looking deeper inside, investigators can see where the \textbf{tube splits}, with one passage continuing toward the back hall and another leading \textbf{upward}, toward the private lounges—including Mercer’s.  	
\end{Example}


Any investigator willing to \textbf{crawl inside} will find it claustrophobic, the \textbf{walls cool and slick} from years of condensation. The air carries a \textbf{faint metallic tang}, mingled with the stale scent of old receipts and spilt brandy. A \Difficult \textbf{Notice} roll will reveal that a few \textbf{scraps of paper} cling to the corners of the passage, suggesting messages were hurriedly sent or torn up mid-transit. If they push forward, they may notice \textbf{a single dark thread caught on a rivet}—a clue that someone in dark clothing passed through recently.  

This passage is the key to unravelling \textbf{how the killer escaped the locked room}, but whether the investigators are willing to \textbf{follow the same route} remains to be seen…  

\begin{CommentBox}{What Can Be Found in the Pneumatic System}  
	\begin{itemize}
		\item \textbf{Signs of tampering:} A bent latch, scratches, and disturbed dust suggest recent use.
		\item \textbf{A broken glass vial shard:} Found inside the tube, confirming poison use.
		\item \textbf{A service ladder with a bent rung:} Indicates someone climbed through in haste.
		\item \textbf{A split passage:} One leading toward the back hall, the other to the private lounges.
		\item \textbf{Traces of the killer’s passage:} A dark thread caught on a rivet, scattered paper scraps.
	\end{itemize}
\end{CommentBox}

\subsubsection{The Back Office}  
Tucked away behind a locked door, the back office is where the club’s finances are managed, and sensitive dealings are recorded. The ledgers here reveal an interesting financial pattern. Mercer had accrued a \textbf{significant gambling debt} at the Brass Orchid over the past year—yet, in the past few weeks, he had begun paying it off in unusually large sums. Where did the money come from?  

\begin{CommentBox}{Clues to Discover}  
	\begin{itemize}  
		\item \textbf{Financial Records:} The ledgers show that Mercer has made \textbf{several large payments} on his debt, suggesting he had recently come into a substantial amount of money. If the investigators follow this trail, they will discover that the timing aligns suspiciously with the time when \textbf{Beatrice} started spending substantially more time with him.  
	\end{itemize}  
\end{CommentBox}  





