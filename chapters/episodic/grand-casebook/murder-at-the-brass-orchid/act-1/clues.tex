\begin{Example}{}
	\subsubsection*{Primary Clues}
	\begin{itemize}
		\item A \textbf{half-finished drink laced with poison}, still resting on the table near Mercer’s body. A faint almond scent lingers, barely noticeable beneath the overpowering aroma of brandy.
		\item The victim’s \textbf{missing pocket watch}, unaccounted for at the crime scene but later discovered in an unexpected location.
		\item A \textbf{scrap of torn paper}, crumpled tightly in Mercer’s hand, as though grasped in his final moments—either in desperation or as a final act of defiance.
		\item The \textbf{pneumatic tube system}, a hidden network connecting various parts of the club, shows signs of recent tampering.
	\end{itemize}

	\subsubsection*{What the clues reveal}
	\begin{itemize}
		\item \textbf{The poisoned drink} confirms the cause of death. The faint almond scent suggests cyanide or a similar fast-acting toxin but without an obvious delivery method.
		\item \textbf{The missing pocket watch}, later found in the servers' area, is not inherently suspicious—but its location is. It suggests that someone, likely a staff member, moved through that area after Mercer’s death. \textbf{Henry "Rigs" Rigby}, the bartender, recovered it but might need some persuasion to reveal the circumstances.
		\item \textbf{The scrap of torn paper} remains tightly clutched in Mercer’s hand. The jagged edge suggests it was ripped from a larger document. Whether Mercer seized it in a moment of panic or it was forcibly torn from him before he collapsed is unclear, but its contents might point to the motive.
		\item \textbf{The tampered pneumatic tube system} is the key to the locked-room mystery. It provides a discreet means of entry and escape, but only staff or someone intimately familiar with the club would know how to exploit it.
	\end{itemize}
\end{Example}