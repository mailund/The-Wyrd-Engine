
\begin{WyrdSettingHeading}
    \WyrdCapLine{L}{ondon}, 1896. A city of gaslit streets, towering factories, and secrets lurking in the shadows. This is an era of progress, where steam and steel reshape the world—but beneath the veneer of industry and refinement, the old mysteries remain. The line between science and the supernatural is thinner than most would dare to believe.

    You are part of The Grand Society of Inquiry, a clandestine organisation of detectives, scholars, and unconventional thinkers dedicated to unravelling the mysteries the world would rather forget. The police may handle mundane crimes, but when the case is impossible, when the authorities turn a blind eye, or when the answers defy reason, that is where you come in.

    The aristocracy hides more than it reveals. The city's underworld knows whispers of truths the elite wish to bury. Strange happenings unfold in laboratories, occult circles, and long-forgotten ruins. It is your job to investigate, to bring truth to light—whether the world is ready for it or not.

    You will encounter murderers whose motives defy logic, inventions beyond their time, secret societies vying for power, and horrors that exist just beyond the veil of reason. Some mysteries should never be solved, but you have chosen to chase the truth regardless.

    London may not thank you for what you uncover. The truth is rarely comforting. But if not you, then who?

    So, tell me: What mystery has found its way to your doorstep tonight?
\end{WyrdSettingHeading}

\section{The Setting}

London in 1896 is a city of contradictions. At its heart lies a tension between progress and tradition, the rational and the arcane. Airships drift over soot-covered rooftops, automata assist in the factories, and steam-powered cabs rattle through the cobbled streets. Yet for all these marvels of industry, old fears still lurk in the fog. Ancient horrors persist in forgotten crypts, and whispers of the occult echo in gentlemen’s clubs and back alley gatherings.

This is a world where gaslight barely holds back the darkness, where rational minds struggle to explain the inexplicable. The Grand Casebook embraces the interplay between Victorian-era crime fiction, steampunk ingenuity, and the gothic supernatural.

\subsection{The Grand Society of Inquiry}

Founded in the wake of the Crimean War, The Grand Society of Inquiry was established by a coalition of scholars, detectives, and adventurers who recognised that certain mysteries lay beyond the reach of conventional authorities. Though their official purpose is to investigate "unusual" occurrences, they are as much a secret society as an investigative body. Their members come from all walks of life—former police officers, rogue academics, disgraced aristocrats, and those who have glimpsed the supernatural and can never return to ignorance.

The Society operates in secrecy, liaising with those who have knowledge of the unseen world—whether they be alchemists, mesmerists, or reformed criminals. Their headquarters, a sprawling archive hidden beneath a London bookshop, contains a wealth of esoteric knowledge that only a select few are permitted to access.

\subsection{The Powers That Be}

While the Society pursues truth, others work to obscure it. Various factions hold sway over London, each with their own stake in its mysteries:

\begin{itemize}
    \item \textbf{Scotland Yard:} The official enforcers of law and order, most officers dismiss the supernatural, though a handful of seasoned inspectors have learned otherwise. The Yard tolerates the Society only when their interests align.
    \item \textbf{The Ministry of Esoteric Affairs:} A shadowy government branch that monitors supernatural activity. Their agents operate with impunity, and their goals often clash with those of the Society.
    \item \textbf{The Order of the Silver Dawn:} An occultist cabal that seeks power through ritual and ancient knowledge. Some claim their origins stretch back to the alchemists of the Elizabethan court.
    \item \textbf{The Industrial Magnates:} The great industrialists of London have their own secrets, from illicit experiments to unspeakable dealings with forces beyond human comprehension.
    \item \textbf{The Underworld Syndicates:} Smugglers and thieves have always known the truth—London's alleys and docks are haunted by more than mere criminals.
\end{itemize}

\subsection{Types of Play}

The Grand Casebook is structured as an episodic mystery-driven setting, where each session presents a new case to unravel. While overarching plots may weave through multiple cases, each game is designed to be a self-contained investigation. The types of mysteries players may face include:

\begin{itemize}
    \item \textbf{Classic Crime:} Murders, thefts, and conspiracies with unexpected twists.
    \item \textbf{Scientific Anomalies:} Unstable inventions, rogue automata, and the consequences of reckless experimentation.
    \item \textbf{Supernatural Encounters:} Hauntings, curses, and beings that should not exist.
    \item \textbf{Political Intrigue:} Power struggles within the aristocracy, blackmail, and espionage.
    \item \textbf{Exploratory Adventures:} Venturing into forgotten catacombs, abandoned asylums, or hidden laboratories.
\end{itemize}

\subsection{Character Roles}

Players take on the roles of Society members, each bringing unique skills to the investigative team. Some possible roles include:

\begin{itemize}
    \item \textbf{The Detective:} A seasoned investigator skilled in deduction and intuition.
    \item \textbf{The Scientist:} A brilliant mind on the cutting edge of technological advancements.
    \item \textbf{The Occultist:} A scholar of the esoteric, familiar with arcane lore.
    \item \textbf{The Rogue:} A streetwise operative connected to the city’s underbelly.
    \item \textbf{The Aristocrat:} A well-connected socialite whose influence opens doors.
    \item \textbf{The Soldier:} A combat-trained veteran, ready to handle more physical threats.
\end{itemize}

\subsection{Rule Adaptations for This Setting}

The Grand Casebook modifies standard play to suit its unique blend of investigation, steampunk technology, and gothic horror. Some adjustments include:

\begin{itemize}
    \item \textbf{Stress and Wounds:} Psychological stress plays a more significant role, with lingering mental consequences affecting future investigations. You can leave out stresses and wounds entirely for most mystery adventures and simply act out any confrontation.
    \item \textbf{Tools of the Trade:} Players may access specialised investigative tools, such as clockwork analysers, ectoplasmic detectors, or enchanted relics.
    \item \textbf{Mystery Structure:} Cases follow a structured flow, focusing on gathering clues, making deductions, and confronting the truth.
    \item \textbf{Supernatural Threats:} Unnatural foes require specific knowledge or preparations to overcome, emphasising research as much as combat.
\end{itemize}


%% TODO: more world building

\section{Adventures}

The following adventures are aimed at 3-5 players and should take 2-4 hours to play. 

\subsection{The Call to Adventure}

At the heart of every investigation lies The Grand Society of Inquiry, an esteemed and enigmatic organisation dedicated to the relentless pursuit of truth. Operating from the opulent halls of the Grand Hall, the society boasts a network of detectives, scholars, and specialists, each possessing a unique skill set vital to solving the most perplexing cases.

When a new case emerges, summons are discreetly dispatched to those deemed most suited for the task at hand. These messages—delivered via courier, pneumatic tube, or even through more esoteric means—call upon select members to assemble and uncover the mystery that awaits. No two groups are ever quite the same, for the \textbf{Grand Analytical Engine}, a vast and intricate steam-powered construct housed in the depths of the Grand Hall, determines the composition of each investigative team.

\begin{CommentBox}{Framing The Call to Adventure}
	The setup for starting adventures is typical for episodic games where the players can vary from session to session. Having an explanation for why the characters vary from case to case means that no further in-game explanation is needed.
\end{CommentBox}

%% TODO: Something about the shared structure to the adventures here (the form of mystery adventures)

%% TODO: get this ito the world description

%The Grand Analytical Engine
%
%This marvel of engineering, a hybrid of Babbage’s Analytical Engine and the finest advancements in mechanical computation, processes a staggering wealth of information. Data is fed into its whirring mechanisms by archivists and clerks, cross-referencing past cases, skills, and affiliations. The result: a meticulously curated team, assembled not by human intuition, but by the cold, logical precision of brass gears and punched cards. Whether by fate or by cold calculation, those summoned are invariably drawn into intrigue, danger, and the pursuit of justice.


\subimport{murder-at-the-brass-orchid}{murder-at-the-brass-orchid}
%% FIXME: proper importing
\subimport{./}{clockmakers-deception}
\subimport{./}{the-silent-courier}


%% TODO: Add three more adventures
