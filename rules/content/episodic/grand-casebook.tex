\chapter{The Grand Casebook}\label{chap:grand-casebook}

\begin{WyrdSettingHeading}
    \DndDropCapLine{L}{ondon}, 1896. A city of gaslit streets, towering factories, and secrets lurking in the shadows. This is an era of progress, where steam and steel reshape the world—but beneath the veneer of industry and refinement, the old mysteries remain. The line between science and the supernatural is thinner than most would dare to believe.

    You are part of The Grand Society of Inquiry, a clandestine organisation of detectives, scholars, and unconventional thinkers dedicated to unravelling the mysteries the world would rather forget. The police may handle mundane crimes, but when the case is impossible, when the authorities turn a blind eye, or when the answers defy reason, that is where you come in.

    The aristocracy hides more than it reveals. The city’s underworld knows whispers of truths the elite wish to bury. Strange happenings unfold in laboratories, occult circles, and long-forgotten ruins. It is your job to investigate, to bring truth to light—whether the world is ready for it or not.

    You will encounter murderers whose motives defy logic, inventions beyond their time, secret societies vying for power, and horrors that exist just beyond the veil of reason. Some mysteries should never be solved, but you have chosen to chase the truth regardless.

    London may not thank you for what you uncover. The truth is rarely comforting. But if not you, then who?

    So, tell me: What mystery has found its way to your doorstep tonight?
\end{WyrdSettingHeading}

\section{The Setting}

London in 1896 is a city of contradictions. At its heart lies a tension between progress and tradition, the rational and the arcane. Airships drift over soot-covered rooftops, automata assist in the factories, and steam-powered cabs rattle through the cobbled streets. Yet for all these marvels of industry, old fears still lurk in the fog. Ancient horrors persist in forgotten crypts, and whispers of the occult echo in gentlemen’s clubs and back alley gatherings.

This is a world where gaslight barely holds back the darkness, where rational minds struggle to explain the inexplicable. The Grand Casebook embraces the interplay between Victorian-era crime fiction, steampunk ingenuity, and the gothic supernatural.

\subsection{The Grand Society of Inquiry}

Founded in the wake of the Crimean War, The Grand Society of Inquiry was established by a coalition of scholars, detectives, and adventurers who recognised that certain mysteries lay beyond the reach of conventional authorities. Though their official purpose is to investigate "unusual" occurrences, they are as much a secret society as an investigative body. Their members come from all walks of life—former police officers, rogue academics, disgraced aristocrats, and those who have glimpsed the supernatural and can never return to ignorance.

The Society operates in secrecy, liaising with those who have knowledge of the unseen world—whether they be alchemists, mesmerists, or reformed criminals. Their headquarters, a sprawling archive hidden beneath a London bookshop, contains a wealth of esoteric knowledge that only a select few are permitted to access.

\subsection{The Powers That Be}

While the Society pursues truth, others work to obscure it. Various factions hold sway over London, each with their own stake in its mysteries:

\begin{itemize}
    \item \textbf{Scotland Yard:} The official enforcers of law and order, most officers dismiss the supernatural, though a handful of seasoned inspectors have learned otherwise. The Yard tolerates the Society only when their interests align.
    \item \textbf{The Ministry of Esoteric Affairs:} A shadowy government branch that monitors supernatural activity. Their agents operate with impunity, and their goals often clash with those of the Society.
    \item \textbf{The Order of the Silver Dawn:} An occultist cabal that seeks power through ritual and ancient knowledge. Some claim their origins stretch back to the alchemists of the Elizabethan court.
    \item \textbf{The Industrial Magnates:} The great industrialists of London have their own secrets, from illicit experiments to unspeakable dealings with forces beyond human comprehension.
    \item \textbf{The Underworld Syndicates:} Smugglers and thieves have always known the truth—London's alleys and docks are haunted by more than mere criminals.
\end{itemize}

\subsection{Types of Play}

The Grand Casebook is structured as an episodic mystery-driven setting, where each session presents a new case to unravel. While overarching plots may weave through multiple cases, each game is designed to be a self-contained investigation. The types of mysteries players may face include:

\begin{itemize}
    \item \textbf{Classic Crime:} Murders, thefts, and conspiracies with unexpected twists.
    \item \textbf{Scientific Anomalies:} Unstable inventions, rogue automata, and the consequences of reckless experimentation.
    \item \textbf{Supernatural Encounters:} Hauntings, curses, and beings that should not exist.
    \item \textbf{Political Intrigue:} Power struggles within the aristocracy, blackmail, and espionage.
    \item \textbf{Exploratory Adventures:} Venturing into forgotten catacombs, abandoned asylums, or hidden laboratories.
\end{itemize}

\subsection{Character Roles}

Players take on the roles of Society members, each bringing unique skills to the investigative team. Some possible roles include:

\begin{itemize}
    \item \textbf{The Detective:} A seasoned investigator skilled in deduction and intuition.
    \item \textbf{The Scientist:} A brilliant mind on the cutting edge of technological advancements.
    \item \textbf{The Occultist:} A scholar of the esoteric, familiar with arcane lore.
    \item \textbf{The Rogue:} A streetwise operative connected to the city’s underbelly.
    \item \textbf{The Aristocrat:} A well-connected socialite whose influence opens doors.
    \item \textbf{The Soldier:} A combat-trained veteran, ready to handle more physical threats.
\end{itemize}

\subsection{Rule Adaptations for This Setting}

The Grand Casebook modifies standard play to suit its unique blend of investigation, steampunk technology, and gothic horror. Some adjustments include:

\begin{itemize}
    \item \textbf{Stress and Wounds:} Psychological stress plays a more significant role, with lingering mental consequences affecting future investigations. You can leave out stresses and wounds entirely for most mystery adventures and simply act out any confrontation.
    \item \textbf{Tools of the Trade:} Players may access specialised investigative tools, such as clockwork analysers, ectoplasmic detectors, or enchanted relics.
    \item \textbf{Mystery Structure:} Cases follow a structured flow, focusing on gathering clues, making deductions, and confronting the truth.
    \item \textbf{Supernatural Threats:} Unnatural foes require specific knowledge or preparations to overcome, emphasising research as much as combat.
\end{itemize}


%% TODO: more world building

\section{Adventures}

The following adventures are aimed at 3-5 players and should take 2-4 hours to play. 

\subsection{The Call to Adventure}

At the heart of every investigation lies The Grand Society of Inquiry, an esteemed and enigmatic organisation dedicated to the relentless pursuit of truth. Operating from the opulent halls of the Grand Hall, the society boasts a network of detectives, scholars, and specialists, each possessing a unique skill set vital to solving the most perplexing cases.

When a new case emerges, summons are discreetly dispatched to those deemed most suited for the task at hand. These messages—delivered via courier, pneumatic tube, or even through more esoteric means—call upon select members to assemble and uncover the mystery that awaits. No two groups are ever quite the same, for the \textbf{Grand Analytical Engine}, a vast and intricate steam-powered construct housed in the depths of the Grand Hall, determines the composition of each investigative team.

\begin{WyrdComment}{Framing The Call to Adventure}
	The setup for starting adventures is typical for episodic games where the players can vary from session to session. Having an explanation for why the characters vary from case to case means that no further in-game explanation is needed.
\end{WyrdComment}

%% TODO: Something about the shared structure to the adventures here (the form of mystery adventures)

%% TODO: get this ito the world description

%The Grand Analytical Engine
%
%This marvel of engineering, a hybrid of Babbage’s Analytical Engine and the finest advancements in mechanical computation, processes a staggering wealth of information. Data is fed into its whirring mechanisms by archivists and clerks, cross-referencing past cases, skills, and affiliations. The result: a meticulously curated team, assembled not by human intuition, but by the cold, logical precision of brass gears and punched cards. Whether by fate or by cold calculation, those summoned are invariably drawn into intrigue, danger, and the pursuit of justice.


\section{Murder at the Brass Orchid}



The investigators are called to The Brass Orchid either by an inside contact, a desperate plea from the club’s owner, or at the behest of an anonymous benefactor. The establishment is filled with wealthy patrons, performers, and staff—each with their own secrets to hide. The club’s reputation is at stake, and the clock is ticking before the police arrive to sweep things under the rug.

The players must piece together the events of the evening, questioning patrons and staff, analysing the crime scene, and determining who had the means, motive, and opportunity to commit the crime. However, the deeper they dig, the more they realise that this murder is just the tip of the iceberg.

\subsection{Premise} 
A high-society soirée at the exclusive cabaret, The Brass Orchid, is cut short when a well-connected financier is found dead in a locked room. The party was attended by the city's elite, but none saw the murder happen—or so they claim. The investigators must navigate a world of secrets, deception, and hidden rivalries to uncover the truth.

\subsection{What Really Happened} 
Beatrice Langley, a hostess at The Brass Orchid, killed the financier, Edward Mercer, to protect herself from blackmail. Mercer had uncovered details about Beatrice’s past life and was threatening to expose her unless she paid a steep price. Desperate and out of options, she poisoned his drink and used the club’s pneumatic tube system to dispose of the evidence. However, a miscalculation led to certain clues being left behind.

\subsection{Act 1: The Crime Scene} 
The investigators arrive at The Brass Orchid and are quickly introduced to the case. The scene is a private lounge where Mercer’s body remains untouched. Clues present in the scene include:
\begin{DndReadAloud}{}
	\begin{itemize}
		\item A half-finished drink laced with poison.
		\item The victim’s missing pocket watch. (Is later found in an unexpected location).
		\item The pneumatic tube system that leads in and out of various club areas.
	\end{itemize}
\end{DndReadAloud}

\begin{DndComment}{What the clues reveal}
	\begin{itemize}
		\item The drink with poison reveals the cause of death.
		\item The missing pocket watch is currently in the hands of \textbf{Henry ``Rigs'' Rigby}, the bartender at the Orchid's lounge. He found it in the servers' area, where Beatrice dropped it after leaving the pneumatic tube system. The watch itself is not a clue, but the location where it was found is.
		\item The pneumatic tube system was used by Beatrice to leave the locked room. It leads to the servers' area and would normally not be noticed by patrons, hinting at a culprit amongst the staff.
	\end{itemize}
\end{DndComment}


Patrons and staff provide conflicting accounts, making it difficult to discern the truth at first. To spice things up, the GM can add further complications:
\begin{DndReadAloud}{\textbf{Complications:}}
	\begin{itemize}
		\item Club owner \textbf{Madame Yvette Duval} insists that the investigation remain discreet—if the scandal spreads, the club’s reputation is finished. 
		\item Meanwhile, \textbf{Inspector Julian Hargrave}, a police detective with little patience for private investigators, arrives and attempts to assert authority over the case.
	\end{itemize}
	
	\noindent
	The players must balance diplomacy, deception, or outright defiance to continue their work.
\end{DndReadAloud}



\subsection{Act 2: The Investigation} 

Players must navigate the web of lies surrounding the Brass Orchid’s elite clientele and staff. Key locations include:
\begin{DndReadAloud}{}
	\begin{itemize}
		\item \textbf{The performers’ dressing rooms}, where whispers of illicit affairs and secret dealings emerge.
		\item \textbf{The club’s bar}, where a bartender, \textbf{Henry ``Rigs'' Rigby}, may know more than he lets on.
		\item \textbf{The back office}, where financial records hint at Mercer’s recent blackmail attempts.
	\end{itemize}
\end{DndReadAloud}

%% TODO: HERE

\noindent
A chase scene or social confrontation may occur if a suspect attempts to flee or cover up crucial evidence. The club’s owner, Madame Yvette Duval, will insist on discretion, urging players to avoid drawing attention.

\subsection{Act 3: The Reveal} 
With all the pieces in place, the investigators must confront \textbf{Beatrice Langley}. She will initially deny involvement but cracks under pressure if presented with compelling evidence:
\begin{DndReadAloud}{}
	\begin{itemize}
		\item \textbf{Traces of the poison} found in her personal belongings.
		\item \textbf{Testimonies from staff} who overheard threats.
		\item \textbf{Inconsistencies in her alibi}.
	\end{itemize}
\end{DndReadAloud}

\noindent
If handled carefully, she may confess outright. However, if the players push too hard or fail to secure proof, she may attempt to escape into the night, leading to a dramatic confrontation.

\subsection{Resolutions} 
Depending on how the investigators handle the case, different outcomes may occur:
\begin{itemize}
	\item \textbf{Justice Served}: Beatrice is arrested or confesses, ensuring the truth is revealed.
	\item \textbf{A Deal in the Shadows}: The investigators allow Beatrice to flee, leveraging her knowledge for future gain.
	\item \textbf{The Wrong Culprit}: A scapegoat is framed, or the authorities arrest someone else entirely.
	\item \textbf{A Mystery Unsolved}: The players fail to piece everything together, leaving The Brass Orchid haunted by unanswered questions.
\end{itemize}

Regardless of the resolution, this case's events ripple across London’s elite, setting the stage for future intrigues.

\begin{WyrdScenarioHeading}[The Clockmaker's Deception]{The Clockmaker's\\ Deception}
    A shocking murder has thrown London’s scientific and industrial circles into disarray. The esteemed inventor, \textbf{Dr Sebastian Thorne}, stands accused of killing a rival engineer, \textbf{Arthur Bellamy}, who was found dead in Thorne’s workshop. The evidence against him seems irrefutable—Bellamy’s body was discovered with blunt force trauma, and the only witness claims that one of Thorne’s own clockwork creations struck the fatal blow.

    But something about the case doesn’t add up. The mechanical automaton, a prototype designed to assist in fine-detail engineering, should be incapable of such an act. Was this an unfortunate accident, or has someone manipulated the scene to frame Thorne? The investigators must untangle the mystery before the city condemns a man who may be innocent—or worse, before a hidden truth shakes the foundations of science itself.

    \subsection*{Premise} 
    A renowned inventor is accused of murder when his latest clockwork creation is found standing over a dead body. The case seems open and shut, but a deeper conspiracy lurks beneath the surface. Was the machine truly responsible, or is someone using technology as a convenient scapegoat?

    \subsection*{What Really Happened} 
    Arthur Bellamy had uncovered a secret — one that threatened powerful interests within London’s scientific community. He arranged a meeting with Thorne under the guise of a professional discussion, intending to share his findings. However, before he could reveal the full truth, an unknown party silenced him.

    The real killer staged the scene, positioning Thorne’s automaton as the culprit. By tampering with the machine’s mechanisms and manipulating witnesses, they ensured that suspicion would fall on Thorne. Now, as the city rushes to condemn him, the investigators must uncover the true murderer, reveal the secret Bellamy died for, and navigate the dangerous underworld of industrial espionage.
\end{WyrdScenarioHeading}

\begin{GmTips}
    As with the previous scenario, you can act out the summoning to \textbf{The Grand Society of Inquiry} as a way to introduce the investigators to the case. If the set of player characters in this scenario differs from the player characters in the previous one, this would give you an excellent way of introducing the new characters to the players.
\end{GmTips}

\begin{GmTips}
    This case provides an excellent opportunity to explore themes of scientific advancement, ethical dilemmas, and the fear of technology gone rogue. The case may also lead into larger conspiracies within London's industrial elite, depending on how deep the investigators choose to dig.
\end{GmTips}

\subsection{Act 1: The Accusation}  
The investigators are summoned to the scene of the crime—the locked workshop of Dr Thorne. The city’s authorities have already decided his guilt, but the inconsistencies in the case suggest a deeper truth.

\begin{Example}{Key Elements of Act 1}
    \begin{itemize}\raggedright
        \item \textbf{Examining the Crime Scene:} Bellamy was struck down in Thorne’s workshop. The automaton is positioned near the body, but no command sequence should have allowed it to act violently.
        \item \textbf{The Automaton:} A marvel of engineering, yet it lacks any known capacity for independent action. Its gears and actuators show signs of tampering.
        \item \textbf{Thorne’s Testimony:} The accused swears he is innocent, claiming he was in another room when the murder occurred.
        \item \textbf{The Witness:} A factory worker insists he saw the automaton move on its own to deliver the fatal strike. But is he telling the full truth?
    \end{itemize}
\end{Example}

\noindent
With the evidence stacked against Thorne, the investigators must uncover what really happened in the workshop that night.

\subsection{Act 2: The Hidden Conflict}  
As the investigation deepens, the players discover that Bellamy’s death was not a simple case of mechanical failure—it was a carefully orchestrated act of sabotage.

\begin{Example}{Key Elements of Act 2}
    \begin{itemize}
        \item \textbf{Bellamy’s Discovery:} The victim had uncovered something significant—plans, a prototype, or a hidden truth that made him a target.
        \item \textbf{The Secret Rivalry:} The industrial elite of London are at war behind closed doors. Bellamy and Thorne were both entangled in a larger battle over technological supremacy.
        \item \textbf{The Sabotaged Automaton:} Someone tampered with the machine’s internal mechanisms. If the players investigate closely, they may find evidence of deliberate reprogramming or mechanical interference.
        \item \textbf{A Race Against Time:} The longer the investigators take, the more pressure mounts to convict Thorne. Influential figures want the case closed quickly, and the truth buried.
    \end{itemize}
\end{Example}

\noindent
By the end of Act 2, the investigators should have a suspect—but proving their guilt will require uncovering their true motive.

\subsection{Act 3: The Mastermind}

With all the pieces in place, the investigators must expose the true murderer before Thorne is sentenced.

\begin{Example}{Key Elements of Act 3}
    \begin{itemize}\raggedright
        \item \textbf{The True Killer:} A rival inventor? A corrupt businessman? Or someone from Thorne’s own inner circle?
        \item \textbf{The Motive:} Bellamy’s research, a dangerous secret, or industrial sabotage? What truth was worth killing for?
        \item \textbf{The Final Confrontation:} The players must gather the final proof, present their case, or prevent another murder before the truth is lost forever.
    \end{itemize}
\end{Example}

\subsection{Resolution: Justice or Cover-Up?}  
The players’ choices will determine the final outcome:

\begin{itemize}
    \item \textbf{If Thorne is cleared:} He is freed, but powerful enemies remain.
    \item \textbf{If the killer is exposed:} The consequences will depend on their connections—justice may not always be served.
    \item \textbf{If the truth is buried:} The industrial elite breathe a sigh of relief, but the players leave knowing they only scratched the surface of something far larger.
\end{itemize}

One thing is certain: the march of progress is unstoppable, but the cost of invention is often paid in blood.
\begin{WyrdScenarioHeading}[The Silent Courier]{The Silent Courier}
    \index{The Silent Courier}
    \index{Scenario!The Silent Courier}

    The investigators are drawn into the case when the body of \textbf{Henry Graves} is discovered in the early hours of the morning; his pockets turned inside out except for the strange, untouched letter. The local police dismiss it as a robbery gone wrong, but those with a keen eye know better.

    The players must follow the trail of clues left behind, track down those involved in the message's delivery, and decipher the meaning of the letter. But they are not the only ones searching for the truth—dangerous individuals are watching their every move, determined to keep the past buried.

    \subsection*{Premise} 
    A messenger is found dead in a foggy alley, clutching a letter sealed in an unknown cypher. The contents of the letter are clearly valuable—valuable enough to kill for. Who was the intended recipient, and what secret was worth a man's life?

    \subsection*{What Really Happened} 
    The messenger, Henry Graves, was delivering a coded message between two rival factions of a secret society. The letter contained evidence of a betrayal within their ranks. However, a third party, fearing exposure, intercepted the courier and silenced him before he could complete his task. The letter remains intact, but its sender and intended recipient remain a mystery—one the investigators must unravel before the killers strike again.
\end{WyrdScenarioHeading}



\subsection{Act 1: The Body and the\\Letter}  
The investigators arrive at the crime scene—a foggy alley where Henry Graves was found dead. The police have ruled it a botched robbery, but subtle inconsistencies suggest otherwise.  

\begin{Example}{Key Elements of Act 1}
    \begin{itemize}
        \item \textbf{Examining the Crime Scene:} Players can search for physical evidence—how was Graves killed? What does the positioning of his body suggest?
        \item \textbf{The Letter:} The only item left untouched in his possession, written in an unfamiliar cipher. Why was it spared when everything else was taken?
        \item \textbf{Witnesses and Leads:} The investigators may find someone who heard or saw something—a vagrant, a night watchman, or a fellow courier. Their accounts might be fragmented, but they hint at someone following Graves before his death.
        \item \textbf{The Silent Pursuers:} A subtle but key element—players may not realize it yet, but they are being watched. The moment they take an interest in the case, their names are added to the list of people who know too much.
    \end{itemize}
\end{Example}

\noindent
Once the investigators realize this was no ordinary mugging, the mystery broadens. Who was Henry Graves delivering the letter to, and what was so important that it was worth his life?

\subsection{Act 2: The Trail of Secrets}  
Following leads from Act 1, the investigators begin piecing together Graves' movements before his death. His route suggests he was in contact with powerful individuals who rarely leave behind traces.  

\begin{Example}{Key Elements of Act 2}
    \begin{itemize}
        \item \textbf{Tracking the Letter’s Origin:} Discovering who wrote the letter is just as crucial as finding its recipient. The players must investigate Graves' recent commissions and any known associates.
        \item \textbf{The Rival Factions:} As the investigation deepens, it becomes clear that the letter is tied to a schism within a secretive society. Who is working against whom, and what information was in the letter?
        \item \textbf{Attempts to Stop the Investigation:} By this point, the players will have drawn attention. Shadowy figures may approach them with offers, threats, or outright attempts on their lives.
        \item \textbf{A Key Betrayal:} An NPC the investigators have relied on may be compromised, leading to a moment where the players question who they can trust.
    \end{itemize}
\end{Example}

\noindent
At the end of Act 2, the players should be closing in on the recipient of the letter. However, the conspiracy is still one step ahead, and the final piece of the puzzle remains missing—the full contents of the letter.

\subsection{Act 3: The Truth Unveiled}  
The final act sees the investigators face their most dangerous challenge yet. The true nature of the letter is revealed, and they must decide what to do with it.

\begin{Example}{Key Elements of Act 3}
    \begin{itemize}
        \item \textbf{The Letter’s Recipient:} At last, the players find the person who was meant to receive the letter. But will they be an ally, or do they have their own agenda?
        \item \textbf{The Real Enemy:} The true mastermind behind the murder emerges—was it a rogue faction leader, a powerful noble, or someone much closer than the players realized?
        \item \textbf{The Final Confrontation:} Whether it’s a chase, a duel of words, or a desperate escape, the players must navigate the resolution carefully. The wrong choice could cost them their lives.
        \item \textbf{The Fate of the Letter:} The letter contains damning evidence—exposing corruption, revealing a dangerous truth, or holding the key to an even larger mystery. What the players choose to do with it will shape the story’s aftermath.
    \end{itemize}
\end{Example}

\subsection{Resolution: The Consequences of Truth}  
The outcome of the scenario depends on how the investigators handle the final confrontation and the letter itself:

\begin{itemize}
    \item \textbf{If the letter is destroyed:} The conspiracy continues, but the players may have made powerful enemies or secret allies.
    \item \textbf{If the letter is revealed:} The truth spreads, but at what cost? Some factions may fall, others may rise, and new threats may emerge.
    \item \textbf{If the letter is delivered to its intended recipient:} The consequences will depend on who the recipient truly is and whether they were acting in good faith.
\end{itemize}

No matter the resolution, one thing is certain: \textbf{The Silent Courier} was only the beginning.


%% TODO: Add three more adventures
