\section{Gear}
\index{Gear}

Unlike other systems that track individual items, inventory weight, and resource management, \emph{The Wyrd Engine} keeps gear streamlined and abstract. Instead of worrying about encumbrance, ammunition, or minor supplies, characters only track \textbf{gear that truly matters}. This means that most mundane equipment is assumed to be available when reasonable, and only items that provide a mechanical or narrative advantage are recorded.

\subsection{Gear as Traits}
Gear in \emph{The Wyrd Engine} functions similarly to Traits. Instead of listing specific damage values or weight, an item has a \textbf{trait} that defines its benefit in play. 

Each character begins with \textbf{three pieces of gear}, and each piece should:
\begin{itemize}
    \item Provide a \emph{specific mechanical advantage} (e.g. \textbf{+2 bonus} to a relevant skill check). This works exactly the way that traits work; gear simply has traits.
    \item Offer a \emph{unique function} that enables new actions.
    \item Be \emph{narratively significant}—not just generic supplies.
\end{itemize}

See the ``Gear in \emph{The Grand Casebook}'' for examples of gear.

\begin{DndSidebar}[float=!t]{Gear in \emph{The Grand Casebook}}
\textbf{Detective’s Magnifying Glass}  
\emph{Gain +2 to Investigate when examining tiny details or analysing documents.}

\textbf{Clockwork Grappling Hook}  
\emph{Once per session, escape or reach a high place instantly.}

\textbf{Masterwork Dueling Pistol}  
\emph{Gain +2 to Shoot in one-on-one confrontations.}

\textbf{Encrypted Notebook}  
\emph{Allows the player to store complex cyphers or hidden information that only they can decode.}

\textbf{Hidden Blade}  
\emph{Use \textbf{Stealth} instead of \textbf{Fight} in a surprise attack.}

\textbf{Reinforced Trench Coat}  
\emph{Gain +2 to \textbf{Physique} when resisting blunt force trauma.}
\end{DndSidebar}

\subsection{Using Gear in Play}
Gear should not be micromanaged but used to define a character’s tools, specialities, and advantages. If an item logically fits a character’s concept—such as a detective having a notebook or a thief carrying lockpicks—it’s assumed to be available without taking up a slot. Only equipment that \emph{enhances gameplay} or \emph{creates narrative opportunities} should be explicitly listed.

\begin{DndComment}{Game Master Tip}
	If a player asks, “Do I have this item?” consider whether it fits their role and background. If it makes sense, they do. If it would provide a major advantage, it should be a tracked piece of gear with a trait.
\end{DndComment}

