\section{Basic Combat in The Wyrd Engine}
\index{Combat}

Combat in \textit{The Wyrd Engine} is designed to be fast, cinematic, and deadly. Instead of tracking minute details like hit points or exact damage values, the system focuses on the flow of action, player choices, and the consequences of combat. Depending on a game's setting and how large a part combat is for any particular scenario, you might want to adjust the combat rules, and for that, we refer to Chapter~\ref{chap: combat}. The combat rules described in this chapter are tailored for games where combat plays a minor role, so combat encounters should be quickly resolved in gaming time.

\subsection{Initiative: Who Acts First?}
\index{Combat!Initiative} 

Combat follows a structured yet flexible turn order:

\begin{DndReadAloud}{Determining Initiative}
	\begin{itemize}
    	\item \textbf{Surprise \& Readiness:} If one side is clearly ambushing the other, they act first.
	    \item \textbf{Tactical Positioning:} If no clear ambush is present, the GM determines turn order based on readiness.
	    \item \textbf{Rolling for Initiative:} If multiple characters are competing to act first, roll \textbf{4dF + Notice} (or another relevant skill). The highest roll acts first, with ties resolved narratively.
	\end{itemize}
\end{DndReadAloud}

\subsection{Taking Actions in Combat}
\index{Combat actions}

On their turn, a character can do the following:
\begin{itemize}
    \item \textbf{One primary action} (Attack, defend, use an item, complex manoeuvre)
    \item \textbf{One minor action} (Draw a weapon, reposition, open a door, shout a command)
    \item \textbf{Free actions} (Speaking briefly, minor environmental interactions)
\end{itemize}

\subsection{Attacking and Defending}
\index{Combat!Attacking}\index{Combat!Defending}

Attacks are resolved using opposed rolls:
\begin{DndReadAloud}{Attack Resolution}
	\begin{itemize}
    	\item The attacker rolls \textbf{4dF + their combat skill} (\textbf{Fight} for melee, \textbf{Shoot} for firearms).
	    \item The defender rolls \textbf{4dF + their defense skill} (\textbf{Athletics} for dodging, \textbf{Fight} for parrying).
	    \item If the attacker’s total exceeds the defender’s, the attack lands and deals damage.
	\end{itemize}
\end{DndReadAloud}

If the defender has a higher score than or equal to the attacker, the attack is averted, and no damage is dealt. Ties are always in the defender's favour. If the attacker scores higher, the damage inflicted on the defender is the attacker's score minus the defender's.

\begin{DndReadAloud}{}
	\subsubsection{Example Attack}
	Jonathan Blackwood swings a cane at an enemy thug. He rolls \textbf{4dF +2 (Fight)}, while the thug rolls \textbf{4dF +1 (Athletics)} to dodge. If Jonathan’s result is higher, the hit lands, otherwise, it is defended. If Jonathan rolls \FudgeRes{+++-} = 2 he gets a score of \textbf{+4} when combined with his \textbf{Fight} skill. If the thug then rolls \FudgeRes{+--0} = -1, giving him a score of \textbf{0} when combined with his \textbf{Athletics}, then Jonathan's score is higher, so he scores a hit, and the damage he inflicts is \textbf{+4 - 0 = +4}. The thug takes \textbf{+4} in damage.
\end{DndReadAloud}


\subsection{Damage: Stress and Wounds}
\index{Damage}
\index{Combat!Soaking up damage}

Instead of traditional hit points, \emph{The Wyrd Engine} uses \textbf{Stress}\index{Damage!Stress}\index{Stress} to represent minor injuries and \textbf{Wounds}\index{Damage!Wounds}\index{Wounds} for more serious, lasting harm. While not precisely the same as hit points, they function similarly—absorbing damage until a character is overwhelmed. However, by splitting damage into ``it's just a flesh wound'' stress boxes and actual wounds, we get a combat system as dynamic as hit point-based systems but with a roleplaying twist. The stresses work as generic hit points, but once a character takes a wound, the mechanics force the player and Game Master to determine which kind of wound and how the character is affected, as this goes into penalties for later actions.

\begin{DndReadAloud}{Stress and Wounds}
	\begin{itemize}
    	\item \textbf{Stress:} Represents minor setbacks, fatigue, or temporary injuries. These are automatically cleared after a fight.
	    \item \textbf{Wounds} come in three levels of severity. They take longer to heal, and adds penalties for future actions.
	\end{itemize}
\end{DndReadAloud}

Any damage inflicted must be soaked up by either \textbf{Stress} or \textbf{Wounds}. Each player has five \emph{Stress boxes}, \StressBoxes, and five \emph{Wounds boxes} where the wounds are split into three categories: three \textbf{Mild Wounds} (\MildWounds), two \textbf{Moderate Wounds} (\ModerateWounds), and one \textbf{Severe Wounds} (\SevereWounds). These boxes, combined, are where a character can soak up damage.

\begin{DndReadAloud}{}
	\\
	\Damage
\end{DndReadAloud}

Damage is soaked up by the boxes left-to-right and top-to-bottom, so the stress boxes will soak up the first five points of damage. After that, the following three damage points are inflicted as mild wounds, then the next two as moderate wounds, and finally, the character suffers a severe wound. If all stresses and wounds are ticked off, the character is \textbf{out of action} (see \textsc{Death and the End of Combat} on page~\pagereftext{core:death}).

%% TODO: Continue from here

Damage is converted to wounds similarly to how damage is converted to stress, using 1-to-1 or some other conversion factor. A \textbf{Mild Wound} absorbs the same as one stress box, a \textbf{Moderate Wound} the same as two stress boxes, and a \textbf{Severe Wound} the same as three stress boxes. In addition, when taking a \textbf{Wound}, one or more relevant skills are affected.

\begin{DndTable}[header=]{ll}
    \textbf{Wound Type} & \textbf{Effect} \\
    \hline
    \textbf{Mild Wound}  & -1 to relevant skill rolls \\
    \textbf{Moderate Wound} & -2 to relevant skill rolls \\
    \textbf{Severe Wound}  & -3 to all physical actions \\
\end{DndTable}

Damage does not have to fill a wound to invoke it. If a character already has a \textbf{Mild Wound} and takes \textbf{+1} in damage, it goes into the \textbf{Moderate Wound} \emph{and fills it}. Even though a \textbf{Moderate Wound} can take two in damage, it is inflicted as soon as it takes \emph{any} damage, and if the \textbf{+1} cannot go into a stress box or the \textbf{Mild Wound}, it goes into the \textbf{Moderate Wound}.

\begin{DndSidebar}[float=!t]{Example Wound}
Josephine Langley is shot during a gunfight and takes \textbf{+2} damage. She has no remaining Stress boxes, so she takes a \textbf{Moderate Wound} (since a \textbf{Moderate Wound} can soak up two damage while a \textbf{Mild Wound} cannot). The GM rules that the injury impairs her movement, applying a \textbf{-2} penalty to \textbf{Athletics} and \textbf{Fight} rolls.
\end{DndSidebar}

If all stress boxes are filled and all three wounds are taken, the character is out of action. What this means is up to the GM, but games are usually more fun if player characters live to fight another day. For one-shot games, it is okay to kill off characters towards the end of the session, but don't do it early in the game.

\subsection{Healing and Recovery}
\index{Healing}\index{Recovery}

\begin{itemize}
    \item \textbf{Stress} clears at the end of a scene.
    \item \textbf{Mild Wounds} require a short rest (a few hours) or first aid.
    \item \textbf{Moderate Wounds} require days of rest or professional medical care.
    \item \textbf{Severe Wounds} require weeks of rest, surgery, or supernatural healing (if applicable).
\end{itemize}

\subsection{Combat Maneuvers and Special Actions}
Instead of simply attacking, players can use tactical manoeuvres:

\begin{DndReadAloud}{Combat Maneuvers}
\begin{itemize}
    \item \textbf{Disarm:} Use Fight to knock a weapon from an opponent’s hands.
    \item \textbf{Grapple:} Use Fight vs. Athletics to restrain an enemy.
    \item \textbf{Push:} Use Athletics to shove an opponent into hazards.
    \item \textbf{Feint:} Use Deceive to trick an enemy into missing a defense.
    \item \textbf{Suppressing Fire:} Use Shoot to force enemies into cover.
    \item \textbf{Intimidate:} Use Provoke to demoralize foes.
\end{itemize}
\end{DndReadAloud}

\emph{The Wyrd Engine} does not have rules for all the myriad ways that actions can be used in combat, but the GM should generally try to convert an action into either an unopposed or opposed obstacle and let the outcome affect bonuses and penalties for future dice rolls.

\subsection{Weapons and Gear in Combat}
Weapons do not deal numeric damage but affect combat through \textbf{Traits}. Weapon traits work the same way as any gear trait and can be used when attacking or defending.

\begin{DndReadAloud}{Types of Weapon Traits}
\begin{itemize}
    \item \textbf{Weapons with Traits} grant \textbf{+2} in relevant situations (e.g., “Mastercrafted Rapier” gives +2 to Fight in duels).
    \item \textbf{Firearms} can inflict instant Wounds if the shot is well-placed.
    \item \textbf{Improvised Weapons} may impose a penalty unless the character is skilled in their use.
\end{itemize}
\end{DndReadAloud}

When a weapon's \textbf{Trait} adds to the attack of a character, it will indirectly affect the damage the attack is inflicting. More interesting uses of weapon traits give other advantages to their wielder.

\begin{DndReadAloud}{Example Weapon Traits}
\begin{itemize}
    \item \textbf{Fine Dueling Sabre} – \textit{+2 to Fight when dueling.}
    \item \textbf{Hidden Derringer} – \textit{Once per scene, draw a concealed firearm unnoticed.}
    \item \textbf{Reinforced Cane} – \textit{Can be used as both a weapon and a defensive tool.}
\end{itemize}
\end{DndReadAloud}

\subsection{Death and the End of Combat}\label{core:death}
When a character suffers a \textbf{Severe Wound} and takes further damage, they are at risk of death. The simplest choice here is to equate all damage boxes ticked off and character death, but this is not always the best option. It might be fine for nameless mooks the players are fighting but for player characters or important (or just interesting) NPCs, it is often more interesting to consider such a character \textbf{defeated} rather than \textbf{dead}.

Instead of killing off characters, take them captured. Beat them up and leave them for death. Anything \emph{interesting} that can still count as a defeat. Of course, depending on their situation and the setting you are playing in. A zombie is unlikely to capture a character, so true to the zombie genre, you might want to kill off characters there. A vampire, on the other hand, could start monologing about vampiric superiority for long enough that the character could be rescued. 

If you do consider the last wound as essentially death, you might still allow:
\begin{itemize}
    \item A final desperate action before succumbing.
    \item A chance to survive if an ally intervenes.
    \item A dramatic consequence, such as permanent injury.
\end{itemize}

Combat ends when one side is defeated, flees, or surrenders. Survivors must then deal with the consequences of their wounds, the choices they made, and the path ahead.

\begin{DndComment}{Game Master Tip}
If a player is at risk of death, consider narrative consequences rather than instant removal. A major wound or permanent injury can be more interesting than a sudden death.
\end{DndComment}

