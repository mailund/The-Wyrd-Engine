\chapter{The Wyrd Engine Core Mechanics}
\label{chap:core}

\DndDropCapLine{T}{he} Wyrd Engine is a lightweight, narrative-driven tabletop roleplaying system designed for quick character creation, streamlined play, and minimal bookkeeping. It aims to provide a simple yet flexible framework that new players can easily pick up while still offering enough depth to engage experienced groups. The system leans into storytelling and improvisation, ensuring that the mechanics never overshadow the unfolding drama of the game.

Unlike more complex RPG systems that emphasise character progression, detailed mechanics, and long-term development, the Wyrd Engine is built for episodic or one-shot adventures where characters are meant to be jumped into and played immediately. This makes it ideal for groups with varying levels of experience, casual game nights, convention settings, or groups that enjoy shifting between different settings and tones without committing to long-term character advancement. By focusing on scene-based resolution, simple skills and traits, and intuitive conflict resolution, the Wyrd Engine keeps the story moving forward while maintaining a satisfying level of challenge and tension.

While the system lacks deep specialisation mechanics, its flexibility allows players to create compelling, unique characters through traits, skills, and equipment that influence their play style. Success in the Wyrd Engine isn’t dictated by meticulous number-crunching but rather by player ingenuity, teamwork, and the creative use of their abilities. Every character is designed to be compelling and memorable right from the start, ensuring they have the tools to make an impact within the narrative. The result is a game that emphasises momentum, character-driven storytelling, and high-action scenarios without getting bogged down in excessive rules.

\section{Conflict resolution at a glance}
\index{Conflict resolution}

Whenever characters encounter an obstacle—be it an unsolvable riddle, a desperate struggle to escape a flooded sewer or a battle against a coven of deadly necromancers—they must find a way to overcome the challenge. Whether through wit, skill, or sheer determination, resolving conflicts is at the heart of the game, driving the story forward and shaping the fate of the characters.

With The Wyrd Engine, all conflict resolution follows the same pattern that combines \textbf{4dF} Fudge Dice, described \pagereftext{core:fudge-dice}, \textbf{Skills} described on \pagereftext{core:skills}, and \textbf{Traits} described on \pagereftext{core:traits}. You combine these three and compare them to a \textbf{Difficulty Levels (DR)}, described on \pagereftext{core:difficulty-levels}, and the result determines the outcome of a conflict.

\begin{Example}[Steps in conflict resolution]
	\begin{itemize}
		\item Roll four Fudge Dice (\textbf{4dF}).
		      Each die has \textbf{+ (plus), - (minus), and 0 (blank) faces}. Add up the plusses and minuses.
		\item The roll result is added to a relevant \textbf{Skill} modifier.
		\item If relevant, \textbf{Traits} can be applied as bonuses.
		\item The final result is compared against a \textbf{difficulty level (DL)} to determine success or failure:
		\begin{itemize}
			\item \textbf{4dF} + \textbf{Skill} + \textbf{Trait} $>$ \textbf{DL} (Success)	
			\item \textbf{4dF} + \textbf{Skill} + \textbf{Trait} $=$ \textbf{DL} (Tie)
			\item \textbf{4dF} + \textbf{Skill} + \textbf{Trait} $<$ \textbf{DL} (Failure)
		\end{itemize}
	\end{itemize}
\end{Example}

A \textbf{Tie}\index{Tie} will usually qualify as a success, but the GM may decide that the outcome is a partial success or a compromise. This is up to the GM's discretion and should be based on the context of the situation and add minor complications to the success.

These steps will always be the general pattern for resolving conflicts, only differing in which skills and traits are involved, how the difficulty level is determined, and what the consequences of success or failure will be.

\section{Fudge dice (4dF)}\index{Fudge dice}\index{4dF}\label{core:fudge-dice}

Fudge dice are dice that can give you one of three values: \FudgeDie{-}, \FudgeDie{}, or \FudgeDie{+}. You can buy this type of dice if you want, but you can also use any normal six-sided die and declare 1 and 2 to be \FudgeDie{-}, 3 and 4 to be \FudgeDie{}, and 5 and 6. to be \FudgeDie{+}.

Whenever we roll dice in The Wyrd Engine, we roll four such dice (we write it as 4dF) and we add up the result, where \FudgeDie{-} counts as -1, \FudgeDie{} as 0, and \FudgeDie{+} as +1. So, for example
	\FudgeRes{++-0} = +1 + 1 - 1 + 0 = 1
	and 
	\FudgeRes{-+--} = -1 + 1 - 1 - 1 = -2.

Using 4dF gives us a distribution of outcomes that look like this:
\Graph[4dF results]{stats/4dF}

You are unlikely to roll the extremes; you should expect to hit $\pm$4 about 1\% of the time (each)---about one time out of a hundred rolls, you should get +4, and about one time in a hundred, you should get -4. You expect to get an outcome above +3 or below -3 about 6\% of the time (each)---about one in twenty for each.

Another way to visualise the outcome of a 4dF is as the chance you have of rolling higher than some threshold value:

\Graph[Success probabilities]{stats/4dF-success}

It is impossible to roll lower than \textbf{-4} with a \textbf{4dF} roll, but you can tie with it (with probability 1/81).
To roll higher than or equal to \textbf{-3}, you just have to avoid \FudgeRes{----}, and this outcome only happens one out of 81 rolls. To roll equal to \textbf{+4} you \emph{have} to roll \FudgeRes{++++}, which also happens with probability 1/81. To roll \emph{higher} than \textbf{+4} is impossible, since this it the highest value you can roll.

In conflict resolution, this graph is relevant as it tells us how likely it is for a character without the necessary skills and relevant traits to succeed at any given difficulty level. It is this graph of success probabilities you should have in mind when setting difficulty levels, and we return to it later. The graph, as it is here, is the probabilities you get if you had to rely on 4dF alone, without any skills or traits.


% !TeX root = ../../../wyrd.tex

\section{Adapting Skills}\label{toolbox:sec:adapting-skills}
\index{Adapting Skills}
\index{Skills!Adapting}
\index{Skills}

Skills are the backbone of most mechanical interactions in \wyrd, and adapting them is one of the most direct ways to tailor the system to your setting. The core rules offer a streamlined and versatile skill list, but you are encouraged to reshape it to fit the needs of your world. Whether you're adjusting for a specific genre, introducing new types of conflict, or simply looking to give characters more specialised roles, modifying the skill system allows you to bring tone and theme to the forefront. In the sections that follow, we explore the different ways you can adapt the skill list—by changing its level of detail, adjusting how skills function, and incorporating setting-specific elements.

\subsection{Custom Skill Effects}\label{toolbox:custom-skill-effects}
\index{Skills!Custom Effects}

Although most skills work the same way—rolling to overcome an obstacle or create an advantage—you can add unique effects or permissions for specific skills. For instance, a \textit{Fear} skill might be usable as an attack in horror-themed games. A \textit{Magic} skill might allow creating temporary advantages with elemental force. A \textit{Politics} skill might interact with faction reputation.

These additions can make your skill list feel more alive and setting-specific, especially when certain skills enable special actions others cannot perform.

\begin{CommentBox}{Skill-Based Worldbuilding}
    If only characters with the \textit{Occult} skill can perceive spirits, that says something about your world.  
    If everyone has \textit{Hacking}, it says something else. Use your skill list to show what’s normal—and what’s extraordinary.
\end{CommentBox}

When it comes to skills, also keep in mind that NPCs don’t always need to follow the same rules as player characters. In fact, giving NPCs unique abilities or expanded uses of existing skills can reinforce the tone of your setting and elevate the drama.
    
For example, in an Urban Fantasy game where the player characters are ordinary humans working for a secret government agency, you might restrict players from using magic. However, supernatural NPCs—such as the Fay—could wield a distinct \textbf{Magic} skill to bend reality, conjure illusions, or enchant the environment. This sharp contrast reminds players that they're operating in a world full of forces they don’t fully understand or control.
    
Alternatively, you can give NPCs unique applications of skills the players do have. A Fay creature might use \textbf{Deception} not merely to lie, but to weave illusions or subtly reshape perception—while the same skill, when used by a player, is limited to mundane falsehoods. This approach lets you reinforce the supernatural as uncanny and dangerous, even when using familiar mechanics.

\subsection{Cultural and Setting-Specific Skills}
Some games thrive when skill lists reflect cultural knowledge, world assumptions, or unique technologies. For example, a post-apocalyptic setting might replace \textit{Technology} with \textit{Scavenging} or \textit{Repurposing}. A faerie-tale world might include \textit{Glamour}, \textit{Wyrd}, or \textit{Bargaining} as standalone skills.

These decisions bring flavour and cohesion to your skill list, but be mindful of how often a given skill will be useful. A skill that only applies once or twice in a campaign may be better represented as a Trait instead.

\subsection{Hybrid and Conditional Skills}
Sometimes a skill might straddle two roles. A skill like \textit{Survival} might be used for physical endurance, wilderness navigation, and resisting fear in certain settings. You can define conditional uses for skills that serve multiple narrative purposes—just be clear with your players about what each skill covers.

You can also let certain Traits expand how a skill functions. For instance, a Trait like \textit{Soldier’s Discipline} might let a player use \textit{Will} to resist physical intimidation or pain, blurring the line between mental and physical endurance.

\subsection{Buying Skills}
\index{Skills!Buying}
\index{Skills!Skill Points}
\index{Skills!Skill Budget}

The first adjustment you can make is to the \textbf{skill budget}—the number of skill ranks available to characters during creation. The core rules assume that player characters have one +3 skill, two +2 skills, and three +1 skills. But instead of fixing the distribution of skills, you can fit the \emph{skill budget}. The default skill lists gives a total of 10 points to distribute across the skill list. You can distribute those 10 points however you like, going for fewer high-level skills or more low-level skills. 

The default budget is 10 points, which allows for a range of character builds and play styles. However, you can increase or decrease this number based on your game’s needs. In many of the adjustment ideas below, we’ll suggest a new budget that fits the changes you’re making.

\subsection{Levels of Detail}\label{toolbox:detailed-skill-lists}
\index{Skills!Granularity}
\index{Skills!Complexity}

The more detailed your skill list, the more mechanical variety your characters can express—but this comes at the cost of speed and simplicity. A short, broad skill list is ideal for games that emphasise narrative flow and improvisation. A longer, more detailed list works better for tactical play, complex investigations, or games where niche expertise matters.

The level of detail you choose will influence character identity, spotlight moments, and the kinds of stories your game is best equipped to tell. A one-shot game about magical investigators might keep things simple with a single \textit{Magic} skill, while a long-running campaign about academic wizards could break that down into \textit{Rituals}, \textit{Alchemy}, \textit{Runes}, and \textit{Summoning}.

\begin{CommentBox}{Quick Tip: Choose Your Skill Scope}
    \begin{itemize}
        \item Use \textbf{broad skills} like \textit{Combat}, \textit{Technology}, or \textit{Magic} for high-level, fast-paced games.
        \item Use \textbf{narrow skills} like \textit{Blades}, \textit{Firearms}, or \textit{Arcane Lore} for games focused on detail, strategy, or realism.
        \item Blend both by starting with broad skills and expanding only the ones that matter to your setting or your players.
    \end{itemize}
\end{CommentBox}

\subsubsection{Example: Broad vs Specific Skills}
\index{Skills!Example}

The detail levels of skills can greatly affect how characters differentiate themselves. In a game with a broad skill list, characters may feel similar even if they have different backgrounds or personalities. This can lead to a lack of distinctiveness in character roles and abilities.

Consider two characters in a swashbuckling adventure game: \textbf{Captain Elise Vaunt} is a charismatic privateer and master duelist. \textbf{Professor Thaddeus Wren} is a scholarly gentleman with a dark past in the royal navy.

If your game uses a broad skill list, both characters might look quite similar:

\begin{SkillsBox}
    \Expert  & Combat \\
    \Skilled & Persuade \\
    \Novice  & Lore
\end{SkillsBox}

You would probably \emph{play} the characters differently, but the \emph{mechanics} would be almost identical. In any situation where one of the characters is likely to succeed of fail, the other is as well. They might have different backgrounds and personalities, but their skills are so broad that they can both do everything equally well. This can lead to a lack of distinctiveness in character roles and abilities.

But with a more granular skill list, their differences become much clearer:

\vspace{0.5\baselineskip}
\noindent
\textbf{Captain Elise Vaunt:}
\begin{SkillsBox}
    \Expert  & Swords \\
    \Skilled & Intimidate \\
    \Novice  & Navigation
\end{SkillsBox}

\noindent
\textbf{Professor Thaddeus Wren:}
\begin{SkillsBox}
    \Expert  & Pistols \\
    \Skilled & Etiquette \\
    \Novice  & History
\end{SkillsBox}

Now, their roles in the story and in gameplay feel much more distinct. Elise dominates in close combat and commands the deck with fearsome presence. Thaddeus excels in duels at a distance and navigates social intrigue with ease. The characters feel more unique because the skills are more focused—and that clarity can help both players and GMs build scenes where each can shine.


\subsection{Adjusting the Skill Budget}
\index{Skills!Budget}
\index{Skills!Detailed Skill Lists}

When you increase the granularity of your skill list, you should consider expanding the \textbf{skill budget}—that is, the number of skill ranks available to characters during creation. The more specific your skills become, the more ranks characters need to remain competent across the same range of activities.

For example, if a broad skill like \textit{Combat} is split into \textit{Swords}, \textit{Pistols}, and \textit{Unarmed}, a character who would have taken \textit{Combat} at +3 might now need to spread their ranks across multiple areas to reflect the same breadth of capability. Without increasing the number of skill picks available, characters become artificially limited—not because of concept or balance, but because of mechanical compression.

The goal of increasing detail isn’t to make characters weaker, but to make their abilities more specific. To maintain the same level of competence, you’ll want to give players more points to distribute when using a longer or more detailed skill list. This ensures characters feel just as capable, while allowing their specialities and limitations to emerge more clearly in play.

As a rough guideline, you might increase the total number of ranks allowed by +2 to +4 when moving from a broad list of around 10 skills to a more detailed list of 15 to 20. You may also want to slightly raise the cap for individual skills (e.g., from +3 to +4) if you want players to be able to achieve strong specialisation without sacrificing versatility.

\subsubsection{Example: Broad vs Detailed Skill Budgets}
\index{Skills!Skill Budget}
\index{Skills!Detailed Skill Lists}

Let’s compare how the same character concept can be expressed using different skill list granularities and budgets. We'll use the example of \textbf{Aria Flint}, an elite thief with a flair for infiltration and social deception.

\noindent\textbf{Using a Broad Skill List (10 Points):}

\begin{SkillsBox}
    \Expert  & Stealth \\
    \Skilled & Deceive, Athletics, Burglary \\
    \Novice  & Notice
\end{SkillsBox}

This version of Aria is quick, sneaky, and good at lying and lockpicking. With only five skill entries, she’s mechanically lean and easy to play, but her abilities are fairly generalised.

\vspace{0.5\baselineskip}

\noindent\textbf{Using a Detailed Skill List (20 Points):}

\begin{SkillsBox}
    \Expert  & Sneaking, Disguise \\
    \Skilled & Climbing, Lockpicking, Pickpocketing, Deception, Escape Artist, Urban Navigation \\
    \Novice  & Observation, Balance
\end{SkillsBox}

With 20 points to distribute across a more granular list, Aria’s skills now paint a much more specific picture. We learn that she’s not just a burglar—she’s a nimble climber, an expert in disguise, and an agile escape artist. This version allows for more detailed storytelling and spotlight moments, but would be unwieldy without the expanded skill budget.

\begin{CommentBox}{Design Principle: Equal Power, More Detail}
    The detailed version of Aria isn't more powerful—she's just more specific. Both builds cover the same narrative ground, but the granular list gives finer control over which exact abilities she excels in. To keep the experience fair and functional, the skill budget increases in proportion to the level of detail.
\end{CommentBox}

A word of caution if you take this route: the more skills you add, the more complex character creation becomes. Picking the default six skills out of a list of maybe 15 is a lot easier to do than picking 10 out of a list of 50. If you have a long list of skills, consider how many of them are likely to be used in play. If you are running a one-shot game, make sure that each skill is likely to come up at least once. If you are running a long-term campaign, consider how many skills are likely to be relevant to the characters' backgrounds and the story you want to tell.


\subsection{Default Skill Levels}
\index{Skills!Default Levels}
\index{Skills!Untrained}

In the standard \wyrd rules, characters are assumed to have a \textbf{default skill level of 0} in any skill they do not explicitly take. This makes sense when using a broad skill list—most characters can attempt common actions like running, persuading, or shooting without specialised training. A +0 represents baseline competence, where a character relies on raw talent or experience rather than honed expertise.

However, as your skill list becomes more detailed, this assumption may no longer hold. In a setting with specific and technical skills, certain tasks may reasonably require training just to attempt. For instance, it might be fair to assume that most characters can drive a ground vehicle (\textit{Driving} at +0), but not everyone knows how to perform mid-flight repairs on a damaged starship (\textit{Hyperdrive Repair} might default to \textbf{–4}, or even be unavailable without a relevant Trait).

You can adjust default values on a skill-by-skill basis. Ask yourself: could an untrained person even attempt this? If so, what’s their chance of success? If not, consider requiring a Trait, narrative justification, or a different approach entirely. You might also group certain advanced skills under prerequisites or special permissions to make their rarity explicit.

\begin{CommentBox}{Guiding Defaults}
    \begin{itemize}
        \item \textbf{+0:} Common knowledge or intuitive actions (\textit{Climb}, \textit{Charm}, \textit{First Aid}).
        \item \textbf{–2:} Specialist tasks with some public awareness (\textit{Surgery}, \textit{Codebreaking}, \textit{Forgery}).
        \item \textbf{–4:} Highly technical or dangerous skills unlikely to be attempted without training (\textit{Hyperdrive Repair}, \textit{Necromancy}, \textit{Nuclear Engineering}).
        \item \textbf{N/A:} Cannot be attempted at all without a specific Trait or background.
    \end{itemize}
\end{CommentBox}

Customising default skill levels adds flavour and helps enforce genre expectations. In a gritty cyberpunk world, even a street-savvy hacker might have no idea how to pilot a corporate dropship. In a mythic fantasy setting, few outside the high temples would dare attempt divine rituals. Use defaults not to punish players, but to shape the world and encourage meaningful choices in character creation.

\subsection{Tiered Skills \& Skill Progression}\label{toolbox:skill-progress}
\index{Tiered Skills}
\index{Skills!Tiered}
\index{Skill progression}
\index{Skills!Progression}
\index{Skills!Prerequisites}

Some settings benefit from \textbf{tiered} skill structures, where broad foundational skills unlock or govern access to more specialised ones. This structure fosters a stronger sense of mastery, progression, and narrative depth—particularly in worlds shaped by formal education, martial training, arcane study, or professional disciplines.

For example, a game might include a general \textbf{Combat} skill, which branches into sub-skills such as \textbf{Melee}, \textbf{Ranged}, and \textbf{Unarmed}. Alternatively, an academic setting might use a broad \textbf{Lore} skill that leads into more focused areas such as \textbf{Occult}, \textbf{Alchemy}, or \textbf{History}. This layered approach encourages specialisation while preserving access to wider fields of knowledge or training.

Advanced skills should offer meaningful additions to their foundational counterparts—otherwise, they risk feeling redundant. This could be as simple as enabling a character to perform tasks beyond the scope of the basic skill: for instance, \textbf{Lore} might cover mundane knowledge, while only the more specialised \textbf{Occult} skill allows understanding of the supernatural.

Alternatively, advanced skills may be more powerful than their parent skills, but with a narrower focus. These specialised skills can provide a greater bonus to specific actions than the broader skill, though only under limited circumstances where their expertise applies.

For example, the general \textbf{Fight} skill might have specialisations such as \textbf{Bladework} for sword fighting and \textbf{Martial Arts} for unarmed combat. A character with \textbf{Fight +1} would gain a \textbf{+1} bonus to all close-combat actions, while someone with \textbf{Bladework +2} would gain a larger bonus—say, double its value—for sword fighting actions, for a total of \textbf{+4}. Similarly, \textbf{Martial Arts +2} would provide \textbf{+4} to unarmed combat actions.

You could also allow the base and advanced skills to stack. For instance, a character with \textbf{Fight +1} and \textbf{Bladework +2} might receive \textbf{+1} to all close-combat actions, but gain a total of \textbf{+5} when using a sword—combining the general bonus with the specialised one.

\subsubsection*{Using Prerequisites}

Tiered skills can be structured so that a character must possess a prerequisite skill before taking the specialised version. For instance:

\begin{itemize}
    \item To gain \textbf{Martial Arts}, the character must first have at least +1 in \textbf{Unarmed}.
    \item To study \textbf{Forbidden Lore}, the character must already possess \textbf{Lore} at +2 or higher.
    \item To take \textbf{Arcane Theory}, the character must have \textbf{Magic} and a related Trait (e.g., \textit{Gifted} or \textit{Apprentice of the Circle}).
\end{itemize}

This approach provides a natural sense of progression and helps reinforce the fiction—characters learn basics before advancing to deeper or more specialised knowledge. It also gives the GM tools to gate certain abilities, reserving them for more experienced characters.


\subsubsection*{Skill Caps}

In addition to acting as a prerequisite, a parent skill can serve as a \textbf{cap} for related sub-skills. For example:

\begin{itemize}
    \item A character cannot raise \textbf{Alchemy} higher than their rank in \textbf{Lore}.
    \item \textbf{Bladework} cannot exceed the character’s \textbf{Combat} skill.
\end{itemize}

This maintains the hierarchy of skills and prevents characters from becoming disproportionately advanced in one area without first investing in the fundamentals.

\subsubsection*{Mechanical Implementation}

To incorporate tiered skills in your game, follow these guidelines:

\begin{enumerate}
    \item \textbf{Define Parent and Sub-Skills:} Clearly identify which skills require prerequisites and what their parent skills are.
    \item \textbf{Set Prerequisite Thresholds:} Decide whether the parent skill is merely required (e.g., must be taken at any rank) or if a specific threshold is needed (e.g., +2 or higher).
    \item \textbf{Apply Caps as Needed:} If using skill caps, specify that the sub-skill cannot be rated higher than its parent.
    \item \textbf{Track During Advancement:} When players advance their characters, ensure they meet all prerequisites before selecting new skills.
\end{enumerate}



\begin{CommentBox}{Tip: Use Tiered Skills to Tell a Story}
Tiered skills are more than just a mechanical tool—they're a storytelling device. When a character advances from \textit{Unarmed} to \textit{Martial Arts}, that tells us something about their growth and training. When a scholar unlocks \textit{Forbidden Lore}, it suggests a shift in worldview or a dangerous breakthrough. Use these moments as narrative milestones.
\end{CommentBox}

Tiered skills introduce additional complexity, so they should be used thoughtfully. They work best in settings where progression, discipline, or mastery are central themes. In one-shots or episodic games, the added complexity often outweighs the benefits of the extra nuance they provide. Not every game needs tiered skills—but when they suit the tone and setting, they can enrich both the mechanics and the narrative depth of your world.


\vspace{1\baselineskip}
That concludes our look at adapting the skill system. The core idea is this: use your skill list as a lever to shape tone, pace, and complexity. Add detail where it matters to your setting; strip it back where speed and clarity are more important. The more intentional your choices, the more your skill list will reinforce your game’s identity.



\section{Adapting Traits}\label{toolbox:sec:adapting-traits}
\index{Adapting Traits}
\index{Traits!Adapting}
\index{Traits}

Traits are what make your character unique. They define your capabilities, limitations, and how you interact with the world. In \wyrd, traits are flexible narrative elements that reflect your identity, backstory, training, and personal style. More than just bonuses or modifiers, traits allow characters to break the normal rules of the setting, rewrite what is possible, and shape the story around their distinctive strengths—or weaknesses.

This section provides guidance on how to adapt traits to suit different genres, campaign styles, or tone, as well as how to design your own traits in a balanced and engaging way.

\subsection{Narrative Permission}\index{Narrative Permission}

In \wyrd, traits usually just provide bonuses to skill rolls under specific circumstances, but they can also grant what is called \textbf{narrative permission}—the ability to do something within the fiction of the game without requiring further justification, rolls, or explanation. This means traits aren't just passive descriptors; they give characters concrete authority to act in certain ways that would otherwise require effort, planning, or approval.

For example:
\begin{itemize}
    \item A character with the trait \textbf{Master of Disguise} can change their appearance convincingly without needing special equipment or extended preparation. Even if other characters would normally need a skill check to disguise themselves, this trait allows the user to do so automatically under reasonable conditions.
    \item A character with \textbf{Fearless} doesn't need to roll to resist fear, intimidation, or supernatural dread. They are simply unaffected—unless the source of fear is so extreme that it might overwhelm even this trait.
    \item A trait like \textbf{Royal Bloodline} might grant access to noble courts, feudal privileges, or ancient knowledge simply because the character is part of a recognised lineage.
\end{itemize}

Narrative permission encourages fast, fluid play by cutting out unnecessary rolls and debates. It also rewards players for creating characters that shape the fiction in interesting ways. If you can do it because of a trait, the world acknowledges it—even if there’s no explicit mechanical bonus attached.

\subsection{Positive and Negative Traits}\index{Traits!Positive}\index{Traits!Negative}

Traits in \wyrd can be either \textbf{positive} or \textbf{negative}, also known as \textbf{advantages} and \textbf{disadvantages}. Both types serve narrative and mechanical purposes, and both can enrich a character.

\subsubsection*{Positive Traits}

Positive traits are the traits from the core mechanics and reflect talents, privileges, or other favourable qualities. These traits can represent training, innate ability, access to resources, social standing, supernatural gifts, or other narrative advantages. Some examples include:

\begin{itemize}
    \item \textbf{Uncanny Aim} – Your shots are unnaturally accurate; \textbf{+2} to \textbf{Shoot} rolls if you take a turn to aim.
    \item \textbf{Well-Connected} – You have an extensive network of contacts and allies; \textbf{+2} to \textbf{Contacts} when on your home turf.
    \item \textbf{Resilient Spirit} – You recover quickly from magical or emotional trauma; \textbf{+2} to \textbf{Resist} rolls against supernatural effects.
\end{itemize}

\subsubsection*{Negative Traits}

Negative traits reflect flaws, limitations, vulnerabilities, or narrative complications. They are not simply penalties; they are tools for drama and depth. These come in two forms, \textbf{narrative traits} and \textbf{penalty traits}. The former are traits that the GM can invoke to create complications during the game while the latter are the simple trait bonuses, just with negative bonuses.

Some examples include:

\begin{itemize}
    \item \textbf{Short Fuse} – You lose your temper easily, often to your own detriment; \textbf{Narrative Trait}---the GM can insist that a player acts out this trait.
    \item \textbf{One Arm} – You’ve adapted well, but certain physical tasks are still challenging; \textbf{-1} to \textbf{Atheletics} and \textbf{Fight}.
    \item \textbf{Marked by the Enemy} – Your presence is easily detected by a specific group or entity; \textbf{-2} to \textbf{Contacts} when the group is involved.
\end{itemize}

The simplest form of negative traits are the \textbf{penalty traits}, which simply impose a negative modifier to a specific skill or action. These are straightforward and easy to understand, but they can feel less engaging than narrative traits.

The more complex form of negative traits are the \textbf{narrative traits}, which are more flexible and can be used to create interesting complications. These traits can be invoked by the GM to create obstacles, but they also grant players opportunities to gain story spotlight, character development, or additional resources.

For novice players, penalty traits are often easier to grasp, as they are more straightforward and less abstract. However, narrative traits can be more rewarding for experienced players, as they allow for greater creativity and engagement with the story.

\subsection{Point Budgets and Trait Balance}

To balance traits during character creation, \wyrd uses a point-based trait budget. Players are given a set number of points to spend on traits (by default three). Positive traits cost points, while negative traits can \textbf{refund} points, allowing players to afford more powerful abilities at the cost of drawbacks.

For example:
\begin{itemize}
    \item A character might have 3 trait points by default.
    \item They take three positive traits that cost 1 points each, but wish to take a fourth trait as well.
    \item To afford the fourth trait, they take a negative trait worth -1 point, giving them 1 extra to spend.
\end{itemize}

The exact values can be customised by the GM depending on the campaign’s tone. More grounded settings might limit characters to 2 or 3 points; more heroic or high-fantasy games might offer 5 or more.

\subsection{Trait Cost}

The core rules give each player character three traits, corresponding to three trait points if each trait costs one point. However, the cost of traits can vary based on their power level, narrative significance, and the overall balance of the game. Here are some general guidelines for assigning point values to traits:

\subsubsection{1-Point Traits}
These represent standard abilities, modest advantages, or situational narrative permissions. They may grant a small mechanical bonus (such as a +1 to a specific type of roll), allow a character to bypass a minor obstacle, or introduce useful resources or contacts.

    \begin{itemize}
        \item \textbf{Night Vision} – You can see clearly in low light without penalty. Ignore penalties for dim or moonlit conditions.
        \item \textbf{Quick Draw} – You may draw or switch weapons as a free action, even when surprised.
        \item \textbf{Wealthy} – You have access to significant personal funds. Once per session, you may declare you have just the right equipment, item, or bribe.
        \item \textbf{Former Soldier} – Gain +1 to \textbf{Tactics} or \textbf{Fight} when acting in structured combat or following chain of command.
        \item \textbf{Trained Tracker} – Gain +1 to \textbf{Survival} or \textbf{Notice} when following trails or identifying signs of movement.
    \end{itemize}

\subsubsection{2-Point Traits}
These traits are more powerful or versatile. They may combine a mechanical bonus with a broad narrative effect, significantly alter the rules for a particular type of action, or grant rare abilities. Traits at this level often define a character’s archetype or signature role in the party.

    \begin{itemize}
        \item \textbf{Unstoppable} – Once per scene, you may ignore the effects of a \textbf{Wound} or a failed roll and continue acting as if you succeeded.
        \item \textbf{Arcane Initiate} – You may cast Rank 1 spells from a chosen magical discipline and sense nearby sources of arcane power.
        \item \textbf{Silver-Tongued} – Gain +2 to \textbf{Rapport} in social conflicts where charm or eloquence is relevant.
        \item \textbf{Combat Mastery} – Choose one weapon type. Gain +1 to \textbf{Fight} and treat all attacks with this weapon as one step harder to block or parry.
        \item \textbf{Psychic Sensitivity} – You can detect strong emotional states and mental influence. Gain +1 to \textbf{Empathy} when reading intent or mood.
    \end{itemize}

\subsubsection{3-Point Traits}
Reserved for potent abilities, unique narrative privileges, or traits that break core assumptions of the setting. These may represent supernatural powers, elite status, ancient artifacts, or other extraordinary capabilities. Most characters will not begin play with traits at this level unless the tone of the campaign allows it.

    \begin{itemize}
        \item \textbf{Immortal} – You cannot die from age or natural causes. You ignore the first deathblow once per session and return later, scarred but alive.
        \item \textbf{Chosen by the Fates} – Once per session, you may reroll any failed roll and treat a partial success as a full success.
        \item \textbf{Bound Djinn} – You possess a powerful spirit in servitude. Once per session, it can perform a miraculous feat (teleportation, destruction, protection).
        \item \textbf{Royal Mandate} – You are recognised as a true heir to a great throne. Gain +2 to \textbf{Command} when dealing with nobility or military forces, and demand safe passage through loyal lands.
        \item \textbf{Reality Bender} – Once per scene, you may alter a small piece of the world’s logic—create a door where there was none, change gravity, or make an object vanish.
    \end{itemize}

\subsubsection{-1 or -2 Point Traits (Drawbacks/Disadvantages)}

Negative traits can be used to gain additional trait points during character creation. A -1 trait introduces a recurring complication, social disadvantage, or mild limitation. A -2 trait should be impactful, with mechanical or narrative consequences that frequently come into play. Negative traits are a great way to build flawed but compelling characters and can help reinforce the tone of darker or grittier settings.

    \begin{itemize}
        \item \textbf{Chronic Pain} (-1) – At the start of each session, roll a die. On a 1 or 2, you suffer -1 to all physical actions for the rest of the scene.
        \item \textbf{Wanted by the Law} (-1) – You are pursued by local authorities. The GM may introduce pursuit, arrest, or bounty complications at any time.
        \item \textbf{Bad Reputation} (-1) – You suffer -2 to \textbf{Charm} or \textbf{Rapport} when dealing with anyone aware of your past.
        \item \textbf{Cursed} (-2) – Once per session, the GM may declare a roll fails dramatically, regardless of the result, due to a malevolent supernatural force.
        \item \textbf{Magical Addiction} (-2) – You must use a magical effect or spell each session or suffer a -2 to all mental actions until you do.
        \item \textbf{Enemy Faction Surveillance} (-2) – A powerful group is always watching you. The GM may introduce spies, traps, or threats in any location you visit.
    \end{itemize}

These examples are not exhaustive, and GMs are encouraged to create custom traits that suit the tone and style of their game. The key is to ensure that each trait is meaningful, impactful, and relevant to the character’s identity and role in the story.

If you allow players to purchase negative traits, establish clear guidelines for how they are used. Negative traits should never render a character unplayable or be treated as a form of punishment. Instead, they should introduce compelling complications, moral dilemmas, or recurring challenges that enhance character development and storytelling.

It is also wise to place a cap on the number of negative traits a player can take. This prevents characters from becoming either too flawed or mechanically overloaded with too many bonuses. Negative traits should matter just as much as positive ones, and the GM should feel empowered to invoke them during play. However, if a character has too many, it becomes difficult to give each one the attention it deserves.


\subsection{Adapting Traits to the Setting}

One of the greatest strengths of the trait system is its adaptability. Traits can be themed to suit the tone, genre, or even specific location of a game. Setting-specific traits can deepen immersion, reinforce tone, and give characters a unique connection to the world they inhabit. Consider the following examples:

\begin{itemize}
    \item \textbf{Gritty Detective Story}
    \begin{itemize}
        \item \textbf{Streetwise} – Gain +1 to \textbf{Contacts} or \textbf{Deception} when navigating criminal circles or shady neighbourhoods.
        \item \textbf{Chronic Insomnia} – You are always alert, even when others sleep. Gain +1 to \textbf{Notice} during night scenes or stakeouts, but recover \textbf{Fatigue} one step more slowly.
        \item \textbf{Undercover Cop} – You may assume a criminal identity without suspicion. Once per session, you may declare a prior undercover relationship with an NPC.
    \end{itemize}

    \item \textbf{Mythic Fantasy}
    \begin{itemize}
        \item \textbf{Dragon-Blooded} – You are resistant to fire and may breathe flame once per session as a magical attack (Rank 2).
        \item \textbf{Voice of the Gods} – Gain +2 to \textbf{Command} or \textbf{Rapport} when delivering divine proclamations or preaching in sacred places.
        \item \textbf{Cursed by Ice} – You are immune to cold and can freeze small amounts of water with a touch, but your presence chills the air and marks you as unnatural.
    \end{itemize}

    \item \textbf{Science Fiction}
    \begin{itemize}
        \item \textbf{Cybernetic Reflexes} – Gain +1 to \textbf{Initiative} and reduce the difficulty of reactions and evasive actions by 1.
        \item \textbf{Zero-G Training} – You do not suffer penalties for operating in low or zero gravity environments. Gain +1 to \textbf{Athletics} in microgravity.
        \item \textbf{Black Market Supplier} – You have access to rare or illegal goods. Once per session, declare that you “already have” a restricted item or contact for contraband.
    \end{itemize}
\end{itemize}

The GM may also provide a curated list of setting-specific traits to guide character creation or spark inspiration. Players are always welcome to propose their own traits, provided they align with the tone of the game and offer meaningful opportunities for roleplay or mechanical impact.

\subsection{Mechanical vs Narrative Traits}

Traits in \wyrd may offer mechanical benefits (e.g. bonus to rolls, rerolls, or new uses for skills), narrative permission (e.g. bypassing obstacles or gaining automatic success), or both. The most memorable traits usually have some narrative hook—even if their primary purpose is mechanical.

For example:
\begin{itemize}
    \item \textbf{Mechanical Only:} \textbf{Combat Reflexes} – Gain +2 to your first initiative roll in a conflict.
    \item \textbf{Narrative Only:} \textbf{Member of the Silver Order} – You are a recognised member of a knightly brotherhood and can call on their aid or protection.
    \item \textbf{Hybrid:} \textbf{Veteran Duelist} – You gain +1 to attack rolls with swords and may demand formal duels in civilised lands.
\end{itemize}

When designing your own traits, try to include a narrative angle that makes the character more vivid, even if the mechanical effect is simple.

\subsection{Design Guidelines for Custom Traits}

If you're creating your own traits, either as a player or a GM---and we think you should---consider the following checklist:

\begin{itemize}
    \item \textbf{Clarity:} Is the trait’s benefit or drawback clearly defined?
    \item \textbf{Consistency:} Does it follow the tone and logic of the setting?
    \item \textbf{Impact:} Will the trait meaningfully affect play without dominating it?
    \item \textbf{Drama:} Does it lead to interesting choices, complications, or character moments?
    \item \textbf{Permission:} What does this trait allow the character to do in the fiction that others can’t?
\end{itemize}

Traits are not meant to be exhaustive or exhaustive rules; they are shorthand for what makes your character extraordinary. Think of them as storytelling fuel.

\subsection{Optional: Trait Ratings}\index{Traits!Rated}\index{Rated Traits}\index{Trait ratings}

In some campaigns, the GM may allow \textbf{rated traits}, where a trait can be taken at multiple levels (e.g., \textbf{Keen Eyesight +1}, \textbf{+2}, or \textbf{+3}). Each level increases the potency of the trait, either by improving mechanical bonuses, enhancing narrative scope, or granting additional uses. In the trait budget, the cost increases with the rating. This system introduces a more granular level of character progression and can support high-powered or mechanically detailed styles of play.

While trait ratings add complexity, they can be useful in campaigns where characters are expected to specialise deeply, develop signature abilities over time, or build toward legendary status. However, rated traits are generally not necessary for most games, and GMs are encouraged to use them only if they suit the tone and pacing of the campaign.

\subsubsection*{Rated Trait Examples}
Here are some examples of traits that scale well with levels:

\begin{itemize}
    \item \textbf{Keen Eyesight +1/+2/+3} – You gain a bonus to all visual perception checks:
    \begin{itemize}
        \item \textbf{+1} to \textbf{Notice} when spotting hidden objects or distant movement.
        \item \textbf{+2} allows you to see in poor lighting and automatically spot hidden enemies within medium range.
        \item \textbf{+3} allows you to detect movement no one else can perceive, such as invisible figures or sniper reflections.
    \end{itemize}

    \item \textbf{Arcane Affinity +1/+2/+3} – Your control over magic improves:
    \begin{itemize}
        \item \textbf{+1} reduces the Fatigue cost of spells by 1 (minimum 1).
        \item \textbf{+2} allows you to reroll one failed spellcasting attempt per session.
        \item \textbf{+3} increases your effective spell rank by +1 for all purposes (e.g., overcoming resistances or effects).
    \end{itemize}

    \item \textbf{Tough as Nails +1/+2/+3} – You can endure physical punishment that would stagger others:
    \begin{itemize}
        \item \textbf{+1} grants +1 to resist Wounds from physical attacks.
        \item \textbf{+2} allows you to ignore the effects of your first Mild Wound in each session.
        \item \textbf{+3} treats all Wounds one step less severe (e.g., Moderate becomes Mild).
    \end{itemize}
\end{itemize}

\subsubsection*{Design Guidelines}
When designing rated traits, keep the following in mind:

\begin{itemize}
    \item \textbf{Scaling Should Be Linear or Thematic.} Avoid exponential power creep. Each level should be a meaningful but manageable improvement.
    \item \textbf{Cap Levels Appropriately.} Most rated traits should max out at +3. Traits beyond that may unbalance the game or blur the line between Traits and narrative powers.
    \item \textbf{Costs Should Rise Accordingly.} A simple cost structure is 1 point per level, but GMs may require higher costs at higher levels (e.g., +2 costs 3 points total, +3 costs 5).
    \item \textbf{Narrative and Mechanical Scaling.} Consider not only bonuses to rolls but increased narrative reach—more uses per session, greater influence, or wider applicability.
\end{itemize}

\subsubsection*{Use Cases}
Rated traits work well in the following types of campaigns:

\begin{itemize}
    \item \textbf{Long-form campaigns} with extended character advancement.
    \item \textbf{High-powered settings} where legendary figures, elite soldiers, or demigods walk the world.
    \item \textbf{Point-buy campaigns} where players want fine-tuned control over power scaling.
    \item \textbf{Settings with prestige paths, guild ranks, or magical mastery} that make progression feel earned.
\end{itemize}

In more narrative or rules-light campaigns, simpler traits with fixed effects are often sufficient. Use rated traits when you want mechanical depth, tactical variation, or rewarding progression that grows with the character’s journey.

\begin{GmTips}
    Not all traits scale well. Only use rating levels for traits where each level clearly improves play in a consistent and balanced way. If you find players always taking a trait to maximum level, consider whether it’s too efficient or undercosted.
\end{GmTips}

\vspace{1\baselineskip}
Traits are the lens through which your character sees the world—and how the world responds in turn. Whether they define supernatural power, deep flaws, specialised training, or noble lineage, they shape every moment of play. Use them boldly and creatively to make characters that live, struggle, and shine.


\section{Gear}
\index{Gear}

Unlike other systems that track individual items, inventory weight, and resource management, \wyrd keeps gear streamlined and abstract. Instead of worrying about encumbrance, ammunition, or minor supplies, characters only track \textbf{gear that truly matters}. This means that most mundane equipment is assumed to be available when reasonable, and only items that provide a mechanical or narrative advantage are recorded.

\subsection{Gear as Traits}
Gear in \emph{The Wyrd Engine} functions similarly to Traits. Instead of listing specific damage values or weight, an item has a \textbf{trait} that defines its benefit in play. 

\wyrd gear should:
\begin{itemize}
    \item Provide a \emph{specific mechanical advantage} (e.g. \textbf{+2 bonus} to a relevant skill check).
    \item Offer a \emph{unique function} that enables new actions.
    \item Be \emph{narratively significant}—not just generic supplies.
\end{itemize}

Notice that the first two requirements closely resemble the description of traits. This is intentional, as it allows gear to have game mechanic effects while reusing the same rules already introduced.

\begin{CommentBox}{Example Gear}
	\textbf{Detective’s Magnifying Glass}  
	\emph{Gain +2 to Investigate when examining tiny details or analysing documents.}

	\textbf{Clockwork Grappling Hook}  
	\emph{Once per session, escape or reach a high place instantly.}

	\textbf{Masterwork Dueling Pistol}  
	\emph{Gain +2 to Shoot in one-on-one confrontations.}

	\textbf{Encrypted Notebook}  
	\emph{Allows the player to store complex cyphers or hidden information that only they can decode.}

	\textbf{Hidden Blade}  
	\emph{Use \textbf{Stealth} instead of \textbf{Fight} in a surprise attack.}

	\textbf{Reinforced Trench Coat}  
	\emph{Gain +2 to \textbf{Physique} when resisting blunt force trauma.}
\end{CommentBox}

\subsection{Using Gear in Play}
Gear should not be micromanaged but used to define a character’s tools, specialities, and advantages. If an item logically fits a character’s concept—such as a detective having a notebook or a thief carrying lockpicks—it’s assumed to be available without taking up a slot. Only equipment that \emph{enhances gameplay} or \emph{creates narrative opportunities} should be explicitly listed.

The trait-like behaviour of gear can also serve a second purpose in \emph{The Wyrd Engine}: Gear provides a way to boost characters abilities---quite substantially---by \textbf{+2} bonuses whenever the gear's requirements are met. For advancing characters when preparing them for a battle with the final boss of a scenario, a Game Master can gift the players with increasingly powerful gear as rewards for minor battles. Using gear is a simple way to handle character advancement in \emph{The Wyrd Engine}.

Once player characters start relying on such powerful items, a Game Master has a second trick to add excitement: unlike traits, gear can be taken away again. Recovering stolen gear necessary for the final confrontation is an excellent way to add side-quests to a game session.

\vspace*{\fill}

\begin{GmTips}
	If a player asks, “Do I have this item?” consider whether it fits their role and background. If it makes sense, they do. If it would provide a major advantage, it should be a tracked piece of gear with a trait.
\end{GmTips}

\vspace*{\fill}
\begin{center}
    \includegraphics[width=\linewidth]{img/separt/gear-logo}
\end{center}
\vspace*{\fill}

% Switching column with slightly nicer balancing
\end{multicols}
\begin{multicols}{2}
\section{Difficulty Levels}
\index{Difficulty levels}

While \emph{The Wyrd Engine} uses a simple resolution mechanic, it is important to establish how difficult a given action is. The Game Master determines the \textbf{Difficulty Rating (DR)} based on the complexity of the task, the environment, and any obstacles the characters may face.

\subsection{Passive Opposition}
\index{Passive opposition}
The \textbf{Difficulty Rating (DR)} represents the challenge level of a task. The simplest tasks involve no active opposition—where success or failure is determined solely by the character’s own abilities. This could be deciphering an ancient cipher, scaling a rocky cliff, or crafting a delicate mechanism—situations where the only obstacle is the task itself, rather than an opposing force.

In these cases, the player rolls \textbf{4dF} + their \textbf{Skill Modifier} and applies any relevant \textbf{Trait} or \textbf{Gear bonus} (Gear Traits). If the total meets or exceeds the DR, the action succeeds.

The GM determines the difficulty rating based on two factors: how inherently challenging the task is and how critical it is to the game’s progression. A well-balanced difficulty keeps the players engaged—offering real challenges without creating dead ends. While setbacks can enrich the story, a GM should never impose an insurmountable barrier that halts progress entirely. Instead, every challenge should be an opportunity for clever thinking, teamwork, and dramatic tension.

The following table can guide you in determining the difficulty rating for a task:

\begin{DndTable}[header=Difficulty Levels in \emph{The Wyrd Engine}]{lX}
    \textbf{Difficulty Rating} & \textbf{Example Task}\\
    \hline
    \Trivial & A task so easy that failure is nearly impossible (walking across a stable floor, recalling your own name). \\
    \Simple & A straightforward action requiring minimal effort (identifying a common herb, climbing a ladder). \\
    \Easy & A minor challenge that most people can accomplish without effort (jumping over a puddle, recalling common knowledge). \\
    \Basic & An ordinary action requiring some attention (spotting a misplaced item, balancing on a narrow beam). \\
    \Challenging & A moderate test of skill or effort (spotting a hidden compartment, climbing a wooden fence). \\
    \Difficult & A task requiring training or experience (tracking footprints in the rain, persuading a sceptical guard). \\
    \Formidable & A demanding task that pushes a character’s limits (picking a complex lock under pressure, leaping between rooftops). \\
    \Arduous & A near-impossible feat requiring mastery (detecting a forged document at a glance, sniping a target from extreme range). \\
    \Extreme & A task on the edge of human capability (convincing a lifelong enemy to trust you, performing surgery in total darkness). \\
    \Impossible & A superhuman achievement defying all odds (dodging bullets mid-air, convincing an ancient dragon to surrender). \\
\end{DndTable}

For levels up to \Basic, rolls are usually unnecessary unless dramatic tension is involved. For characters with appropriate skills, \Basic tasks can also be handled without rolls.

We can superimpose the difficulty levels on the 4dF success rate graph to directly visualise how difficult it will be with just dice rolls to reach a given level:

\begin{center}
	\includegraphics{stats/4dF-DR.pdf}
\end{center}

The graph tells us that even \Trivial tasks can fail if you are unskilled and unlucky enough, and \Challenging tasks will fail a third of the time for someone without the necessary skills.

Adding skills effectively shifts the difficulty levels. When playing the game, we add skill levels to the 4dF rolls, as this is the easiest way to calculate the result, but when setting difficulty levels, it is easier to think in terms of how difficult an unskilled character would find a task, and then shift the difficulty levels down by one for each skill level a character has.

A skill level of \Novice adds one to the 4dF, which effectively shifts the difficulties down by one. If we are adding \textbf{+1} to a roll, the unmodified range of \textbf{-4} to \textbf{+4} for a \Untrained character instead becomes the shifted range of \textbf{-3} to \textbf{+5}, for example. With this switch, the difficulty with which a \Novice character hits a \Challenging level will be the same as if he only had to reach the \Basic level.

\begin{center}
	\includegraphics{stats/shifted-DR.pdf}
\end{center}

A \Basic task, which has a 2/3 chance of success for an \Untrained character will be a success one out of twenty for a \Novice and a guaranteed success for an \Expert character. An \Extreme task, which will be impossible for an \Untrained and not much easier for a \Novice, has a one-in-five chance of success for an \Expert. Add in a \textbf{Trait (+2)}---which shifts the range by an additional two points---and an \Expert character will, under the right circumstances, have a one-in-three chance of doing the impossible.

The table below shows the probability of success for the different difficulty levels at different skill levels:

\begin{DndTable}[header=Success probability per skill level]{lrrrr}
    \textbf{Difficulty} & \textbf{0} & \textbf{+1} & \textbf{+2} & \textbf{+3} \\
    \hline
    \Trivial     & 98.8\% & 100.0\% & 100.0\% & 100.0\% \\
    \Simple      & 93.8\% &  98.8\% & 100.0\% & 100.0\% \\
    \Easy        & 81.5\% &  93.8\% &  98.8\% & 100.0\% \\
    \Basic       & 61.7\% &  81.5\% &  93.8\% & 98.8\% \\
    \Challenging & 38.7\% &  61.7\% &  81.5\% & 93.8\% \\
    \Difficult   & 18.5\% &  38.7\% &  61.7\% & 81.5\% \\
    \Formidable  &  6.2\% &  18.5\% &  38.7\% & 61.7\% \\
    \Arduous     &  1.2\% &   6.2\% &  18.5\% & 38.7\% \\
    \Extreme     &  -     &   1.2\% &   6.2\% & 18.5\% \\
    \Impossible  &  -     &   -     &   1.2\% &  6.2\% \\
\end{DndTable}

Players will not need to consult this table during a game---in \emph{The Wyrd Engine} we are not keen on using tables for game mechanics---but it should give a Game Master a rough idea of how to set difficulty levels when planning a game session.


\begin{DndComment}{Game Master Tip}
	When deciding on difficulty levels, you should focus on the narrative aspects of the game rather than realism in difficulty. You want to give the players exciting challenges, but any conflict resolution should have narrative relevance. Don't ask for dice rolls if you can act out a scene instead, and don't ask for dice rolls unless both failure and success will have exciting consequences. It is okay to have automatic wins and automatic losses if the alternative will break the story you are trying to tell, and it is okay to set unrealistically low or high difficulty levels if that is what it takes to tell a good story.
	\end{DndComment}



\subsection{Active Opposition}
\index{Active opposition}
When two characters compete directly but are not in combat (for that, see below), both roll \textbf{4dF + their relevant skill}. The highest result wins.

\begin{itemize}
    \item If one character beats the other by \textbf{1 or 2 points}, they succeed with a minor advantage.
    \item If they beat the other by \textbf{3 or more points}, their success is so impressive that the GM can, at their discretion, provide the winning character with a \textbf{boon}.
\end{itemize}

A \textbf{boon} is a one-use trait invented for the situation at hand. It is only active for the current scene and is lost if not used after the scene ends.

\subsection{Ties and Partial Successes}
\index{Ties}\index{Partial successes}
Not every roll results in a clean success or failure. When a roll \textbf{ties} the Difficulty Rating, or when failure would halt progress entirely, the GM may introduce a \textbf{complication}:

\begin{itemize}
    \item \textbf{Success with a Cost:} The action succeeds, but at a price (e.g., escaping a pursuer but losing an important clue).
    \item \textbf{Mixed Success:} The character achieves part of their goal, but not completely (e.g., unlocking a door but setting off an alarm).
    \item \textbf{A New Complication:} The failure introduces an unexpected twist (e.g., picking a lock only to find guards already inside).
\end{itemize}

\subsection{Interpreting Failure}
\index{Interpreting failure}
A failed roll doesn’t necessarily mean the character is incompetent—it simply means their approach didn’t work this time. The GM should ensure failures lead to new choices, not dead ends.

\begin{DndComment}{Game Master Tip}
    If a failed roll would stop the story in its tracks, offer the player an alternative: “You can still succeed but at a cost.” This keeps the momentum going while making failure meaningful.
\end{DndComment}

\subsection{Boosts: Optional Rule for Increasing Success}\index{Boosts}

As an optional rule, you can allow players to create \textbf{Boosts}—temporary numerical bonuses such as +1 or +2 that can be applied to a relevant roll. Boosts represent situational advantages, quick thinking, or clever tactics that enhance a character’s chance of success.  

Boosts can take different forms, including:  

\begin{itemize}
    \item \textbf{Preparation}: Taking extra time to study a problem, setting up tools, or laying a trap.  
    \item \textbf{Tactical Advantage}: Gaining higher ground, flanking an enemy, or exploiting a distraction.  
    \item \textbf{Environmental Factors}: Using dim lighting for stealth, a rainstorm to obscure movement, or an echoing chamber to amplify a command.  
    \item \textbf{Teamwork}: Coordinating efforts with allies, assisting with a skill check, or providing cover in combat.  
\end{itemize}

To gain a Boost, a player must describe how their actions create an advantage and roll an appropriate skill or trait check. If successful, they gain a Boost that applies to their next relevant roll. Boosts typically last for a single action but may persist longer if narratively justified.  

Boosts are a simple way to reward creativity, reinforce teamwork, and give players more control over their success in \textit{The Wyrd Engine}.


\subsection{Teamwork: Optional Rule for Assisting Allies}

In \textit{The Wyrd Engine}, collaboration can be just as important as individual skill. As an optional rule, players may assist one another to increase the chances of success in a task or conflict. When a character helps an ally, they provide a \textbf{Teamwork Bonus}, a small numerical boost that enhances the primary actor’s roll.  

Teamwork Bonuses can take different forms, including:  

\begin{itemize}
    \item \textbf{Direct Assistance}: Actively working alongside an ally, such as two people lifting a heavy object or multiple minds solving a puzzle.  
    \item \textbf{Tactical Coordination}: Calling out enemy movements in battle, providing covering fire, or distracting an opponent.  
    \item \textbf{Shared Knowledge}: Using past experiences or expertise to guide another character’s actions, such as an engineer giving instructions to a less skilled mechanic.  
    \item \textbf{Moral Support}: Bolstering an ally’s resolve with encouragement, inspiration, or leadership.  
\end{itemize}

To assist, the supporting player must describe how they are helping and roll an appropriate skill or trait check. If successful, they grant the primary actor a \textbf{+1 bonus} to their roll. In special cases—such as exceptional teamwork, well-planned strategies, or group efforts—the GM may allow the bonus to increase to \textbf{+2}.  

Only one character can provide a Teamwork Bonus per roll unless the GM rules that multiple participants are required. This system encourages cooperation and allows players to combine their strengths to overcome greater challenges.

% !TeX root = ../../../wyrd.tex


\WyrdCapLine{T}{he} core combat system outlined previously is sufficient for settings where combat isn’t a significant focus. In Agatha Christie-style mysteries, detailed combat rules would only clutter gameplay.

However, the importance and style of combat vary greatly between settings. Some games favour \textbf{quick, brutal encounters}, where a precise sniper shot or assassin’s blade swiftly ends a confrontation. Others emphasize \textbf{heroic, extended battles}, featuring characters bravely facing overwhelming odds.

The \textbf{tone and pacing of combat} should align with your game’s themes. A gritty setting might make injuries devastating and every choice critical, while a cinematic action game might allow daring heroics, letting characters survive improbable scenarios.

Players seeking \textbf{tactical complexity} may enjoy detailed positioning, cover, and resource management, rewarding careful planning. Alternatively, a more \textbf{freeform style} abstracts combat into dramatic narrative exchanges.

Furthermore, combat can significantly shape character development and storytelling. The outcomes of battles—both victories and defeats—can profoundly influence character arcs, relationships, and the broader narrative. A character who narrowly survives a deadly encounter might grapple with newfound fears or vulnerabilities, adding emotional depth to your story.

Additionally, combat encounters present opportunities for memorable narrative moments. A tense standoff, a heroic last stand, or a daring escape can become pivotal scenes that players recall long after the game ends. Thoughtfully designed combat can thus enrich the overall storytelling experience, providing dramatic stakes and moments of intense emotional engagement.

No matter your preferred style, \wyrd provides adaptable combat mechanics to suit your story and gameplay. That flexibility is the focus of this chapter.



\section{Combat Statistics}

Combat outcomes in \wyrd depend primarily on the skill difference between attackers and defenders, though dice rolls introduce some variability. Table \pagereftext{tbl:damage-probability} shows probabilities of inflicting damage based on skill disparities, expected damage per round, and average rounds needed to inflict 7+ damage (taking out a core character). The graphic on this page visually illustrates these probabilities.

\Graph[Damage per Attacker-Defender levels]{stats/damage_distribution.png}

Skill differences dominate combat outcomes by design. Each round favours the defender slightly (since ties do not deal damage). Within multiple rounds, the character with initiative attacks first, giving them a slight edge as well. Small differences in skill levels (1-2 levels) can have a large effect. A difference where the attacker has one level higher than the defender will not substantially shorten a combat --- it is expected to cut the rounds by half, from 11.1 to 5.5 --- but the probability of a character with +1 in attack and 0 in defence defeating a character with 0 in both attack and defence is 87.6\% compared to only 53.0\% if the two were evenly matched.

This emphasis on skills over randomness ensures predictable yet engaging gameplay, reinforcing the strategic importance of positioning and skill management. Any combat bonuses, for either attack or defence, can swing the battle. The long expected combat for equally skilled characters is also intentional. It prevents unfortunate characters from being eliminated in a single blown, reducing the randomness of combat. It does, however, mean that combat can be drawn out if the only combat actions are attacks and defending. But it generally shouldn't be.



\end{multicols}
\clearpage
\begin{DndTable}[header=Damage probability by relative skill level (Attack - Defence)]{crrrrrrrr}\label{tbl:damage-probability}
    \textbf{Attack - Defence} & \textbf{0 stress} & \textbf{1 stress} & \textbf{2 stress} & \textbf{3 stress} & \textbf{4 stress} & \textbf{5 stress} & \textbf{6 stress} & \textbf{7+ stress} \\
    -4 &  97.6\% &   1.7\% &   0.5\% &   0.1\% &   - &   - &   - &   - \\
    -3 &  93.6\% &   4.1\% &   1.7\% &   0.5\% &   0.1\% &   - &   - &   - \\
    -2 &  85.9\% &   7.7\% &   4.1\% &   1.7\% &   0.5\% &   0.1\% &   - &   - \\
    -1 &  73.9\% &  11.9\% &   7.7\% &   4.1\% &   1.7\% &   0.5\% &   0.1\% &   - \\
     0 &  58.4\% &  15.5\% &  11.9\% &   7.7\% &   4.1\% &   1.7\% &   0.5\% &   0.1\% \\
    +1 &  41.6\% &  16.9\% &  15.5\% &  11.9\% &   7.7\% &   4.1\% &   1.7\% &   0.6\% \\
    +2 &  26.1\% &  15.5\% &  16.9\% &  15.5\% &  11.9\% &   7.7\% &   4.1\% &   2.3\% \\
    +3 &  14.1\% &  11.9\% &  15.5\% &  16.9\% &  15.5\% &  11.9\% &   7.7\% &   6.4\% \\
    +4 &   6.4\% &   7.7\% &  11.9\% &  15.5\% &  16.9\% &  15.5\% &  11.9\% &  14.1\% \\
    +5 &   2.4\% &   4.1\% &   7.7\% &  11.9\% &  15.5\% &  16.9\% &  15.5\% &  26.0\% \\
    +6 &   0.7\% &   1.7\% &   4.1\% &   7.7\% &  11.9\% &  15.5\% &  16.9\% &  41.5\% \\
\end{DndTable}

\begin{DndTable}[header=Expected Damage in One Round]{lrrrrrrrrrrr}
    \textbf{Attacker - Defender} & \textbf{-4} & \textbf{-3} & \textbf{-2} & \textbf{-1} & \textbf{0} & \textbf{+1} & \textbf{+2} & \textbf{+3} & \textbf{+4} & \textbf{+5} & \textbf{+6} \\
    \textbf{Expected Damage}     & 0.0365      & 0.108       & 0.260       & 0.530        & 0.950       & 1.53        & 2.26         & 3.07         & 3.92         & 4.74         & 5.46      \\
\end{DndTable}


\begin{DndTable}[header=Expected Rounds to Accumulate 7+ Damage]{lrrrrrrrrrrr}
    \textbf{Attacker - Defender} & \textbf{-4} & \textbf{-3} & \textbf{-2} & \textbf{-1} & \textbf{0} & \textbf{+1} & \textbf{+2} & \textbf{+3} & \textbf{+4} & \textbf{+5} & \textbf{+6} \\
    \textbf{Expected Rounds}            & 210.5      & 89.3      & 40.3      & 19.3      & 11.1     & 5.5      & 3.8      & 2.8      & 2.3      & 1.9      & 1.6      \\
\end{DndTable}

The expected damage is the average damage that a player can expect to inflict in one round of combat, assuming that the player has the initiative and attacks first. The expected rounds to accumulate 7+ damage is the average number of rounds that it would take for a player to inflict 7+ damage on an opponent, assuming that the player has the initiative and attacks first.

\begin{DndTable}[header=Probability of player with initiative winning]{lrrrrrrrrr}
    &  \textbf{P2(0,0)} & \textbf{P2(0,1)} & \textbf{P2(0,2)} & \textbf{P2(1,0)} & \textbf{P2(1,1)} & \textbf{P2(1,2)} & \textbf{P2(2,0)} & \textbf{P2(2,1)} & \textbf{P2(2,2)}  \\
    \textbf{P1(0,0):} &  53.0\% &   9.3\% &   0.1\% &  17.6\% &   1.2\% &   0.0\% &   4.8\% &   0.2\% &   0.0\% \\
    \textbf{P1(0,1):} &  92.3\% &  51.2\% &   2.7\% &  53.0\% &   9.3\% &   0.1\% &  17.6\% &   1.2\% &   0.0\% \\
    \textbf{P1(0,2):} &  99.9\% &  97.5\% &  50.4\% &  92.3\% &  51.2\% &   2.7\% &  53.0\% &   9.3\% &   0.1\% \\
    \textbf{P1(1,0):} &  87.6\% &  53.0\% &   9.3\% &  56.2\% &  17.6\% &   1.2\% &  27.0\% &   4.8\% &   0.2\% \\
    \textbf{P1(1,1):} &  99.2\% &  92.3\% &  51.2\% &  87.6\% &  53.0\% &   9.3\% &  56.2\% &  17.6\% &   1.2\% \\
    \textbf{P1(1,2):} & 100.0\% &  99.9\% &  97.5\% &  99.2\% &  92.3\% &  51.2\% &  87.6\% &  53.0\% &   9.3\% \\
    \textbf{P1(2,0):} &  97.7\% &  87.6\% &  53.0\% &  85.0\% &  56.2\% &  17.6\% &  60.9\% &  27.0\% &   4.8\% \\
    \textbf{P1(2,1):} &  99.9\% &  99.2\% &  92.3\% &  97.7\% &  87.6\% &  53.0\% &  85.0\% &  56.2\% &  17.6\% \\
    \textbf{P1(2,2):} & 100.0\% & 100.0\% &  99.9\% &  99.9\% &  99.2\% &  92.3\% &  97.7\% &  87.6\% &  53.0\% \\
\end{DndTable}

Notation \textbf{Pn(A,D)} should be read as player \emph{n} has attack skills \emph{A} and defence skills \emph{D}. Player 1 has the initiative and attacks first. Evenly matched, the player that attacks first has a slight advantage. The probability that the second player wins is one minus the probability that the first player wins.

For all tables, we have not taken into account the effect of wound penalties or the use of combat maneuvers.

The tables are not intended to be used as a reference during play, but rather to give you an idea of the expected outcomes of combat. This can help the GM design combat encounters that are challenging but not impossible for the players.

\vspace{\baselineskip}
\hrule
\begin{multicols}{2}







\section{Making Combat Interesting}

Combat shouldn’t merely be a predictable dice-rolling exercise. \wyrd balances the active opposition mechanics used elsewhere, for both determining when an attack is successful and how much damage is inflicted, with a few additional mechanics to keep combat engaging. And these mechanics are well known as well: \textbf{using traits and gear} to obtain offensive or defensive bonuses, and using combat manoeuvres as \textbf{boosts} to gain additional advantages.

But before we consider applying these mechanics in combat, let us consider the alternative which is to let characters slug it out with no modifiers. This is a valid option, but it can lead to combat being a simple exercise in rolling dice and adding numbers, and unless the two characters are evenly matched, the outcome is strongly skewed in one direction or the other.

We will use the example of \emph{Anna the Assassin} and \emph{Brian the Barbarian} to illustrate this. Both characters have a \textbf{+1 Fight} skill, which they use for both attacking and defending. Initially, they are evenly matched, so the outcome of their combat is almost entirely dependent on the dice rolls, with only a slight advantage to the player with the initiative, in this case Anna.

\begin{Example}{Combat without modifiers}
    \emph{Anna the Assassin} jumps on top of her table at the \emph{Rusty Dagger Tavern}, blades gleaming in the flickering lantern light. Across the room, \emph{Brian the Barbarian} rises with a growl, knocking over his ale as he draws his enormous axe. 

    Anna has the initiative and attacks first. She rolls a \FudgeRes{+0--} = \textbf{-1} and Brian rolls a \FudgeRes{++--} = \textbf{0}. They both add their \textbf{Fight +1} but they cancel out. Since Anna's attack is below Brian's defence, she does not inflict damage. 

    \vspace{0.5\baselineskip}
    \begin{tcolorbox}[
        damageboxbase,
        title=Damage Boxes
    ]
    \begin{tabular}{@{}l l@{ } l@{ } l@{ } l@{ }}
        \textbf{Anna the Assassin} & \FatigueBoxes[0][3] & \MildWounds[0][1] & \ModerateWounds[0][1] & \SevereWounds[0][1] \\
        \textbf{Brian the Barbarian} & \FatigueBoxes[0][3] & \MildWounds[0][1] & \ModerateWounds[0][1] & \SevereWounds[0][1]
    \end{tabular}
    \end{tcolorbox}

    Brian retaliates with his own attack, rolling a \FudgeRes{++00} = \textbf{+2} against Anna's defence of \FudgeRes{+00-} = \textbf{0}. This time the attack is successful, and Brian inflicts \textbf{2 damage} on Anna.
   
    \begin{tcolorbox}[
        damageboxbase,
        title=Damage Boxes
    ]
    \begin{tabular}{@{}l l@{ } l@{ } l@{ } l@{ }}
        \textbf{Anna the Assassin} & \FatigueBoxes[2][3] & \MildWounds[0][1] & \ModerateWounds[0][1] & \SevereWounds[0][1] \\
        \textbf{Brian the Barbarian} & \FatigueBoxes[0][3] & \MildWounds[0][1] & \ModerateWounds[0][1] & \SevereWounds[0][1]
    \end{tabular}
    \end{tcolorbox}

    Now it is Anna's turn again. She rolls a \FudgeRes{++00} = \textbf{+2} against Brian's defence of \FudgeRes{++0-} = \textbf{+1}. This time, Anna's attack causes \textbf{1 damage} to Brian.

    \begin{tcolorbox}[
        damageboxbase,
        title=Damage Boxes
    ]
    \begin{tabular}{@{}l l@{ } l@{ } l@{ } l@{ }}
        \textbf{Anna the Assassin} & \FatigueBoxes[2][3] & \MildWounds[0][1] & \ModerateWounds[0][1] &\SevereWounds[0][1] \\
        \textbf{Brian the Barbarian} & \FatigueBoxes[1][3] & \MildWounds[0][1] & \ModerateWounds[0][1] &\SevereWounds[0][1]
    \end{tabular}
    \end{tcolorbox}

    Now Brian swings his axe again, rolling a \FudgeRes{+00-} = \textbf{0} against Anna's defence of \FudgeRes{+++0} = \textbf{+3}. The attack is smaller than the defence, so Brian does not inflict any damage.

    \begin{tcolorbox}[
        damageboxbase,
        title=Damage Boxes
    ]
    \begin{tabular}{@{}l l@{ } l@{ } l@{ } l@{ }}
        \textbf{Anna the Assassin} & \FatigueBoxes[2][3] & \MildWounds[0][1] & \ModerateWounds[0][1] & \SevereWounds[0][1] \\
        \textbf{Brian the Barbarian} & \FatigueBoxes[1][3] & \MildWounds[0][1] & \ModerateWounds[0][1] & \SevereWounds[0][1]
    \end{tabular}
    \end{tcolorbox}

\end{Example}

We could go on here, and there is close to a 50\% chance for both of the opponents to win, so some uncertainty in the outcome, but it is not very exciting to play out a battle this way.

We can vary the situation slightly using just traits. Anna the Assassin has a \textbf{Blade of the Night} trait that gives her a +2 bonus to attack rolls in the dark.

\begin{Example}{Exploiting Traits}
    \emph{Anna the Assassin} followed \emph{Brian the Barbarian} as he left the \emph{Rusty Dagger Tavern}, waiting for the right moment to strike. As Brian stepped into the dark alley to releave himself, Anna leapt from the shadows.

    The GM judges that the alley is dark enough for Anna to use her \textbf{Blade of the Night} trait, giving her a +2 bonus to attack rolls.

    She rolls a \FudgeRes{++00} = \textbf{+2} and adds her trait \textbf{+2}. Brian's defence is \FudgeRes{++0-} = \textbf{+1}. The difference is \textbf{+3}, so Anna inflicts \textbf{3 damage} on Brian.

    \vspace{0.5\baselineskip}
    \begin{tcolorbox}[
        damageboxbase,
        title=Damage Boxes
    ]
    \begin{tabular}{@{}l l@{ } l@{ } l@{ } l@{ }}
        \textbf{Anna the Assassin} & \FatigueBoxes[0][3] & \MildWounds[0][1] & \ModerateWounds[0][1] &\SevereWounds[0][1] \\
        \textbf{Brian the Barbarian} & \FatigueBoxes[3][3] & \MildWounds[0][1] & \ModerateWounds[0][1] & \SevereWounds[0][1]
    \end{tabular}
    \end{tcolorbox}

    Brian, now aware of Anna's presence, retaliates with a roar. He rolls a \FudgeRes{++0-} = \textbf{+1} against Anna's defence of \FudgeRes{+00-} = \textbf{0}. Anna's trait is only applicable for attacks, so she cannot add it here. The difference is \textbf{+1}, so Brian inflicts \textbf{1 damage} on Anna.

    \begin{tcolorbox}[
        damageboxbase,
        title=Damage Boxes
    ]
    \begin{tabular}{@{}l l@{ } l@{ } l@{ } l@{ }}
        \textbf{Anna the Assassin} & \FatigueBoxes[1][3] & \MildWounds[0][1] &\ModerateWounds[0][1] &\SevereWounds[0][1] \\
        \textbf{Brian the Barbarian} & \FatigueBoxes[3][3] & \MildWounds[0][1] &\ModerateWounds[0][1] &\SevereWounds[0][1]
    \end{tabular}
    \end{tcolorbox}

    Anna attacks again, rolling a \FudgeRes{++0-} = \textbf{+1} and adds \textbf{+2} for an attack of \textbf{+3} against Brian's defence of \FudgeRes{++0-} = \textbf{+1}. The difference is \textbf{+2}.

    \begin{tcolorbox}[
        damageboxbase,
        title=Damage Boxes
    ]
    \begin{tabular}{@{}l l@{ } l@{ } l@{ } l@{ }}
        \textbf{Anna the Assassin} & \FatigueBoxes[1][3] & \MildWounds[0][1] & \ModerateWounds[0][1] &\SevereWounds[0][1] \\
        \textbf{Brian the Barbarian} & \FatigueBoxes[3][3] &\MildWounds[1][1] &\ModerateWounds[1][1] &\SevereWounds[0][1]
    \end{tabular}
    \end{tcolorbox}

    At this point, Brian conceeds the fight.
\end{Example}

It is not that adding traits to make the battle more uneven also makes it more interesting --- if anything, it makes it less interesting since the chance of the outclassed character winning is so low. But at least such a combat encounter is over quickly, and the players can move on to the next scene. The point is not that skill or trait bonuses adds excitement to combat, however, but the use of traits and gear can make choosing the battlefield, the time and place, a strategicly important decision, which \emph{can} add excitement to combat.

\subsection{Changing the Battlefield}

Once a combat encounter is underway, the players might not be able to change the conditions to activate a trait, but sometimes they can --- if Anna and Brian were fighting in the tavern and Anna had the chance to throw the room into darkness, for example. If the players \emph{can} change the conditions they are fighting in, then that becomes a tactical goal. Increasing the attack or defence stats by one or two levels can be a significant advantage, and the players should be encouraged to use their traits and gear to gain that advantage.

\begin{Example}{Changing the Battlefield}
    \emph{Anna the Assassin} and \emph{Brian the Barbarian} find themselves locked in combat inside the \emph{Rusty Dagger Tavern}. The room is lit by swaying oil-lamps, and Anna's \textbf{Blade of the Night} trait—granting +2 to attacks in the dark—is currently useless.

    Anna decides to act. On her turn, instead of attacking, she uses an action to snuff out the main lantern by flipping a table into it. The GM calls for an \textbf{Athletics} \DL{2} check. Anna rolls \FudgeRes{+0+-} = \textbf{+1} and adds it to her \textbf{Athletics +2} skill. The lantern crashes to the floor, plunging the room into shadow.

    Brian roars in frustration and swings blindly, rolling a \FudgeRes{+00-} = \textbf{0}, but Anna defends with \FudgeRes{+++0} = \textbf{+3}, easily dodging in the darkness.

    Now it's Anna’s turn. With the room dark, her \textbf{Blade of the Night} activates. She attacks, rolling \FudgeRes{++0-} = \textbf{+1}, adds +2 from the trait, for a total of \textbf{+3}. Brian defends with \FudgeRes{+0--} = \textbf{-1}, giving Anna a difference of \textbf{+4}.

    \vspace{0.5\baselineskip}
    \begin{tcolorbox}[
        damageboxbase,
        title=Damage Boxes
    ]
    \begin{tabular}{@{}l l@{ } l@{ } l@{ } l@{ }}
        \textbf{Anna the Assassin} & \FatigueBoxes[0][3] & \MildWounds[0][1] & \ModerateWounds[0][1] & \SevereWounds[0][1] \\
        \textbf{Brian the Barbarian} & \FatigueBoxes[3][3] & \MildWounds[1][1] & \ModerateWounds[0][1] & \SevereWounds[0][1]
    \end{tabular}
    \end{tcolorbox}

    Realising he's completely outmatched in the dark, Brian stumbles toward the door, seeking light—or surrender.
\end{Example}


\subsection{Combat Maneuvers}

If the players cannot invoke their existing traits (or the traits of their gear), then they can still use combat maneuvers to gain bonuses to their attacks or defences.

Combat maneuvers are special actions that can be used to gain a temporary advantage in combat. If you are changing the battlefield to gain a bonus from a trait, you already posses the trait, but you need to change the situation to gain the bonus. Traits are narrow in scope, and not all situations will enable you to exploit them, even after taking actions to change the battlefield. Combat maneuvers are always available, however. At any time, you can spend an action to perform a combat maneuver, which will give you a bonus to your next attack or defence, but unlike traits, combat maneuver bonuses are transient and lost as soon as you use them, or as soon as an attempt to increase them fails.

In any round, instead of attacking, a character can
\begin{itemize}
    \item Do an \textbf{attack} combat maneuver to gain a \textbf{+2 bonus} to their next attack.
    \item Do a \textbf{defend} combat maneuver to gain a \textbf{+2 bonus} to their next defence.
\end{itemize}

Bonuses accumulated until they are used, or until the character fails a combat maneuver, in which case the entire accumulated bonus is lost. The two bonuses accumulate independently, and a failed maneuver does not affect the other bonus.

Doing a combat maneuver works like normal opposition rolls. A character should always be allowed to use theh skill they use for attacking or defending against a difficulty level of \textbf{2}, with \textbf{ties reducing the bonus to +1}, but the GM should also allow inventive players to use other skills if they can justify it. In that case, the GM should judge whether the opposition roll is passive or active and set appropriate difficulty levels for passive rolls. In the case of ties, the GM should judge whether the tie is a success or a failure, and what the consequences are, i.e., whether a tie reduces the bonus to +1 or whether it is a failure that doesn't remove the accumulated bonus.

\begin{Example}{Attack Combat Maneuvers}
    \emph{Anna the Assassin} is sneaking up on \emph{Brian the Barbarian}. She intends to jump him, which would be an attack, but her player figures that if she sneaks up close and stabs him in the back, she should get an attack bonus. The GM aggrees, but requires an active opposition roll, Anna's \textbf{Stealth} against Brian's \textbf{Notice}. Upon success, she will get a +2 bonus to her stab attack, but on failure or a tie Brian would get to attack with initiative.

    Anna rolls a \FudgeRes{++00} = \textbf{+2} and adds her \textbf{Stealth +2} against Brian's \FudgeRes{++0-} + \textbf{Notice +1}. The total is \textbf{+4} against \textbf{+2}, so the roll is a success, so she gains a +2 bonus to her next attack, an attack she immidiately makes.

    Anna attacks, rolling a \FudgeRes{++0-} = \textbf{+1} and adds her \textbf{Fight +1} and the \textbf{+2 bonus} from the combat maneuver for a total of \textbf{+4}. Brian defends with \FudgeRes{+000} = \textbf{+1} plus his \textbf{Fight +1} for a total of \textbf{+2}, giving Anna a difference of \textbf{+2}.
\end{Example}

This example shows that you can use a normal opposition roll to gain a combat maneuver bonus. Strictly speaking, the combat hadn't started yet, but preparing for battle is a valid combat maneuver, and the GM should allow it.

\begin{Example}{Defence Combat Maneuvers}
    \emph{Brian the Barbarian}, screaming from being stabbed in the back, throws himself behind a dumbster, trying to take cover. 

    This is a \textbf{defend} combat maneuver --- he is doing the action instead of attacking -- and Brian will use his \textbf{Athletics +2} against a \textbf{+2} difficulty level. He rolls \FudgeRes{++00} = \textbf{+2} and adds his \textbf{Athletics +2} for a total of \textbf{+4}, so he succeeds and gets a +2 bonus to his next defence.
    
    Anna attacks and rolls a \FudgeRes{+++-} = \textbf{+2} and adds her \textbf{Fight +1} for a total of \textbf{+3} (she no longer has the bonus she used for her stealth attack). Brian rolls \FudgeRes{++0-} + \textbf{Fight +1} plus the \textbf{+2} defence bonus for a total of \textbf{+4}. Being in cover behind the dumpster saved him from the attack.
\end{Example}

Even with combat manoeuvres, there is still a risk that a fight may drag on, with one character steadily building up attack bonuses while the other accumulates defensive ones—leaving their relative positions unchanged. This is mitigated somewhat by the chance of losing a bonus when a manoeuvre fails.

The base difficulty of \DL{2} means a character with a relevant skill of \textbf{+1} will only succeed about a third of the time (38.7\%), while a character with \textbf{+2} will succeed just under two-thirds of the time (61.7\%). Success is far from guaranteed, and the risk of failure—and losing the bonus—is significant. As a result, the tactic of simply stacking bonuses is not a reliable long-term strategy.

However, the ability to use non-combat skills to perform manoeuvres allows characters to play to their strengths—if they can be creative and the GM permits it. This opens up new tactical options for players, which is the true purpose of combat manoeuvres. They are not just a way to gain bonuses to attack or defence, but a tool for players to engage the system creatively and leverage a broader range of skills to gain the upper hand in combat.

When multiple characters are involved in combat, maneuvers also add a layer of tactical complexity. If two characters are attacking a third, the defender is effectively prevented from building up defence bonuses. The defence bonus they have will be expended on the first attack, so the attackers can decide to have one build up attack bonuses while the other attacks, and the defender cannot build a defence bonus against the boosted attack that will eventually come.

In larger battles, deciding who fights who, and how to use combat maneuvers, can be a tactical decision. If the players are fighting a group of enemies, they can choose to attack one at a time, or they can split up and attack multiple enemies at once. Their choices will determine how they can build up their own bonuses and what choices their opponents can make for their own combat maneuvers.

\section{Weapons and Armour}

Gear traits can enhance combat just like any other opposition rolls, and it’s natural to model weapons and armour as such traits. Fists are less effective than knives, which are in turn less effective than swords. Similarly, leather armour offers less protection than chainmail, which is weaker than full plate.

The level of detail you apply depends on the setting and how often combat arises in your game. In a setting where combat is rare, you might avoid complex rules altogether. But if combat is a central part of the game, then weapon and armour choice can become an important part of both character identity and tactical planning.

Below are examples of how gear traits can be used to model the effectiveness of different weapons and armour across different settings. These examples are not exhaustive but should serve as a helpful baseline. 

\begin{CommentBox}{A note of caution}
    Adding bonuses to weapons and armour can easily lead to an arms race. If every opponent and player continually escalates their gear bonuses, you may end up with excessive bookkeeping but no meaningful change to the gameplay. To avoid this, ensure players face enemies both less and more well-equipped than themselves. Gaining a powerful weapon to overcome a challenge can make for a compelling story—but simply scaling weapons in parallel with enemies leads to stagnation.
\end{CommentBox}

\subsection{Weapons}

Weapons can be modelled as gear traits that provide a bonus to attack rolls. Light weapons may grant +1, while heavier or more advanced weapons may grant +2 or more. However, excessive stacking of bonuses should be avoided—encourage variety in use and tactical application instead.

\subsubsection*{Fantasy}

\begin{itemize}
  \item \textbf{Unarmed / Improvised Weapon (0)} – Fists, chairs, tankards.
  \item \textbf{Dagger / Club (+1)} – Small, quick weapons that are easy to conceal or use in close quarters.
  \item \textbf{Sword / Axe / Spear (+2)} – Standard martial weapons with a reliable combat bonus.
  \item \textbf{Greatsword / Polearm (+3)} – Two-handed or powerful weapons with greater reach or impact.
  \item \textbf{Legendary Weapon (+4)} – Rare magical or mythic weapons with narrative weight. These should be plot-relevant.
\end{itemize}

\subsubsection*{Modern}

\begin{itemize}
  \item \textbf{Fist / Stun Baton (0)} – Non-lethal or improvised.
  \item \textbf{Knife / Pistol (+1)} – Standard sidearms or melee tools.
  \item \textbf{Shotgun / Assault Rifle (+2)} – Tactical weapons for combat scenarios.
  \item \textbf{Sniper Rifle / Heavy Weapon (+3)} – Long-range or high-calibre weapons; often slower or bulkier.
  \item \textbf{Prototype or Military-Grade Weapon (+4)} – Restricted or experimental tech, used sparingly.
\end{itemize}

\subsubsection*{Sci-Fi}

\begin{itemize}
  \item \textbf{Plasma Dagger / Energy Whip (+1)} – Futuristic melee weapons.
  \item \textbf{Laser Rifle / Gauss Gun (+2)} – Common energy weapons with precise or powerful shots.
  \item \textbf{Plasma Cannon / Anti-Matter Lance (+3)} – Devastating weapons, difficult to wield or maintain.
  \item \textbf{Relic of the Ancients (+4)} – Rare and potent alien or ancient technology, central to plot arcs.
\end{itemize}

\subsection{Armour}

Armour provides a bonus to defence rolls, reducing the chance of taking damage. Unlike weapons, armour often comes with trade-offs—such as reduced mobility, attention-drawing bulk, or limited availability in certain settings.

\subsubsection*{Fantasy}

\begin{itemize}
  \item \textbf{None / Clothing (0)} – Offers no real protection.
  \item \textbf{Leather Armour (+1)} – Light, flexible, and common among rogues or rangers.
  \item \textbf{Chainmail / Scale Armour (+2)} – Heavier protection at the cost of agility.
  \item \textbf{Plate Armour (+3)} – Full-body protection, often worn by elite knights.
  \item \textbf{Enchanted Armour (+4)} – Rare magical items that may confer additional narrative effects.
\end{itemize}

\subsubsection*{Modern}

\begin{itemize}
  \item \textbf{None / Casual Wear (0)} – No protective value.
  \item \textbf{Kevlar Vest (+1)} – Light ballistic protection against small arms.
  \item \textbf{Tactical Body Armour (+2)} – Offers improved coverage and resistance.
  \item \textbf{Bomb Suit / Riot Gear (+3)} – Maximum protection, but heavy and cumbersome.
  \item \textbf{Prototype Armour (+4)} – Advanced gear from research labs or special forces.
\end{itemize}

\subsubsection*{Sci-Fi}

\begin{itemize}
  \item \textbf{Nano-Weave Undersuit (+1)} – Flexible and stylish, useful for infiltration or agents.
  \item \textbf{Combat Exosuit (+2)} – Reinforced armour with HUD and power support.
  \item \textbf{Powered Armour (+3)} – Heavy-duty suits with strength amplification and shielding.
  \item \textbf{Void Armour (+4)} – Ancient or alien tech that defies conventional damage.
\end{itemize}



\section{Fighting Styles}

Not all combatants fight the same way. Some rely on brute strength, others on speed, cunning, or honed discipline. In \wyrd, you can represent different forms of combat using \textbf{fighting styles}—distinct techniques, schools, or traditions that combine specific skills, weapons, and tactics into recognisable approaches to battle.

Fighting styles can be purely narrative, or they can provide mechanical bonuses when used strategically. A style may work well against some opponents but poorly against others, introducing a natural system of strengths and weaknesses—like rock-paper-scissors, but more flexible and open to creative interpretation.

Fighting styles can be expressed using \textbf{traits}, or defined narratively by the GM and players. Some styles may grant a bonus in certain situations (e.g., against heavy armour, while surrounded, or in darkness), while others are designed to counter particular styles or skills.

\subsection*{Combining Skills and Weapons}

In a flexible system like \wyrd, fighting styles can be built by combining different skills with specific types of gear. Some examples:

\begin{itemize}
  \item A duelist might use \textbf{Rapport} with a rapier, turning insults and flourishes into distractions that act as boosts.
  \item A berserker could rely on \textbf{Physique} and heavy weapons to overwhelm foes, gaining bonuses when ignoring defence or attacking multiple opponents.
  \item A street brawler might combine \textbf{Deceive} with improvised weapons to create unexpected openings or feints.
  \item A monk could use \textbf{Will} to resist pain and channel inner focus into precise strikes.
\end{itemize}

The GM should encourage players to define how their fighting style works and reward creative combinations that match the character’s concept. A style should inform tactics and scene flavour, not just provide flat bonuses.

\subsection*{Style Counters and Technique Matchups}

To create a richer tactical space, you may define style interactions—some fighting styles are naturally strong or weak against others. For example:

\begin{itemize}
  \item \textbf{Iron Wall Style} (shield and spear, defensive posture) is effective against aggressive melee attackers but struggles against agile ranged foes.
  \item \textbf{Whispering Fang} (dagger and cloak, deception-based) excels at breaking enemy focus but is vulnerable to disciplined or intuitive fighters.
  \item \textbf{Stone Fist Boxing} (brute-force strikes) overpowers finesse-based styles but lacks adaptability against tricksters or feints.
  \item \textbf{Storm Serpent Form} (fluid motion, staff work) can counter slower styles, but is disrupted by grapplers or sudden aggressive charges.
\end{itemize}

These interactions do not need precise mechanics. Instead, treat them as situational modifiers, boosts, or justification for compelling outcomes in contested rolls. If one style clearly counters another in the fiction, grant the player a temporary boost or invoke a free aspect reflecting the advantage.

\subsection*{Style as Trait}

You may formalise a fighting style as a trait, such as:

\begin{itemize}
  \item \textbf{Trained in the Windblade School} — Gain +2 to create an advantage when using twin blades in open spaces.
  \item \textbf{Master of Red Lotus Fist} — Once per scene, ignore one point of damage when fighting unarmed.
  \item \textbf{Practitioner of the Twelve Strikes} — Gain a boost when successfully predicting and countering a known style.
\end{itemize}

As with other gear and character traits, these bonuses should be conditional and narratively grounded. A style becomes more meaningful when it shapes how a character approaches combat, not just what numbers they use.

\subsection*{Creating Your Own Styles}

Encourage players to invent styles suited to the setting. In a fantasy world, schools of swordplay may rival one another like noble houses. In modern settings, street-fighting techniques might evolve from urban subcultures. In sci-fi, martial forms might be adapted to zero-gravity or cybernetic bodies.

The goal is not to add complexity, but depth. A good fighting style helps define a character, enriches combat scenes, and offers opportunities for drama, rivalry, and growth.


\section{Designing Combat Encounters}

A good combat scene is more than a series of dice rolls. It should feel dynamic, cinematic, and full of opportunities for player creativity. In \wyrd, combat works best when it serves the story, engages the players' imagination, and gives everyone a chance to use their unique abilities. If every fight ends up as two characters exchanging blows until one runs out of boxes, something important is missing.

This section offers guidance on how to build more compelling encounters—ones that are not only balanced and mechanically interesting but also rich with narrative possibilities.

\subsection*{Leverage Traits and Narrative Hooks}

The simplest way to make combat more engaging is to ensure that the players’ traits are relevant. Each trait represents a part of the character’s identity or background. Design encounters where players can bring these traits into play:

\begin{itemize}
  \item A stormy rooftop chase where a trait like \textbf{Born on the Streets} might apply.
  \item A duel before a crowd where \textbf{Performer at Heart} can earn boosts through showmanship.
  \item A darkened tomb where a character with \textbf{Eyes Adjusted to the Dark} gains a crucial edge.
\end{itemize}

Encourage players to look for narrative justification to invoke their traits, and create situations where the fiction invites those connections. Even a simple skirmish can become memorable if it feels personal.

\subsection*{Terrain as a Tactical Resource}

Combat becomes more than trading attacks when the environment offers opportunities—and dangers.

Design the battlefield with features that can be used to gain advantage, such as:

\begin{itemize}
  \item \textbf{Cover}: Crates, statues, or vehicles that provide defensive bonuses.
  \item \textbf{Hazards}: Fires, cliffs, swinging chains, or unstable walkways that add tension.
  \item \textbf{Interactive objects}: Chandeliers, levers, crumbling walls, or magical artefacts.
  \item \textbf{Elevation or bottlenecks}: Platforms, narrow bridges, or spiral staircases that favour certain tactics.
\end{itemize}

Include aspects or situational advantages the players can discover or create—like “Loose Floorboards” or “Broken Balcony”—to encourage experimentation. Let clever use of the terrain grant boosts, free invokes, or even shift the course of battle.

\subsection*{Opponents With Personality}

Enemies should do more than just roll to hit. Make each foe feel unique by giving them:

\begin{itemize}
  \item \textbf{A defining trait or tactic}: e.g. “Shields of the Moon Guard” may always defend in formation.
  \item \textbf{A specific goal}: Instead of fighting to the death, maybe the villain is trying to escape, complete a ritual, or delay the players.
  \item \textbf{A weakness to discover}: An enemy may be immune to standard attacks but vulnerable to clever tactics or specific effects.
  \item \textbf{A dramatic flair}: Use monologues, emotional stakes, or surprise reinforcements to raise tension.
\end{itemize}

Opponents should also be capable of using the environment and creating their own advantages. A good enemy might throw a lantern to ignite the room, or use a grappling hook to flee across a rooftop.

\subsection*{Goals Beyond “Defeat All Enemies”}

If every combat ends when the last opponent falls, fights can feel repetitive. Introduce alternative or secondary objectives:

\begin{itemize}
  \item \textbf{Survive for a number of rounds} until backup arrives.
  \item \textbf{Protect a location or NPC} from waves of enemies.
  \item \textbf{Reach a lever, seal, or portal} while under fire.
  \item \textbf{Delay the enemy ritual} long enough for an ally to complete their task.
  \item \textbf{Retrieve an item} from the battlefield and escape.
\end{itemize}

Victory conditions that shift mid-fight—such as an enemy revealing a second form or reinforcements arriving—can also create surprise and momentum.

\subsection*{Use Boosts and Temporary Aspects}

Encourage players and enemies to create \textbf{boosts} and \textbf{temporary aspects}. These fleeting advantages make the flow of combat feel more dynamic and tactical.

Examples:
\begin{itemize}
  \item \textbf{Disarmed!} — After a clever create advantage action.
  \item \textbf{Pinned Behind Cover} — Created with a well-placed shot.
  \item \textbf{Thrown Off Balance} — A boost from a successful feint or trip.
\end{itemize}

By rewarding clever play with tangible benefits—even short-lived ones—you make the moment-to-moment action of combat more engaging.

\subsection*{Let the Players Shape the Fight}

Combat should never feel like the GM is simply executing a script. Let players influence the battlefield, shift the stakes, and change the conditions. Encourage actions like:

\begin{itemize}
  \item \textbf{Creating distractions} to split enemy forces.
  \item \textbf{Changing the environment}, such as plunging a room into darkness or collapsing a walkway.
  \item \textbf{Calling on allies} mid-fight through a trait or resource.
  \item \textbf{Escalating the situation}, e.g. drawing more guards, triggering alarms, or starting fires.
\end{itemize}

A combat scene becomes exciting when everyone at the table contributes ideas, builds on each other’s moves, and feels like they’re shaping the outcome together.

\subsection*{Escalation and Pacing}

Even well-designed fights can become stale if they drag on too long. Keep things moving by:

\begin{itemize}
  \item Tracking the fight’s \textbf{emotional stakes}—what changes if the players win or lose?
  \item Introducing \textbf{timed complications}, such as a door that must be unlocked while fighting.
  \item Raising the tension with \textbf{mid-combat twists}: reinforcements, betrayal, an unexpected monster.
  \item Letting enemies \textbf{retreat or surrender} if the tide turns.
\end{itemize}

Think of each combat as a narrative beat, not just a mechanical challenge. If the outcome no longer matters or the momentum is lost, consider wrapping up the scene with a concession or a dramatic finish.

\subsection*{Combat as a Conversation}

Finally, remember that combat in \wyrd is not a war game—it’s a storytelling conversation. The dice add suspense, but the story is what gives the fight meaning. The best combat encounters aren't just about who hits harder, but who risks something, who grows, and what changes because of it.



\section{Character Creation}
\index{Character creation}

Creating a character in \wyrd is a quick and streamlined process, designed to get players into the game with minimal preparation. Each character is defined by a small but meaningful set of attributes that shape their role in the story. Unlike systems with long-term progression, \wyrd prioritises narrative impact over mechanical advancement, making character creation simple yet flexible.

Every player character is built using the following elements:

\subsection{Step 1: Concept}

Before assigning mechanics, players should develop a brief \textbf{character concept}. This is a short description of who the character is, their role in the story, and what makes them interesting. Concepts should be evocative but flexible, helping guide both roleplay and mechanical choices.

\begin{CommentBox}{Example Character Concepts}
    \begin{itemize}
        \item \emph{A disgraced noble turned detective, haunted by his past.}
        \item \emph{An eccentric engineer whose inventions are as brilliant as they are dangerous.}
        \item \emph{A silver-tongued con artist who survives by wit and charm.}
        \item \emph{A fearless occult investigator seeking forbidden knowledge.}
    \end{itemize}
\end{CommentBox}

\subsection{Step 2: Choose Skills}

Each character has a set of \textbf{Skills} that determine their strengths and weaknesses. Skills represent broad areas of expertise rather than hyper-specialised talents, ensuring versatility.

Characters receive a total of \textbf{six skill ranks}, distributed as follows:

\begin{itemize}
    \item \textbf{1 \Expert} skill
    \item \textbf{2 \Skilled} skills
    \item \textbf{3 \Novice} skills
\end{itemize}

All unselected skills default to \Untrained.

\begin{Example}{}
	When assigning skills, players should consider their character’s background and expertise. A veteran detective might prioritise \textbf{Investigate} and \textbf{Notice}, while a rogue might favour \textbf{Stealth} and \textbf{Deceive}.
\end{Example}

The total sum of skill ranks should equal \textbf{10}. This ensures that every character is balanced in overall competence while allowing for specialisation.

\subsection{Step 3: Select Traits}

Every character has exactly \textbf{three Traits}. Traits represent exceptional abilities, personal quirks, or special training that set a character apart. 

\textbf{Traits provide one of three benefits:}
\begin{itemize}
    \item A \textbf{+2 bonus} when applied to a relevant skill check.
    \item A \textbf{special ability} that can be used \emph{once per scene or session}.
    \item A \textbf{narrative permission} to attempt actions that would normally be impossible.
\end{itemize}

\begin{CommentBox}{Example Traits}
    \begin{itemize}
        \item \textbf{Master Duelist} – \emph{Gain +2 to Fight when using a rapier or fencing techniques.}
        \item \textbf{Inventive Genius} – \emph{Can craft unique gadgets that defy conventional mechanics.}
        \item \textbf{Unshakable Will} – \emph{Once per session, ignore the effects of fear or mind control.}
        \item \textbf{Underworld Connections} – \emph{Gain +2 to Contacts when dealing with criminals.}
        \item \textbf{The Cards Never Lie} – \emph{Use Lore instead of Investigate when predicting an outcome.}
    \end{itemize}
\end{CommentBox}

Traits should enhance a character’s strengths and provide unique advantages in play. They should not be overly broad or cover multiple unrelated areas.

\subsection{Step 4: Select Gear}

\wyrd does not track mundane items or encumbrance. Instead, \textbf{gear} is used to track items that have a significant impact on gameplay. Unlike traits, gear is not inherent to a character but can be aquired or lost during play. At the Game Master's discression, players can start out with a fixed number of gear items, say three per character. Alternatively, important gear can work as plot devices, with the Game Master deciding when and how to introduce them into the game.

Each piece of gear functions like a Trait, providing either:
\begin{itemize}
    \item A \textbf{+2 bonus} when used appropriately.
    \item A \textbf{special ability} usable once per scene or session.
    \item A \textbf{narrative permission} to perform unique actions.
\end{itemize}

\begin{CommentBox}{Example Gear}
    \begin{itemize}
        \item \textbf{Clockwork Lockpick} – \emph{+2 to Burglary when opening mechanical locks.}
        \item \textbf{Enchanted Mirror} – \emph{Once per session, reveal a hidden truth.}
        \item \textbf{Mastercrafted Rapier} – \emph{+2 to Fight in one-on-one duels.}
        \item \textbf{Detective’s Notebook} – \emph{Use Investigate instead of Rapport when questioning suspects.}
        \item \textbf{Hidden Derringer} – \emph{Once per scene, draw a concealed firearm unnoticed.}
    \end{itemize}
\end{CommentBox}

\subsection{Step 5: Stress and Wounds}

Characters have a limited ability to absorb harm before suffering long-term effects. A standard character has:
\begin{itemize}
    \item \textbf{Four Stress Boxes} – Used to absorb minor failures.
    \item \textbf{Mild, Moderate, and Severe Wounds} – Represent lasting harm or setbacks.
\end{itemize}
\DamageBox

Wounds replace traditional hit points and can reflect physical, mental, or social strain. A "Mild" consequence might be a bruised rib, while a "Severe" consequence could be a permanent injury or a shattered reputation.

\subsection{Step 6: Final Details}

With mechanics in place, players can now define their characters':
\begin{itemize}
    \item \textbf{Name} – Fitting for the setting and character concept.
    \item \textbf{Appearance} – Distinctive traits, clothing, and demeanour.
    \item \textbf{Personality} – Key personality traits, motivations, or quirks.
    \item \textbf{Backstory} – A brief origin story or notable past experiences.
\end{itemize}

\begin{CommentBox}{Final Advice for Players}
    \textbf{Focus on character over numbers.} \wyrd is designed for narrative-driven play, so build a character that fits the story rather than optimising for maximum efficiency.
\end{CommentBox}

Once these steps are complete, the character is ready for play!

\WyrdFooterImage{img/pageart/bottom-gear-2}

\end{multicols}
\newpage
\section{Key NPCs}

In the following pages, you will find a selection of key NPCs designed to serve as recurring figures. Each character includes a brief description, a glimpse into their background, and a set of traits that can be used to enrich their presence, deepen interactions, and support the unfolding mystery in your game.

\begin{description}
    \item[Mr Alton Merriweather (page \pageref{npc:alton-merriweather})] --- The Chief Steward of the Grand Hall of Inquiry, Mr. Merriweather is a master of decorum and logistics. He oversees the estate’s day-to-day operations with clockwork precision, ensuring that investigators are well supplied and guests properly screened.

    \item[Inspector Quentin Hale (page \pageref{npc:inspector-hale})] --- A rising figure in the Metropolitan Police, Inspector Hale is known for his unwavering belief in procedure and a deep mistrust of private investigators. He often finds himself at odds with the Grand Society of Inquiry, viewing them as a disruptive influence on lawful investigation.
    
    \item[Kip “Knuckles” Mallory (page \pageref{npc:kip-mallory})] --- A streetwise information broker with a network of contacts throughout the city, Knuckles trades in secrets, half-truths, and debts too dirty for polite society. He is quick with a grin and quicker to vanish when the heat is on.

    \item[Dr Octavius Wren (page \pageref{npc:octavius-wren})] --- A brilliant but eccentric scientist, convinced that automata are gaining sentience. Publicly the leader of \emph{The Aetheric Liberty Assembly}, advocating for the rights of sentient machines, but secretly runs \emph{The Automata Liberation Army}—a radical group that seeks to free automata from oppression. He is a master of aetheric technology and has a knack for creating bizarre inventions.

\end{description}

\clearpage
    \subsection{{\small Chief Steward of the Grand Hall}\\ Mr Alton Merriweather}
    \label{npc:alton-merriweather}

        \emph{Unflappable, efficient, and eternally composed, Mr. Merriweather has served the Grand Society for over four decades—and he has never once been surprised.}
        \vspace{.5\baselineskip}
      
    \columnratio{0.375,0.375,0.25}
    \begin{paracol}{3}
        \subsubsection*{Background:}
        Mr Alton Merriweather has served as the chief butler and steward of the Grand Hall of Inquiry since the Society's early days. A master of decorum and logistics, he oversees the estate’s day-to-day operations with clockwork precision. Few know that he was once a field agent himself—though those who glimpse the faint scars beneath his cuffs might suspect a deeper past.
      
        Mr Merriweather maintains the perfect balance of discretion and authority. He ensures that investigators are well supplied, guests properly screened, and that no detail in the Grand Hall ever falls into disorder. While he speaks in clipped, courteous tones, there is steel behind his gaze and loyalty in every action.
      
        \switchcolumn
        \subsubsection*{Using in Play:}
        Mr Merriweather is an anchor NPC—reliable, ever-present, and a point of continuity between investigations. He can:
        \begin{itemize}
          \item Deliver mission briefings or dossiers from the Society’s analysts.
          \item Provide subtle guidance or nudge players toward overlooked details.
          \item Secure equipment, lodgings, or discreet transport.
          \item Offer cryptic remarks hinting at the Society’s deeper secrets.
        \end{itemize}
        He is not meant to overshadow the players, but rather to support them—like the butler in a mystery novel who knows more than he lets on. In times of need, he may reveal surprising resourcefulness, especially if the Grand Hall is ever under threat.
      
        \switchcolumn      
        \subsubsection{Skills}
            \noindent\Expert: Etiquette \\
            \noindent\Skilled: Insight, Logistics \\
            \noindent\Novice: Stealth, Medicine, Presence \\
        \subsubsection{Traits}
          \textbf{Unseen, Unshaken} — Once per session, appear at just the right moment—regardless of obstacles or distance.
      
    \end{paracol}
    \vspace{.5\baselineskip}
    \hrule
    \vspace{.5\baselineskip}

    \subsection{{\small By-the-Book Investigator}\\ Inspector Quentin Hale}
    \label{npc:inspector-hale}
    
    \emph{A stern and rising figure in the Metropolitan Police, Inspector Hale is known for his unwavering belief in procedure and a deep mistrust of private investigators.}
    \vspace{.5\baselineskip}
    
    \columnratio{0.375,0.375,0.25}
    \begin{paracol}{3}
        \subsubsection*{Background:}
        Inspector Quentin Hale is a career man with aspirations of high office. Intelligent, meticulous, and unyielding, he considers the Grand Society of Inquiry a disruptive influence on lawful investigation. While not antagonistic out of malice, his dedication to procedure and political advancement frequently puts him at odds with the Society’s methods. Despite this, he may occasionally seek their help when a case falls outside conventional explanation—grudgingly, of course.
        
        \switchcolumn
        \subsubsection*{Using in Play:}
        Inspector Hale works best as a recurring foil or rival—an NPC who applies pressure, raises stakes, and reminds players that their investigations exist within a broader system of law and politics. He may:
        \begin{itemize}
          \item Attempt to take over a case or block access to key evidence.
          \item Arrest a scapegoat if the players delay or antagonize him.
          \item Undermine the Grand Society’s reputation with the authorities.
          \item Call on the players in private when a case becomes “irregular.”
        \end{itemize}
        Use Hale to inject conflict, force clever diplomacy, or complicate scenes where the players operate in the open.
        
        \switchcolumn      
        \subsubsection{Skills}
            \noindent\Expert: Reasoning \\
            \noindent\Skilled: Discipline, Command \\
            \noindent\Novice: Awareness, Presence, Investigation \\
        \subsubsection{Traits}
            \textbf{Procedure is Power} — Gains a bonus when solving problems by following official protocols to the letter.\\
            \noindent\textbf{Authoritative Glare} — Can reroll when using rank or command presence to compel obedience.\\
    \end{paracol}

    \clearpage
    \subsection{{\small Whisper Broker}\\ Kip “Knuckles” Mallory}
    \label{npc:kip-mallory}
    
    \emph{Quick with a grin and quicker to vanish, Knuckles trades in secrets, half-truths, and debts too dirty for polite society.}
    \vspace{.5\baselineskip}
    
    \columnratio{0.375,0.375,0.25}
    \begin{paracol}{3}
        \subsubsection*{Background:}
        Kip Mallory, known on the streets as “Knuckles,” is an ex-pickpocket turned information broker. With a network of urchins, cabbies, and dockhands, he collects the underbelly’s whispers about crimes, scandals, and disappearances. Though rough around the edges, he is clever, pragmatic, and loyal to those who pay fairly and ask the right way.
        
        \switchcolumn
        \subsubsection*{Using in Play:}
        Knuckles is ideal for providing street-level intel—revealing cryptic leads and information the authorities overlook. He can:
        \begin{itemize}
          \item Drop hints about recent events or persons of interest.
          \item Offer minor favors in exchange for coin or a promise of future assistance.
          \item Connect players with the criminal underworld or serve as a bridge to dubious allies.
          \item Betray the party if their reputation becomes too dangerous.
        \end{itemize}
        Use him to add local color, steer investigations, and introduce tension from the shadows.
        
        \switchcolumn      
        \subsubsection{Skills}
            \noindent\Expert: Streetwise \\
            \noindent\Skilled: Deception, Awareness \\
            \noindent\Novice: Stealth, Presence, Mobility \\
        \subsubsection{Traits}
            \textbf{Too Quick to Catch} — Can reroll when evading capture or disappearing into a crowd.\\
            \noindent\textbf{Favour for a Favour} — Once per session, declare a helpful contact or resource—but you’ll owe Knuckles for it later.
    \end{paracol}
    \vspace{.5\baselineskip}
    \hrule
    \vspace{.5\baselineskip}

    \subsection{{\small Grand Artificer of the Aetheric Liberty Assembly}\\ Dr Octavius Wren}
\label{npc:octavius-wren}

    \emph{Visionary, rebel, and scholar of the forbidden spark. Dr Wren dreams not of progress, but of liberation through invention.}
    \vspace{.5\baselineskip}
  
    \columnratio{0.375,0.375,0.25}
    \begin{paracol}{3}
    \subsubsection*{Background:}
    Once a lauded professor at the Royal College of Natural Philosophy, Dr Octavius Wren vanished from public life after his controversial treatises on sentient automata and free energy were suppressed by the Crown. Years later, he re-emerged as the charismatic leader of the Aetheric Liberty Assembly—a coalition of inventors, exiles, and rogue thinkers who believe true freedom lies in decentralised aetheric technology.

    Secretly, he has been building the Automata Liberation Army from the more radical members of the Assembly. He believes that automata are gaining sentience and that they deserve the same rights as humans, and he and the Liberation Army are willing to go to any lengths to achieve this goal.

    \switchcolumn
    \subsubsection*{Using in Play:}
    Dr Wren can serve as:
    \begin{itemize}
      \item A philosophical foil to players who favour order over innovation.
      \item A source of illicit information, rare devices, or urgent warnings.
      \item A wildcard ally when a common threat emerges—though always on his own terms.
      \item The architect of a grand aetheric event that spirals out of control.
    \end{itemize}
    Though he speaks of liberty, Wren may sacrifice much in the name of progress—including people.

    Wren sees the Grand Society not as enemies, but as blind custodians of a decaying system. His speeches blend poetic fervour with mechanical insight, and his presence inspires fierce loyalty among his followers. Though soft-spoken and refined, there’s an unmistakable intensity in his eyes—a man who has glimpsed a world remade.

    \switchcolumn
    \subsubsection{Skills}
        \noindent\Expert: Engineering \\
        \noindent\Skilled: Rhetoric, Lore \\
        \noindent\Novice: Deception, Insight, Resources \\
    \subsubsection{Traits}
        \textbf{Voice of the Future} — When addressing a crowd or debating ideology, Wren gains advantage and can shift the mood of a scene. \\

    \end{paracol}
    % \vspace{.5\baselineskip}
    % \hrule
    % \vspace{.5\baselineskip}

