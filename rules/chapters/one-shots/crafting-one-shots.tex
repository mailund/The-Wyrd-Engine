A one-shot is a complete story told in a single session. Whether it's used for convention play, introducing new players, or exploring side stories, one-shots offer a focused, low-commitment narrative experience. Unlike ongoing campaigns or episodic series, they demand tight storytelling, strong hooks, and clear structure. Done well, they leave a lasting impact—and just enough mystery to haunt players after the final scene fades.

\section[What Makes One-Shots Unique]{What Makes\\ One-Shots Unique}

\subsection*{Contained Storytelling}
A one-shot begins and ends within a single session (typically 2–4 hours). There may be loose ends, but the central conflict must resolve within that time.

\subsection*{Limited Time, Focused Impact}
There is no room for sprawling subplots or excessive setup. Instead, the narrative must deliver immediate intrigue and fast escalation.

\subsection*{Character Simplicity}
Characters often have fewer complications or long-term arcs. Strong archetypes and clear motivations help players engage quickly.

\subsection*{Tone and Pacing}
One-shots often lean into strong tonal choices—horror, comedy, tragedy, or pulp adventure—because they don’t need to support tonal variation over time.

\section[Strengths and Limitations]{Strengths and\\ Limitations}

\subsection*{Strengths}
\begin{itemize}
    \item Easy to run with rotating or new players.
    \item Ideal for playtesting ideas, systems, or settings.
    \item Lower commitment encourages experimentation.
    \item Great for introducing your setting without overwhelming detail.
    \item Allows for high-impact, high-risk storytelling.
\end{itemize}

\subsection*{Limitations}
\begin{itemize}
    \item Limited time for character development or emotional depth.
    \item Harder to incorporate slow-build mysteries or subtle foreshadowing.
    \item Players unfamiliar with the system may need more support.
    \item Can feel rushed if poorly paced.
\end{itemize}

\section[Designing for One-Shots]{Designing for\\ One-Shots}

\subsection*{1. Strong Opening Hook}
Begin in the middle of the action or mystery. Skip slow setup—players should be invested within the first ten minutes.

\subsection*{2. Simple, Compelling Premise}
Keep the premise tight and clear. Examples:
\begin{itemize}
    \item A murder at midnight during a storm.
    \item An ancient vault opens for one night only.
    \item A ritual has begun—someone must stop it, or finish it.
\end{itemize}

\subsection*{3. Manageable Scope}
Limit the number of major NPCs, scenes, or factions. Three to five major beats is a good rule of thumb.

\subsection*{4. Clear Stakes and Urgency}
Give the players a reason to act now—time limits, escalating threats, or personal consequences.

\subsection*{5. Flexible Endings}
Prepare for a few possible outcomes, but don’t over-prepare. Be ready to adapt the resolution to the players’ choices.

\subsection*{6. Evocative Setting with Minimal Exposition}
Use bold, sensory descriptions. Establish tone with a few well-chosen details rather than lore dumps.

\section{Tools for Success}

\subsection*{Pre-Generated Characters}
Provide pre-built characters with short backstories, defined goals, and ties to the central conflict. This speeds up onboarding and ensures every PC has a stake in the story.

\subsection*{Scene Structure}
Use a modular outline:
\begin{itemize}
    \item \textbf{Scene I – The Hook:} Drop players into a mystery or conflict.
    \item \textbf{Scene II – Investigation or Complication:} Uncover clues or escalate the threat.
    \item \textbf{Scene III – Revelation or Confrontation:} Force a decision, battle, or twist.
    \item \textbf{Scene IV – Fallout or Resolution:} End on a strong note—closure, tragedy, victory, or a haunting question.
\end{itemize}

\subsection*{Running Tips}
\begin{itemize}
    \item Keep things moving—cut slow scenes quickly.
    \item Let players shape the tone and pace where possible.
    \item Use flashbacks or mid-game revelations to tie characters into the story.
    \item Embrace bold player decisions—they’re often the most memorable part of the session.
\end{itemize}


\section{Conclusion}

One-shots are like ghost stories told around a fire—brief, powerful, and unforgettable when done right. They reward clarity, creativity, and bold decisions. Whether you’re running a single night of suspense or laying the groundwork for something bigger, crafting a one-shot is an art worth mastering.