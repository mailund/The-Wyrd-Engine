% !TeX root = ../../wyrd.tex


A one-shot is a complete story told in a single session, typically designed to last between two and four hours. Whether run at a convention, as a standalone evening of entertainment, or to introduce new players to the world of tabletop roleplaying, one-shots provide a focused, low-commitment narrative experience. They are an ideal format for groups who want a compelling story without the long-term investment of a full campaign.

Unlike ongoing campaigns or episodic series—where narratives can unfold gradually, subplots evolve over weeks, and characters undergo slow-burn development—a one-shot demands immediacy. The story must hook players quickly, progress cleanly, and resolve within a tight timeframe. This brevity encourages bold decisions, dramatic reversals, and heightened stakes, often resulting in intense, cinematic sessions that stick in the memory long after the final dice have fallen.


\section[What Makes One-Shots Unique]{What Makes\\ One-Shots Unique}

The beauty of a one-shot lies in its constraints. There is little room for filler scenes, wandering digressions, or downtime. Every moment must serve the core story. Because of this, one-shots reward focused design: a strong premise, clear player objectives, and a defined structure that builds tension and momentum. When done well, a one-shot leaves players satisfied—but also curious. It’s not uncommon for a tightly written one-shot to linger in a group’s imagination, tempting players to return to those characters, revisit that world, or ask what would’ve happened if they had made a different choice.

One-shots are a distinctive storytelling format within tabletop gaming. They are not simply shorter versions of campaigns, but a format with its own strengths, challenges, and design philosophy. Understanding what makes one-shots unique will help GMs craft more effective and satisfying sessions.

\subsection*{Contained Storytelling}

A one-shot begins and ends within a single session. There may be thematic connections to other stories or hints of a broader world, but the central conflict must be introduced, explored, and resolved within a tight narrative window. This sense of containment allows players to commit more fully to riskier choices, explore unfamiliar characters, or embrace tragic endings without long-term consequences. While the story may leave a few mysteries unresolved—sometimes deliberately—it should feel whole and meaningful as a standalone experience.

\subsection*{Limited Time, Focused Impact}

Time is the most precious resource in a one-shot. With only a few hours to work with, there’s no room for elaborate exposition, slow builds, or aimless wandering. The session must begin with immediate tension or curiosity, and each scene must push the story forward. One-shots benefit from a clear goal, strong pacing, and a structure that avoids downtime. This focused design creates space for big character moments, daring actions, and impactful resolutions—all in a single sitting.

\subsection*{Character Simplicity}

Because one-shots don’t allow for long-term arcs, character design should emphasise clarity and intent. Strong archetypes and bold personalities help players make fast decisions and stay engaged. Pre-generated characters often work best, especially if they’re tied directly into the story through motivations, secrets, or relationships. Depth isn’t sacrificed—it’s condensed. A single moment of doubt, sacrifice, or revelation can define a one-shot character more than hours of gradual growth in a campaign.

\subsection*{Tone and Pacing}

One-shots work best when they commit to a specific tone. Whether it’s gothic horror, screwball comedy, gritty noir, or romantic tragedy, a one-shot has the freedom to lean hard into style without needing to balance it across multiple sessions. This makes one-shots excellent vehicles for experimenting with genre, mood, and emotional intensity. Pacing should match the tone—frantic in a heist, methodical in an investigation, eerie and lingering in horror. The best one-shots feel like a short story or a single, memorable episode of television: self-contained, stylish, and confident in what they’re trying to do.


\section[Strengths and Limitations]{Strengths and\\ Limitations}

One-shots are not just shorter adventures—they are a distinct narrative form with their own design strengths and structural limitations. Understanding what they do best (and where they can falter) will help GMs make the most of the format and avoid common pitfalls.

\subsection*{Strengths}

One-shots offer a number of unique advantages that make them especially appealing for both new and experienced groups. Their flexibility, accessibility, and high-impact storytelling potential are ideal for fast, focused play.

\begin{itemize}
    \item \textbf{Easy to run with rotating or new players.}  
    One-shots require no long-term commitment and minimal backstory, making them perfect for players who are new to the hobby or only available for a single session. Their self-contained nature means players can jump in and out without disrupting a larger arc.

    \item \textbf{Ideal for playtesting ideas, systems, or settings.}  
    Running a one-shot is a great way to test a new ruleset, character option, or narrative concept. GMs can experiment freely without needing to rebalance for a long campaign or worry about continuity between sessions.

    \item \textbf{Lower commitment encourages experimentation.}  
    Both players and GMs can try bold or unconventional approaches in a one-shot. Characters can take big risks, explore flawed or extreme personalities, or even embrace tragic fates—all without worrying about long-term consequences.

    \item \textbf{Great for introducing your setting without overwhelming detail.}  
    Because one-shots work best when focused, they offer an excellent way to showcase a setting in bite-sized pieces. You can introduce themes, factions, or locations without requiring players to memorise pages of lore.

    \item \textbf{Allows for high-impact, high-risk storytelling.}  
    The time limit encourages fast pacing and big emotional or narrative moments. Characters might die, betray each other, or uncover terrifying truths. These stories can leave lasting impressions precisely because they don't have to be safe or sustainable over time.
\end{itemize}

\subsection*{Limitations}

Of course, one-shots also come with certain constraints. Some of these are creative limitations; others are logistical or structural. Being aware of them helps ensure they don’t become stumbling blocks in play.

\begin{itemize}
    \item \textbf{Limited time for character development or emotional depth.}  
    Without multiple sessions to build relationships or inner conflicts, characters may feel flatter or more archetypal. It’s up to the players and GM to pack meaning into fewer scenes.

    \item \textbf{Harder to incorporate slow-build mysteries or subtle foreshadowing.}  
    Complex plotlines that rely on gradually seeded clues, long-term suspense, or evolving dynamics often don’t work in a one-shot. Stories must be front-loaded with intrigue and provide satisfying payoff without overcomplication.

    \item \textbf{Players unfamiliar with the system may need more support.}  
    If the system is new to the group, valuable time can be lost to rules explanations. This is especially true if character creation isn’t handled beforehand. Pre-generated characters and cheat sheets are strongly recommended.

    \item \textbf{Can feel rushed if poorly paced.}  
    A one-shot with too many moving parts—or no clear goal—can collapse under its own weight. Without a tight structure, the story may lose momentum, leading to either an abrupt ending or an unsatisfying conclusion.
\end{itemize}


\section[Designing for One-Shots]{Designing for\\ One-Shots}

Designing a one-shot is an exercise in precision. With limited time and a clear endpoint, every choice—plot, pacing, characters, setting—needs to serve the story efficiently. What follows are six key principles that will help you structure memorable, engaging one-shots that feel complete, even within a few short hours of play.

\subsection*{1. Strong Opening Hook}

The first ten minutes of a one-shot are crucial. You don’t have time to build tension slowly or establish elaborate backstories. Begin in the middle of something already happening: a body on the floor, a letter in hand, a fire in the distance, a scream from the next room. Let the players start with a question they urgently want answered.

Opening “in media res” works especially well—drop them into a scene and ask how they got there. For example:
\begin{Example}{}
    You’re standing in the centre of a locked theatre. The lights have just gone out. There’s blood on the stage. What do you do?
\end{Example}

A strong hook creates buy-in. It immediately signals that something is happening, that it matters, and that the players have agency in how it unfolds.

\subsection*{2. Simple, Compelling Premise}

Your core premise should be easy to explain in a sentence or two. Avoid sprawling setups, multiple unrelated mysteries, or complex histories that require lengthy exposition. Instead, centre your story around a clear conflict or question. Good one-shot premises often contain a ticking clock or inherent mystery. For example:
\begin{itemize}
    \item A murder occurs at a noble’s estate just before a storm traps everyone inside.
    \item A long-sealed vault opens at sunset—and stays open for only one night.
    \item A ritual is underway. The players must decide whether to stop it, complete it, or escape before it finishes.
\end{itemize}

A simple premise isn’t shallow—it’s focused. It allows room for character development and twist endings without getting lost in setup.

\subsection*{3. Manageable Scope}

It’s tempting to include everything: rich worldbuilding, dozens of NPCs, multiple red herrings, and a twisty plot. Resist that urge. One-shots thrive on tight focus. Choose one primary conflict, three to five major beats (scenes or locations), and a handful of meaningful NPCs. Any more risks bloating the runtime or confusing the players.

Instead of building a sprawling world, suggest it with detail. An overheard rumour, a newspaper headline, or a coded message can hint at a larger setting without stealing time from the current story.

\subsection*{4. Clear Stakes and Urgency}

To keep the story moving, players need a reason to act—and act soon. Stakes should be personal, immediate, or irreversible. Give the players something to care about: a friend in danger, a mysterious disease spreading, a deadline that can’t be ignored.

Urgency can take many forms:
\begin{itemize}
    \item A ritual completing at midnight.
    \item A train leaving in one hour—with the killer on board.
    \item A crumbling structure where every moment increases the risk of collapse.
\end{itemize}

Whether dramatic or subtle, the stakes should always give the players something to lose and something to gain.

\subsection*{5. Flexible Endings}

One-shots often live or die by their final scene. Aim to prepare two or three possible outcomes, but remain ready to improvise based on player choices. A satisfying ending doesn’t mean wrapping up every detail—just resolving the core conflict or answering the central question.

Let endings reflect the tone of the session. Horror one-shots may end in ambiguity or dread. Comedic ones might end in chaos. Don’t be afraid of tragic outcomes, moral ambiguity, or unanswered questions—especially if they tie into character choices.

\subsection*{6. Evocative Setting\\ with Minimal Exposition}

You only have a few scenes to establish your world—make them count. Use vivid sensory detail to ground players: the smell of wet parchment in a forgotten archive, the flicker of blue flame beneath a cracked cathedral dome, the screech of iron wheels on cobblestone tracks.

Avoid lore dumps. Instead, let the setting emerge through what the players see, hear, and interact with. A well-crafted description or strange encounter can do more than a page of backstory. Trust players to infer the world—they don’t need to know everything, just enough to believe in it.





\section{Tools for Success}

Even the best story concept can stumble without the right preparation. Fortunately, one-shots don’t require hours of worldbuilding or elaborate stat blocks. What they do require is clarity, pacing, and tools that support fast, immersive play. The following techniques will help you get the most out of your session while minimising prep and maximising player engagement.

\subsection*{Pre-Generated Characters}

One of the most effective ways to streamline a one-shot is by using pre-generated characters. These should be more than stat blocks—they should come with defined goals, short backstories, and, most importantly, a connection to the central conflict of the session.

A good pre-gen character:
\begin{itemize}
    \item Has a clear personality or archetype (e.g., “the bitter ex-soldier,” “the naive apprentice,” “the sceptical scholar”).
    \item Possesses a reason to care about the unfolding events.
    \item May have secrets, suspicions, or goals that introduce tension or opportunity.
\end{itemize}

Offer players 4–6 pre-gens with distinct roles and tones. Encourage them to pick quickly—part of the magic of one-shots is diving into a character you didn’t overthink.

\subsection*{Scene Structure}

While improvisation is a core strength of many games, one-shots benefit greatly from having a modular outline. Think of it like a four-act structure—enough to provide guidance without locking you into a rigid script.

\begin{itemize}
    \item \textbf{Scene I – The Hook:} Start with a compelling problem or inciting incident. It should immediately raise questions and give the players something to react to.
    \item \textbf{Scene II – Investigation or Complication:} The players follow leads, uncover hidden truths, or realise things are worse than they seemed. Introduce new stakes or challenge their assumptions.
    \item \textbf{Scene III – Revelation or Confrontation:} This is the climax—whether it's a dramatic confrontation, a moral dilemma, or a terrifying discovery. Choices made here shape the ending.
    \item \textbf{Scene IV – Fallout or Resolution:} Close on a note that suits the tone—whether satisfying or unsettling. Let players reflect on what they achieved… or unleashed.
\end{itemize}

This structure keeps the pacing tight while giving you the flexibility to adapt in response to player actions.

\subsection*{Running Tips}

Even with strong design, the GM’s role in guiding tone and flow is essential. The following tips help ensure your one-shot delivers on its promise:

\begin{itemize}
    \item \textbf{Keep things moving.} Don’t be afraid to cut away from slow moments. If a scene drags, fast-forward to the next turning point. Momentum matters more than completeness.
    
    \item \textbf{Let players shape the tone.} While you may have a vision for a dark horror story or light-hearted caper, players will bring their own energy. Adapt to it when possible. One-shots thrive when the group buys into the tone together.
    
    \item \textbf{Use flashbacks or revelations.} If characters feel disconnected, introduce flashbacks, visions, or sudden memories to tie them into the story. These techniques can deepen character engagement with minimal setup.
    
    \item \textbf{Embrace bold choices.} The best one-shots often veer off-course. Don’t resist it—lean in. Let the story twist around the players' decisions. That’s where the real magic happens.
\end{itemize}

\section{Conclusion}

One-shots are like ghost stories told around a fire—brief, powerful, and unforgettable when done right. They reward bold decisions, clear design, and creative risks. Whether you're crafting a tense investigation, a surreal magical mystery, or a tale of tragic heroism, the one-shot format invites you to say something sharp and lasting, all in a single sitting.

Think of one-shots not as smaller stories, but as distilled ones. They are opportunities to explore strange ideas, give players dramatic moments, and create a complete narrative arc in a fraction of the time. Whether you’re running a stand-alone adventure, testing a setting, or setting the stage for something larger, crafting a one-shot is an art worth mastering—and one of the most satisfying forms of play in tabletop gaming.

In the following pages you will find a set of one-shot adventures designed to showcase \wyrd. Each adventure is self-contained, with pre-generated characters, a clear premise, and a focused structure. They are designed to be run in a single session, but can be expanded or adapted to fit your group’s playstyle. Use them as a starting point, a template, or a source of inspiration for your own one-shot creations. And remember: the best one-shots are the ones that linger in the imagination long after the final dice have fallen.