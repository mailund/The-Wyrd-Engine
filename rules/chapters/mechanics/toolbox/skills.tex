% !TeX root = ../../../wyrd.tex

\section{Adapting Skills}\label{toolbox:sec:adapting-skills}
\index{Adapting Skills}
\index{Skills!Adapting}
\index{Skills}

Skills are the backbone of most mechanical interactions in \wyrd, and adapting them is one of the most direct ways to tailor the system to your setting. The core rules offer a streamlined and versatile skill list, but you are encouraged to reshape it to fit the needs of your world. Whether you're adjusting for a specific genre, introducing new types of conflict, or simply looking to give characters more specialised roles, modifying the skill system allows you to bring tone and theme to the forefront. In the sections that follow, we explore the different ways you can adapt the skill list—by changing its level of detail, adjusting how skills function, and incorporating setting-specific elements.

\subsection{Custom Skill Effects}\label{toolbox:custom-skill-effects}
\index{Skills!Custom Effects}

Although most skills work the same way—rolling to overcome an obstacle or create an advantage—you can add unique effects or permissions for specific skills. For instance, a \textit{Fear} skill might be usable as an attack in horror-themed games. A \textit{Magic} skill might allow creating temporary advantages with elemental force. A \textit{Politics} skill might interact with faction reputation.

These additions can make your skill list feel more alive and setting-specific, especially when certain skills enable special actions others cannot perform.

\begin{CommentBox}{Skill-Based Worldbuilding}
    If only characters with the \textit{Occult} skill can perceive spirits, that says something about your world.  
    If everyone has \textit{Hacking}, it says something else. Use your skill list to show what’s normal—and what’s extraordinary.
\end{CommentBox}

When it comes to skills, also keep in mind that NPCs don’t always need to follow the same rules as player characters. In fact, giving NPCs unique abilities or expanded uses of existing skills can reinforce the tone of your setting and elevate the drama.
    
For example, in an Urban Fantasy game where the player characters are ordinary humans working for a secret government agency, you might restrict players from using magic. However, supernatural NPCs—such as the Fay—could wield a distinct \textbf{Magic} skill to bend reality, conjure illusions, or enchant the environment. This sharp contrast reminds players that they're operating in a world full of forces they don’t fully understand or control.
    
Alternatively, you can give NPCs unique applications of skills the players do have. A Fay creature might use \textbf{Deception} not merely to lie, but to weave illusions or subtly reshape perception—while the same skill, when used by a player, is limited to mundane falsehoods. This approach lets you reinforce the supernatural as uncanny and dangerous, even when using familiar mechanics.

\subsection{Cultural and Setting-Specific Skills}
Some games thrive when skill lists reflect cultural knowledge, world assumptions, or unique technologies. For example, a post-apocalyptic setting might replace \textit{Technology} with \textit{Scavenging} or \textit{Repurposing}. A faerie-tale world might include \textit{Glamour}, \textit{Wyrd}, or \textit{Bargaining} as standalone skills.

These decisions bring flavour and cohesion to your skill list, but be mindful of how often a given skill will be useful. A skill that only applies once or twice in a campaign may be better represented as a Trait instead.

\subsection{Hybrid and Conditional Skills}
Sometimes a skill might straddle two roles. A skill like \textit{Survival} might be used for physical endurance, wilderness navigation, and resisting fear in certain settings. You can define conditional uses for skills that serve multiple narrative purposes—just be clear with your players about what each skill covers.

You can also let certain Traits expand how a skill functions. For instance, a Trait like \textit{Soldier’s Discipline} might let a player use \textit{Will} to resist physical intimidation or pain, blurring the line between mental and physical endurance.

\subsection{Buying Skills}
\index{Skills!Buying}
\index{Skills!Skill Points}
\index{Skills!Skill Budget}

The first adjustment you can make is to the \textbf{skill budget}—the number of skill ranks available to characters during creation. The core rules assume that player characters have one +3 skill, two +2 skills, and three +1 skills. But instead of fixing the distribution of skills, you can fit the \emph{skill budget}. The default skill lists gives a total of 10 points to distribute across the skill list. You can distribute those 10 points however you like, going for fewer high-level skills or more low-level skills. 

The default budget is 10 points, which allows for a range of character builds and play styles. However, you can increase or decrease this number based on your game’s needs. In many of the adjustment ideas below, we’ll suggest a new budget that fits the changes you’re making.

\subsection{Levels of Detail}\label{toolbox:detailed-skill-lists}
\index{Skills!Granularity}
\index{Skills!Complexity}

The more detailed your skill list, the more mechanical variety your characters can express—but this comes at the cost of speed and simplicity. A short, broad skill list is ideal for games that emphasise narrative flow and improvisation. A longer, more detailed list works better for tactical play, complex investigations, or games where niche expertise matters.

The level of detail you choose will influence character identity, spotlight moments, and the kinds of stories your game is best equipped to tell. A one-shot game about magical investigators might keep things simple with a single \textit{Magic} skill, while a long-running campaign about academic wizards could break that down into \textit{Rituals}, \textit{Alchemy}, \textit{Runes}, and \textit{Summoning}.

\begin{CommentBox}{Quick Tip: Choose Your Skill Scope}
    \begin{itemize}
        \item Use \textbf{broad skills} like \textit{Combat}, \textit{Technology}, or \textit{Magic} for high-level, fast-paced games.
        \item Use \textbf{narrow skills} like \textit{Blades}, \textit{Firearms}, or \textit{Arcane Lore} for games focused on detail, strategy, or realism.
        \item Blend both by starting with broad skills and expanding only the ones that matter to your setting or your players.
    \end{itemize}
\end{CommentBox}

\subsubsection{Example: Broad vs Specific Skills}
\index{Skills!Example}

The detail levels of skills can greatly affect how characters differentiate themselves. In a game with a broad skill list, characters may feel similar even if they have different backgrounds or personalities. This can lead to a lack of distinctiveness in character roles and abilities.

Consider two characters in a swashbuckling adventure game: \textbf{Captain Elise Vaunt} is a charismatic privateer and master duelist. \textbf{Professor Thaddeus Wren} is a scholarly gentleman with a dark past in the royal navy.

If your game uses a broad skill list, both characters might look quite similar:

\begin{SkillsBox}
    \Expert  & Combat \\
    \Skilled & Persuade \\
    \Novice  & Lore
\end{SkillsBox}

You would probably \emph{play} the characters differently, but the \emph{mechanics} would be almost identical. In any situation where one of the characters is likely to succeed of fail, the other is as well. They might have different backgrounds and personalities, but their skills are so broad that they can both do everything equally well. This can lead to a lack of distinctiveness in character roles and abilities.

But with a more granular skill list, their differences become much clearer:

\vspace{0.5\baselineskip}
\noindent
\textbf{Captain Elise Vaunt:}
\begin{SkillsBox}
    \Expert  & Swords \\
    \Skilled & Intimidate \\
    \Novice  & Navigation
\end{SkillsBox}

\noindent
\textbf{Professor Thaddeus Wren:}
\begin{SkillsBox}
    \Expert  & Pistols \\
    \Skilled & Etiquette \\
    \Novice  & History
\end{SkillsBox}

Now, their roles in the story and in gameplay feel much more distinct. Elise dominates in close combat and commands the deck with fearsome presence. Thaddeus excels in duels at a distance and navigates social intrigue with ease. The characters feel more unique because the skills are more focused—and that clarity can help both players and GMs build scenes where each can shine.


\subsection{Adjusting the Skill Budget}
\index{Skills!Budget}
\index{Skills!Detailed Skill Lists}

When you increase the granularity of your skill list, you should consider expanding the \textbf{skill budget}—that is, the number of skill ranks available to characters during creation. The more specific your skills become, the more ranks characters need to remain competent across the same range of activities.

For example, if a broad skill like \textit{Combat} is split into \textit{Swords}, \textit{Pistols}, and \textit{Unarmed}, a character who would have taken \textit{Combat} at +3 might now need to spread their ranks across multiple areas to reflect the same breadth of capability. Without increasing the number of skill picks available, characters become artificially limited—not because of concept or balance, but because of mechanical compression.

The goal of increasing detail isn’t to make characters weaker, but to make their abilities more specific. To maintain the same level of competence, you’ll want to give players more points to distribute when using a longer or more detailed skill list. This ensures characters feel just as capable, while allowing their specialities and limitations to emerge more clearly in play.

As a rough guideline, you might increase the total number of ranks allowed by +2 to +4 when moving from a broad list of around 10 skills to a more detailed list of 15 to 20. You may also want to slightly raise the cap for individual skills (e.g., from +3 to +4) if you want players to be able to achieve strong specialisation without sacrificing versatility.

\subsubsection{Example: Broad vs Detailed Skill Budgets}
\index{Skills!Skill Budget}
\index{Skills!Detailed Skill Lists}

Let’s compare how the same character concept can be expressed using different skill list granularities and budgets. We'll use the example of \textbf{Aria Flint}, an elite thief with a flair for infiltration and social deception.

\noindent\textbf{Using a Broad Skill List (10 Points):}

\begin{SkillsBox}
    \Expert  & Stealth \\
    \Skilled & Deceive, Athletics, Burglary \\
    \Novice  & Notice
\end{SkillsBox}

This version of Aria is quick, sneaky, and good at lying and lockpicking. With only five skill entries, she’s mechanically lean and easy to play, but her abilities are fairly generalised.

\vspace{0.5\baselineskip}

\noindent\textbf{Using a Detailed Skill List (20 Points):}

\begin{SkillsBox}
    \Expert  & Sneaking, Disguise \\
    \Skilled & Climbing, Lockpicking, Pickpocketing, Deception, Escape Artist, Urban Navigation \\
    \Novice  & Observation, Balance
\end{SkillsBox}

With 20 points to distribute across a more granular list, Aria’s skills now paint a much more specific picture. We learn that she’s not just a burglar—she’s a nimble climber, an expert in disguise, and an agile escape artist. This version allows for more detailed storytelling and spotlight moments, but would be unwieldy without the expanded skill budget.

\begin{CommentBox}{Design Principle: Equal Power, More Detail}
    The detailed version of Aria isn't more powerful—she's just more specific. Both builds cover the same narrative ground, but the granular list gives finer control over which exact abilities she excels in. To keep the experience fair and functional, the skill budget increases in proportion to the level of detail.
\end{CommentBox}

A word of caution if you take this route: the more skills you add, the more complex character creation becomes. Picking the default six skills out of a list of maybe 15 is a lot easier to do than picking 10 out of a list of 50. If you have a long list of skills, consider how many of them are likely to be used in play. If you are running a one-shot game, make sure that each skill is likely to come up at least once. If you are running a long-term campaign, consider how many skills are likely to be relevant to the characters' backgrounds and the story you want to tell.


\subsection{Default Skill Levels}
\index{Skills!Default Levels}
\index{Skills!Untrained}

In the standard \wyrd rules, characters are assumed to have a \textbf{default skill level of 0} in any skill they do not explicitly take. This makes sense when using a broad skill list—most characters can attempt common actions like running, persuading, or shooting without specialised training. A +0 represents baseline competence, where a character relies on raw talent or experience rather than honed expertise.

However, as your skill list becomes more detailed, this assumption may no longer hold. In a setting with specific and technical skills, certain tasks may reasonably require training just to attempt. For instance, it might be fair to assume that most characters can drive a ground vehicle (\textit{Driving} at +0), but not everyone knows how to perform mid-flight repairs on a damaged starship (\textit{Hyperdrive Repair} might default to \textbf{–4}, or even be unavailable without a relevant Trait).

You can adjust default values on a skill-by-skill basis. Ask yourself: could an untrained person even attempt this? If so, what’s their chance of success? If not, consider requiring a Trait, narrative justification, or a different approach entirely. You might also group certain advanced skills under prerequisites or special permissions to make their rarity explicit.

\begin{CommentBox}{Guiding Defaults}
    \begin{itemize}
        \item \textbf{+0:} Common knowledge or intuitive actions (\textit{Climb}, \textit{Charm}, \textit{First Aid}).
        \item \textbf{–2:} Specialist tasks with some public awareness (\textit{Surgery}, \textit{Codebreaking}, \textit{Forgery}).
        \item \textbf{–4:} Highly technical or dangerous skills unlikely to be attempted without training (\textit{Hyperdrive Repair}, \textit{Necromancy}, \textit{Nuclear Engineering}).
        \item \textbf{N/A:} Cannot be attempted at all without a specific Trait or background.
    \end{itemize}
\end{CommentBox}

Customising default skill levels adds flavour and helps enforce genre expectations. In a gritty cyberpunk world, even a street-savvy hacker might have no idea how to pilot a corporate dropship. In a mythic fantasy setting, few outside the high temples would dare attempt divine rituals. Use defaults not to punish players, but to shape the world and encourage meaningful choices in character creation.

\subsection{Tiered Skills \& Skill Progression}\label{toolbox:skill-progress}
\index{Tiered Skills}
\index{Skills!Tiered}
\index{Skill progression}
\index{Skills!Progression}
\index{Skills!Prerequisites}

Some settings benefit from \textbf{tiered} skill structures, where broad foundational skills unlock or govern access to more specialised ones. This structure fosters a stronger sense of mastery, progression, and narrative depth—particularly in worlds shaped by formal education, martial training, arcane study, or professional disciplines.

For example, a game might include a general \textbf{Combat} skill, which branches into sub-skills such as \textbf{Melee}, \textbf{Ranged}, and \textbf{Unarmed}. Alternatively, an academic setting might use a broad \textbf{Lore} skill that leads into more focused areas such as \textbf{Occult}, \textbf{Alchemy}, or \textbf{History}. This layered approach encourages specialisation while preserving access to wider fields of knowledge or training.

Advanced skills should offer meaningful additions to their foundational counterparts—otherwise, they risk feeling redundant. This could be as simple as enabling a character to perform tasks beyond the scope of the basic skill: for instance, \textbf{Lore} might cover mundane knowledge, while only the more specialised \textbf{Occult} skill allows understanding of the supernatural.

Alternatively, advanced skills may be more powerful than their parent skills, but with a narrower focus. These specialised skills can provide a greater bonus to specific actions than the broader skill, though only under limited circumstances where their expertise applies.

For example, the general \textbf{Fight} skill might have specialisations such as \textbf{Bladework} for sword fighting and \textbf{Martial Arts} for unarmed combat. A character with \textbf{Fight +1} would gain a \textbf{+1} bonus to all close-combat actions, while someone with \textbf{Bladework +2} would gain a larger bonus—say, double its value—for sword fighting actions, for a total of \textbf{+4}. Similarly, \textbf{Martial Arts +2} would provide \textbf{+4} to unarmed combat actions.

You could also allow the base and advanced skills to stack. For instance, a character with \textbf{Fight +1} and \textbf{Bladework +2} might receive \textbf{+1} to all close-combat actions, but gain a total of \textbf{+5} when using a sword—combining the general bonus with the specialised one.

\subsubsection*{Using Prerequisites}

Tiered skills can be structured so that a character must possess a prerequisite skill before taking the specialised version. For instance:

\begin{itemize}
    \item To gain \textbf{Martial Arts}, the character must first have at least +1 in \textbf{Unarmed}.
    \item To study \textbf{Forbidden Lore}, the character must already possess \textbf{Lore} at +2 or higher.
    \item To take \textbf{Arcane Theory}, the character must have \textbf{Magic} and a related Trait (e.g., \textit{Gifted} or \textit{Apprentice of the Circle}).
\end{itemize}

This approach provides a natural sense of progression and helps reinforce the fiction—characters learn basics before advancing to deeper or more specialised knowledge. It also gives the GM tools to gate certain abilities, reserving them for more experienced characters.


\subsubsection*{Skill Caps}

In addition to acting as a prerequisite, a parent skill can serve as a \textbf{cap} for related sub-skills. For example:

\begin{itemize}
    \item A character cannot raise \textbf{Alchemy} higher than their rank in \textbf{Lore}.
    \item \textbf{Bladework} cannot exceed the character’s \textbf{Combat} skill.
\end{itemize}

This maintains the hierarchy of skills and prevents characters from becoming disproportionately advanced in one area without first investing in the fundamentals.

\subsubsection*{Mechanical Implementation}

To incorporate tiered skills in your game, follow these guidelines:

\begin{enumerate}
    \item \textbf{Define Parent and Sub-Skills:} Clearly identify which skills require prerequisites and what their parent skills are.
    \item \textbf{Set Prerequisite Thresholds:} Decide whether the parent skill is merely required (e.g., must be taken at any rank) or if a specific threshold is needed (e.g., +2 or higher).
    \item \textbf{Apply Caps as Needed:} If using skill caps, specify that the sub-skill cannot be rated higher than its parent.
    \item \textbf{Track During Advancement:} When players advance their characters, ensure they meet all prerequisites before selecting new skills.
\end{enumerate}



\begin{CommentBox}{Tip: Use Tiered Skills to Tell a Story}
Tiered skills are more than just a mechanical tool—they're a storytelling device. When a character advances from \textit{Unarmed} to \textit{Martial Arts}, that tells us something about their growth and training. When a scholar unlocks \textit{Forbidden Lore}, it suggests a shift in worldview or a dangerous breakthrough. Use these moments as narrative milestones.
\end{CommentBox}

Tiered skills introduce additional complexity, so they should be used thoughtfully. They work best in settings where progression, discipline, or mastery are central themes. In one-shots or episodic games, the added complexity often outweighs the benefits of the extra nuance they provide. Not every game needs tiered skills—but when they suit the tone and setting, they can enrich both the mechanics and the narrative depth of your world.


\vspace{1\baselineskip}
That concludes our look at adapting the skill system. The core idea is this: use your skill list as a lever to shape tone, pace, and complexity. Add detail where it matters to your setting; strip it back where speed and clarity are more important. The more intentional your choices, the more your skill list will reinforce your game’s identity.


