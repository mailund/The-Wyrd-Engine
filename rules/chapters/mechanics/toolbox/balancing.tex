\section{Balancing}\index{Balancing}

Before diving into the techniques for adapting the core mechanics, it’s worth reflecting on the role of \textbf{balance} in your game. Many traditional RPG systems place heavy emphasis on mechanical fairness—ensuring that all player characters are equally powerful, or that monsters and challenges scale precisely to match the players. \wyrd takes a different path.

\subsection{Balancing Player Characters}

In many systems, strict mechanics—such as classes, levels, or point-buy systems—are used to maintain parity between characters. These frameworks aim to ensure that a wizard and a warrior of the same level contribute equally, especially in combat. Yet in practice, true balance is elusive. Some builds naturally excel in certain situations, while others may struggle.

\wyrd does not assume mechanical equality is either achievable or desirable. The goal is not for every character to be equally powerful on paper, but for every player to feel central to the unfolding story.

Balance in \wyrd is achieved at the \textbf{narrative level}, not the numerical one. What matters is that every character has meaningful moments to shine, and each player gets a fair share of the spotlight. A party consisting of a hulking ogre and three nimble goblins may seem wildly uneven in terms of raw power, yet be perfectly balanced in terms of narrative focus and player engagement. These contrasts often lead to the most compelling and memorable stories.

As a GM, your job isn’t to enforce symmetry—it’s to make sure every character matters. A physically weak scholar might be the only one able to decipher ancient runes. A combat-averse negotiator might defuse conflicts before they erupt. So long as each character is woven into the story and given space to act, you’ve achieved balance where it counts.

\subsection{Balancing Encounters}

Traditional RPGs often tie progression to a constant escalation of power—characters level up, enemies scale up, and encounters are carefully tuned to match. \wyrd doesn’t require this kind of calibration. Characters may grow over time, but their growth is usually narrative or situational rather than exponential.

The system is intentionally scale-independent. Since skill modifiers are relative, any improvement can be reflected by adjusting the difficulty of tasks or the competence of opposition. A +2 bonus is a +2 bonus, whether it belongs to a player or a monster—the math stays the same, but the context defines the challenge.

There’s no need for experience levels, hit dice, or challenge ratings. Just decide how hard an encounter should feel, then design your scene accordingly. The fiction comes first; the numbers support it.

\subsection{Balancing the Game}

In the adaptations and rule variants throughout this chapter—and those that follow—you’ll often see recommendations for assigning “points” to characters during creation. These are intended as flexible guidelines, not rigid frameworks. Using a consistent point budget can help players build characters with similar scope, but this should never be mistaken for true balance.

Equal point totals don’t guarantee equal narrative presence. What matters is that every character feels important to the story, is invited to act meaningfully, and receives time in the spotlight. A fragile but clever investigator may be just as vital as a hardened warrior, depending on the nature of the scene.

When designing your own content, aim for variety in the challenges you present. Let different moments favour different characters. That’s where real balance emerges—from story structure, not strict mechanics.

\begin{CommentBox}{A Note to GMs: Spotlight over Symmetry}
    It can be tempting to obsess over keeping characters mathematically equal—but don’t. Your goal isn’t to ensure everyone has the same numbers; it’s to ensure everyone has a reason to be at the table. One player might solve puzzles, another might command in battle, and a third might charm their way through a tense negotiation. As long as the story makes space for them all, you've achieved balance where it counts.
\end{CommentBox}

Ultimately, balancing a game in \wyrd means trusting your players and embracing the strengths of the narrative. Don’t worry if some characters seem stronger or weaker on paper. Focus instead on crafting stories that offer a range of challenges, emotional beats, and spotlight moments. Let the fiction breathe, and let your players surprise you. That’s the kind of balance that lasts.
