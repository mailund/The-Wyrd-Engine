
\WyrdCapLine{E}{xtending} the game rules to fit your own settings and temperament, also known as \textbf{homebrewing}\index{homebrewing}, is part and parcel of the roleplaying experience, and \wyrd is designed with this in mind.  

The core rules are intentionally light, providing a solid foundation that can support a wide variety of genres and play styles. Whether you're running gritty pulp noir, whimsical faerie tales, or a post-apocalyptic dieselpunk odyssey, the core mechanics should serve you well with minimal adjustment.

That said, different settings often call for subtle (or not-so-subtle) variations in emphasis. A cyberpunk setting may need rules for hacking and digital warfare. A high fantasy world might benefit from expanded magic systems or creature creation tools. A campaign focused on interstellar diplomacy might want more structure around social interaction, negotiation, or influence mechanics. Likewise, a game centred on intense action could benefit from more detailed combat options or gear-related traits.

\wyrd will never be a hyper-detailed or simulationist system—but it doesn't need to be. It is a flexible engine. Think of it not as a finished machine, but as a well-stocked toolbox. Pick the tools that suit your table, refine them to your taste, and don't be afraid to build new ones when the need arises. 

If you treat the rules as a starting point, rather than a strict framework, you can adapt \wyrd to power almost any story you want to tell.


\subimport{./}{tools}
\subimport{./}{balancing}
\subimport{./}{skills}
\subimport{./}{traits}
\subimport{./}{stress}
\subimport{./}{advancement}




