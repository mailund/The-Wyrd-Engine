\section{Adapting Traits}\label{toolbox:sec:adapting-traits}
\index{Adapting Traits}
\index{Traits!Adapting}
\index{Traits}

Traits are what make your character unique. They define your capabilities, limitations, and how you interact with the world. In \wyrd, traits are flexible narrative elements that reflect your identity, backstory, training, and personal style. More than just bonuses or modifiers, traits allow characters to break the normal rules of the setting, rewrite what is possible, and shape the story around their distinctive strengths—or weaknesses.

This section provides guidance on how to adapt traits to suit different genres, campaign styles, or tone, as well as how to design your own traits in a balanced and engaging way.

\subsection{Narrative Permission}\index{Narrative Permission}

In \wyrd, traits usually just provide bonuses to skill rolls under specific circumstances, but they can also grant what is called \textbf{narrative permission}—the ability to do something within the fiction of the game without requiring further justification, rolls, or explanation. This means traits aren't just passive descriptors; they give characters concrete authority to act in certain ways that would otherwise require effort, planning, or approval.

For example:
\begin{itemize}
    \item A character with the trait \textbf{Master of Disguise} can change their appearance convincingly without needing special equipment or extended preparation. Even if other characters would normally need a skill check to disguise themselves, this trait allows the user to do so automatically under reasonable conditions.
    \item A character with \textbf{Fearless} doesn't need to roll to resist fear, intimidation, or supernatural dread. They are simply unaffected—unless the source of fear is so extreme that it might overwhelm even this trait.
    \item A trait like \textbf{Royal Bloodline} might grant access to noble courts, feudal privileges, or ancient knowledge simply because the character is part of a recognised lineage.
\end{itemize}

Narrative permission encourages fast, fluid play by cutting out unnecessary rolls and debates. It also rewards players for creating characters that shape the fiction in interesting ways. If you can do it because of a trait, the world acknowledges it—even if there’s no explicit mechanical bonus attached.

\subsection{Positive and Negative Traits}\index{Traits!Positive}\index{Traits!Negative}

Traits in \wyrd can be either \textbf{positive} or \textbf{negative}, also known as \textbf{advantages} and \textbf{disadvantages}. Both types serve narrative and mechanical purposes, and both can enrich a character.

\subsubsection*{Positive Traits}

Positive traits are the traits from the core mechanics and reflect talents, privileges, or other favourable qualities. These traits can represent training, innate ability, access to resources, social standing, supernatural gifts, or other narrative advantages. Some examples include:

\begin{itemize}
    \item \textbf{Uncanny Aim} – Your shots are unnaturally accurate; \textbf{+2} to \textbf{Shoot} rolls if you take a turn to aim.
    \item \textbf{Well-Connected} – You have an extensive network of contacts and allies; \textbf{+2} to \textbf{Contacts} when on your home turf.
    \item \textbf{Resilient Spirit} – You recover quickly from magical or emotional trauma; \textbf{+2} to \textbf{Resist} rolls against supernatural effects.
\end{itemize}

\subsubsection*{Negative Traits}

Negative traits reflect flaws, limitations, vulnerabilities, or narrative complications. They are not simply penalties; they are tools for drama and depth. These come in two forms, \textbf{narrative traits} and \textbf{penalty traits}. The former are traits that the GM can invoke to create complications during the game while the latter are the simple trait bonuses, just with negative bonuses.

Some examples include:

\begin{itemize}
    \item \textbf{Short Fuse} – You lose your temper easily, often to your own detriment; \textbf{Narrative Trait}---the GM can insist that a player acts out this trait.
    \item \textbf{One Arm} – You’ve adapted well, but certain physical tasks are still challenging; \textbf{-1} to \textbf{Atheletics} and \textbf{Fight}.
    \item \textbf{Marked by the Enemy} – Your presence is easily detected by a specific group or entity; \textbf{-2} to \textbf{Contacts} when the group is involved.
\end{itemize}

The simplest form of negative traits are the \textbf{penalty traits}, which simply impose a negative modifier to a specific skill or action. These are straightforward and easy to understand, but they can feel less engaging than narrative traits.

The more complex form of negative traits are the \textbf{narrative traits}, which are more flexible and can be used to create interesting complications. These traits can be invoked by the GM to create obstacles, but they also grant players opportunities to gain story spotlight, character development, or additional resources.

For novice players, penalty traits are often easier to grasp, as they are more straightforward and less abstract. However, narrative traits can be more rewarding for experienced players, as they allow for greater creativity and engagement with the story.

\subsection{Point Budgets and Trait Balance}

To balance traits during character creation, \wyrd uses a point-based trait budget. Players are given a set number of points to spend on traits (by default three). Positive traits cost points, while negative traits can \textbf{refund} points, allowing players to afford more powerful abilities at the cost of drawbacks.

For example:
\begin{itemize}
    \item A character might have 3 trait points by default.
    \item They take three positive traits that cost 1 points each, but wish to take a fourth trait as well.
    \item To afford the fourth trait, they take a negative trait worth -1 point, giving them 1 extra to spend.
\end{itemize}

The exact values can be customised by the GM depending on the campaign’s tone. More grounded settings might limit characters to 2 or 3 points; more heroic or high-fantasy games might offer 5 or more.

\subsection{Trait Cost}

The core rules give each player character three traits, corresponding to three trait points if each trait costs one point. However, the cost of traits can vary based on their power level, narrative significance, and the overall balance of the game. Here are some general guidelines for assigning point values to traits:

\subsubsection{1-Point Traits}
These represent standard abilities, modest advantages, or situational narrative permissions. They may grant a small mechanical bonus (such as a +1 to a specific type of roll), allow a character to bypass a minor obstacle, or introduce useful resources or contacts.

    \begin{itemize}
        \item \textbf{Night Vision} – You can see clearly in low light without penalty. Ignore penalties for dim or moonlit conditions.
        \item \textbf{Quick Draw} – You may draw or switch weapons as a free action, even when surprised.
        \item \textbf{Wealthy} – You have access to significant personal funds. Once per session, you may declare you have just the right equipment, item, or bribe.
        \item \textbf{Former Soldier} – Gain +1 to \textbf{Tactics} or \textbf{Fight} when acting in structured combat or following chain of command.
        \item \textbf{Trained Tracker} – Gain +1 to \textbf{Survival} or \textbf{Notice} when following trails or identifying signs of movement.
    \end{itemize}

\subsubsection{2-Point Traits}
These traits are more powerful or versatile. They may combine a mechanical bonus with a broad narrative effect, significantly alter the rules for a particular type of action, or grant rare abilities. Traits at this level often define a character’s archetype or signature role in the party.

    \begin{itemize}
        \item \textbf{Unstoppable} – Once per scene, you may ignore the effects of a \textbf{Wound} or a failed roll and continue acting as if you succeeded.
        \item \textbf{Arcane Initiate} – You may cast Rank 1 spells from a chosen magical discipline and sense nearby sources of arcane power.
        \item \textbf{Silver-Tongued} – Gain +2 to \textbf{Rapport} in social conflicts where charm or eloquence is relevant.
        \item \textbf{Combat Mastery} – Choose one weapon type. Gain +1 to \textbf{Fight} and treat all attacks with this weapon as one step harder to block or parry.
        \item \textbf{Psychic Sensitivity} – You can detect strong emotional states and mental influence. Gain +1 to \textbf{Empathy} when reading intent or mood.
    \end{itemize}

\subsubsection{3-Point Traits}
Reserved for potent abilities, unique narrative privileges, or traits that break core assumptions of the setting. These may represent supernatural powers, elite status, ancient artifacts, or other extraordinary capabilities. Most characters will not begin play with traits at this level unless the tone of the campaign allows it.

    \begin{itemize}
        \item \textbf{Immortal} – You cannot die from age or natural causes. You ignore the first deathblow once per session and return later, scarred but alive.
        \item \textbf{Chosen by the Fates} – Once per session, you may reroll any failed roll and treat a partial success as a full success.
        \item \textbf{Bound Djinn} – You possess a powerful spirit in servitude. Once per session, it can perform a miraculous feat (teleportation, destruction, protection).
        \item \textbf{Royal Mandate} – You are recognised as a true heir to a great throne. Gain +2 to \textbf{Command} when dealing with nobility or military forces, and demand safe passage through loyal lands.
        \item \textbf{Reality Bender} – Once per scene, you may alter a small piece of the world’s logic—create a door where there was none, change gravity, or make an object vanish.
    \end{itemize}

\subsubsection{-1 or -2 Point Traits (Drawbacks/Disadvantages)}

Negative traits can be used to gain additional trait points during character creation. A -1 trait introduces a recurring complication, social disadvantage, or mild limitation. A -2 trait should be impactful, with mechanical or narrative consequences that frequently come into play. Negative traits are a great way to build flawed but compelling characters and can help reinforce the tone of darker or grittier settings.

    \begin{itemize}
        \item \textbf{Chronic Pain} (-1) – At the start of each session, roll a die. On a 1 or 2, you suffer -1 to all physical actions for the rest of the scene.
        \item \textbf{Wanted by the Law} (-1) – You are pursued by local authorities. The GM may introduce pursuit, arrest, or bounty complications at any time.
        \item \textbf{Bad Reputation} (-1) – You suffer -2 to \textbf{Charm} or \textbf{Rapport} when dealing with anyone aware of your past.
        \item \textbf{Cursed} (-2) – Once per session, the GM may declare a roll fails dramatically, regardless of the result, due to a malevolent supernatural force.
        \item \textbf{Magical Addiction} (-2) – You must use a magical effect or spell each session or suffer a -2 to all mental actions until you do.
        \item \textbf{Enemy Faction Surveillance} (-2) – A powerful group is always watching you. The GM may introduce spies, traps, or threats in any location you visit.
    \end{itemize}

These examples are not exhaustive, and GMs are encouraged to create custom traits that suit the tone and style of their game. The key is to ensure that each trait is meaningful, impactful, and relevant to the character’s identity and role in the story.

If you allow players to purchase negative traits, establish clear guidelines for how they are used. Negative traits should never render a character unplayable or be treated as a form of punishment. Instead, they should introduce compelling complications, moral dilemmas, or recurring challenges that enhance character development and storytelling.

It is also wise to place a cap on the number of negative traits a player can take. This prevents characters from becoming either too flawed or mechanically overloaded with too many bonuses. Negative traits should matter just as much as positive ones, and the GM should feel empowered to invoke them during play. However, if a character has too many, it becomes difficult to give each one the attention it deserves.


\subsection{Adapting Traits to the Setting}

One of the greatest strengths of the trait system is its adaptability. Traits can be themed to suit the tone, genre, or even specific location of a game. Setting-specific traits can deepen immersion, reinforce tone, and give characters a unique connection to the world they inhabit. Consider the following examples:

\begin{itemize}
    \item \textbf{Gritty Detective Story}
    \begin{itemize}
        \item \textbf{Streetwise} – Gain +1 to \textbf{Contacts} or \textbf{Deception} when navigating criminal circles or shady neighbourhoods.
        \item \textbf{Chronic Insomnia} – You are always alert, even when others sleep. Gain +1 to \textbf{Notice} during night scenes or stakeouts, but recover \textbf{Fatigue} one step more slowly.
        \item \textbf{Undercover Cop} – You may assume a criminal identity without suspicion. Once per session, you may declare a prior undercover relationship with an NPC.
    \end{itemize}

    \item \textbf{Mythic Fantasy}
    \begin{itemize}
        \item \textbf{Dragon-Blooded} – You are resistant to fire and may breathe flame once per session as a magical attack (Rank 2).
        \item \textbf{Voice of the Gods} – Gain +2 to \textbf{Command} or \textbf{Rapport} when delivering divine proclamations or preaching in sacred places.
        \item \textbf{Cursed by Ice} – You are immune to cold and can freeze small amounts of water with a touch, but your presence chills the air and marks you as unnatural.
    \end{itemize}

    \item \textbf{Science Fiction}
    \begin{itemize}
        \item \textbf{Cybernetic Reflexes} – Gain +1 to \textbf{Initiative} and reduce the difficulty of reactions and evasive actions by 1.
        \item \textbf{Zero-G Training} – You do not suffer penalties for operating in low or zero gravity environments. Gain +1 to \textbf{Athletics} in microgravity.
        \item \textbf{Black Market Supplier} – You have access to rare or illegal goods. Once per session, declare that you “already have” a restricted item or contact for contraband.
    \end{itemize}
\end{itemize}

The GM may also provide a curated list of setting-specific traits to guide character creation or spark inspiration. Players are always welcome to propose their own traits, provided they align with the tone of the game and offer meaningful opportunities for roleplay or mechanical impact.

\subsection{Mechanical vs Narrative Traits}

Traits in \wyrd may offer mechanical benefits (e.g. bonus to rolls, rerolls, or new uses for skills), narrative permission (e.g. bypassing obstacles or gaining automatic success), or both. The most memorable traits usually have some narrative hook—even if their primary purpose is mechanical.

For example:
\begin{itemize}
    \item \textbf{Mechanical Only:} \textbf{Combat Reflexes} – Gain +2 to your first initiative roll in a conflict.
    \item \textbf{Narrative Only:} \textbf{Member of the Silver Order} – You are a recognised member of a knightly brotherhood and can call on their aid or protection.
    \item \textbf{Hybrid:} \textbf{Veteran Duelist} – You gain +1 to attack rolls with swords and may demand formal duels in civilised lands.
\end{itemize}

When designing your own traits, try to include a narrative angle that makes the character more vivid, even if the mechanical effect is simple.

\subsection{Design Guidelines for Custom Traits}

If you're creating your own traits, either as a player or a GM---and we think you should---consider the following checklist:

\begin{itemize}
    \item \textbf{Clarity:} Is the trait’s benefit or drawback clearly defined?
    \item \textbf{Consistency:} Does it follow the tone and logic of the setting?
    \item \textbf{Impact:} Will the trait meaningfully affect play without dominating it?
    \item \textbf{Drama:} Does it lead to interesting choices, complications, or character moments?
    \item \textbf{Permission:} What does this trait allow the character to do in the fiction that others can’t?
\end{itemize}

Traits are not meant to be exhaustive or exhaustive rules; they are shorthand for what makes your character extraordinary. Think of them as storytelling fuel.

\subsection{Optional: Trait Ratings}\index{Traits!Rated}\index{Rated Traits}\index{Trait ratings}

In some campaigns, the GM may allow \textbf{rated traits}, where a trait can be taken at multiple levels (e.g., \textbf{Keen Eyesight +1}, \textbf{+2}, or \textbf{+3}). Each level increases the potency of the trait, either by improving mechanical bonuses, enhancing narrative scope, or granting additional uses. In the trait budget, the cost increases with the rating. This system introduces a more granular level of character progression and can support high-powered or mechanically detailed styles of play.

While trait ratings add complexity, they can be useful in campaigns where characters are expected to specialise deeply, develop signature abilities over time, or build toward legendary status. However, rated traits are generally not necessary for most games, and GMs are encouraged to use them only if they suit the tone and pacing of the campaign.

\subsubsection*{Rated Trait Examples}
Here are some examples of traits that scale well with levels:

\begin{itemize}
    \item \textbf{Keen Eyesight +1/+2/+3} – You gain a bonus to all visual perception checks:
    \begin{itemize}
        \item \textbf{+1} to \textbf{Notice} when spotting hidden objects or distant movement.
        \item \textbf{+2} allows you to see in poor lighting and automatically spot hidden enemies within medium range.
        \item \textbf{+3} allows you to detect movement no one else can perceive, such as invisible figures or sniper reflections.
    \end{itemize}

    \item \textbf{Arcane Affinity +1/+2/+3} – Your control over magic improves:
    \begin{itemize}
        \item \textbf{+1} reduces the Fatigue cost of spells by 1 (minimum 1).
        \item \textbf{+2} allows you to reroll one failed spellcasting attempt per session.
        \item \textbf{+3} increases your effective spell rank by +1 for all purposes (e.g., overcoming resistances or effects).
    \end{itemize}

    \item \textbf{Tough as Nails +1/+2/+3} – You can endure physical punishment that would stagger others:
    \begin{itemize}
        \item \textbf{+1} grants +1 to resist Wounds from physical attacks.
        \item \textbf{+2} allows you to ignore the effects of your first Mild Wound in each session.
        \item \textbf{+3} treats all Wounds one step less severe (e.g., Moderate becomes Mild).
    \end{itemize}
\end{itemize}

\subsubsection*{Design Guidelines}
When designing rated traits, keep the following in mind:

\begin{itemize}
    \item \textbf{Scaling Should Be Linear or Thematic.} Avoid exponential power creep. Each level should be a meaningful but manageable improvement.
    \item \textbf{Cap Levels Appropriately.} Most rated traits should max out at +3. Traits beyond that may unbalance the game or blur the line between Traits and narrative powers.
    \item \textbf{Costs Should Rise Accordingly.} A simple cost structure is 1 point per level, but GMs may require higher costs at higher levels (e.g., +2 costs 3 points total, +3 costs 5).
    \item \textbf{Narrative and Mechanical Scaling.} Consider not only bonuses to rolls but increased narrative reach—more uses per session, greater influence, or wider applicability.
\end{itemize}

\subsubsection*{Use Cases}
Rated traits work well in the following types of campaigns:

\begin{itemize}
    \item \textbf{Long-form campaigns} with extended character advancement.
    \item \textbf{High-powered settings} where legendary figures, elite soldiers, or demigods walk the world.
    \item \textbf{Point-buy campaigns} where players want fine-tuned control over power scaling.
    \item \textbf{Settings with prestige paths, guild ranks, or magical mastery} that make progression feel earned.
\end{itemize}

In more narrative or rules-light campaigns, simpler traits with fixed effects are often sufficient. Use rated traits when you want mechanical depth, tactical variation, or rewarding progression that grows with the character’s journey.

\begin{GmTips}
    Not all traits scale well. Only use rating levels for traits where each level clearly improves play in a consistent and balanced way. If you find players always taking a trait to maximum level, consider whether it’s too efficient or undercosted.
\end{GmTips}

\vspace{1\baselineskip}
Traits are the lens through which your character sees the world—and how the world responds in turn. Whether they define supernatural power, deep flaws, specialised training, or noble lineage, they shape every moment of play. Use them boldly and creatively to make characters that live, struggle, and shine.

