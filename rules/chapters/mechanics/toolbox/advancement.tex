\section{Advancement}
\index{Advancement}

Characters in \wyrd grow through experience, hardship, and meaningful change. Advancement is not simply a matter of accumulating power—it reflects how the events of the story have shaped your character, what they’ve learned, and how they’ve adapted. The following guidelines cover how skills, traits, and stress tracks can evolve over time.

\subsection{Advancing Skills}
\index{Advancement!Skills}
\index{Skills!Advancement}

Skills in \wyrd represent learned knowledge, training, or instinctive talent. Players may improve their skills at the end of significant narrative arcs or milestones, as determined by the GM. This could be after completing a major mission, surviving a personal trial, or resolving a long-running conflict.

When advancement is granted, the GM can award players a number of points to spend on skills. The number of points should be small—typically 1-3 points per session or milestone. Players can use these points to advance existing skills or acquire new ones.

\begin{Example}{Increasing a Skill Level}
    After a harrowing escape from a collapsing mine, a character might gain a point in \textbf{Athletics} for their quick thinking and agility.
\end{Example}

\begin{Example}{Acquiring a New Skill}
    Alternatively, a character who has spent time in a library might gain a point in \textbf{Research} for their newfound knowledge.
\end{Example}

\noindent
The GM may also allow players to trade points between skills, reflecting a character’s shifting focus or priorities.

\begin{Example}{Swapping Points}
    After months surviving in the wilds, a former diplomat might trade a point in \textbf{Persuade} for a new point in \textbf{Survival}, reflecting hard-earned experience.
\end{Example}

\subsection{Advancing Traits}
\index{Advancement!Traits}
\index{Traits!Advancement}

Unlike skills, traits often represent defining aspects of a character—magical ancestry, personal connections, lifelong habits, or supernatural curses. As such, trait advancement should be more deliberate and narratively justified.

To gain a new trait, a character must reach a significant milestone and demonstrate that the trait has emerged organically from the story. The GM may require:
\begin{itemize}
    \item A key event or transformation (e.g., surviving a divine trial, forging a bond with a dragon).
    \item The completion of a quest or ritual (e.g., reclaiming a family heirloom, overcoming a long-standing flaw).
    \item A narrative arc of personal change (e.g., finally facing a fear, accepting one’s destiny).
\end{itemize}

Likewise, traits may evolve. A minor trait might be upgraded to a more potent version, or one trait may be replaced by another that better reflects a new identity.

\begin{Example}{Upgrading a Trait}
    \textbf{Example:} A character with the trait \textbf{Touched by Fire} might upgrade it to \textbf{Flamebound Champion} after willingly entering a volcano to awaken a sleeping spirit.
\end{Example}

Advancement through negative traits is also possible. A character may overcome or alter a flaw through meaningful growth—but doing so should cost something. Removing a drawback may require giving up a bonus, completing a personal trial, or replacing it with a new trait that reflects the internal conflict resolved.

\subsection{Advancing Stress Tracks}
\index{Advancement!Stress}
\index{Stress Tracks!Advancement}

Stress represents a character’s ability to withstand physical harm (\textbf{Wounds}) and emotional or narrative pressure (\textbf{Fatigue}). It is a measure of their resilience in the face of adversity—not just toughness, but endurance of the spirit.

Increasing a stress track should be rare and meaningful. It should only occur:
\begin{itemize}
    \item After a major personal transformation (e.g., spiritual awakening, rigorous training, surviving a near-death event).
    \item Through a specific trait that explicitly increases a stress track (e.g., \textbf{Hardened Veteran} might grant an additional Moderate Wound).
    \item As part of a campaign milestone that raises the overall danger level, and even then, sparingly.
\end{itemize}

Avoid turning stress into an arms race. If both players and adversaries continually increase their stress tracks, conflict becomes longer but not more dramatic. Instead, use stress advancement as a way to reinforce character arcs and turning points.

\begin{Example}{Increasing a Stress Track}
    After enduring great emotional loss and confronting their past, a character gains a permanent +1 to \textbf{Fatigue}, symbolising a newfound inner strength.
\end{Example}

\subsection{Final Thoughts on Advancement}
Advancement in \wyrd should be character-driven, not mechanical. It is not about levelling up, but about evolving with the story. Encourage players to tie growth to their characters’ personal arcs—what they’ve suffered, what they’ve gained, and what they’ve become.

The GM should treat advancement as an opportunity to highlight change, celebrate character moments, and deepen the connection between mechanics and narrative.

\begin{CommentBox}{A Note to GMs: Meaningful Advancement}
    Advancement is not just about numbers; it’s about the story. Encourage players to think about how their characters have changed, what they’ve learned, and how they’ve grown. This will make the game more engaging and meaningful for everyone involved.
\end{CommentBox}
