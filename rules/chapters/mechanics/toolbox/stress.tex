\section{Adapting Stress}\label{toolbox:sec:adapting-stress}
\index{Adapting Stress}
\index{Stress!Adapting}
\index{Stress}

Stress is a core mechanic in \wyrd, representing the physical, emotional, and narrative toll that situations take on a character. The system is highly flexible and can be adapted to suit a variety of genres, themes, and playstyles. Stress tracks can even be repurposed to represent other limited resources, such as magical energy, reputation, or social standing.

\subsection{Adapting Hit Points}
\index{Stress!Hit Points}

The number of \textbf{Fatigue} and \textbf{Wound} boxes available to a character determines how much stress they can endure before suffering consequences. The default structure is intentionally flexible, but you can adjust it to better reflect the tone of your game.

\subsubsection{Cinematic Action}
For fast-paced, heroic games (such as pulp adventures or 90s action movies), increase the number of Fatigue or Minor Wound boxes. You might even rename Minor Wounds to \textbf{Flesh Wounds} to reinforce the tone. Characters can absorb more hits and keep fighting, suffering only mild penalties (e.g., -1 to rolls) as they go.

\subsubsection{Gritty Realism}
In a darker or more grounded game, turn the Minor Wound into an additional Moderate or Severe Wound. This makes characters more vulnerable to injury but with the same total stress, which lets them fight on but with greater penalties. This creates tension and highlights the cost of violence.

\subsubsection{Simplified Play}
If you don’t want to track multiple stress types, you can unify all stress into a single \textbf{Hit Point} track. In this case, every box simply represents physical wear, and damage ticks off boxes regardless of source.

\subsection{Alternative Stress Tracks}
\index{Stress!Alternative Tracks}

Stress tracks can be adapted to represent more than just physical and emotional strain. You can use them for any limited, depletable resource that plays a meaningful role in your setting. The core idea remains the same: when you use the resource, you mark off boxes. Some boxes (like Fatigue) carry no penalty; others (like Wounds) do.

\subsubsection{Magic Track}
Represents a character’s available magical energy. 
    \begin{itemize}
        \item 5 \textbf{Fatigue} boxes – represent routine spellcasting with no penalty.
        \item 3 \textbf{Drained} boxes – impose a \textbf{-1} or \textbf{-2} penalty to future casting while marked.
    \end{itemize}

\subsubsection{Sanity Track}
Suitable for horror games. 
    \begin{itemize}
        \item 3 \textbf{Stress} boxes – represent temporary fear, confusion, or shock.
        \item 2 \textbf{Madness} boxes – impose \textbf{-2} to all mental skill rolls or cause hallucinations and narrative complications.
    \end{itemize}

\subsubsection{Reputation Track:} Represents how much social capital a character can spend. 
    \begin{itemize}
        \item 4 \textbf{Favour} boxes – can be spent to gain bonuses to \textbf{Rapport}, \textbf{Contacts}, or \textbf{Command}.
        \item 2 \textbf{Scandal} boxes – impose \textbf{-1 or -2} penalties to social interactions until recovered.
    \end{itemize}

\begin{ExampleGame}{Spending Reputation Points}
    \line[Player] “I call in a favour from the guildmaster. I want them to get us through the checkpoint without questions.” \\
    \line[GM]     “Alright, mark a \textbf{Favour} box on your reputation track. You gain +2 to your \textbf{Rapport} roll with the guards.” \\
    \line[Player] “Perfect. That gives me just enough to beat the DR.”
\end{ExampleGame}

Stress tracks like these are powerful tools for structuring tension and resource management. They reward players who spend carefully—and offer compelling consequences when limits are reached.

\subsection{Stress as a Narrative Element}
\index{Stress!Narrative Use}

Stress tracks don’t have to be tied to harm or exhaustion—they can also represent a character’s capacity to shape fate, luck, or destiny.

For example, you could give each character a \textbf{Luck} track with three boxes. At any point, a player can tick a box to:
\begin{itemize}
    \item Reroll a failed check.
    \item Add +2 to any roll.
    \item Avoid a consequence or twist—\emph{if} they can justify the stroke of fortune narratively.
\end{itemize}

\begin{GmTips}
    Let players narrate how their luck manifests—a fortunate misfire, a slippery escape, a forgotten coin underfoot. This builds dramatic tension and invests them in the outcome.
\end{GmTips}

You could also use similar tracks for \textbf{Hope}, \textbf{Instinct}, \textbf{Courage}, or \textbf{Willpower}, giving players limited narrative tools to push beyond normal limits in thematically appropriate ways.

\subsection{Recovery}
\index{Stress!Recovery}

Recovery is just as important as depletion, and the pace of recovery should match the tone of your campaign.

\subsubsection{Cinematic Recovery}
In light-hearted or heroic games, allow Fatigue and Mild Wounds to recover quickly—perhaps after a short rest or between scenes. Moderate and Severe Wounds may heal fully between episodes, or require only brief narrative justification.
    
\subsubsection{Gritty Recovery} In harsher settings, recovery is slow and costly. Fatigue may require full rest, while Wounds demand medical care, time, or even consequences. Recovery may involve rolls against skills like \textbf{Healing}, or depend on sanctuary or safe downtime.

\subsubsection{Track-Specific Recovery:} 
    \begin{itemize}
        \item \textbf{Magic Track:} Fatigue boxes might recover with rest or meditation, while Drained boxes require full downtime, magical aid, or rare ingredients.
        \item \textbf{Reputation Track:} Favour may recover in friendly environments or after good deeds; Scandal may require public apologies, counter-rumours, or letting time pass.
        \item \textbf{Sanity Track:} Stress might recover after sleep or therapy, while Madness could linger unless addressed through roleplay, rituals, or personal breakthroughs.
    \end{itemize}

\subsection*{Final Thoughts on Stress}

Adapted stress tracks are a powerful design tool. They let you focus play on the kinds of pressure that matter most in your setting—whether that’s physical, emotional, magical, social, or something else entirely. Keep them clear, thematic, and impactful, and they’ll enrich both narrative and mechanics across every session.