\newcolumn
\section[Skills Based Magic: The Known and the Named]{Skills Based Magic:\\ The Known\\ and the Named}

\begin{Example}{}
    Mireya traces the final sigil into the chalk-drawn circle. Each stroke is deliberate, each angle precise. The name must be perfect. Not spoken aloud—never aloud—but inscribed with full understanding.

    “To perceive the flame,” she whispers, “you must know more than heat. You must know hunger.”

    Around her, candles gutter as unseen winds stir. She doesn’t flinch. She lays her hand on the page beside her—six glyphs, etched in gold leaf, glowing faintly.

    \textit{Perceive. Flame. Memory. Self.} 

    She speaks the final syllable.

    For a heartbeat, she sees the fire’s history: who lit it, why, the pain it consumed. The flame flickers back, knowing it has been seen.

    Magic is not in the word alone—but in the one who names it rightly.
\end{Example}

The Known and the Named is \emph{medium to very hard magic} with an elaborate system of rules and mechanics. It adds complexity and depth to the game, making it suitable for settings with a focus on magic, such as high fantasy or magical academies. It is ideal for players who enjoy tactical gameplay and character progression through mastery of magic. It is best avoided for one-shots.
