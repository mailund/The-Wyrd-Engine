
\section{Building Magic with Core Mechanics}

\wyrd is built on three simple yet powerful components: \textbf{Skills}, \textbf{Traits}, and \textbf{Stress}. These same tools can be used to create a wide variety of magic systems, from subtle enchantments to world-shaking sorcery. By using the existing mechanics in flexible ways, you can design a magic system that fits seamlessly into your setting without adding complexity for its own sake.

\subsection*{Using Skills for Magic}

One of the simplest ways to represent magic is to introduce a dedicated \textit{Magic} skill (or multiple skills for different magical traditions). This allows characters to roll to cast spells, channel energy, decipher magical texts, or sense supernatural forces.

How much you rely on skills depends on how "hard" your magic system is:
\begin{itemize}
    \item In a \textbf{soft magic} system, a Magic skill roll might be used to determine success when calling on mysterious forces or interpreting omens. The effects are largely narrative.
    \item In a \textbf{hard magic} system, you might define clear actions or effects that can be performed with a Magic skill roll, possibly using fixed difficulties or cost thresholds.
\end{itemize}

You can also split the skill into multiple domains for more granularity—\textit{Ritual Magic}, \textit{Elementalism}, \textit{Divination}, etc.—depending on how central magic is to your game.

\subsection*{Using Traits for Magic}

Traits are ideal for granting magical capabilities and defining the flavour of magic in your world. A Trait can do any of the following:
\begin{itemize}
    \item Provide a \textbf{+2 bonus} when using a skill to perform a magical action (e.g., \textit{Fire Adept} might give +2 to Magic when wielding flame).
    \item Allow a character to perform a unique magical action others cannot (e.g., \textit{Speak with the Dead}).
    \item Grant a \textbf{once per scene/session} magical effect (e.g., teleporting a short distance, summoning a spectral ally).
\end{itemize}

By combining Traits with appropriate skills, you can model everything from specialised spellcasters to innate magical creatures. Traits also work well in soft magic systems—serving as vague, evocative powers that offer narrative permission to do magical things without strict limitations.

You can also create themed Trait sets—such as schools of magic, elemental affinities, or bloodlines—to further flavour your system and character options.

\subsection*{Using Stress for Magic Costs and Risks}

Stress represents the toll magic takes on the caster. The toll doing magic does to the caster's system might be damage that the usual stress boxes will have to absorbe. Casting a powerful spell might deal 2 points of Fatigue, or cause Wounds on a botched ritual. This reinforces the idea that magic is dangerous or exhausting, and creates tension when players must choose between casting and conserving energy. Alternatively, you can introduce a separate track of stress boxes, or ``mana points'', to represent magical energy. This allows for a more tactical approach to magic use, where players must manage their resources carefully.

In more structured systems, you can assign stress costs to specific spells or magical effects. You can also track magical corruption, instability, or backlash using separate stress tracks or consequences. For example:
\begin{itemize}
    \item \textit{Casting from life force:} Wounds as cost.
    \item \textit{Psychic strain:} Fatigue or a separate “Mind” track.
    \item \textit{Chaotic magic:} On failure, take stress or roll for a side effect.
\end{itemize}

Stress can also be used to limit magic-use in a more freeform system. Instead of spell slots or mana, the caster simply takes stress each time they cast—and must choose when to risk pushing too far.

\subsection*{Tying It All Together}

Most magic systems built in \wyrd will use all three , skills, traits, and stress, in some way. For example:

\begin{itemize}
    \item A character has the Trait \textit{Stormcaller}, granting +2 to Magic when controlling weather.
    \item They use their Magic skill to attempt to summon lightning during a scene.
    \item The GM assigns a difficulty based on conditions and scope, and on a success, the spell works.
    \item Casting the storm drains 2 Fatigue, and if the character pushes further, they risk a Consequence.
\end{itemize}

With just a few consistent mechanics, you can create highly flexible and thematic magic systems that feel integrated with the rest of the game.

In the following sections, we’ll explore different examples of how to structure magic systems, both soft and hard.
