\section[Design Goals for Magic Systems]{Design Goals for\\Magic Systems}

Before diving into the mechanics of magic, it's important to consider what kind of magic best fits the story you want to tell. Not all magic systems are created equal, and not all of them need the same level of structure. In designing a system for your game, you should consider tone, genre expectations, and how much emphasis you want to place on magical abilities during play.

To help with this, we’ll frequently refer to two ends of a spectrum: \textbf{soft} magic and \textbf{hard} magic.

\subsection{Soft Magic vs Hard Magic}

A \textbf{soft magic system} is mysterious, unpredictable, and often unexplained. Magic might appear as divine will, ancient curses, or the unknowable power of nature. Its role in the story is usually thematic or atmospheric, and it's more likely to serve as a narrative device than a mechanical tool. Soft magic works well in games that lean into horror, wonder, or mythic storytelling, where the unknown is part of the appeal.

A \textbf{hard magic system}, by contrast, is defined, repeatable, and governed by rules. Players understand what magic can do and what it can’t, and their characters are often trained practitioners who rely on clear mechanics. Hard magic systems shine in tactical or high-fantasy games, where magic is a tool to be mastered, and players want to build characters who use it with precision and strategy.

Most games fall somewhere in between. A setting might use soft magic for gods and ancient powers, but provide a hard magic system for player spellcasters. Or it might begin with soft, mysterious magic that gradually becomes more structured as players learn its secrets.

\begin{CommentBox}{Soft vs. Hard Magic in Fiction}
    \subsection*{Soft Magic Examples}
    \begin{itemize}
        \item \textbf{\emph{The Lord of the Rings}} – Gandalf’s magic is powerful but undefined. We never know exactly what he can or cannot do; his power serves the story and themes rather than a consistent rule set.
        \item \textbf{\emph{A Song of Ice and Fire}} – Magic is rare, ancient, and often unknowable. Prophecies, shadowy rituals, and dragons contribute to an atmosphere of mystery.
        \item \textbf{\emph{Princess Mononoke}} – Spirits and curses operate on symbolic and emotional logic more than mechanical rules. Magic enhances the mythic tone rather than offering player-like abilities.
    \end{itemize}
    
    \subsection*{Hard Magic Examples}
    \begin{itemize}
        \item \textbf{\emph{Fullmetal Alchemist}} – Alchemy follows strict rules based on equivalent exchange. Characters learn and master the system, and much of the story turns on its limitations.
        \item \textbf{\emph{Avatar: The Last Airbender}} – Bending is tied to clear disciplines and elements. While fantastical, it has well-defined boundaries and is learned like martial arts.
        \item \textbf{\emph{Mistborn}} (Brandon Sanderson) – The magic system is fully explained, involving specific metals and predictable effects. Characters strategically plan how to use it in conflicts.
    \end{itemize}
    
    \noindent
    \textit{Tip:} Soft magic enhances wonder and mystery. Hard magic enables strategy and player agency. Choose the flavour that suits your story—or blend them.
\end{CommentBox}
    

\subsection{The Goals of Magic in \wyrd}

The Wyrd Engine treats magic as just another kind of narrative power—like stealth, combat, or persuasion. It should support character expression, meaningful choices, and dramatic moments. Whether magic is rare and ritualistic or common and codified, the goals for any magic system in \wyrd are:

\begin{itemize}
    \item \textbf{Flexibility:} Magic should adapt to your setting. The system should be easy to customise, whether you’re building druidic rites, psychic powers, arcane science, or divine miracles.
    \item \textbf{Narrative Focus:} Magic should enhance the story, not overwhelm it. Magical abilities should feel impactful, but they should also create interesting complications, choices, and consequences.
    \item \textbf{Player Agency:} Magic should be something players engage with actively. Whether casting spells or dealing with magical effects, player characters should have tools to shape outcomes and influence the world.
    \item \textbf{Simplicity:} While magic can be powerful and varied, it should not require pages of rules or countless exceptions. The system should be simple to run and easy to learn.
\end{itemize}

In the following sections, you'll find guidance for building both soft and hard magic systems using the tools provided by \wyrd. You can use the included example systems as written, or treat them as a foundation to craft something unique for your world.
