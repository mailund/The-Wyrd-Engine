\section[Boosts-Based Magic]{Boosts-Based Magic}

\begin{Example}{}
    The Warden’s Path winds through deep forests, high peaks, sunlit plains, and storm-wracked shores. It is not a road of cities or empires, but of roots, rivers, and stones warmed by ancient fire. Those who walk it do not command the elements—they listen to them, learn from them, and earn their trust.

    A Warden feels the tremor beneath the ground before it speaks. They know when the rain is mercy and when it is warning. They do not cast spells—they shape their will through discipline, ritual, and connection to the world around them.

    And when the balance is broken, they are the ones who rise to restore it.
\end{Example}

This section describes a \emph{medium-soft to medium-hard} magic system built on balance, focus, and elemental harmony. It is ideal for ritual guardians, wandering monks, or nature-bound mystics. Power flows through alignment—not domination—and magic is shaped through action, breath, and will.

Where the trait based system described earlier is mysterious and entirely narrative, the system in this section is also narratively rich, but more mechanically grounded. Magical effects are built through the core system’s \textbf{boosts} mechanic (\pagereftext{core:boosts}), creating a structured but flexible toolkit. This system emphasises preparation over spontaneity, making it less suited to fast-paced or reactive combat spells—but perfect for building tension and releasing it in dramatic, cinematic moments.

\begin{GmTips}
    The system in this section is ideal for players who want magic that is rhythmic, grounded, and tactical. The build-up mechanic supports dramatic timing, while elemental attunement reinforces a character’s identity and worldview.
    \begin{itemize}
        \item \textbf{Pros:} Evocative, balanced; supports big moments and careful planning.
        \item \textbf{Cons:} Requires forethought; less spontaneous than freeform systems.
        \item \textbf{Best For:} Elemental guardians, ritual casters, mystics, and nature-based traditions.
        \item \textbf{Not For:} Chaotic or academic spellcasters (see the spell based magic system for those).
    \end{itemize}
\end{GmTips}

\subsection{The Mechanics}

Magic in this system is built from two components: \textbf{Elemental Traits} and \textbf{Skills}.

\begin{itemize}\raggedright
    \item \textbf{Elemental Traits} represent the character’s attunement to a specific element, such as fire, earth, water, or air.
    \item \textbf{Skills} determine how that element is directed—whether to attack, defend, reshape the environment, or endure hardship. These are standard skills from the core rules.
\end{itemize}

\begin{Example}{Elemental Traits}
    \begin{itemize}
        \item \textit{Heart of the Flame} — attuned to fire and heat  
        \item \textit{Stonebound} — attuned to earth and endurance  
        \item \textit{Voice Like Thunder} — attuned to air and storm  
        \item \textit{Dancer of Tides} — attuned to water and flow  
    \end{itemize}
\end{Example}

Together, the elemental trait and skill allow the character to channel magic through action. There are two primary modes of use: \textbf{Build-Up} and \textbf{Release}. These function similarly to the optional \textbf{boosts} system (\pagereftext{core:boosts}).

\subsubsection{Build-Up}

A \textbf{Build-Up} action lets the caster use an element combined with a skill to channel magical energy into a spell. Each successful build-up adds a \textbf{+2 bonus}. These bonuses accumulate and are stored until the spell is released. If a roll fails, the bonuses are lost. (Optionally, this may also cause a magical misfire.) A tie grants no bonus but does not cause failure.

\begin{ExampleGame}{Building Up Magic}
    \line[Player] “I want to raise a stone wall between the villagers and the raiders.”
    \line[GM]     "That sounds like a difficulty of \textbf{+4}. What are you using?”
    \line[Player] “I’m using my \textbf{Stonebound} trait and \textbf{Craft} skill.”
    \line[GM]     “Sounds good. You can build up now or release with just your skill.”
    \line[Player] “I’ll build up—try to raise something strong.”
    \line[GM]     “Great. Roll \textbf{Craft}. Success gives you a +2 for when you release.”
    \line[Player] “I got a +4!”
    \line[GM]     “Success! The stones tremble at your call. You may build up further or release next round with a +2.”
\end{ExampleGame}

The magic user can use any appropriate skill to build up magic, as long as the GM agrees. As long as the player can justify it narratively, the GM should usually allow it.

\begin{ExampleGame}{Building Up Magic}
    \line[Player] “I want to build up the magic more to reach \textbf{+4}. Can I use \textbf{Notice +3} to find how the rocks in the ground can support the wall?”
    \line[GM]     “You sure can. Roll away.”
    \line[Player] “Success! I rolled a \textbf{+3}.”
    \line[GM]     “Great. Your total bonus is now \textbf{+4}!”
\end{ExampleGame}

\subsubsection{Release}

A \textbf{Release} action uses a skill and element to produce a magical effect (a standard skill roll), augmented by the built-up bonus. The element adds narrative flavour, but no further mechanical effect.

While the magic user may use tangential skills when building up, the skill used to release must be directly related to the spell’s effect.

\begin{ExampleGame}{Releasing Magic}
    \line[Player] “I’m ready to release. I have \textbf{Craft +2} and the \textbf{+4} bonus.”
    \line[GM]     “Great—you needed \textbf{+4} to succeed.”
    \line[Player] “I rolled a \textbf{+1}, so total is \textbf{+7}!”
    \line[GM]     “You call up the wall. Rocks rise and earth surges, forming a formidable barrier that shields the village!”
\end{ExampleGame}

\subsection{Elemental Limits}

Magic users can only manipulate elements they are attuned to. A character with only \textit{Dancer of Tides} cannot control fire or earth—unless they acquire a new trait through training, a quest, or mystical insight.

This encourages thematic specialisation and reinforces character identity. No two magic users need be alike.

\subsection{Optional Mechanics}

To increase risk, tension, or magical drama, you may apply any of the following optional rules:

\begin{itemize}
    \item \textbf{Burnout:} On a failed build-up, the magic user takes 1 Fatigue.
    \item \textbf{Unstable Cast:} If the magic user builds up more than \textbf{+4} and fails their release roll, the spell misfires.
    \item \textbf{Elemental Stress:} Instead of rolling, the magic user can choose to take Stress to build up magic automatically.
\end{itemize}

\subsection{In Summary}

The system in this chapter is a disciplined, balanced form of elemental magic. It rewards foresight, patience, and creative expression. The build-up and release mechanic allows players to shape tension over time and deliver dramatic payoffs. For players who enjoy strategy, elemental symbolism, and poetic precision, this system offers a deeply satisfying path.