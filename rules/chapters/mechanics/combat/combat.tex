% !TeX root = ../../../wyrd.tex


\WyrdCapLine{T}{he} core combat system outlined previously is sufficient for settings where combat isn’t a significant focus. In Agatha Christie-style mysteries, detailed combat rules would only clutter gameplay.

However, the importance and style of combat vary greatly between settings. Some games favour \textbf{quick, brutal encounters}, where a precise sniper shot or assassin’s blade swiftly ends a confrontation. Others emphasize \textbf{heroic, extended battles}, featuring characters bravely facing overwhelming odds.

The \textbf{tone and pacing of combat} should align with your game’s themes. A gritty setting might make injuries devastating and every choice critical, while a cinematic action game might allow daring heroics, letting characters survive improbable scenarios.

Players seeking \textbf{tactical complexity} may enjoy detailed positioning, cover, and resource management, rewarding careful planning. Alternatively, a more \textbf{freeform style} abstracts combat into dramatic narrative exchanges.

Furthermore, combat can significantly shape character development and storytelling. The outcomes of battles—both victories and defeats—can profoundly influence character arcs, relationships, and the broader narrative. A character who narrowly survives a deadly encounter might grapple with newfound fears or vulnerabilities, adding emotional depth to your story.

Additionally, combat encounters present opportunities for memorable narrative moments. A tense standoff, a heroic last stand, or a daring escape can become pivotal scenes that players recall long after the game ends. Thoughtfully designed combat can thus enrich the overall storytelling experience, providing dramatic stakes and moments of intense emotional engagement.

No matter your preferred style, \wyrd provides adaptable combat mechanics to suit your story and gameplay. That flexibility is the focus of this chapter.



\section{Combat Statistics}

Combat outcomes in \wyrd depend primarily on the skill difference between attackers and defenders, though dice rolls introduce some variability. Table \pagereftext{tbl:damage-probability} shows probabilities of inflicting damage based on skill disparities, expected damage per round, and average rounds needed to inflict 7+ damage (taking out a core character). The graphic on this page visually illustrates these probabilities.

\Graph[Damage per Attacker-Defender levels]{stats/damage_distribution.png}

Skill differences dominate combat outcomes by design. Equal skills favour defenders slightly, making prolonged stalemates common. A character significantly outmatched (four levels lower) has negligible chances of success, whereas a character significantly superior (four levels higher) has a substantial likelihood (41.5\%) of defeating the opponent swiftly (1.6 rounds on average).

This emphasis on skills over randomness ensures predictable yet engaging gameplay, reinforcing the strategic importance of positioning and skill management across all aspects of the game.

\section{Making Combat Interesting}

Combat shouldn’t merely be a predictable dice-rolling exercise. \wyrd balances simplicity with optional strategic depth by utilizing the same active opposition mechanics found elsewhere in the game: the attacker’s roll minus the defender’s roll directly determines damage.

Characters and gear traits influence outcomes significantly, modelling weapon effectiveness, defensive capabilities, and situational advantages realistically and intuitively.

Strategically, players should seek combat conditions that leverage their traits while minimizing their opponents’.

Tactically, \wyrd introduces combat manoeuvres: actions taken instead of direct attacks to accumulate bonuses for future strikes or defences. Defensive bonuses apply immediately upon the next attack received, presenting risk if wasted against a weak strike, while attack bonuses remain unused until deployed, creating strategic decision-making around timing and preparation. Players must balance between immediate action and building up strength for decisive, powerful moves, adding meaningful choices and tension to combat encounters.

%% FIXME: give some examples of combat maneuvers


\section{Weapons and Armour}


\section{Designing Combat Encounters}




\end{multicols}
\clearpage
\begin{DndTable}[header=Damage probability by relative skill level]{crrrrrrrr}\label{tbl:damage-probability}
    \textbf{Attack - Defence} & \textbf{0 stress} & \textbf{1 stress} & \textbf{2 stress} & \textbf{3 stress} & \textbf{4 stress} & \textbf{5 stress} & \textbf{6 stress} & \textbf{7+ stress} \\
    -4 &  97.6\% &   1.7\% &   0.5\% &   0.1\% &   - &   - &   - &   - \\
    -3 &  93.6\% &   4.1\% &   1.7\% &   0.5\% &   0.1\% &   - &   - &   - \\
    -2 &  85.9\% &   7.7\% &   4.1\% &   1.7\% &   0.5\% &   0.1\% &   - &   - \\
    -1 &  73.9\% &  11.9\% &   7.7\% &   4.1\% &   1.7\% &   0.5\% &   0.1\% &   - \\
     0 &  58.4\% &  15.5\% &  11.9\% &   7.7\% &   4.1\% &   1.7\% &   0.5\% &   0.1\% \\
    +1 &  41.6\% &  16.9\% &  15.5\% &  11.9\% &   7.7\% &   4.1\% &   1.7\% &   0.6\% \\
    +2 &  26.1\% &  15.5\% &  16.9\% &  15.5\% &  11.9\% &   7.7\% &   4.1\% &   2.3\% \\
    +3 &  14.1\% &  11.9\% &  15.5\% &  16.9\% &  15.5\% &  11.9\% &   7.7\% &   6.4\% \\
    +4 &   6.4\% &   7.7\% &  11.9\% &  15.5\% &  16.9\% &  15.5\% &  11.9\% &  14.1\% \\
    +5 &   2.4\% &   4.1\% &   7.7\% &  11.9\% &  15.5\% &  16.9\% &  15.5\% &  26.0\% \\
    +6 &   0.7\% &   1.7\% &   4.1\% &   7.7\% &  11.9\% &  15.5\% &  16.9\% &  41.5\% \\
\end{DndTable}

% skill_diff prob_damage
% <fct>            <dbl>
% 1 -4                 2.3
% 2 -3                 6.4
% 3 -2                14.1
% 4 -1                26  
% 5 0                 41.5
% 6 1                 58.4
% 7 2                 73.9
% 8 3                 85.8
% 9 4                 93.5
% 10 5                 97.6
% 11 6                 99.3

\begin{DndTable}[header=Expected Damage in One Round]{lrrrrrrrrrrr}
    \textbf{Attacker - Defender} & \textbf{-4} & \textbf{-3} & \textbf{-2} & \textbf{-1} & \textbf{0} & \textbf{+1} & \textbf{+2} & \textbf{+3} & \textbf{+4} & \textbf{+5} & \textbf{+6} \\
    \textbf{Expected Damage}     & 0.0365      & 0.108       & 0.260       & 0.530        & 0.950       & 1.53        & 2.26         & 3.07         & 3.92         & 4.74         & 5.46      \\
\end{DndTable}


\begin{DndTable}[header=Expected Rounds to Accumulate 7+ Damage]{lrrrrrrrrrrr}
    \textbf{Attacker - Defender} & \textbf{-4} & \textbf{-3} & \textbf{-2} & \textbf{-1} & \textbf{0} & \textbf{+1} & \textbf{+2} & \textbf{+3} & \textbf{+4} & \textbf{+5} & \textbf{+6} \\
    \textbf{Expected Rounds}            & 210.5      & 89.3      & 40.3      & 19.3      & 11.1     & 5.5      & 3.8      & 2.8      & 2.3      & 1.9      & 1.6      \\
\end{DndTable}


\begin{multicols}{2}
