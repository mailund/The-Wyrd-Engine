
\WyrdCapLine{R}{aces} are a staple of fantasy roleplaying games but also frequently used in science fiction settings such as space operas. From noble elves and hulking orcs to insectoid aliens and sentient machines, the inclusion of unique peoples and beings brings depth, diversity, and wonder to a game world. They shape culture, mythology, and conflict—and offer players a means of exploring identities that challenge or reflect their own.

In \wyrd, the terms \textbf{race} and \textbf{creature} are not rigid mechanical categories. Rather, they reflect a narrative function: \textbf{races} are player character origins—backgrounds that offer depth and thematic flavour—while \textbf{creatures} typically refer to non-player entities, ranging from allies and fauna to threats and mysteries. Mechanically, they follow the same rules: a small number of Traits that define what makes them unique, and a clear role in the fiction.

This flexible approach encourages creative worldbuilding while avoiding reductive or deterministic interpretations of culture or biology. Whether you're designing a lineage, a species, or a strange spirit with no defined form, this chapter provides tools to breathe life into them through narrative concept and evocative Traits.

Above all, the goal is to create beings—human or otherwise—that feel grounded in your world, fit the tone of your game, and invite rich storytelling.

\section{Design Philosophy}

Designing races and creatures begins with purpose and theme. What role does this being serve in your setting? Is it a cultural anchor, a living enigma, a mirror to humanity, or a monster lurking in the mist? In \wyrd, Traits do most of the mechanical lifting, allowing your creative focus to rest on ideas, imagery, and narrative significance. Traits should express \textbf{story}, not simulate biology.

In practice, designing a race or creature often means identifying a small set of Traits that capture its essence. While Traits normally highlight what makes a particular \emph{character} distinct from others, racial or creature Traits define what makes one \emph{type of being} distinct—while creating a thematic throughline among its members.

Avoid over-defining what a race or creature can or cannot do. Let individuals vary. Not every member of a species will share the same talents, values, or fate. A race or creature should suggest \textbf{possibilities}—not impose limitations. Think in terms of archetypes, myths, and cultural patterns rather than hard rules or fixed biology.

In this system, \textbf{races} represent origins for player characters. They are defined not by rigid stat blocks, but by narrative elements—culture, appearance, beliefs—and a small number of iconic Traits. A race might be shaped by the myths it tells, the lands it inhabits, or the legacy it bears.

\textbf{Creatures}, by contrast, are typically non-player entities: animals, spirits, constructs, ghosts, monsters, and everything in between. Mechanically, creatures are treated the same as NPCs in \wyrd—defined by their Traits, which convey their essence and narrative role. A creature might serve as an ally, a threat, or simply part of the world’s texture. Some are simple and bestial; others possess intelligence, goals, and depth.

Traditional RPGs often use stat blocks for monsters, and \wyrd’s approach to mook NPCs serves a similar purpose. However, creatures in this system can also be fully realised characters. The term \emph{creature}, in context, usually refers to a more generic or simplified NPC—often one intended for brief encounters or symbolic roles.

The boundary between races, creatures, and NPCs is intentionally soft. A playable character could originate from a so-called creature species, and a race could easily serve as the foundation for a major antagonist. What matters most is not the label, but the story you wish to tell.



\section{Creating a Race}

The process of creating a race in \wyrd begins with narrative intention. Before touching the mechanics, take a moment to consider what makes this being interesting, memorable, and meaningful within your setting. Is it a remnant of a fallen empire? A species adapted to live in a world of perpetual twilight? A being birthed from the dreams of dying gods? The more clearly you define its purpose in the story, the more confidently you can give it shape at the table.

This narrative-first approach emphasises theme, role, and flavour. Rather than constructing beings from a checklist of biological traits, you focus on what the race or creature represents, how it interacts with the world, and what kind of stories it invites. Your design might evoke mystery, tragedy, nobility, horror, whimsy—or some blend of many tones.

Once the concept is clear, you can translate it into mechanics using the core building blocks of the system: Traits. Each Trait should express a core truth about the being—not just how it functions, but how it feels. Mechanical power is secondary to narrative clarity and flavour. You can always adjust balance later, but evocative design will inspire deeper stories.

The following steps offer a reliable structure for designing races and creatures, suitable for both player character origins and non-player encounters:

\begin{itemize}
  \item \textbf{Name and Concept:} Start with a vivid phrase that captures the essence of the being. This could describe its nature, origin, or purpose. For example: ``Twilight-born nomads of the salt dunes,'' ``Bone-dancers from the grave-cities,'' or ``Winged sentinels shaped by the storm.'' Keep it short but rich in imagery.

  \item \textbf{Core Traits:} Select 1–3 Traits that define what makes the race or creature unique. Traits should be evocative, flexible, and suggest multiple uses. For example, \emph{Sings to Forgotten Stones} might imply a magical affinity for ancient places, communication with spirits, or unlocking sealed doors. Traits should do narrative work while offering mechanical hooks.

  \item \textbf{Optional Abilities or Rules:} If the being has a special feature not easily captured by a Trait, you may add a simple rule or suggested Skill bonus. These should be kept minimal—only use them when they clarify something essential. For example, ``Characters from this race gain +1 to \textbf{Lore} when navigating ruins'' is fine, but avoid creating full subsystems. Traits should remain the primary expression of the being.

  \item \textbf{Narrative Hooks:} Consider how this race or creature fits into your setting and your stories. Are they widely known or hidden from the world? What relationships do they have with other groups or places? Provide a few examples of how the being might appear in play—as a guide, an antagonist, a mysterious traveller, or even a rumour. These hooks ground your design in actual gameplay.
\end{itemize}

When designing races, it is standard practice to use the default stress values from the core rules, ensuring that all player characters—regardless of race—can withstand a similar amount of damage. This maintains mechanical balance and simplifies character creation. However, if a particular race has a strong narrative justification for being more fragile or more durable (such as ethereal spirits or stone-skinned giants), you may choose to adjust their stress boxes accordingly.

In the examples at the end of this chapter, we have not provided stress values for races—unlike creatures—because we assume that all player characters begin with the same baseline stress unless the group agrees to a specific exception.

Races designed for player use may also benefit from cultural flavour: shared beliefs, naming conventions, rituals, or outlooks that enrich roleplaying. Creatures, especially those used as threats or wonders, should have strong visual or behavioural cues—things that players can latch onto quickly and remember later.

Ultimately, the best race and creature designs blend strong narrative identity with concise, expressive mechanics. Let story lead, and the rest will follow.


\section{Creating a Creature}

Creatures in \wyrd are typically non-player beings that populate the world, provide challenges, or add colour to a scene. Unlike races, which serve as origins for player characters, creatures are often designed for limited appearances—whether as mysterious threats, symbolic figures, or flavourful encounters. As such, they are usually simpler in mechanical scope, but just as important in terms of narrative presence.

Creatures that you only intent to use once or twice are better modelled by making a non-player character. The mechanics are the same, after all. But if you wish to populate your world with strange beings your players will run into repeatedly, then you should consider creating a creature. Think of creatures as templates for non-player beings the same way as races are templates for player characters.

A creature might be a lurking horror in an abandoned chapel, a loyal beast bonded to a lost civilisation, or a malfunctioning automaton repeating an ancient command. These beings shape mood and tone, push characters into action, or represent larger themes in the story. A creature doesn’t need a complete backstory—but it should always spark curiosity, tension, or wonder.

Mechanically, creatures are usually built with 1–3 Traits. These Traits represent the creature’s instincts, powers, or strange behaviours, rather than skills or complex rules. The Traits should offer cues for how the creature acts, what it wants, and how players might overcome, avoid, or understand it. When used in combat, most creatures function as mooks—defeated with a single solid hit—but they can still leave a strong impression.

The following process can help you create compelling creatures quickly and effectively:

\begin{itemize}
  \item \textbf{Name and Role:} Describe what the creature is and why it exists in the scene. This could be literal (``Ash-Wolf of the Charred Hills'') or thematic (``Warden of Broken Promises''). Its name alone should raise questions or evoke imagery.

  \item \textbf{Traits:} Choose 1–3 Traits that define the creature’s nature and behaviour. These Traits should not just describe its abilities but imply how it moves, thinks, or feels. For example, \emph{Hungers for Forgotten Names} implies more than just appetite—it suggests a metaphysical hunger and a means of interacting with the world.

  \item \textbf{Motive and Instinct:} What drives the creature? Is it territorial, curious, bound by duty, or simply lashing out in pain? Most creatures act on instinct rather than dialogue. Defining their behaviour pattern helps the GM improvise in play.

  \item \textbf{Weaknesses and Vulnerabilities:} Optional, but useful. These could be narrative vulnerabilities (``flees from firelight'') or mechanical (``can only be harmed by silver or sound''). These limitations make encounters more dynamic and encourage creative problem-solving.

  \item \textbf{Encounter Use:} How is this creature likely to appear in play? As a sudden ambush, a guardian of a location, or a strange companion that follows the group? Is it a one-time horror, or part of a recurring motif? This anchors the creature in the story’s structure.
\end{itemize}

Creatures are most effective when they leave an impression—through an eerie sound, a haunting silhouette, or an incomprehensible gesture. They don’t need elaborate stats to be memorable. If you can describe them in a single sentence that sticks in a player’s mind, you’ve done your job well.

Finally, remember that not all creatures need to be threats. Some may evoke pity, wonder, or even amusement. Let your setting’s tone guide how weird, whimsical, or terrifying your creations should be.


\section{Increasing the Trait Budget}

In the core rules of \wyrd, each character begins with 3 Traits. This default is ideal for one-shot adventures, quick character creation, and situations where all player characters share a similar background or origin. With only a few Traits, players can focus on strong, flavourful choices without being overwhelmed, and GMs can quickly assess the unique abilities present at the table.

However, in settings where multiple distinct races exist—and especially when those races are mechanically represented through pre-defined Traits—this default Trait budget may feel limiting. If a race comes with 1–3 fixed Traits, players have fewer opportunities to customise their characters and express individuality through their Trait selections. In effect, character uniqueness is sacrificed to emphasise race uniqueness.

To balance this trade-off, you can increase the Trait budget to 4 or even 5 Traits per character. This allows players to retain meaningful personal expression, even when some Traits are pre-assigned by race. For example, if a character’s race provides 2 Traits by default, the player might still choose 2 or 3 Traits freely, depending on the total Trait budget used. This maintains both narrative flavour and mechanical flexibility.

Importantly, this expanded budget works best in campaigns where players are already familiar with the setting. When players understand the role and themes of each race, pre-defined Traits become useful shorthand rather than a source of cognitive overload. Picking up a new character—especially in drop-in/drop-out play—becomes easier when some of their Traits are baked into the race concept.

\subsection*{Generalist Races}

Some races, such as humans in many settings, are designed as \textbf{generalists}. Rather than providing fixed Traits, they offer flexibility—often allowing players to choose all their Traits freely or select from a broader list. This design reinforces the narrative idea that such races are adaptable, diverse, or unbound by tradition. It also gives players more room to create distinctive concepts within the same racial group.

In contrast, \textbf{specialist} races may define 2–3 Traits from the outset, leaving little or no room for individual choice. These races typically embody a strong thematic identity or cultural cohesion. For instance, a race of obsidian-skinned subterranean sages might always include \emph{Eyes That Pierce the Dark} and \emph{Stonebound Memory} as part of their mechanical definition.

Both approaches are valid, and the choice depends on your setting's themes. Generalist races provide freedom and narrative flexibility; specialist races reinforce a shared cultural or mythic image. Increasing the Trait budget allows you to support both approaches without compromising player creativity.

Ultimately, the goal is to ensure that race enhances a character's story, rather than restricting it. Adjust the Trait budget to reflect the needs of your group, your setting, and the tone of the game you want to create.



\section{Example Races}

The following examples illustrate how to create races using the narrative-first approach outlined earlier. Each one includes a core concept, signature Traits, and narrative hooks that can be adapted to your setting. These races may be used as-is, modified to suit your world, or simply serve as inspiration for designing your own.

\subsection*{Elves – The Twilight-Born}\index{Elves}

Ancient, elegant, and bound to fading realms, Elves are the children of twilight and memory. They dwell in the borderlands of the world—deep forests, forgotten ruins, and moonlit vales where the veil between past and present grows thin. Once proud stewards of great empires, many now live as wanderers, artists, or quiet observers, their longevity burdened with loss.

Elves are not simply long-lived humans with pointy ears. They experience time differently, speak in metaphors, and carry fragments of forgotten songs in their blood. Their connection to the natural and arcane makes them seem strange or aloof to other races, but to an Elf, the world is a layered tapestry of echoes, omens, and beauty just out of reach.

\subsubsection{Name and Concept}

Twilight-born stewards of the old world, caught between remembrance and renewal.

\subsubsection*{Core Traits}
\begin{itemize}
  \item \textbf{Grace of the Twilight Realms} — You gain a +2 bonus to \textbf{Stealth} when moving through dim light, forests, ruins, or other liminal spaces.
  \item \textbf{Whispers of the Forgotten} — You may use \textbf{Empathy} instead of \textbf{Lore} when sensing the emotional imprint left behind in old places or ancient artifacts.
\end{itemize}

\subsubsection{Optional Rules:} Elves may add +1 to \textbf{Lore} when dealing with ancient places, lost civilizations, or matters of fate and prophecy.

\subsubsection{Narrative Hooks:}
\begin{itemize}
  \item A lone Elf musician plays a song only the forest remembers—and something stirs in response.
  \item An ancient Elf scholar offers to help the players unlock a sealed tomb but demands a memory in return.
  \item An Elf knight, exiled for breaking tradition, seeks redemption among mortals—but cannot shake the past.
\end{itemize}



\subsection*{Dwarves – The Stone-Kin}\index{Dwarves}

Stalwart and enduring, Dwarves are the children of stone and fire, shaped by the pressure of deep places and the weight of tradition. They build cities beneath mountains, carve stories into granite, and measure time not in seasons, but in generations. Their craftsmanship is legendary, and their sense of duty runs as deep as the mines they dig.

Dwarves are not just short, bearded miners. They are living echoes of the earth—resilient, methodical, and bound by oaths older than empires. To a Dwarf, identity is forged, not born. Every tool, every wall, every word carries weight. Outsiders may see them as gruff or stubborn, but beneath their stoicism lies a fierce loyalty to kin, craft, and clan.

\subsubsection{Name and Concept}

Stone-forged traditionalists who shape the world as much as they are shaped by it.

\subsubsection*{Core Traits}
\begin{itemize}
  \item \textbf{Stoneblood Resilience} — You gain a +2 bonus to \textbf{Physique} when resisting fatigue, poison, or environmental hazards such as cold, heat, or suffocating air.

  \item \textbf{Built to Endure} — You may use \textbf{Craft} instead of \textbf{Athletics} when overcoming physical challenges involving endurance, weight, or structure (such as lifting gates, reinforcing walls, or enduring collapses).
\end{itemize}

\subsubsection{Optional Rules:} Dwarves gain +1 to \textbf{Craft} when working with stone, metal, or underground structures.

\subsubsection{Narrative Hooks:}
\begin{itemize}
  \item A Dwarf mason seeks to finish a stone bridge his ancestors abandoned generations ago—despite warnings that the mountain is cursed.
  \item A Dwarf exile wanders the surface world, carrying a hammer with a broken handle and a shame no one will speak of.
  \item A Dwarf ambassador offers aid in return for a favour: recovering a sacred relic buried beneath enemy territory.
\end{itemize}

\subsection*{Virelians – The Shard-Minds}\index{Virelians}

The Virelians are crystalline beings born from the shifting mineral storms of the rogue planet Virex. They possess no flesh or blood, but instead grow luminous lattice-structures over time, each node of their body acting as both storage and processor. Thought, memory, and even personality can be transferred between cores, allowing Virelians to transcend death—though not always without cost.

To outsiders, they seem cold and inhuman, speaking in harmonic vibrations and measuring morality in patterns of efficiency. Yet among themselves, the Virelians are contemplative, communal, and devoted to what they call the “Chorus of Data”—a collective memory that binds their culture across time and space.

\subsubsection{Name and Concept}

Crystalline minds evolved for data preservation and long-form logic, shaped by deep time and cold stars.

\subsubsection*{Core Traits}
\begin{itemize}
  \item \textbf{Cognitive Node Matrix} — You may use \textbf{Craft} instead of \textbf{Lore} when analysing systems, deciphering alien tech, or interacting with information-based environments.

  \item \textbf{Resonant Frame} — You gain a +2 bonus to \textbf{Will} when resisting mental intrusion, emotional manipulation, or attempts to deceive you through social means.
\end{itemize}

\subsubsection{Optional Rules:} Virelians do not require air or food, but are vulnerable to sonic disruption effects (at GM discretion).

\subsubsection{Narrative Hooks:}
\begin{itemize}
  \item A damaged Virelian arrives on a frontier station, its memory lattice corrupted—and its mission unclear.
  \item A Virelian archivist offers access to forgotten data in exchange for help recovering a missing shard of itself.
  \item A rogue Virelian begins replicating itself illegally, fracturing its identity—and the Chorus—in the process.
\end{itemize}


\subsection*{Zerathi – The Starborn Aristocracy}\index{Zerathi}

The Zerathi are a proud and radiant species who claim descent from the first sentient beings to walk among the stars. Towering, luminous, and draped in gravitic robes and radiant metals, they consider themselves the rightful stewards of galactic destiny. Their homeworld, Zerath Prime, is a crystalline megastructure orbiting a binary sun—and serves as both temple and throne.

Zerathi culture blends mysticism and monarchy, science and prophecy. They speak in grand proclamations, wage philosophical duels, and often look down on "less awakened" species with a mix of pity and disdain. Yet some among them have broken from tradition—wandering the stars as exiles, prophets, or agents of change.

\subsubsection{Name and Concept}

Luminous, telepathic philosopher-nobles who see themselves as the rightful rulers of the cosmos.

\subsubsection*{Core Traits}
\begin{itemize}
  \item \textbf{Voice of the Cosmic Mandate} — Gain a +2 bonus to \textbf{Provoke} when delivering commands, ultimatums, or declarations of ideological superiority.

  \item \textbf{Born of Solar Light} — You may use \textbf{Will} instead of \textbf{Physique} to resist heat, radiation, or exhaustion in high-energy environments.
\end{itemize}

\subsubsection{Optional Rules:} Zerathi may add +1 to \textbf{Lore} when dealing with ancient galactic histories, cosmic mysteries, or alien philosophies.

\subsubsection{Narrative Hooks:}
\begin{itemize}\raggedright
  \item A Zerathi emissary demands the players surrender an artefact deemed too dangerous for "lesser minds."
  \item An exiled Zerathi prince seeks allies in a desperate bid to halt a corrupt royal prophecy.
  \item A Zerathi relic awakens in the hands of a player—suggesting forgotten ancestry or cosmic favour.
\end{itemize}




\section{Example Creatures}

The following creatures showcase how to design non-player entities using Traits and stress values to create distinct behaviours and challenges. Unlike races, creatures are often designed for specific narrative roles—threats to be overcome, enigmas to be unravelled, or denizens that deepen the world’s texture. Some are simple and expendable; others might reappear across multiple encounters.

Each example includes a core concept, defining Traits (both strengths and flaws), stress and consequence boxes, and guidance on how to use the creature in play. These entries are intended to be modular—feel free to adapt them to your own setting or use them as templates for building new beings.

\subsection*{Goblins–Scavengers of the Cracks}\index{Goblins}

Goblins lurk on the fringes of society and civilization—dwelling in sewer tunnels, forgotten stations, shattered ruins, and hollowed-out machines. Small, quick, and numerous, they thrive in chaos and clutter, cobbling together survival from the debris of greater powers. They are clever in a crude sort of way, but easily startled, poorly organized, and prone to infighting.

Despite their reputation as pests, goblins are not inherently evil—just desperate, opportunistic, and shaped by generations of scarcity. Left alone, they squabble and scavenge. Pushed, they bite.

\subsubsection*{Core Traits}
\begin{itemize}
  \item \textbf{Scrap-Smart} — Gain a +2 bonus to \textbf{Craft} when improvising tools, traps, or repairs from junk or discarded materials.

  \item \textbf{Cowardly Instinct} — Suffer a –2 penalty to \textbf{Will} when resisting fear, intimidation, or overwhelming odds.
\end{itemize}

\subsubsection*{Stress and Consequences}
\begin{itemize}
  \item \textbf{Fatigue:} \FatigueBox\FatigueBox
  \item \textbf{Wounds:} Mild \MildWound\MildWound
\end{itemize}

\subsubsection*{Use in Play:}
\begin{itemize}
  \item Goblins rarely attack alone. They appear in swarms, flee when outmatched, and often return in greater numbers.
  \item One or two may be clever enough to speak, trade, or even beg—especially if cornered.
  \item Goblin traps are crude but dangerous, especially in cluttered environments.
\end{itemize}


\subsection*{Trolls – Hulks Beneath the Hills}\index{Trolls}

Trolls are massive, stubborn creatures of stone and sinew, found in caverns, under bridges, and in the deep wilderness. Their leathery skin is thick as bark, their bones hard as granite, and their patience measured in centuries. Though slow to act, once roused to anger, a troll is nearly unstoppable—swatting aside attackers, shrugging off wounds, and bellowing with raw fury.

Trolls are not mindless brutes. They are ancient, solitary beings with long memories and strange customs. Some speak in riddles, others hoard treasures or bones, and a few are even capable of bitter humour. They care little for the affairs of mortals, unless disturbed.

\subsubsection*{Core Traits}
\begin{itemize}
  \item \textbf{Unyielding Mass} — Gain a +2 bonus to \textbf{Physique} when resisting force, pain, or attempts to move or restrain you.

  \item \textbf{Dull but Determined} — You may use \textbf{Force} instead of \textbf{Notice} to react to sudden movement, loud noise, or perceived threats.
\end{itemize}

\subsubsection*{Stress and Consequences}
\begin{itemize}
  \item \textbf{Fatigue:} \FatigueBox \FatigueBox \FatigueBox \FatigueBox \FatigueBox
  \item \textbf{Wounds:} 
  Mild \MildWound \MildWound \MildWound \MildWound \\
  Moderate \ModerateWound \ModerateWound \ModerateWound \\
  Severe \SevereWound \SevereWound
\end{itemize}

\subsubsection*{Use in Play:}
\begin{itemize}
  \item Trolls can endure prolonged combat and should be treated as serious threats unless the players are well-prepared.
  \item They often demand tribute rather than initiate violence—until provoked.
  \item A troll may guard something of value without even realising it, such as a hidden passage or a long-lost artefact.
\end{itemize}

\subsection*{Zombies – The Restless Dead}\index{Zombies}

Zombies are corpses reanimated by foul magic, cursed infection, or unknown energies. Driven by base instinct and hunger, they shamble through the world in search of the living. While slow and dim-witted, they are tireless and unfeeling—immune to pain, fear, and persuasion. A single zombie is a nuisance; a horde is a nightmare.

Though they lack higher thought, zombies may retain echoes of their former selves—a limp, a half-remembered gesture, a name moaned in the dark. In some cases, their origin may offer clues to how they can be stopped—or what unfinished business binds them to the world.

\subsubsection*{Core Traits}
\begin{itemize}
  \item \textbf{Unliving Endurance} — Gain a +2 bonus to \textbf{Physique} when resisting physical harm, poisons, or conditions that would exhaust or disable a living creature.

  \item \textbf{Rotting Reflexes} — Suffer a –2 penalty to \textbf{Notice} when reacting to fast movement, sudden changes, or anything requiring alert awareness.
\end{itemize}

\subsubsection*{Stress and Consequences}
\begin{itemize}
  \item \textbf{Fatigue:} \FatigueBox \FatigueBox
  \item \textbf{Wounds:} Mild \MildWound \quad Moderate \ModerateWound
\end{itemize}

\subsubsection*{Use in Play:}
\begin{itemize}
  \item Zombies are ideal mooks: they can be defeated easily one-on-one but become overwhelming in groups.
  \item They often ignore injuries that would drop another creature—crawling with shattered limbs or continuing to fight while impaled.
  \item Consider describing their lingering humanity: a wedding ring on a bony hand, a tattered uniform, a child's toy in their pack.
\end{itemize}

\subsection*{Thresher Mites – Hive Shadows of the Void}\index{Thresher Mites}

Thresher Mites are semi-intelligent arthropod swarms found drifting in the debris fields of derelict ships and asteroid belts. Roughly the size of a fist, each mite is individually harmless—but in swarms, they consume hull plating, insulation foam, and even organic tissue with frightening efficiency. Their clicking communication creates eerie rhythms, and some xeno-biologists believe they are the scouts of a larger, hidden intelligence.

While they typically prey on derelicts, swarms have been known to infest spaceports, sabotage cryopods, or hijack communication relays. No known central mind has been identified—though patterns in their movements suggest eerie coordination. Worse, they sometimes leave survivors... changed.

\subsubsection*{Core Traits}
\begin{itemize}
  \item \textbf{Swarm Logic} — Gain a +2 bonus to \textbf{Stealth} when attacking or infiltrating through vents, crawlspaces, or blind spots created by disarray or structural damage.

  \item \textbf{Fragile Exoskeletons} — Suffer a –2 penalty to \textbf{Physique} when resisting blunt force, fire, or wide-area effects.
\end{itemize}

\subsubsection*{Stress and Consequences}
\begin{itemize}
  \item \textbf{Fatigue:} \FatigueBox \FatigueBox \FatigueBox
  \item \textbf{Wounds:} Mild \MildWound \quad Moderate \ModerateWound
\end{itemize}

\subsubsection*{Use in Play:}
\begin{itemize}
  \item Mites infiltrate, disable, and overwhelm—not with strength, but with numbers and cunning. They are especially dangerous in tight quarters.
  \item A swarm may serve as an environmental hazard, escalating over time as players delay or disturb their habitat.
  \item Players may uncover strange patterns in their behaviour, hinting at a hive intelligence or encoded message.
\end{itemize}

\subsection*{Nymari – Lurkers Between Frequencies}\index{Nymari}

The Nymari are shimmering entities glimpsed only in moments of electrical distortion or when signals fail. Believed to originate from a higher-dimensional stratum of reality, they phase in and out of existence, drawn to patterns of thought and electromagnetic noise. Their forms flicker like broken holograms, and their voices sound like garbled static across failing radios.

Though they do not appear to possess conventional intelligence, they exhibit complex, predatory behaviour—stalking sentient lifeforms across abandoned space stations, derelict vessels, and communications arrays. Some believe they feed on cognition itself. Others suggest they are lost travellers, seeking to understand our dimension by disassembling the minds they find.

\subsubsection*{Core Traits}
\begin{itemize}
  \item \textbf{Out-of-Phase Anatomy} — You gain a +2 bonus to \textbf{Stealth} when surrounded by electronic interference, machinery, or active communication signals.

  \item \textbf{Cognitive Distortion Field} — You may use \textbf{Will} instead of \textbf{Provoke} when attempting to disorient or frighten sentient beings with your unnatural presence.
\end{itemize}

\subsubsection*{Stress and Consequences}
\begin{itemize}
  \item \textbf{Fatigue:} \FatigueBox \FatigueBox \FatigueBox
  \item \textbf{Wounds:} 
  Mild \MildWound \MildWound \\
  Moderate \ModerateWound \\
  Severe \SevereWound
\end{itemize}

\subsubsection*{Use in Play:}
\begin{itemize}\raggedright
  \item Nymari can pass through walls, vanish into flickering monitors, or ride radio signals across long distances.
  \item They do not communicate, but may mimic voices from intercepted transmissions or the minds they've touched.
  \item Damaging one requires disrupting its resonance—such as through feedback loops, magnetic fields, or sensory overload.
\end{itemize}


\subsection*{Dragon – Scourge of Sky and Flame}\index{Dragon}

Dragons are the terror of legends made flesh—vast, intelligent, and ancient beyond reckoning. Their wings blot out the sun, their voices shake the stones, and their breath turns cities to cinders. Each dragon is unique, shaped by the hoard it guards, the land it claims, and the centuries it has endured. Some are tyrants, others slumbering gods of ruin—but all are forces of nature, not mere beasts.

A dragon is not simply a monster to be slain. It is a test of will, wit, and courage—a creature that demands respect, and rarely shows mercy.

\subsubsection*{Core Traits}
\begin{itemize}
  \item \textbf{Breath of the Inferno} — Gain a +5 bonus when using \textbf{Force} to unleash fire breath in combat, incinerate obstacles, or suppress entire groups.

  \item \textbf{Wings Like Tempests} — Gain a +2 bonus to \textbf{Athletics} when flying, diving, or using sudden gusts of wingbeat to disorient or displace enemies.

  \item \textbf{Primordial Intellect} — Gain a +2 bonus to \textbf{Lore} when dealing with ancient magic, interpreting forgotten languages, or recognising long-lost symbols.

  \item \textbf{Scales Like Forged Iron} — You may use \textbf{Physique} instead of \textbf{Will} to resist mental attacks, intimidation, or fear-based effects—your physical dominance is its own defence.
\end{itemize}

\subsubsection*{Special Rules}
\begin{itemize}
  \item \textbf{Legendary Presence:} Dragons may make social attacks with \textbf{Provoke} against all enemies in the scene as a single action once per conflict.
  \item \textbf{Unnatural Toughness:} Reduce all incoming damage by 1 (to a minimum of 1) unless the attack exploits a known vulnerability (such as enchanted weapons or a dragon’s specific bane).
\end{itemize}

\subsubsection*{Stress and Consequences}
\begin{itemize}
  \item \textbf{Fatigue:} 
  \FatigueBox\FatigueBox\FatigueBox\FatigueBox\FatigueBox
  \FatigueBox\FatigueBox\FatigueBox\FatigueBox\FatigueBox 
  \item \textbf{Wounds:} 
  Mild \MildWound\MildWound\MildWound\MildWound
       \MildWound\MildWound\MildWound\MildWound \\
  Moderate \ModerateWound\ModerateWound\ModerateWound\ModerateWound \\
  Severe \SevereWound\SevereWound\SevereWound\SevereWound 
\end{itemize}

\subsubsection*{Use in Play:}
\begin{itemize}
  \item A dragon should be a defining encounter. It should not be treated as a mook or mere obstacle—use it sparingly and dramatically.
  \item Dragons are cunning. They may parley, deceive, threaten, or manipulate as easily as they destroy.
  \item Their lair should be an extension of their personality—cursed hoards, volcanic chambers, enchanted stormclouds—providing both environmental hazards and narrative clues.
\end{itemize}