\section{Social \& Influence}

This category covers a character’s ability to interact with others—whether through charm, deceit, empathy, intimidation, or diplomacy. These skills govern how characters navigate conversations, influence opinions, earn trust, or manipulate perception. Social \& Influence skills are essential in scenarios involving negotiation, interrogation, persuasion, or political maneuvering, and can often be just as powerful as any weapon.

Whether brokering peace between factions, bluffing your way past a guard, or reading a suspect’s reactions, these abilities shape how your character leaves an impression on the world—and how the world responds in kind.

\subsection{Command}\index{Command!Skill}\index{Skills!Command}\label{skill:command}
\textbf{Settings:} Military, Political, Urban, Steampunk\\
\textbf{Scope:} Narrow\\
\textbf{Other Names:} \emph{Leadership}, \emph{Authority}, \emph{Orders}\\
\vspace{\baselineskip}

\emph{Command} is the skill of giving clear, forceful instructions in high-pressure situations. It is used to coordinate groups, issue orders, rally morale, or assert authority over subordinates. Unlike \emph{Provoke}, which compels through intimidation, \emph{Command} motivates through structure and responsibility.

Characters with high \emph{Command} are often officers, captains, tacticians, or experienced leaders who know how to take charge when others falter. This skill is particularly important when dealing with trained personnel, military units, or large operations.

\vspace{0.5\baselineskip}
\noindent\textbf{Related Skills:} \emph{Rapport} (p. \pageref{skill:rapport}), \emph{Provoke} (p. \pageref{skill:provoke})


\subsection{Contacts}\index{Contacts!Skill}\index{Skills!Contacts}\label{skill:contacts}
\textbf{Settings:} All\\
\textbf{Scope:} Broad\\
\textbf{Other Names:} \emph{Networking}, \emph{Connections}, \emph{Social Circles}\\
\vspace{\baselineskip}

\emph{Contacts} represents a character’s personal and professional network—the people they know and can call upon. It’s used to find information, arrange favours, acquire illicit goods, or gain access to restricted places. A high \emph{Contacts} rating means you know someone who knows someone, whether in high society or the criminal underworld.

This skill doesn’t measure charm or social finesse (see \emph{Rapport}), but rather the reach and reliability of your network. It’s ideal for fixers, journalists, black marketeers, or anyone who knows how to grease the right palms.

\vspace{0.5\baselineskip}
\noindent\textbf{Related Skills:} \emph{Resources} (p. \pageref{skill:resources}), \emph{Rapport} (p. \pageref{skill:rapport})


\subsection{Deceive}\index{Deceive!Skill}\index{Skills!Deceive}\label{skill:deceive}
\textbf{Settings:} All\\
\noindent\textbf{Scope:} Broad\\
\textbf{Other Names:} \emph{Lying}, \emph{Bluff}, \emph{Falsehood}\\
\vspace{\baselineskip}

\emph{Deceive} is the skill of lying convincingly, creating false impressions, and covering up the truth. It includes forging identities, feigning innocence, hiding one’s intentions, and creating convincing distractions. Unlike \emph{Rapport}, which builds trust honestly, \emph{Deceive} manipulates it.

This skill can be used for impersonation, slipping misinformation into conversation, or faking emotions to gain sympathy. Characters with high \emph{Deceive} are con artists, spies, illusionists, and anyone who lives behind a mask.

\vspace{0.5\baselineskip}
\noindent\textbf{Base of:} \emph{Disguise} (p. \pageref{skill:disguise}), \emph{Forgery} (p. \pageref{skill:forgery})



\subsection{Diplomacy}\index{Diplomacy!Skill}\index{Skills!Diplomacy}\label{skill:diplomacy}
\textbf{Settings:} Political, Urban, Fantasy\\
\textbf{Scope:} Narrow\\
\textbf{Other Names:} \emph{Statecraft}, \emph{Protocol}, \emph{Formal Negotiation}\\
\vspace{\baselineskip}

\emph{Diplomacy} is a specialised form of \emph{Rapport} used in formal, political, or cross-cultural negotiations. It covers knowledge of protocol, tactful phrasing, and navigating delicate situations without giving offense. It is essential for emissaries, courtiers, ambassadors, and anyone who represents a group in high-stakes discussion.

This skill differs from general persuasion by focusing on structure, tone, and tradition—whether mediating between warring nations or brokering a truce with the fae.

\vspace{0.5\baselineskip}
\noindent\textbf{Requires:} \emph{Rapport} (p. \pageref{skill:rapport})


\subsection{Empathy}\index{Empathy!Skill}\index{Skills!Empathy}\label{skill:empathy}
\textbf{Settings:} All\\
\textbf{Scope:} Broad\\
\textbf{Other Names:} \emph{Insight}, \emph{Emotional Intelligence}, \emph{Intuition}\\
\vspace{\baselineskip}

\emph{Empathy} is the ability to understand what others are feeling, even when they try to hide it. It allows characters to read body language, interpret tone, and sense underlying emotions—whether used for comfort, leverage, or subtle observation. It’s essential for detecting lies, easing tensions, or recognising when someone is in distress.

This skill is often paired with social interaction but also plays a key role in interrogation, negotiation, or even detecting enchantments that manipulate emotions. Characters with high \emph{Empathy} may be counsellors, diplomats, interrogators—or simply very hard to fool.

\vspace{0.5\baselineskip}
\noindent\textbf{Base of:} \emph{Insight} (p. \pageref{skill:insight}) if used as a narrower skill.

\subsection{Rapport}\index{Rapport!Skill}\index{Skills!Rapport}\label{skill:rapport}
\textbf{Settings:} All\\
\textbf{Scope:} Broad\\
\textbf{Other Names:} \emph{Charm}, \emph{Persuasion}, \emph{Diplomacy}\\
\vspace{\baselineskip}

\emph{Rapport} is the skill of creating trust and goodwill—whether through charm, honesty, or likability. It’s used to build friendships, calm tensions, negotiate deals, or convince others through open dialogue. Unlike \emph{Deceive}, Rapport relies on sincerity, even if it’s strategic.

Characters with high \emph{Rapport} are natural diplomats, peacemakers, and public faces. In some settings, this skill can turn tense interrogations into cooperative discussions, or open doors that would otherwise remain closed.

\vspace{0.5\baselineskip}
\noindent\textbf{Base of:} \emph{Negotiation} (p. \pageref{skill:negotiation}), \emph{Diplomacy} (p. \pageref{skill:diplomacy})



\subsection{Etiquette}\index{Etiquette!Skill}\index{Skills!Etiquette}\label{skill:etiquette}
\textbf{Settings:} Urban, Political, Historical, Fantasy\\
\textbf{Scope:} Narrow\\
\textbf{Other Names:} \emph{Protocol}, \emph{Courtesy}, \emph{Manners}\\
\vspace{\baselineskip}

\emph{Etiquette} is the skill of knowing and following social customs, traditions, and formal behaviour. It allows a character to avoid offense, demonstrate respect, and move gracefully through different social environments—be it a royal court, a merchant guild, or a sacred temple.

This skill is especially important when interacting with nobility, clergy, or foreign cultures where missteps can have diplomatic consequences. It differs from \emph{Rapport} in that it reflects knowledge of decorum and ritual, not just likeability.

\vspace{0.5\baselineskip}
\noindent\textbf{Related Skills:} \emph{Diplomacy} (p. \pageref{skill:diplomacy}), \emph{Rapport} (p. \pageref{skill:rapport})



\subsection{Insight}\index{Insight!Skill}\index{Skills!Insight}\label{skill:insight}
\textbf{Settings:} All\\
\textbf{Scope:} Narrow\\
\textbf{Other Names:} \emph{Intuition}, \emph{Emotional Perception}, \emph{Read People}\\
\vspace{\baselineskip}

\emph{Insight} is a focused application of \emph{Empathy}, used to interpret others' emotions and motivations in real time. It allows characters to detect lies, recognise unspoken tension, or tell when someone is hiding something. While \emph{Empathy} has broader social utility, \emph{Insight} hones in on the internal states of others—especially in tense or deceptive situations.

Characters with high \emph{Insight} are excellent judges of character. They may not always know the full story, but they can sense when something is off.

\vspace{0.5\baselineskip}
\noindent\textbf{Requires:} \emph{Empathy} (p. \pageref{skill:empathy})



\subsection{Intimidate}\index{Intimidate!Skill}\index{Skills!Intimidate}\label{skill:intimidate}
\textbf{Settings:} All\\
\textbf{Scope:} Broad\\
\textbf{Other Names:} \emph{Threaten}, \emph{Menace}, \emph{Presence}\\
\vspace{\baselineskip}

\emph{Intimidate} is the skill of applying pressure through fear, aggression, or imposing presence. It can be used to coerce cooperation, force confessions, silence opposition, or establish dominance in tense interactions. This may take the form of veiled threats, open hostility, or a cold, unblinking stare.

Unlike \emph{Provoke}, which pushes for an emotional outburst, \emph{Intimidate} suppresses reaction through dread. It’s ideal for interrogators, enforcers, or anyone who wants to end a conversation before it begins.

\vspace{0.5\baselineskip}
\noindent\textbf{Related Skills:} \emph{Provoke} (p. \pageref{skill:provoke}), \emph{Command} (p. \pageref{skill:command})



\subsection{Negotiation}\index{Negotiation!Skill}\index{Skills!Negotiation}\label{skill:negotiation}
\textbf{Settings:} All\\
\textbf{Scope:} Narrow\\
\textbf{Other Names:} \emph{Bargaining}, \emph{Deal-Making}, \emph{Haggling}\\
\vspace{\baselineskip}

\emph{Negotiation} is a focused form of \emph{Rapport} used when two or more parties seek a mutually beneficial agreement. It includes reading the other side’s priorities, making persuasive offers, and applying pressure without breaking trust. This skill applies to business deals, hostage talks, treaty proposals, or even informal trades.

Characters with high \emph{Negotiation} understand timing, leverage, and value. They know when to press an advantage, when to offer a concession, and how to close a deal in their favour.

\vspace{0.5\baselineskip}
\noindent\textbf{Requires:} \emph{Rapport} (p. \pageref{skill:rapport})

\subsection{Politics}\index{Politics!Skill}\index{Skills!Politics}\label{skill:politics}
\textbf{Settings:} Political, Urban, Historical, Steampunk\\
\textbf{Scope:} Narrow\\
\textbf{Other Names:} \emph{Statecraft}, \emph{Intrigue}, \emph{Court Lore}\\
\vspace{\baselineskip}

\emph{Politics} is the skill of understanding and influencing formal power structures—governments, councils, guilds, noble houses, and factions. It covers knowledge of titles, legal systems, court etiquette, political history, and the ever-shifting dynamics of influence.

While \emph{Rapport} or \emph{Deceive} may help in a conversation, \emph{Politics} helps you grasp the consequences of that conversation in a broader context. Characters with this skill can anticipate rival manoeuvres, interpret laws to their advantage, and manipulate institutions from within.

\vspace{0.5\baselineskip}
\noindent\textbf{Related Skills:} \emph{Diplomacy} (p. \pageref{skill:diplomacy}), \emph{Etiquette} (p. \pageref{skill:etiquette})

\subsection{Provoke}\index{Provoke!Skill}\index{Skills!Provoke}\label{skill:provoke}
\textbf{Settings:} All\\
\textbf{Scope:} Broad\\
\textbf{Other Names:} \emph{Taunt}, \emph{Agitate}, \emph{Challenge}\\
\vspace{\baselineskip}

\emph{Provoke} is the skill of deliberately inciting emotional reactions—anger, fear, shame, or panic. It’s used to goad enemies into reckless action, rattle someone’s composure, or escalate a situation. Where \emph{Intimidate} imposes silence or control, \emph{Provoke} seeks to draw out a response.

This skill shines in verbal duels, interrogation rooms, and social sabotage. Characters skilled in \emph{Provoke} can make others lash out, slip up, or abandon reason—sometimes without ever raising their voice.

\vspace{0.5\baselineskip}
\noindent\textbf{Related Skills:} \emph{Intimidate} (p. \pageref{skill:intimidate}), \emph{Command} (p. \pageref{skill:command})

