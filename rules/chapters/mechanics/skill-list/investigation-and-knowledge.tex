% !TeX root = ../../../wyrd.tex



\section[Investigation \& Knowledge]{Investigation \&\\ Knowledge}

Investigation skills represent a character’s ability to uncover hidden information. These skills focus on the process of learning what is not already known—whether through observation, deduction, questioning, or research. They differ from knowledge skills, which measure what a character already knows or understands. Knowledge skills, on the other hand, usually reflect what a character already knows. Knowledge skills are often more academic or theoretical, while investigation skills are practical and action-oriented.

\subsection{Forbidden Lore}\index{Forbidden Lore!Skill}\index{Skills!Forbidden Lore}\label{skill:forbidden-lore}
\textbf{Settings:} Horror, Supernatural, Occult\\
\textbf{Scope:} Narrow\\
\textbf{Other Names:} \emph{Eldritch Knowledge}, \emph{Dark Secrets}, \emph{The Unspeakable}\\
\vspace{\baselineskip}

\emph{Forbidden Lore} covers knowledge that was meant to stay buried—things whispered in dead tongues, rituals etched in bone, or truths too terrible to name. It includes awareness of cosmic horrors, cursed tomes, forgotten gods, and truths that erode the sanity of those who learn them. This skill goes beyond the occult—it deals with the truly dangerous and profane.

Characters with this skill may understand non-Euclidean diagrams, recognise ancient signs of corruption, or read fragments of a language not spoken for millennia. Using this knowledge often comes at a price, whether social, spiritual, or psychological.

\vspace{0.5\baselineskip}
\noindent\textbf{Requires:} \emph{Lore} (p. \pageref{skill:lore})

\subsection{Forensics}\index{Forensics!Skill}\index{Skills!Forensics}\label{skill:forensics}
\textbf{Settings:} Modern, Mystery, Horror\\
\textbf{Scope:} Narrow\\
\textbf{Other Names:} \emph{Crime Scene Analysis}, \emph{Pathology}, \emph{Trace Evidence}\\
\vspace{\baselineskip}

\emph{Forensics} is a specialised application of \emph{Investigate}, focused on analysing physical evidence to reconstruct events. This includes identifying substances, examining wounds, determining causes of death, and interpreting blood spatter or residue. It bridges science and investigation, often requiring laboratory equipment or knowledge of anatomy and chemistry.

Characters skilled in \emph{Forensics} are the experts called when a mystery has left physical remains or strange residues behind. Whether in a modern autopsy room, a gothic apothecary’s lab, or a makeshift workstation aboard a starship, they bring clarity to questions that corpses and chemicals can answer.

\vspace{0.5\baselineskip}
\noindent\textbf{Requires:} \emph{Investigate (p. \pageref{skill:investigate})}

\subsection{History}\index{History!Skill}\index{Skills!History}\label{skill:history}
\textbf{Settings:} All\\
\textbf{Scope:} Narrow\\
\textbf{Other Names:} \emph{Historical Knowledge}, \emph{Antiquity}, \emph{Legends}\\
\vspace{\baselineskip}

\emph{History} is the study of the past—its cultures, wars, monarchs, revolutions, and forgotten ages. Characters with this skill can recall key events, timelines, or sociopolitical movements, and interpret the significance of monuments, ruins, or historical artefacts.

This skill may be used to determine the origin of a relic, understand how a present situation echoes past conflicts, or identify the customs of a long-lost empire. It can also provide insight into famous battles, political shifts, or even ancient myths, depending on the genre.

\vspace{0.5\baselineskip}
\noindent\textbf{Requires:} \emph{Lore} (p. \pageref{skill:lore})

\subsection{Investigate}\index{Investigate!Skill}\index{Skills!Investigate}\label{skill:investigate}
\textbf{Settings:} All\\
\textbf{Scope:} Broad\\
\textbf{Other Names:} \emph{Research}, \emph{Search}, \emph{Deduce}\\
\vspace{\baselineskip}

\emph{Investigate} is the skill of uncovering hidden truths through careful observation, logical reasoning, and persistent inquiry. It covers activities such as examining evidence at a crime scene, poring over dusty archives, following a suspect’s trail, or cross-referencing records to verify alibis. While physical clues and paper trails are its bread and butter, this skill also includes structured questioning and methodical analysis.

Characters with high \emph{Investigate} are adept at connecting dots others miss. Whether you're a seasoned detective, a curious scholar, or an occult investigator probing into secrets best left buried, this skill allows you to see patterns in the chaos.

\vspace{0.5\baselineskip}
\noindent\textbf{Base of:} \emph{Forensics} (p. \pageref{skill:forensics})

\subsection{Lore}\index{Lore!Skill}\index{Skills!Lore}\label{skill:lore}
\textbf{Settings:} All\\
\textbf{Scope:} Broad\\
\textbf{Other Names:} \emph{Academics}, \emph{Knowledge}, \emph{Theory}\\
\vspace{\baselineskip}

\emph{Lore} represents a character’s depth of knowledge in scholarly, theoretical, or esoteric fields. It includes general education, historical facts, scientific principles, and obscure supernatural theories, depending on the setting. A character with high \emph{Lore} might recall arcane rituals, ancient legends, or the properties of rare minerals without consulting a book.

This skill doesn’t involve gathering new information (see \emph{Investigate} for that), but rather what the character already knows or can deduce from learned knowledge. In fantasy, it may encompass myth and magic; in sci-fi, quantum theory or alien biology.

\vspace{0.5\baselineskip}
\noindent\textbf{Base of:} \emph{Occult} (p. \pageref{skill:occult}), \emph{History} (p. \pageref{skill:history}), \emph{Forbidden Lore} (p. \pageref{skill:forbidden-lore}), \emph{Science} (p. \pageref{skill:science})

\subsection{Notice}\index{Notice!Skill}\index{Skills!Notice}\label{skill:notice}
\textbf{Settings:} All\\
\textbf{Scope:} Broad\\
\textbf{Other Names:} \emph{Perception}, \emph{Alertness}, \emph{Awareness}\\
\vspace{\baselineskip}

\emph{Notice} is the skill of perceiving things in the moment—spotting movement in the shadows, hearing a faint sound, or sensing when someone is watching. It governs how quickly a character detects danger, uncovers surface-level details, or realises something is out of place. Unlike \emph{Investigate}, which involves deliberate searching, \emph{Notice} is about immediate, reactive perception.

It is commonly used to determine who acts first in a conflict, to detect hidden enemies, or to uncover basic clues without extended analysis. In tense or dangerous environments, a high \emph{Notice} score often means the difference between walking into a trap—or spotting it just in time.

\subsection{Occult}\index{Occult!Skill}\index{Skills!Occult}\label{skill:occult}
\textbf{Settings:} Horror, Fantasy, Supernatural\\
\textbf{Scope:} Narrow\\
\textbf{Other Names:} \emph{Esoterica}, \emph{Mysticism}, \emph{Arcane Knowledge}\\
\vspace{\baselineskip}

\emph{Occult} is the study of hidden and supernatural forces—rituals, symbols, pacts, and forbidden truths. It includes folklore about fae and demons, knowledge of summoning rites, cursed artifacts, ley lines, and ancient prophecies. This skill is used to identify magical phenomena, interpret grimoires, or sense when something has violated natural law.

While many settings treat \emph{Occult} as superstition, in worlds where magic and spirits are real, this skill is a vital tool. Characters with a high \emph{Occult} rating are the ones who know when to salt a threshold, when not to speak a name, and how to tell real power from parlor tricks.

\vspace{0.5\baselineskip}
\noindent\textbf{Requires:} \emph{Lore} (p. \pageref{skill:lore})

\subsection{Science}\index{Science!Skill}\index{Skills!Science}\label{skill:science}
\textbf{Settings:} Modern, Sci-Fi, Steampunk\\
\textbf{Scope:} Narrow\\
\textbf{Other Names:} \emph{Scientific Knowledge}, \emph{Theory}, \emph{Experimental Methods}\\
\vspace{\baselineskip}

\emph{Science} represents formal training in natural, physical, and theoretical disciplines—biology, chemistry, physics, geology, astronomy, and beyond. It allows characters to conduct experiments, interpret lab data, and apply the scientific method to solve problems or confirm hypotheses.

In high-tech or steampunk settings, this skill is essential for designing experiments, operating advanced machinery, or understanding strange new phenomena. In pulp adventures, it may also let you whip up unstable compounds, analyse alien DNA, or predict the eruption of a volcano.

\vspace{0.5\baselineskip}
\noindent\textbf{Requires:} \emph{Lore} (p. \pageref{skill:lore})

