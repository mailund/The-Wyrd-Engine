
\section{Character Creation}
\index{Character creation}

Creating a character in \wyrd is a quick and streamlined process, designed to get players into the game with minimal preparation. Each character is defined by a small but meaningful set of attributes that shape their role in the story. Unlike systems with long-term progression, \wyrd prioritises narrative impact over mechanical advancement, making character creation simple yet flexible.

Every player character is built using the following elements:

\subsection{Step 1: Concept}

Before assigning mechanics, players should develop a brief \textbf{character concept}. This is a short description of who the character is, their role in the story, and what makes them interesting. Concepts should be evocative but flexible, helping guide both roleplay and mechanical choices.

\begin{CommentBox}{Example Character Concepts}
    \begin{itemize}
        \item \emph{A disgraced noble turned detective, haunted by his past.}
        \item \emph{An eccentric engineer whose inventions are as brilliant as they are dangerous.}
        \item \emph{A silver-tongued con artist who survives by wit and charm.}
        \item \emph{A fearless occult investigator seeking forbidden knowledge.}
    \end{itemize}
\end{CommentBox}

\subsection{Step 2: Choose Skills}

Each character has a set of \textbf{Skills} that determine their strengths and weaknesses. Skills represent broad areas of expertise rather than hyper-specialised talents, ensuring versatility.

Characters receive a total of \textbf{six skill ranks}, distributed as follows:

\begin{itemize}
    \item \textbf{1 \Expert} skill
    \item \textbf{2 \Skilled} skills
    \item \textbf{3 \Novice} skills
\end{itemize}

All unselected skills default to \Untrained.

\begin{Example}{}
	When assigning skills, players should consider their character’s background and expertise. A veteran detective might prioritise \textbf{Investigate} and \textbf{Notice}, while a rogue might favour \textbf{Stealth} and \textbf{Deceive}.
\end{Example}

The total sum of skill ranks should equal \textbf{10}. This ensures that every character is balanced in overall competence while allowing for specialisation.

\subsection{Step 3: Select Traits}

Every character has exactly \textbf{three Traits}. Traits represent exceptional abilities, personal quirks, or special training that set a character apart. 

\textbf{Traits provide one of three benefits:}
\begin{itemize}
    \item A \textbf{+2 bonus} when applied to a relevant skill check.
    \item A \textbf{special ability} that can be used \emph{once per scene or session}.
    \item A \textbf{narrative permission} to attempt actions that would normally be impossible.
\end{itemize}

\begin{CommentBox}{Example Traits}
    \begin{itemize}
        \item \textbf{Master Duelist} – \emph{Gain +2 to Fight when using a rapier or fencing techniques.}
        \item \textbf{Inventive Genius} – \emph{Can craft unique gadgets that defy conventional mechanics.}
        \item \textbf{Unshakable Will} – \emph{Once per session, ignore the effects of fear or mind control.}
        \item \textbf{Underworld Connections} – \emph{Gain +2 to Contacts when dealing with criminals.}
        \item \textbf{The Cards Never Lie} – \emph{Use Lore instead of Investigate when predicting an outcome.}
    \end{itemize}
\end{CommentBox}

Traits should enhance a character’s strengths and provide unique advantages in play. They should not be overly broad or cover multiple unrelated areas.

\subsection{Step 4: Select Gear}

\wyrd does not track mundane items or encumbrance. Instead, \textbf{gear} is used to track items that have a significant impact on gameplay. Unlike traits, gear is not inherent to a character but can be aquired or lost during play. At the Game Master's discression, players can start out with a fixed number of gear items, say three per character. Alternatively, important gear can work as plot devices, with the Game Master deciding when and how to introduce them into the game.

Each piece of gear functions like a Trait, providing either:
\begin{itemize}
    \item A \textbf{+2 bonus} when used appropriately.
    \item A \textbf{special ability} usable once per scene or session.
    \item A \textbf{narrative permission} to perform unique actions.
\end{itemize}

\begin{CommentBox}{Example Gear}
    \begin{itemize}
        \item \textbf{Clockwork Lockpick} – \emph{+2 to Burglary when opening mechanical locks.}
        \item \textbf{Enchanted Mirror} – \emph{Once per session, reveal a hidden truth.}
        \item \textbf{Mastercrafted Rapier} – \emph{+2 to Fight in one-on-one duels.}
        \item \textbf{Detective’s Notebook} – \emph{Use Investigate instead of Rapport when questioning suspects.}
        \item \textbf{Hidden Derringer} – \emph{Once per scene, draw a concealed firearm unnoticed.}
    \end{itemize}
\end{CommentBox}

\subsection{Step 5: Stress and Wounds}

Characters have a limited ability to absorb harm before suffering long-term effects. A standard character has:
\begin{itemize}
    \item \textbf{Four Stress Boxes} – Used to absorb minor failures.
    \item \textbf{Mild, Moderate, and Severe Wounds} – Represent lasting harm or setbacks.
\end{itemize}
\DamageBox

Wounds replace traditional hit points and can reflect physical, mental, or social strain. A "Mild" consequence might be a bruised rib, while a "Severe" consequence could be a permanent injury or a shattered reputation.

\subsection{Step 6: Final Details}

With mechanics in place, players can now define their characters':
\begin{itemize}
    \item \textbf{Name} – Fitting for the setting and character concept.
    \item \textbf{Appearance} – Distinctive traits, clothing, and demeanour.
    \item \textbf{Personality} – Key personality traits, motivations, or quirks.
    \item \textbf{Backstory} – A brief origin story or notable past experiences.
\end{itemize}

\begin{CommentBox}{Final Advice for Players}
    \textbf{Focus on character over numbers.} \wyrd is designed for narrative-driven play, so build a character that fits the story rather than optimising for maximum efficiency.
\end{CommentBox}

Once these steps are complete, the character is ready for play!

% \WyrdFooterImage{img/pageart/bottom-gear-2}
