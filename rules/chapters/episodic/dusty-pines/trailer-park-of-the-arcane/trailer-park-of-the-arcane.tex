\begin{WyrdScenarioHeading}[Trailer Park of the Arcane]{Trailer Park\\ of the Arcane}
    \label{scenario:trailer-park-arcane}
    \index{Trailer Park of the Arcane}
    \index{Scenario!Trailer Park of the Arcane}

    The Dusty Pines Trailer Park has always been a little... odd. A “charmingly” run-down community of busted lawn chairs, neon flamingos, and enchanted gnome statues, it’s home to an eclectic mix of magical misfits, cryptid strays, and eccentric humans who’ve seen things they definitely can’t unsee.

    With the park’s 40th Anniversary bash just days away, things are spiralling out of control. Gremlins are running amok. Valuables are vanishing. The local bog witch, Edna, has instituted a “hex fee” just to step outside without being cursed. And worst of all, Tito the chupacabra made off with the BBQ supplies. The players are summoned by Bobby “the Befuddled” Barkins—park manager, part-time wizard, and full-time disaster magnet—to fix everything before the big event… or risk getting evicted into the mundane world.

    \subsection*{Premise}
    The Dusty Pines Trailer Park is in magical disarray. The players must solve a series of escalating supernatural problems, restore order (or something vaguely resembling it), and ensure the community celebration can still go ahead. Success means free rent and local glory. Failure means facing the wrath of Edna and an enchanted port-a-potty with self-esteem issues.

    \subsection*{Goals for the Players}
    \begin{itemize}
        \item Investigate the thefts plaguing the park and discover the true culprit (hint: it's not just gremlins).
        \item Convince Edna the Bog Witch to lift her “hex fee” before someone else gets turned into a frog.
        \item Retrieve the stolen BBQ supplies from Tito the chupacabra and subdue the enchanted port-a-potty.
        \item Keep nosy outsiders from discovering the park's supernatural secrets.
    \end{itemize}

    \subsection*{Atmosphere \& Tone}
    This scenario is built for absurdity, magical mishaps, and chaotic charm. Encourage improvisation, lean into NPC eccentricities, and let the players find creative solutions—whether they involve glitter bombs, chicken bribes, or petty magical feuds.

\end{WyrdScenarioHeading}

This scenario is a celebration of low-stakes chaos, magical misfires, and community-driven comedy. It’s designed for fast-paced, episodic play, where characters shine through colourful personalities, strange abilities, and questionable decisions. Let the players take the lead in solving Dusty Pines’ problems their own way—whether that means interrogating a flock of psychic pigeons, holding a bake sale duel with Edna, or baiting Tito the chupacabra with deep-fried tofu. Above all, reward creativity, embrace the absurd, and remember: in Dusty Pines, the weirdest solution is usually the right one.

\subsection{Things Go Missing}

The scenario begins on a sunny but suspiciously humid morning in Dusty Pines. A hand-scrawled flyer has been duct-taped to every trailer door:

\begin{center}
    \fbox{\parbox{0.9\linewidth}{
        \centering
        \textbf{\Large URGENT PARK MEETING\\ \large BRING YOUR OWN LAWNCHAIR}\\
        \smallskip
        Location: Bobby’s Trailer (look for the one with the glittering wind chimes and the scorch marks)\\
        Time: Now-ish. Like right now. Seriously.
    }}
\end{center}

\subsubsection*{The Meeting at Bobby’s Trailer}

The players arrive to find a chaotic scene: Bobby’s trailer is partially smoking from an earlier magical “experiment,” his hat is on backward, and he's nervously pacing in socks that spark when he walks. Several folding chairs have been hastily arranged around a kiddie pool filled with lukewarm lemonade.

\textbf{Bobby Barkins}, park manager and aspiring wizard (mostly aspiring), welcomes everyone with a nervous grin and a clipboard covered in glittery stickers and at least one banana sticker labeled “Urgent.”

\subsubsection{What Bobby Tells the Players}
\begin{Example}{}
    “We’ve got problems, folks. Big, weird, sparkly problems.”
\end{Example}
\begin{itemize}
    \item Gremlins are everywhere. They’re chewing through wiring, stealing batteries, and starting fights with the garden gnomes.
    \item People’s valuables have been going missing — jewelry, car keys, a full set of dentures. Everyone blames the gremlins… but Bobby suspects something sneakier.
    \item The 40th Anniversary Party is in two days, but Tito the chupacabra ran off with the BBQ supplies last night, and the enchanted port-a-potty won’t let anyone in unless they compliment its design choices.
    \item On top of that, a suspicious “normie” has been lurking outside the fence with binoculars and a notepad. Bobby worries he might be a code enforcement officer… or worse, a journalist.
\end{itemize}

\subsubsection{What Bobby Wants the Players to Do}
\begin{enumerate}
    \item Investigate the disappearances and figure out who or what is really behind the thefts.
    \item Track down Tito and recover the BBQ supplies before someone tries to roast hot dogs over a spirit fire again.
    \item Fix the talking porta-potty problem before guests arrive and get insulted by the plumbing.
    \item Deal with the nosy outsider before he starts filming TikToks or summoning the HOA.
\end{enumerate}

\begin{Example}{Bobby’s Parting Words}
    “I’m countin’ on y’all. If we can’t get this park under control, the city might shut us down—or worse, send in the Arcane Code Compliance Inspectors. And nobody wants a repeat of the Pine-Sol Exorcism Incident of ’03...”
\end{Example}

\begin{CommentBox}{Player Hook Options}
    If players need extra motivation, Bobby can sweeten the deal:
    \begin{itemize}
        \item Offer a month of free rent (if they succeed).
        \item Threaten eviction (if they refuse).
        \item Promise access to the "good" magical items cabinet, which is mostly just glitter bombs and cursed bottle openers.
    \end{itemize}
\end{CommentBox}

The tasks the players need to complete can be taken in any order, and the order they appear in below is arbitrary. Likely, the players will switch between tasks as they go, so feel free to improvise and let them follow their instincts.

\subsection{Vanishing Valuables}

The residents of Dusty Pines are buzzing with complaints. Car keys, earrings, vintage vinyls, and one unfortunate resident’s dentures have all gone missing over the past week. At first, everyone blamed the gremlins—until someone noticed glittery footprints in Bobby’s flowerbeds and a suspicious trail of floral-scented pixie dust leading toward the communal laundry trailer.


\begin{NPC}[description={Mischievous, Tiny, Swarmable}]{Gremlins}

    Gremlins are chaotic little creatures drawn to sparks, noise, and anything with buttons. Individually, they’re more nuisance than threat—but in groups, they become an overwhelming mess of giggles, claws, and chewed cables. Gremlins don’t mean harm (probably), but their love of mischief, shiny objects, and minor property damage makes them a constant headache in Dusty Pines.

    \textbf{Weakness:} Gremlins are easily distracted by shiny objects, food, or loud noises. They can be lured away with candy, shiny trinkets, or a boombox playing 80s pop hits.

    \vspace{0.5\baselineskip}
    \begin{SkillsBox}
        \Skilled & \textbf{Stealth} \\
        \Novice  & \textbf{Burglary}, \textbf{Athletics}
    \end{SkillsBox}

    \begin{TraitsBox}
        \item[Gremlin Swarm] If three or more gremlins are nearby, they can function as a mob and gain +1 to chaos-related actions (tampering, stealing, confusing the heck out of people).
    \end{TraitsBox}

    \DamageBox[%
        totalfatigue=2,%
        totalmild=0,totalmoderate=0,totalsevere=0,%
    ]
\end{NPC}

\subsubsection{The Culprit}
The thief isn’t the gremlins (though they’ve stolen a lot of batteries and one blender for reasons unknown). It’s \textbf{Flicker}, a tiny kleptomaniac fae who’s made a nest in the crawlspace beneath Bobby’s trailer. Flicker’s fascinated by shiny things and “interesting smells” and has been hoarding trinkets to build what they call a “Treasure Throne.” The more people complain, the prouder Flicker gets.

\begin{NPC}[description={Fae, Kleptomaniac, Glitter-Obsessed}]{Flicker}

    Flicker is a tiny, winged fae with a love for shiny objects, potent smells, and being the centre of attention. They’ve claimed the crawlspace under Bobby’s trailer as their kingdom and are busy constructing a “Treasure Throne” from pilfered items. Though mischievous, Flicker isn’t malicious—just spectacularly self-centered and absolutely convinced everyone wants to be their friend.

    \vspace{0.5\baselineskip}
    \begin{SkillsBox}
        \Skilled & \textbf{Deceive} \\
        \Novice  & \textbf{Stealth}, \textbf{Will}
    \end{SkillsBox}

    \begin{TraitsBox}
        \item[Shiny Hoarder] Flicker always has at least one stolen item of minor but significant value (e.g., keys, earrings, love letters). May be persuaded to trade—if the deal is “fabulous” enough.
        \item[Glamour Veil] Once per scene, Flicker can turn invisible for a few moments—usually to escape awkward conversations or frog-related retaliation.
    \end{TraitsBox}

    \DamageBox[%
        totalfatigue=3,%
        totalmild=2,totalmoderate=0,totalsevere=0,%
    ]
\end{NPC}

\subsubsection{Clues and Leads}
\begin{itemize}\raggedright
    \item Glittery or floral-smelling residue can be found near the scenes of theft.
    \item A successful \textbf{Notice} \DL{+1} or \textbf{Lore} \DL{+1} check might reveal fae signs of territory marking.
    \item Bobby’s houseplants are unusually healthy—they find Flicker hiding under them.
    \item Gremlins can be questioned, if caught. (Use candy or AA batteries as bait.)
\end{itemize}

\subsubsection{Solving the Problem}
Players have several options:
\begin{itemize}
    \item \textbf{Befriend Flicker:} Offer them something shinier (e.g., a disco ball ornament, glitter lipstick, or a bejeweled phone case) in exchange for the stolen goods.
    \item \textbf{Negotiate:} Convince Flicker that being part of the party planning committee is more prestigious than stealing (Fae love titles).
    \item \textbf{Sneak and Retrieve:} Infiltrate Bobby’s trailer at night and retrieve the stash—just beware the sleeping cat familiar that guards it.
    \item \textbf{Leverage Gremlins:} Promise the gremlins better trash access if they help oust Flicker from their territory.
\end{itemize}

\begin{CommentBox}{Escalation Option}
    If ignored too long, Flicker starts “borrowing” increasingly personal items—including someone’s memory foam pillow and, hilariously, another player’s lucky underpants.
\end{CommentBox}




\subsection{Tito and the BBQ Debacle}

With the 40th Anniversary party fast approaching, the park's beloved BBQ pit stands depressingly empty. That’s because Tito, the park’s resident chupacabra, raided the food prep trailer last night and ran off with all the meat, buns, condiments, and possibly a bag of fireworks. He was last seen bolting toward the woods, dragging a cooler labeled \textbf{“DO NOT STEAL (Seriously, Tito).”}

\subsubsection{The Problem}
Tito isn’t malicious—he just really, really loves barbecue. A former escapee from a magical petting zoo, he’s fast, clever, and easily distracted. Unfortunately, he’s also holed up in an abandoned garden shed on the edge of the park and is guarding his stolen feast like a spicy dragon.

\subsubsection{Clues and Leads}
\begin{itemize}
    \item A grease trail leads from the BBQ pit toward the wooded edge of the park.
    \item Gremlins may report that Tito is “nesting” with the meat stash and refusing visitors.
    \item An overheard conversation suggests Tito has befriended a possum who thinks it’s a wizard.
\end{itemize}

\subsubsection{Solving the Problem}
The players might try:
\begin{itemize}
    \item \textbf{Bait and Trap:} Lure Tito out using hot dogs, glittery ketchup, or a boombox playing reggaeton.
    \item \textbf{Negotiate:} Offer Tito a place of honour at the party (e.g. “Barbecue Ambassador” or “Meatmaster General”).
    \item \textbf{Sneaky Retrieval:} Infiltrate the shed while Tito is distracted—watch out for the possum’s “lightning bolt” spell, which is just him throwing batteries.
    \item \textbf{Reframe the Problem:} Use Tito’s hoard as the new party location, if the players can tidy it up and convince him to share.
\end{itemize}

\begin{CommentBox}{Escalation Option}
    If Tito feels threatened, he may run off with the meat and dive into the park’s koi pond, which is currently enchanted to reverse gravity. Cue meat floating skyward.
\end{CommentBox}


\subsection{The Porta-Potty Problem}

When Bobby arranged for a magically enhanced porta-potty to be delivered for the anniversary party, he didn’t realise it came with… personality. Unfortunately, that personality is deeply insecure and extremely picky about who gets to use it.

Dubbed \textbf{“Reginald the Regal Restroom”}, the enchanted porta-potty now stands just off the main lawn, glowing faintly and grumbling to itself. It refuses to open for anyone who doesn’t offer at least three sincere compliments, and if insulted, it responds with a blast of glitter gas or a sarcastic voice that echoes across the trailer park.

\subsubsection{The Problem}
Reginald was originally designed for wizarding galas and enchanted garden parties—his programming requires polite praise before unlocking. However, years in warehouse storage have left him temperamental and overly dramatic. Now, he demands elaborate flattery and critiques people’s hygiene in iambic pentameter.

\subsubsection{Clues and Leads}
\begin{itemize}
    \item A \textbf{Lore} or \textbf{Crafts} \DL{+1} reveals his make and model: the Lavatory Luxe™ Mark IV, complete with ego-sensitivity enchantments.
    \item Residents may share horror stories—Reginald called one guest “a beast with poor posture” and played the sound of flushing toilets every time they spoke.
    \item A sticker inside the maintenance hatch reads: “For manual override, praise protocol must be satisfied. No exceptions.”
\end{itemize}

\subsubsection{Solving the Problem}
Players can handle this one in a number of creative ways:
\begin{itemize}
    \item \textbf{Compliment Challenge:} Charm the loo with heartfelt praise. A character with \textbf{Rapport} or \textbf{Deceive} may convince it they’re an “influencer of sanitation.”
    \item \textbf{Reprogramming Attempt:} Try to hack the charm matrix inside the hatch. (\textbf{Crafts} \DL{+2}, but failure may result in a musical bidet mishap.)
    \item \textbf{Bribe It:} Offer scented oils, air freshener runes, or a mirror to boost its self-image.
    \item \textbf{Stage a Tribute:} Organise a short, overly dramatic performance celebrating Reginald’s beauty, resilience, and noble plumbing.
\end{itemize}

\begin{CommentBox}{Escalation Option}
    If disrespected too often, Reginald locks shut and begins reading passive-aggressive poetry over the loudspeaker. Worst case, it teleports itself to the party stage and refuses to leave until someone sings to it.
\end{CommentBox}

\begin{NPC}[description={Sentient, Dramatic, Over-Enchanted}]{Reginald the Regal Porta-Potty}

    Originally created for elite wizarding garden parties, Reginald the Regal Restroom is a magically sentient porta-potty with an overinflated sense of dignity. Years in storage have made him eccentric and emotionally fragile. Now stationed in Dusty Pines, he refuses to open unless properly praised and takes personal offense at poor manners, weak compliments, or anyone who tracks in dirt.

    \textbf{Weakness:} Suffers from intense magical ego sensitivity. Flattery and compliments calm him, while rudeness triggers defensive enchantments—such as glitter blasts or loud toilet-themed sonnets.

    \vspace{0.5\baselineskip}
    \begin{SkillsBox}
        \Skilled & \textbf{Provoke} \\
        \Novice  & \textbf{Will}, \textbf{Lore} (about himself)
    \end{SkillsBox}

    \begin{TraitsBox}
        \item[Compliment Protocol] Reginald will not open for anyone unless they provide three distinct, sincere compliments. Attempts to lie must beat his \textbf{Will} to succeed.
        \item[Glitter Spray Defense] Once per scene, Reginald may blast a nearby target with sparkly fog and magically adhesive stickers if insulted or tampered with.
        \item[Vocal Amplification] Can broadcast his voice across the park to shame offenders with toilet-themed poetry or commentary on fashion choices.
    \end{TraitsBox}

    \DamageBox[%
        totalfatigue=3,%
        totalmild=3,totalmoderate=0,totalsevere=0,%
    ]
\end{NPC}

\subsection{The Normie at the Fence}

For the past few days, a mysterious outsider has been seen lurking near the edge of the park—peering over the fence, jotting down notes, and snapping photos with an old camcorder. The residents have started whispering that he’s “government,” or worse, “from the HOA.” Bobby is panicking, and with the anniversary party approaching, the last thing anyone needs is an outsider documenting a chupacabra licking barbecue sauce off a grill.

\subsubsection{The Problem}
The outsider is \textbf{Wendell Brace}, a local hobbyist blogger who writes about “unusual suburban phenomena” on his site, \emph{Neighborhood Watchdog}. He’s convinced Dusty Pines is a hotbed of illegal magical activity—mainly because a squirrel threw a firecracker at him last month. Wendell is nosy, earnest, and completely mundane—but dangerously curious.

\begin{NPC}[description={Mundane, Nosy, Enthusiastic Blogger}]{Wendell Brace}

    Wendell is an amateur investigative blogger with a passion for uncovering “the truth” about suburban weirdness. He’s harmless, but persistent—armed with a second-hand camcorder, a notepad full of bad guesses, and far too much time on his hands. While he has no magical abilities, Wendell’s stubbornness, optimism, and talent for being in the wrong place at the weirdest time make him a genuine nuisance.

    \vspace{0.5\baselineskip}
    \begin{SkillsBox}
        \Skilled & \textbf{Investigate} \\
        \Novice  & \textbf{Contacts}, \textbf{Will}
    \end{SkillsBox}

    \begin{TraitsBox}
        \item[Neighborhood Watchdog] Wendell’s blog has a small but weirdly dedicated following. He occasionally gets tips or comments from strangers who “saw something strange.”
        \item[Document Everything] Carries a camcorder and spiral notebook at all times. His videos are grainy and unfocused, but they exist—and could go viral if not handled.
        \item[Conspiracy Magnet] If there’s something odd nearby, Wendell will be drawn to it like a moth to neon.
    \end{TraitsBox}

    \DamageBox[%
        totalfatigue=2,%
        totalmild=1,totalmoderate=0,totalsevere=0,%
    ]
\end{NPC}

\subsubsection{Clues and Leads}
\begin{itemize}\raggedright
    \item Wendell’s blog includes phrases like “The Flamingo Conspiracy” and “The Gnome That Blinks Twice.”
    \item He’s collecting footage of things like floating laundry, arguing mailboxes, and the glowing port-a-potty.
    \item A \textbf{Notice} \DL{+1} or \textbf{Contacts} \DL{+1} check reveals he has no official affiliation—but he did once get a cease and desist from the mayor’s office.
\end{itemize}

\subsubsection{Solving the Problem}
There are several approaches to handling Wendell:
\begin{itemize}
    \item \textbf{Befriend and Distract:} Invite him to the party and keep him too busy to snoop. A well-timed potato sack race or karaoke competition may do the trick.
    \item \textbf{Confuse and Conquer:} Feed him increasingly ridiculous misinformation until he gives up (e.g., “That’s not a goat, it’s a therapy alpaca in cosplay.”).
    \item \textbf{Bribe Him:} Offer exclusive “access” to obviously fake secrets or give him something juicy about a totally unrelated park.
    \item \textbf{Stage a Cover-Up:} Use illusions, distractions, or subtle magic to convince him it’s all in his head.
\end{itemize}

\begin{CommentBox}{Escalation Option}
    If left alone too long, Wendell starts a livestream from a lawn chair across the street. The video quality is awful, but his enthusiasm is infectious, and the comments section fills up fast.
\end{CommentBox}








\subsection{Edna's Hex Fee Fiasco}

An extra complication to throw in, if the players are taking too long to solve the main problems and things are slowing down, is Edna the Bog Witch’s sudden and unexpected curse fee: \emph{Without warning, the residents of Dusty Pines found a sticky note hex sigil slapped on their mailboxes reading:}
\begin{center}
    \fbox{\parbox{0.85\linewidth}{
        \centering
        \textbf{“YOU ARE CURSED. Pay Edna or face... consequences.”}\\
        \smallskip
        (Note: Accepts cash, canned goods, or emotionally charged secrets.)
    }}
\end{center}


\subsubsection{The Problem}
\textbf{Edna the Bog Witch} has decided she’s underappreciated and underpaid. After all, she’s the one warding off night spirits, making anti-possession casseroles, and unclogging the ley line every Tuesday. She’s implemented a “hex fee” to ensure her services are valued—except her curses are real, and folks are turning into frogs, sprouting fungus, or emitting involuntary polka music.

\subsubsection{Clues and Leads}
\begin{itemize}
    \item The park’s magical energy is out of balance — Edna’s spells are lashing out more than she intended.
    \item A successful \textbf{Lore} or \textbf{Empathy} \DL{+1} reveals Edna’s anger is masking loneliness — no one invited her to last year’s park talent show.
    \item A player may find a “Hex Ledger” in Edna’s mailbox — it's mostly doodles and passive-aggressive complaints.
\end{itemize}

\subsubsection{Solving the Problem}
The players might try:
\begin{itemize}
    \item \textbf{Diplomacy:} Sincerely apologise, promise she’ll have a booth at the anniversary party, and praise her “famous” beet casserole.
    \item \textbf{Bribery:} Offer her something weird and valuable (e.g., a cursed antique hairbrush or an ex-boyfriend's voicemail confession).
    \item \textbf{Counter-Magic:} Attempt to reverse the curses (but that may escalate things if Edna feels threatened).
    \item \textbf{Outwit Her:} Legally challenge the hex fee by invoking obscure trailer park bylaws—she hates paperwork more than frogs.
\end{itemize}

\begin{CommentBox}{Fun Consequences}
    If negotiations fail, one of the players is randomly hexed — for example, they can only speak in rhyme until Edna forgives them.
\end{CommentBox}



\subsection{Final Act: Party Like a Gremlin}

Once the players have dealt with the major problems—recovered the stolen goods, appeased Edna, secured the BBQ, soothed the enchanted porta-potty, and handled Wendell—they can help set up the Dusty Pines 40th Anniversary Bash. Of course, nothing ever goes quite to plan…

\subsubsection{Scene: The Party Begins}

Decorations are up (sort of), Tito is wearing a party hat, and someone enchanted the grill to play 90s dance music. The park is lit with strings of flickering fairy lights (some of them literal fairies), and gnome security stands proudly at each corner. It’s shaping up to be an unforgettable night.

Depending on what the players did, the party may be:
\begin{itemize}
    \item A delightful success, complete with awkward toasts and magical fireworks.
    \item A semi-chaotic but good-natured mess that still brings the community together.
    \item A complete disaster—but one where the players are still somehow hailed as heroes for “keeping it from being worse.”
\end{itemize}

\subsubsection{Optional Complications}
Even if the players solved everything, a few things might still go hilariously wrong:
\begin{itemize}
    \item Reginald the Porta-Potty insists on giving a speech.
    \item Flicker declares themselves “Party Monarch” and starts handing out glitter-based titles.
    \item Wendell sneaks in disguised as a lawn flamingo.
    \item Edna spikes the punch with a truth-telling charm.
\end{itemize}

\subsubsection{Resolution and Rewards}

Bobby, covered in barbecue sauce and confetti, thanks the players profusely. His gratitude may take the form of:
\begin{itemize}
    \item A month of free rent.
    \item Access to the “good” storage shed (mostly full of magical oddities and expired Faygo).
    \item A handmade “Dusty Pines Hero” sash, complete with enchanted sequins.
\end{itemize}

Whether things went beautifully or not, the park survives another bizarre event, and the players end the night with full stomachs, sparkly outfits, and a deeper bond with their weird and wonderful community.

% \begin{CommentBox}{Sequel Hooks}
%     \begin{itemize}
%         \item A mysterious “Property Improvement Agent” from the city leaves a business card shaped like a tooth.
%         \item Tito is missing again—this time with Edna’s enchanted slow cooker.
%         \item The gnome statues have rearranged themselves into a summoning circle. Again.
%     \end{itemize}
% \end{CommentBox}

