\end{multicols}
\newpage
\section{Recurring Cast}

\subsection{{\small Park Manager}\\ Bobby “the Befuddled” Barkins}
\label{npc:bobby-barkins}

\emph{Well-meaning, magically underqualified, and terminally overwhelmed, Bobby is the beating (and often panicking) heart of Dusty Pines.}
\vspace{.5\baselineskip}

\columnratio{0.375,0.375,0.25}
\begin{paracol}{3}
    \subsubsection*{Background:}
    Bobby inherited the role of park manager from his uncle, along with a dusty spellbook, a broken wand, and a list of residents longer than the eviction notices he keeps forgetting to file. He genuinely wants Dusty Pines to be safe, functional, and hex-free—he just doesn’t have the magical chops to make it happen.

    Despite his magical mishaps and scatterbrained ways, Bobby is loved by most of the park’s residents. He’s always first to show up in a crisis (usually shouting “What did I miss?!”) and last to give up on even the most cursed of trailers. Deep down, Bobby wants to be a hero. Most days, surviving until sunset will do.

    \switchcolumn
    \subsubsection*{Using in Play:}
    Bobby is a key source of comedy and quest hooks. He can:
    \begin{itemize}
      \item Beg the players for help when things go weird.
      \item Accidentally trigger supernatural events.
      \item Offer bizarre magical items from his “supply shed.”
      \item Be the emotional core when the story gets personal.
    \end{itemize}
    Use Bobby to introduce a problem, defuse tension, or give the players someone to protect, pity, or laugh at—often all at once.

    \switchcolumn
    \subsubsection{Skills}
      \noindent\Expert: Community \\
      \noindent\Skilled: Improvisation \\
      \noindent\Novice: Repair, Luck, Disguise \\
      \noindent\Useless: Magical Theory \\
    \subsubsection{Traits}
      \textbf{One Spell Away From Disaster} — Once per session, Bobby casts a spell with unpredictable (GM-chosen) results. It always works—just not how he intended.
\end{paracol}
\vspace{.5\baselineskip}
\hrule
\vspace{.5\baselineskip}

\subsection{{\small Local Hexslinger}\\ Edna the Bog Witch}
\label{npc:edna-bogwitch}

\emph{Clever, cantankerous, and maybe just a little cursed herself, Edna is the trailer park’s resident magical entrepreneur—and terror.}
\vspace{.5\baselineskip}

\columnratio{0.375,0.375,0.25}
\begin{paracol}{3}
    \subsubsection*{Background:}
    Edna arrived at Dusty Pines decades ago and never left. Her trailer is perched on stilts, half-sunken into the marshy backlot, filled with glowing jars and croaking things in bathtubs. She trades in curses, charms, and magical “solutions” that often create as many problems as they solve.

    While Edna acts grumpy and reclusive, she secretly considers the park her domain and its residents her business—whether she likes them or not. Her magic is old, strange, and surprisingly potent. And if she charges an arm and a leg for help, well, that’s just capitalism, baby.

    \switchcolumn
    \subsubsection*{Using in Play:}
    Edna is a wildcard NPC. She can:
    \begin{itemize}
      \item Provide magical items, favours, or (expensive) assistance.
      \item Complicate the plot with her side schemes.
      \item Act as a reluctant mentor or magical consultant.
      \item Be bribed, blackmailed, or convinced—but rarely cheaply.
    \end{itemize}
    She’s useful, dangerous, and deeply inconvenient. Keep her unpredictable, and never let the players feel entirely sure she’s on their side.

    \switchcolumn
    \subsubsection{Skills}
      \noindent\Expert: Curses \\
      \noindent\Skilled: Potioncraft, Intuition \\
      \noindent\Novice: Cooking, Gardening, Gossip \\
    \subsubsection{Traits}
      \textbf{Curses and Casseroles} — Once per session, Edna may invoke a charm, ward, or curse with a one-sentence description. The GM determines the exact effect—half hex, half hospitality.
\end{paracol}
% \vspace{.5\baselineskip}
% \hrule
% \vspace{.5\baselineskip}

\newpage

\subsection{{\small Mascot Menace}\\ Tito the Chupacabra}
\label{npc:tito-chupacabra}

\emph{Equal parts cryptid and chaos gremlin, Tito is a four-legged menace with a big appetite and an even bigger reputation.}
\vspace{.5\baselineskip}

\columnratio{0.375,0.375,0.25}
\begin{paracol}{3}
    \subsubsection*{Background:}
    Tito was once part of a magical petting zoo. He escaped, found Dusty Pines, and decided never to leave. He’s mostly harmless—unless you're a chicken. Or a barbecue. Or anyone trying to get some peace and quiet.

    Rumours swirl that Tito is actually a fae creature in disguise, a government experiment gone wrong, or the reincarnation of a forgotten trickster god. Tito refuses to confirm or deny anything. He just winks and runs off with your leftovers.

    \switchcolumn
    \subsubsection*{Using in Play:}
    Tito is an agent of chaos. He can:
    \begin{itemize}
      \item Appear in the middle of a scene causing trouble.
      \item Steal or hide important items.
      \item Alert players to supernatural danger with uncanny instincts.
      \item Serve as bait, distraction, or reluctant sidekick.
    \end{itemize}
    Use Tito to liven up a dull scene, introduce a new twist, or provide comedic relief. He’s always a surprise—usually messy.

    \switchcolumn
    \subsubsection{Skills}
      \noindent\Expert: Stealth \\
      \noindent\Skilled: Scavenging, Escape \\
      \noindent\Novice: Mimicry, Petting Zoo Diplomacy \\
    \subsubsection{Traits}
      \textbf{Fastest Thing on Four Legs} — Tito may flee, fetch, or vanish from any scene once per session, bypassing obstacles with unnatural speed and uncanny instinct.
\end{paracol}
\vspace{.5\baselineskip}
\hrule
\vspace{.5\baselineskip}

\subsection{{\small Gnome Whisperer}\\ “Rusty” Turnbuckle}
\label{npc:rusty-turnbuckle}

\emph{Eccentric, enthusiastic, and questionably lucid, Rusty claims to speak to lawn ornaments—and the worst part is, he might not be wrong.}
\vspace{.5\baselineskip}

\columnratio{0.375,0.375,0.25}
\begin{paracol}{3}
    \subsubsection*{Background:}
    Rusty arrived at Dusty Pines with two suitcases, a tinfoil hat, and a wheelbarrow full of gnomes. Since then, he’s made it his mission to commune with the “Tiny Folk” and protect the ornamental underclass from neglect and mockery.

    Most residents consider Rusty harmless—if deeply weird. But when the lawn ornaments start moving on their own, he’s suddenly everyone’s first call. Whether he actually hears them or just knows their patterns is anyone’s guess.

    \switchcolumn
    \subsubsection*{Using in Play:}
    Rusty can:
    \begin{itemize}
      \item Provide cryptic insights about magical goings-on.
      \item Act as comic relief, a red herring, or unexpected savant.
      \item Befriend unlikely allies (especially small, inanimate ones).
      \item Lend his “gnome army” to unusual tasks.
    \end{itemize}
    Rusty adds flavour, eccentricity, and a little mystery. Let him be the trailer park oracle no one asked for—but everyone needs.

    \switchcolumn
    \subsubsection{Skills}
      \noindent\Expert: Gnome Lore \\
      \noindent\Skilled: Tinkering, Foraging \\
      \noindent\Novice: Diplomacy, Animal Handling \\
    \subsubsection{Traits}
      \textbf{Underestimated Ally} — Once per session, Rusty may provide the exact bizarre insight or item the players need—delivered with zero explanation and uncanny timing.
\end{paracol}
% \vspace{.5\baselineskip}
% \hrule
% \vspace{.5\baselineskip}


\begin{multicols}{2}
