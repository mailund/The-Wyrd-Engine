% !TeX root = ../../../wyrd.tex


\begin{WyrdSettingHeading}
    \WyrdCapLine{D}{usty Pines}, present day. A sun-bleached trailer park at the edge of a forgettable town, where time ambles, lawn chairs creak in the breeze, and the strange is just part of the scenery. It’s the kind of place where the plumbing hisses with more than water, and the neighbours include witches, cryptids, and that one guy who definitely talks to gnomes.

    You are a resident—by choice or by accident—in a community that’s equal parts magical sanctuary, chaotic campground, and deeply dysfunctional found family. The world outside might ignore Dusty Pines, but weirdness has a way of clumping here like dryer lint. And when things go sideways (and they always do), someone has to step up. That someone is you.

    The manager’s a wizard with more optimism than talent. The local bog witch charges a hex toll. The chupacabra keeps getting into the bins. And the fae? Well, let’s just say they’ve been showing interest in the park’s boundaries again. Every week brings new trouble—sentient lawn ornaments, rogue furniture curses, glamour-addicted elves, or even the IRS.

    There’s no grand society backing you. No badge. No budget. Just your wits, a few magical favours, and the knowledge that if you don’t deal with this, the party’s getting cancelled, the park’s going up in flames, or Bobby’s going to cry again.

    The world isn’t ending—yet. But it is very annoyed. And it’s knocking on your trailer door.

    So, tell me: What weirdness has wandered into Dusty Pines this time?
\end{WyrdSettingHeading}

\section[Welcome to the Dusty Pines Trailer Park]{Welcome to the Dusty Pines Trailer Park}

Dusty Pines is a supernatural sitcom-meets-mystery setting for one-shot play using \textit{The Wyrd Engine}. Each game takes place in a lightly cursed trailer park full of magical oddballs, mischievous gremlins, and eldritch lawn ornaments. This chapter describes how the setting works and how to run episodic play with maximum chaos and charm.

\subsection{Magical Mayhem}

Tucked on the edge of nowhere (and just slightly west of Weird), the \textbf{Dusty Pines Trailer Park} is a proudly ramshackle community where rust meets ritual and lawn ornaments occasionally talk back. Officially, it’s zoned as low-income housing. Unofficially, it’s a sanctuary for supernatural misfits, magical mishaps, and beings with nowhere else to go---or no intention of ever leaving.

Whether you’re a retired werewolf, a gremlin wrangler, or just someone who thinks neon flamingos bring good luck, Dusty Pines is home. The plumbing’s iffy, the gossip flows faster than electricity, and strange things happen every full moon. But the rent’s (usually) paid, and the community sticks together---especially when things get weird.

\section{How the Game Works}

Games set in Dusty Pines are short, episodic one-shots built around a single magical disaster, mystery, or community event gone spectacularly wrong. Each story centres the players as residents or long-term visitors---the sort of people who might not be normal, but are \textit{just} competent enough to solve whatever supernatural shenanigans have struck this week.

Scenarios start with a problem---gremlins in the wiring, a dragon in the quarry, or Edna the bog witch raising her curse prices---and escalate toward hilarity, chaos, and heartfelt resolution. Most episodes end with a party, a potluck, or a suspiciously smoking barbecue. Sometimes all three.

\subsection{Who Are the Characters?}

Characters in Dusty Pines are \textbf{permanent or semi-permanent residents} of the trailer park --- supernatural beings trying to lay low, mundanes who’ve adapted to weirdness, or oddballs who found a place where they finally fit in. They might be:

\begin{itemize}
  \item A disgraced fae prince hiding out in a trailer shaped like a giant mushroom.
  \item A gremlin mechanic who can’t stop ``upgrading'' the park’s plumbing.
  \item A nosy neighbour who’s totally normal (except for the haunted toaster).
  \item Or just the one person Bobby can count on when the port-a-potty grows legs.
\end{itemize}

Players should expect recurring roles, but each session can also introduce new oddballs or temporary guests---useful if you're rotating players.

\subsection{Why Do They Get Involved?}

There’s no secret society, no formal summons---just the raw, inevitable gravitational pull of supernatural nonsense. In Dusty Pines, weird finds \textit{you}. And once you’ve dealt with a cursed couch or a glamour-happy elf invasion, you’re on Bobby’s short list of “People Who Can Probably Handle This.”

Whether it’s guilt, habit, curiosity, or just wanting your home not to burn down again, you get involved. And hey, if you save the day, Bobby might even waive your rent. Maybe.

\section{Tone and Style}

Dusty Pines walks the line between \textbf{urban fantasy sitcom} and \textbf{magical misadventure}. The tone is comedic and heartfelt, but don’t be fooled---there's room for real emotion beneath the glitter, grease, and gnome wars. Episodes should feel like supernatural slice-of-life adventures, with a strong focus on:

\begin{itemize}
  \item Eccentric characters
  \item Magical nonsense
  \item Found family
  \item Creative problem-solving
  \item And the occasional exploding barbecue grill
\end{itemize}

\subsection{Skills in Dusty Pines}

Dusty Pines uses the standard skill list from \textit{The Wyrd Engine}, but some skills show up more often in backyard mayhem than they do in haunted manors or secret societies. Here’s a brief summary of the most commonly used skills in this setting:

\begin{itemize}
  \item \textbf{Charm} --- Winning people over, talking your way out of magical contracts, or hosting an impromptu potluck to calm tensions.
  
  \item \textbf{Crafts} --- Fixing trailers, rigging up magical defenses with duct tape, or baking a pie so good it disrupts a glamour.
  
  \item \textbf{Empathy} --- Understanding when someone’s cursed, emotionally possessed, or just badly in need of a casserole and a hug.
  
  \item \textbf{Grit} --- Staying calm during supernatural nonsense, holding your ground against fae royalty, or resisting swamp-induced hysteria.
  
  \item \textbf{Lore} --- Knowing which gnome is cursed, what phase the moon must be in to banish a couch spirit, and how to barter at the Faerie Flea Market.
  
  \item \textbf{Notice} --- Spotting magical residue, missing lawn ornaments, or when your neighbour has been glamoured into thinking he’s an elf king.
  
  \item \textbf{Provoke} --- Shouting down a sentient port-a-potty, intimidating enchanted furniture, or scaring off a bureaucratic vampire auditor.
  
  \item \textbf{Rapport} --- Calming down magical guests, negotiating with the bog witch, or keeping the community barbecue from descending into hex-fueled chaos.
  
  \item \textbf{Stealth} --- Sneaking past gremlins, hiding from the IRS, or tiptoeing past a sleeping dragon with a stolen grill in tow.
  
  \item \textbf{Will} --- Shaking off curses, resisting glamour, or telling Edna “no” and meaning it.
\end{itemize}

Other skills may show up depending on the story, but in Dusty Pines, emotional intelligence, magical know-how, and a high tolerance for absurdity go further than fists or firearms.



\subsection{Playing Comedy Games}

Unlike structured mysteries or dramatic campaigns, \textbf{comedy games thrive on looseness, improvisation, and player creativity}. In Dusty Pines, the goal isn't to carefully uncover a web of clues or solve a grim puzzle. Instead, the players are thrown into a chaotic magical situation and trusted to make a mess---and clean it up in their own wonderfully ridiculous way.

Each scenario typically revolves around a single central obstacle: a cursed appliance, a magical misunderstanding, or a supernatural guest who’s overstayed their welcome. Rather than guiding players toward a singular solution, these games provide a \textit{problem space}---a weird, unpredictable situation the players are free to wrangle however they see fit.

To keep momentum and encourage engagement, scenarios often include a handful of light \textbf{side-quests or distractions}: a neighbour who needs help, an NPC with conflicting goals, or a magical object behaving badly. These don't need resolution to "win" the game, but they provide flavour and opportunities for player-driven antics.

\textbf{Improvisation is key.} As a GM, you’re more of a comedy collaborator than an architect. Let the players suggest solutions, cause complications, and bounce off each other’s ideas. If someone wants to negotiate with a sentient barbecue grill, let them. If someone uses a glamour spell to win a lawn flamingo beauty pageant---that’s a highlight, not a derailment.

When in doubt, follow the funny. Let characters shine. Embrace magical nonsense. And remember: in Dusty Pines, it doesn’t have to make sense, it just has to be entertaining.


\subsection{Running a Session}

To run a Dusty Pines episode:

\begin{enumerate}
  \item Pick a central conflict or weird event (see the included scenarios).
  \item Introduce it early with a low-stakes opening scene.
  \item Let the players explore, negotiate, and improvise solutions.
  \item Throw in a few magical complications and local NPCs.
  \item End with resolution, usually at a park gathering or spontaneous celebration.
\end{enumerate}

The Dusty Pines format encourages narrative improvisation, community dynamics, and character-led hilarity. Keep the stakes personal, the weirdness high, and the humour flowing.

\subimport{./}{recurring-cast}
\subimport{./}{player-characters}

%% Scenarios
\subimport{trailer-park-of-the-arcane}{trailer-park-of-the-arcane}
\subimport{revenge-of-the-lawn-ornaments}{revenge-of-the-lawn-ornaments}
\subimport{curse-of-the-couch-potato}{curse-of-the-couch-potato}
\subimport{fangs-fur-and-a-fugitive}{fangs-fur-and-a-fugitive}



