\begin{WyrdScenarioHeading}[Revenge of the Lawn Ornaments]{Revenge of the\\ Lawn Ornaments}
    \label{scenario:revenge-lawn-ornaments}
    \index{Revenge of the Lawn Ornaments}
    \index{Scenario!Revenge of the Lawn Ornaments}

    It’s been one week since the infamous Dusty Pines 40th Anniversary Bash, and life in the trailer park is—briefly—quiet. Until someone sees a plastic flamingo peeking in their trailer window. Then a garden gnome trips Bobby with a hose. And Myrtle, the old mannequin by the community shed? She’s been spotted brandishing a plastic fork and rallying the ornaments like a general at war.

    It’s clear something—or someone—is enchanting the trailer park’s lawn décor. The players must investigate the cause, contain the mischief, and deal with Myrtle’s uprising before Dusty Pines is overrun by PVC patriots and resin renegades.

    \subsection*{Premise}
    Something is animating Dusty Pines’ lawn ornaments. The players must uncover the source, calm the chaos, and confront Myrtle the Mannequin before she leads a full-blown plastic rebellion. But first, they’ll need to win over some very attached residents—and maybe dodge a rake-wielding gnome or two.

    \subsection*{Goals for the Players}
    \begin{itemize}
        \item Investigate the source of the magical activity animating the ornaments.
        \item Contain the wandering and mischievous lawn décor.
        \item Track down and confront Myrtle the Mannequin, the apparent ringleader.
    \end{itemize}

    \subsection*{Atmosphere \& Tone}
    This scenario leans hard into surreal comedy and offbeat problem-solving. Let players approach the situation however they like—whether that’s laying magical traps, impersonating statues, or throwing a yard sale-themed ambush. The weirder, the better.

\end{WyrdScenarioHeading}

This scenario is about low-stakes magical chaos and how even the smallest enchanted troublemakers can cause a big stir. Encourage players to use strange tactics, get emotionally invested in weird objects, and have heartfelt conversations with plastic flamingos if necessary. The tone is lighthearted, but the mystery still gives the players something to untangle—with plenty of laughs along the way.

\subsection{Setup: Bobby’s Meltdown}

The game begins with the players being summoned—once again—to Bobby’s trailer, where he’s hiding under a beach umbrella and drinking coffee from a flower pot. He’s twitchy, tired, and convinced the gnomes are plotting something. (He’s not wrong.)

\begin{Example}{What Bobby Tells the Players}
    “They’re everywhere. The flamingos. The gnomes. Even Myrtle’s on the move. And I swear to all that is holy, someone stuck googly eyes on the propane tank.”
\end{Example}

\begin{itemize}
    \item Lawn ornaments have been moving around at night—spooking residents, hiding shoes, and prank-calling Bobby’s walkie-talkie.
    \item Bobby suspects magical contamination. He blames Edna’s “Self-Confidence Potion” or possibly a curse.
    \item Some residents are weirdly defensive of their now-mobile ornaments.
    \item Myrtle, an old mannequin Bobby was holding on to ``for sentimental reasons'' was last seen leading a group of gnomes into the toolshed while muttering about “liberation.”
\end{itemize}

Bobby pleads with the players to investigate quietly—he can’t handle another PR nightmare like the “enchanted picnic table incident of ’22.”

\begin{CommentBox}{Player Hook Options}
    Bobby offers the players:
    \begin{itemize}
        \item First dibs on any magical knick-knacks recovered.
        \item One week of “weirdness immunity” (no chores, no Edna errands).
        \item A coupon for “1 Free Potion of Your Choice (No Refunds).”
    \end{itemize}
\end{CommentBox}

\subsection{Investigating the Magical Source}

Something is clearly enchanting the lawn ornaments—but who (or what) is behind it? The players must begin by collecting clues, talking to the locals, and checking for residual magic. Suspicion falls immediately on Edna (as usual), but there are other possibilities… including a certain “Gnome Whisperer” who’s new in town.

\subsubsection{Clues and Leads}
\begin{itemize}
    \item \textbf{Edna’s Potions:} Edna has recently brewed a batch of “Self-Confidence Potions,” intended to empower her customers emotionally. However, a spill near the communal garden may have soaked into the lawn… and a few gnomes. She denies any connection but is acting suspicious and keeps buying more flowerpots.
    \item \textbf{Rusty the Gnome Whisperer:} Rusty claims he’s been “communing” with the ornaments during his evening meditations. He may have accidentally awakened them with a ritual involving moonlight, glitter, and gas station wine.
    \item \textbf{Magical Residue:} A \textbf{Lore} or \textbf{Notice} \DL{+1} check around the park reveals faint enchantment energy around the garden, lawn flamingos, and Myrtle’s old perch. The magic smells like confidence, sass, and potpourri.
    \item \textbf{The Spellbook Page:} A torn page from one of Edna’s old spellbooks can be found fluttering in a bush. It references “Animating Dormant Household Spirits” and includes scribbled notes about “plastic as an excellent vessel for misplaced ego.”
\end{itemize}

\subsubsection{How Players Can Proceed}
\begin{itemize}
    \item \textbf{Question suspects:} Interview Edna, Rusty, and even Bobby. Each has partial information (and wild theories).
    \item \textbf{Use magic detection:} A player using \textbf{Lore}, a magical trait, or a gadget may scan the lawn for active enchantments.
    \item \textbf{Stakeout at night:} Ornaments become more active after sundown. A nighttime watch might catch Myrtle or her minions mid-mischief.
    \item \textbf{Test the potion theory:} If players experiment with a diluted Self-Confidence Potion and a decorative item (e.g. a garden gnome or a plastic pumpkin), it begins to twitch after a few minutes.
\end{itemize}

\begin{CommentBox}{Twist Possibility}
If you want to mix things up, reveal that Myrtle was never enchanted—she just \emph{always} had a spark of life. The potion merely amplified her ambition.
\end{CommentBox}

\subsection{Investigating the Magical Source}

Something is clearly enchanting the lawn ornaments—but who (or what) is behind it? The players must begin by collecting clues, talking to the locals, and checking for residual magic. Suspicion falls immediately on Edna (as usual), but there are other possibilities… including a certain “Gnome Whisperer” who’s new in town.

\subsubsection{Clues and Leads}
\begin{itemize}
    \item \textbf{Edna’s Potions:} Edna has recently brewed a batch of “Self-Confidence Potions,” intended to empower her customers emotionally. However, a spill near the communal garden may have soaked into the lawn… and a few gnomes. She denies any connection but is acting suspicious and keeps buying more flowerpots.
    \item \textbf{Rusty the Gnome Whisperer:} Rusty claims he’s been “communing” with the ornaments during his evening meditations. He may have accidentally awakened them with a ritual involving moonlight, glitter, and gas station wine.
    \item \textbf{Magical Residue:} A \textbf{Lore} or \textbf{Notice} \DL{+1} check around the park reveals faint enchantment energy around the garden, lawn flamingos, and Myrtle’s old perch. The magic smells like confidence, sass, and potpourri.
    \item \textbf{The Spellbook Page:} A torn page from one of Edna’s old spellbooks can be found fluttering in a bush. It references “Animating Dormant Household Spirits” and includes scribbled notes about “plastic as an excellent vessel for misplaced ego.”
\end{itemize}

\subsubsection{How Players Can Proceed}
\begin{itemize}
    \item \textbf{Question suspects:} Interview Edna, Rusty, and even Bobby. Each has partial information (and wild theories).
    \item \textbf{Use magic detection:} A player using \textbf{Lore}, a magical trait, or a gadget may scan the lawn for active enchantments.
    \item \textbf{Stakeout at night:} Ornaments become more active after sundown. A nighttime watch might catch Myrtle or her minions mid-mischief.
    \item \textbf{Test the potion theory:} If players experiment with a diluted Self-Confidence Potion and a decorative item (e.g. a garden gnome or a plastic pumpkin), it begins to twitch after a few minutes.
\end{itemize}

\begin{CommentBox}{Twist Possibilities}
    \begin{itemize}
        \item If you want to mix things up, reveal that Myrtle was never enchanted—she just \emph{always} had a spark of life. The potion merely amplified her ambition.
        \item If the players delay too long, Myrtle starts leaving tiny, passive-aggressive notes on residents’ doorsteps: \emph{“We demand sun hats and respect.”}
    \end{itemize}
\end{CommentBox}


\subsubsection{Who’s Really Behind It (and How to Stop It)}

The true origin of the magical uprising is a combination of two things: \textbf{Edna’s spilled potion} and \textbf{Myrtle the Mannequin’s latent sentience}. The potion acted like a magical catalyst, awakening dormant personalities in plastic items throughout the park. But Myrtle? She didn’t need the potion—she just needed an excuse.

Once Myrtle saw the other ornaments gaining mobility, she took charge, rallying them with dramatic speeches and yard-sale charisma. She’s been subtly feeding them phrases like “liberation,” “independence,” and “viva la revolution!” from her perch near the toolshed.

\subsubsection{Stopping the Spread}
The players can take one or more of the following approaches:
\begin{itemize}
    \item \textbf{Neutralise the magical field:} A ritual can be performed at the potion spill site using salt, moon water, and a yard gnome’s tiny hat. This will stop new items from animating.
    \item \textbf{Confront Myrtle directly:} See the section on \emph{Finding the Ringleader}. Myrtle may be talked down, tricked, or diplomatically flattered into retirement.
    \item \textbf{Reverse the Potion’s Effects:} If the players can convince Edna to help, she can brew a “Self-Reflection Tonic” that un-boosts inflated egos—though it tastes like boiled pennies and shoe polish.
    \item \textbf{Appease the Ornaments:} An alternative route—if the players are feeling sentimental—is to organise an “Appreciation Parade” for the ornaments. If they feel seen and celebrated, they may calm down on their own.
\end{itemize}


\subsection{Finding the Ringleader}

At the heart of the lawn ornament chaos is a single plastic visionary: \textbf{Myrtle}, a slightly faded department store mannequin who once stood proudly outside the park’s communal toolshed in a sundress and sunhat. Myrtle has always had a spark of strange awareness, but the recent magical surge has awakened her ambition—and her grudge.

She now sees herself as the “Queen of the Yard Folk” and has rallied other animated ornaments under her banner. They call themselves “The Fellowship of Decorative Liberation,” and their demands include better visibility, equal treatment with human residents, and permanent sunscreen access.

\begin{NPC}[description={Plastic, Melodramatic, Revolutionary}]{Myrtle the Mannequin}

    Once a humble garden display in a second-hand sundress, Myrtle has evolved into the self-proclaimed leader of the “Fellowship of Decorative Liberation.” She delivers rousing speeches to lawn ornaments, dreams of founding her own micronation, and resents being replaced by a gargoyle birdbath. While she doesn’t want to hurt anyone, she \emph{does} want respect—and a bigger sun hat.

    \vspace{0.5\baselineskip}
    \begin{SkillsBox}
        \Skilled & \textbf{Provoke} \\
        \Novice  & \textbf{Will}, \textbf{Rapport}
    \end{SkillsBox}

    \begin{TraitsBox}
        \item[Plastic But Proud] Immune to most physical attacks (she’s hollow), giving her a \textbf{+2} to all defence rolls. However, her pride makes her prone to dramatic collapses when offended.
        \item[Charismatic Dictator] Can inspire other lawn ornaments to act on her behalf, gaining a \textbf{+1} when leading a swarm.
    \end{TraitsBox}

    \DamageBox[%
        totalfatigue=2,%
        totalmild=1,totalmoderate=0,totalsevere=0,%
    ]
\end{NPC}

\subsubsection{Tracking Myrtle Down}
Myrtle doesn’t stay in one place for long, but she has been seen:
\begin{itemize}
    \item Holding late-night “strategy sessions” near the toolshed.
    \item Giving speeches to gnome audiences from atop a milk crate stage.
    \item Leaving tiny motivational signs like \textbf{“No More Shelf Life!”} and \textbf{“Plastic is People Too!”}
\end{itemize}

Players can use:
\begin{itemize}
    \item \textbf{Notice} or \textbf{Stealth} to follow trails of glitter, plastic scuff marks, or tiny protest placards.
    \item \textbf{Rapport} or \textbf{Empathy} to persuade a gnome, flamingo, or confused lawn frog to reveal Myrtle’s whereabouts.
    \item \textbf{Contacts} or \textbf{Lore} to consult with Rusty the Gnome Whisperer, who believes Myrtle is “channeling the spirit of a forgotten store clerk from 1987.”
\end{itemize}

\subsubsection{Myrtle’s Motivation}
Myrtle isn’t evil—just melodramatic and deeply offended. She used to be Edna’s favorite lawn decoration until Edna replaced her with a gargoyle-shaped birdbath. Myrtle wants respect, admiration, and possibly a podcast.

A player using \textbf{Empathy} \DL{+1} may learn that Myrtle:
\begin{itemize}
    \item Feels abandoned and underappreciated.
    \item Sees herself as a leader of the “voiceless plastic masses.”
    \item Secretly just wants someone to say she looks fabulous in her new thrift-store blazer.
\end{itemize}

\subsubsection{Dealing with Myrtle}
Here are several ways to resolve the Myrtle situation:
\begin{itemize}
    \item \textbf{Appeal to Her Ego:} Compliment her leadership, promise her a float in the next Dusty Pines parade, and offer to let her write the park newsletter.
    \item \textbf{Negotiate a Retirement Deal:} Set her up with a new spot—perhaps in Edna’s garden, on a pedestal, with monthly tribute flowers.
    \item \textbf{Challenge Her to a Debate:} If players are feeling bold, they can publicly argue her vision in front of the other ornaments.
    \item \textbf{Use a Deactivation Tonic:} Brewed by Edna (grudgingly), this will sap Myrtle’s magical boost—but players must get close enough to apply it (e.g., splash it, bait her into drinking tea, soak it into a scarf).
\end{itemize}

\begin{CommentBox}{Escalation Option}
    If the players mock her or try to capture her by force, Myrtle may order a “full plastic uprising,” prompting a showdown with dozens of bitey gnomes and flailing flamingos.
\end{CommentBox}

\end{multicols}
\clearpage
\begin{multicols}{2}
\subsection{Containing the Lawn Ornaments}

Once the players understand what’s animating the lawn ornaments and who’s leading them, it’s time to deal with the roaming décor itself. Gnomes are blocking trailer doors. Flamingos are forming conga lines across the street. A ceramic frog has been hiding in Bobby’s sock drawer for three days. The players must find a way to safely gather up or neutralise these miniature menaces before they cause even more chaos.

\subsubsection{Ornament Behaviours}
Each group of ornaments acts differently:
\begin{itemize}\raggedright
    \item \textbf{Gnomes} operate in squads, stacking on each other to reach door handles, steal tools, and write graffiti like \emph{“Resin Will Rise.”}
    \item \textbf{Flamingos} are graceful and erratic, prone to performing impromptu interpretive dances and pecking at reflective surfaces.
    \item \textbf{Ceramic Frogs and Turtles} are slower but sneakier—often hiding in drawers or mailboxes and croaking ominously at night.
\end{itemize}

\subsubsection{Capture Methods}
Encourage player creativity, but here are a few options:
\begin{itemize}
    \item \textbf{Magic Circles:} Drawing salt or glitter circles around lawn ornaments can contain them temporarily.
    \item \textbf{Loud Music:} Blasting upbeat music (disco works best) disrupts their coordination and causes flamingos to freeze mid-pose.
    \item \textbf{The “Yard Sale Trap”:} Bait the ornaments with a fake yard sale full of mirrors, plastic party favours, or dollar-store tiaras.
    \item \textbf{Rusty’s Help:} If convinced to assist, Rusty the Gnome Whisperer can deliver a stirring speech that lulls gnomes into passive cooperation.
\end{itemize}

\subsubsection{Rehoming or Deactivation}
Once captured, the players can:
\begin{itemize}
    \item \textbf{Convince residents to let them go:} Especially if some have grown attached (see “Ornament Allies” complication).
    \item \textbf{Store them safely:} A magically shielded shed or box lined with expired Edna potions will keep them dormant.
    \item \textbf{Apply Reversal Magic:} Edna’s “Self-Reflection Tonic” can deactivate the enchantment (though side effects may include mild judgmental croaking).
\end{itemize}

\begin{CommentBox}{Escalation Option}
If containment fails, Myrtle may attempt a “midnight liberation raid” to free her followers. This can lead to a slapstick chase scene through the park under moonlight.
\end{CommentBox}

\begin{NPC}[description={Grumpy, Stabby, Collectible}]{Animated Gnomes}

    These plastic warriors may be knee-high, but they pack a lot of sass. Each gnome acts like a grumpy old man with a toolbelt and a mission. In groups, they’re bold, bitey, and just coordinated enough to be a problem.

    \textbf{Weakness:} Weak to being flipped on their backs or lured with snacks meant for raccoons.

    \vspace{0.5\baselineskip}
    \begin{SkillsBox}
        \Skilled & \textbf{Fight} \\
        \Novice  & \textbf{Stealth}, \textbf{Athletics}
    \end{SkillsBox}

    \begin{TraitsBox}
        \item[Stack 'n' Scramble] When gnomes act in a group of 3 or more, they gain +1 to Fight.
        \item[Garden Guerrilla Tactics] Can set crude traps (tripwires, sprinkler ambushes) with surprising success.
    \end{TraitsBox}

    \DamageBox[%
        totalfatigue=1,%
        totalmild=1,totalmoderate=0,totalsevere=0,%
    ]
\end{NPC}

\begin{NPC}[description={Elegant, Flightless, Unnervingly Stylish}]{Animated Flamingos}

    Once just ornamental distractions, these pink plastic icons now strut with purpose. They glide eerily across lawns, perform strange dances, and sometimes just stand and stare until someone trips. They are weirdly graceful and impossible to intimidate.

    \textbf{Weakness:} Confused by mirrors and obsessed with rhythmic music.

    \vspace{0.5\baselineskip}
    \begin{SkillsBox}
        \Skilled & \textbf{Athletics} \\
        \Novice  & \textbf{Stealth}, \textbf{Will}
    \end{SkillsBox}

    \begin{TraitsBox}
        \item[Choreographed Mayhem] Gain +1 to movement-based actions when moving in sync with other flamingos.
        \item[Plastic Vogue] Once per scene, may cause a distraction by striking an unreasonably fabulous pose.
    \end{TraitsBox}

    \DamageBox[%
        totalfatigue=1,%
        totalmild=0,totalmoderate=0,totalsevere=0,%
    ]
\end{NPC}


\end{multicols}
\clearpage
\begin{multicols}{2}

\subsection{Optional Complication: Tito vs. Gnarly Joe}

While most of the animated ornaments are mischievous but harmless, one gnome has taken things personally. \textbf{Gnarly Joe}—a chipped, battle-scarred garden gnome with a broken rake and a “NO HUMANS ALLOWED” belt—has developed a one-sided feud with Tito the Chupacabra. No one is entirely sure why.

Joe has been spotted stalking Tito’s trailer, leaving tiny traps in his path, and attempting to ride a remote-control truck into battle. Tito, for his part, has been uncharacteristically evasive, hiding behind trash cans and yowling defensively whenever anyone mentions “the gnome.”

\subsubsection{What’s Going On?}
It turns out Gnarly Joe was once stationed in front of the park’s dumpster—Tito’s favourite haunt—and sees the chupacabra as a territorial invader. The potion awakened Joe’s latent protector instinct, and now he’s taken it upon himself to “reclaim the bins for the plastic people.”

\subsubsection{Obstacles and Opportunities}
\begin{itemize}
    \item \textbf{Gnarly Joe’s Traps:} Players may need to disarm (or fall victim to) prank traps like upside-down buckets, tripwires, or glitter bombs in the recycling bins.
    \item \textbf{Tito’s Distress:} Tito may approach the players for help, leaving cryptic claw marks or dragging a lawn chair across the path as an SOS.
    \item \textbf{Night Ambushes:} Joe prefers night attacks. At least one player might be mistaken for Tito and ambushed by a gang of gnome commandos riding a plastic wagon.
\end{itemize}

\subsubsection{Resolving the Feud}
Players might:
\begin{itemize}\raggedright
    \item \textbf{Broker a Peace Treaty:} Convince Joe and Tito to divide territory and promise non-aggression, possibly with dramatic speeches and peace offerings (e.g. glitter-free trash access).
    \item \textbf{Redirect Joe’s Focus:} Give him a new mission—perhaps defending the park against “the real enemy” (e.g. squirrels, wind chimes, the HOA).
    \item \textbf{Stage a Duel:} Set up a ridiculous competition (e.g. a bin-decorating contest or raccoon-calling challenge) to settle the matter with flair.
    \item \textbf{Reassign Joe:} Trap him in a novelty snow globe and declare it his new outpost.
\end{itemize}

\begin{CommentBox}{Optional Twist}
If the players do nothing, Joe escalates into full war mode—building a cardboard fort and declaring Tito a “foe of the realm,” complete with a hand-drawn wanted poster.
\end{CommentBox}

\begin{NPC}[description={Scrappy, Tactical, Bent on Revenge}]{Gnarly Joe}

    Once just another forgotten gnome near the dumpster, Gnarly Joe has become a one-gnome militia thanks to the enchantment surge. With a chipped beard, a bent plastic rake, and a belt made of twist ties, Joe has declared a one-sided war on Tito the Chupacabra and any resident he deems a “bin trespasser.” His motto? “This lawn ain’t big enough for the both of us.”

    \vspace{0.5\baselineskip}
    \begin{SkillsBox}
        \Skilled & \textbf{Fight} \\
        \Novice  & \textbf{Stealth}, \textbf{Crafts} (for trap-making)
    \end{SkillsBox}

    \begin{TraitsBox}
        \item[Miniature Mayhem] Gnarly Joe counts as a mob of one. He rolls +1 to Fight when attacking from cover or with an improvised weapon (plastic rake, bottle cap shield, etc).
        \item[Trash Fort Commander] Once per scene, can deploy a crude defensive position from whatever debris is nearby. Provides +1 to defense rolls until dismantled or knocked over.
    \end{TraitsBox}

    \DamageBox[%
        totalfatigue=2,%
        totalmild=1,totalmoderate=0,totalsevere=0,%
    ]
\end{NPC}



\subsection{Optional Complication: Nighttime Shenanigans}

The enchanted ornaments of Dusty Pines may be odd by day, but after sunset, they become downright uncanny. Under moonlight, their movements grow smoother, their coordination improves, and some even develop strange magical quirks. Flamingos glide like dancers. Gnomes whisper battle plans. Myrtle’s silhouette appears in three places at once.

It’s not just mischief anymore—it’s performance art, rebellion, and low-grade magical chaos.

\subsubsection{What Happens at Night}
\begin{itemize}
    \item \textbf{Enhanced Mobility:} Ornaments move faster, farther, and more fluidly after dark. Some hover a few inches above the ground or leap from trailer to trailer.
    \item \textbf{Magical Buffs:} The lingering enchantment is stronger at night. Ornaments gain access to weird powers like:
    \begin{itemize}
        \item Temporary invisibility (gnomes only).
        \item Holographic duplication (Myrtle, obviously).
        \item Glitter teleportation (don’t ask).
    \end{itemize}
    \item \textbf{Bold Tactics:} Decorations rearrange park signage, relocate residents’ shoes to trees, and form eerie parades that march silently under the stars.
\end{itemize}

\subsubsection{Challenges for the Players}
\begin{itemize}
    \item \textbf{Nighttime Capture Risks:} Attempting to corral the ornaments after dusk raises the Difficulty Level of most actions by +1 due to enhanced ornament powers and overall spookiness.
    \item \textbf{Unsettling Sightings:} Players may witness a dozen gnomes saluting Myrtle from a rooftop, or a flamingo standing motionless in the middle of a streetlamp beam for hours.
    \item \textbf{Sleep Disruption:} If not addressed, nighttime activity may keep residents awake and cranky, increasing tension and lowering cooperation with the players.
\end{itemize}

\subsubsection{Dealing with Night Effects}
Players might try:
\begin{itemize}
    \item \textbf{Luring ornaments into traps before sunset}—or preparing daylight-based defenses like enchanted floodlights or disco-ball wards.
    \item \textbf{Suppressing the enchantment} during peak activity hours with dampening wards, potions, or rituals (Rusty the Gnome Whisperer (p. \pageref{npc:rusty-turnbuckle}) recommends “covering everything in glitter and howling at the moon,” but he’s not a reliable source).
    \item \textbf{Embracing the chaos} and using nighttime to strike dramatic bargains with Myrtle or catch her during a “state address” to her followers.
\end{itemize}

\begin{CommentBox}{Optional Scene}
    Stage a moonlit montage of ornament antics and player responses: a stealthy flamingo dash, a player falling into a trap while holding a net, a magical gnome jailbreak—and a triumphant, slow-motion capture moment set to elevator music.
\end{CommentBox}


\subsection{Final Act: Parade of Peace (or Pieces)}

With Myrtle confronted, the lawn ornaments contained (or embraced), and the magical disturbance resolved, the trailer park can finally begin to recover. Dusty Pines returns—mostly—to its usual, manageable level of weird. Bobby exhales for the first time in days. Edna mutters something about “fickle magic.” A lone flamingo salutes from a rooftop.

Whether the players solved the problem through charm, containment, chaos, or compromise, they’ve once again saved the day in their own wonderfully unconventional way.

\subsubsection{How It Might End}
Depending on the players’ choices, the scenario can conclude in a few different directions:
\begin{itemize}
    \item \textbf{Peaceful Resolution:} Myrtle accepts her role as Park Mascot, and the ornaments return to being mostly decorative—with occasional midnight dance sessions.
    \item \textbf{Heroic Capture:} The players trap Myrtle in a glitter-sealed Tupperware container, and the remaining decorations go dormant… for now.
    \item \textbf{Community Embrace:} The residents hold a “Lawn Appreciation Day” parade to celebrate their animated décor. Myrtle leads the march. Tito reluctantly plays tambourine.
    \item \textbf{Oops, It's Worse:} The magic isn’t fully dispelled, and now the garden hoses have started whispering. But that’s next week’s problem.
\end{itemize}

\subsubsection{Rewards and Fallout}
The players may receive:
\begin{itemize}
    \item A commemorative “Decor Defender” badge made from recycled gnome hats.
    \item A personal thank-you card from Myrtle (written in glitter glue).
    \item One (1) week of official Dusty Pines “No Weirdness” immunity, redeemable never.
    \item A favor from Edna—good for one potion of their choosing. Side effects not included.
\end{itemize}

