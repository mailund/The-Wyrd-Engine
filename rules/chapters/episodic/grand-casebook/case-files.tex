\newpage
\section[Case Files: The Scenarios]{Case Files: The Scenarios}

The following adventures are designed for 3–5 players and typically run between 2–4 hours.

\begin{multicols}{2}

\subsection{The Call to Adventure}

At the heart of every investigation lies the Grand Society of Inquiry, an esteemed and enigmatic organisation dedicated to the relentless pursuit of truth. Operating from the opulent halls of the Grand Hall, the Society employs a network of investigators, scholars, and specialists—each summoned based on their particular expertise.

When a new case arises, messages are discreetly dispatched via courier, pneumatic tube, or stranger means. These summons are determined by the \textbf{Grand Analytical Engine}, a vast, steam-powered machine housed in the Grand Hall’s lower levels. This device analyses a multitude of factors—past case data, personnel availability, skill profiles—and selects an ideal investigative team for each assignment.

\begin{CommentBox}{Framing The Call to Adventure}
    This framing device helps explain varying character rosters from session to session. The Grand Analytical Engine provides an in-universe reason for episodic play with a rotating cast.
\end{CommentBox}
\end{multicols}

\subsection{Cases}

The entries that follow are but a cog’s turn in the vast machinery of mystery—isolated case files drawn from the humming memory banks of the Grand Analytical Engine. Each stands alone, a puzzle wrapped in smoke and shadow, yet they may be engaged in any sequence your chronometers allow. The only constant is this: the investigators must be ready to heed the summons, wind the mainspring of curiosity, and brave the gears of the unknown.

\begin{description}\raggedright
    \item[The Murder at the Brass Orchid (page \pageref{scenario:murder-at-the-brass-orchid})] --- A locked room mystery in the prestigious \emph{Brass Orchid} cabaret. This scenario works both as a simple one-shot mystery and as part of the episodic setting. If you want to test if the type of mystery games in this setting is to your taste, this is a good starting point.
    
    \item[The Clockmaker's Deception (page \pageref{scenario:clockmakers-deception})] --- Is another murder mystery, but this time involving an automaton that may or may not be the culprit. The case thus draws the attention of the \emph{Aetheric Liberty Assembly} and the scenario can thus serve as an introduction to this faction of the setting.
    
    %% FIXME: Gears of Rebellion -- ALA
    
\end{description}



\newpage
\begin{multicols}{2}

\subimport{murder-at-the-brass-orchid}{murder-at-the-brass-orchid} % no setting ties :(
\subimport{clockmakers-deception}{clockmakers-deception} % The Aetheric Liberty Assembly
\subimport{the-silent-courier}{the-silent-courier} % ???

%% TODO: Add three more scenarios to round out the first volume of the Grand Casebook.