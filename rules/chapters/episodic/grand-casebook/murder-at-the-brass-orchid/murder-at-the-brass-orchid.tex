
\begin{WyrdScenarioHeading}{Murder at the Brass Orchid}
	\raggedright
	
	The investigators are called to \textbf{The Brass Orchid}. The establishment is filled with wealthy patrons, performers, and staff—each with their own secrets to hide. The club’s reputation is at stake, and the clock is ticking before the police arrive to sweep things under the rug.

	The players must piece together the events of the evening, question patrons and staff, analyse the crime scene, and determine who had the means, motive, and opportunity to commit the crime. However, the deeper they dig, the more they realise that this murder is just the tip of the iceberg.

	\subsection*{Premise} 
	A high-society soirée at the exclusive cabaret, The Brass Orchid, is cut short when a well-connected financier is found dead in a locked room. The party was attended by the city's elite, but none saw the murder happen—or so they claim. The investigators must navigate a world of secrets, deception, and hidden rivalries to uncover the truth.

	\subsection*{What Really Happened} 
	\textbf{Beatrice Langley}, a hostess at The Brass Orchid, killed the financier, \textbf{Edward Mercer}, to protect herself from blackmail. Mercer had uncovered details about Beatrice’s past life and was threatening to expose her unless she paid a steep price. Desperate and out of options, she poisoned his drink and used the club’s pneumatic tube system to dispose of the evidence. However, a miscalculation led to certain clues being left behind.
\end{WyrdScenarioHeading}

\begin{center}
	\includegraphics[width=.5\linewidth]{img/separt/detective-glass}
\end{center}

\begin{GmTips}
	The suggested passive opposition rolls in the following are only that, suggestions. Feel free to adjust the difficulty based on the investigators' actions, skills, and the pace of the game. Remember that the goal is to keep the story moving forward, not to bog it down with unnecessary obstacles.
\end{GmTips}
\newcolumn


\subsection{Act 0: Into the Fray}

At the Game Master’s discretion, the summons to the \textbf{Grand Hall} may be role-played, allowing players to experience firsthand how the \textbf{Grand Society of Inquiry} assigns cases and selects its agents. The Grand Hall, with its towering bookcases, softly ticking machines, and ever-present scent of aged parchment, serves as a fitting backdrop for such moments. A Society Official—impeccably dressed and radiating an air of quiet authority—steps forward to present the latest mystery: a locked-room murder at the prestigious \textbf{Brass Orchid}, a cabaret favoured by nobles, artists, and the elite. According to the report, the club’s owner, \textbf{Madame Yvette Duval}, contacted the Society in desperation, recognising that only the most capable investigators could unravel the enigma before her reputation—and her high-paying clientele—are irreparably damaged by scandal.

In episodic settings, the \textbf{Call to Adventure} often renders such introductory scenes optional, particularly when players are already invested in the campaign’s rhythm. However, in the first few sessions—when characters are new to the world and the tone is still being established—engaging in a scene outside the primary investigation can add richness and immersion. Receiving a case assignment is a natural opportunity to set the mood, introduce memorable NPCs, and reinforce the Society’s role in orchestrating these investigations, acting as both a guiding hand and an enigmatic presence behind the scenes. These moments can anchor the players in the setting, reminding them that every case is more than a puzzle—it is a mission, a responsibility, and a glimpse into the grand machinery of the world they now inhabit.

Once the players are gathered, the Society Official will provide a brief overview of the case, including the victim's identity, the circumstances surrounding the case. The Brass Orchid remains under lockdown, its golden doors barred to the public while the mystery remains unsolved. But such restrictions cannot last indefinitely. Its wealthy and influential patrons grow restless, and they will not tolerate confinement for long unless official investigators take charge. The pressure mounts: the players must reach the crime scene swiftly, before key witnesses slip away, memories fade, and vital evidence is lost beneath a veil of gossip, misdirection, or intentional sabotage.


\BeginBoxPage
	\begin{multicols}{2}
		\subimport{npcs}{beatrice-langley}
		\vspace{2\baselineskip}
		\subimport{npcs}{edward-mercer}
		\newcolumn
		\subimport{npcs}{madame-yvette-duval}
	\end{multicols}
	\WyrdFooterImage{img/pageart/gears-reverse-L}
\EndBoxPage

\subimport{./}{act-1}
\subimport{./}{act-2}
\subimport{./}{act-3}

\newcolumn
\subsection{Resolutions} 
Depending on how the investigators handle the case, different outcomes may occur:
\begin{itemize}
	\item \textbf{Justice Served}: Beatrice is arrested or confesses, ensuring the truth is revealed.
	\item \textbf{A Deal in the Shadows}: The investigators allow Beatrice to flee, leveraging her knowledge for future gain.
	\item \textbf{The Wrong Culprit}: A scapegoat is framed, or the authorities arrest someone else entirely.
	\item \textbf{A Mystery Unsolved}: The players fail to piece everything together, leaving The Brass Orchid haunted by unanswered questions.
\end{itemize}

Regardless of the resolution, this case's events ripple across London’s elite, setting the stage for future intrigues.
