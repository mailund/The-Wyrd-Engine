
\newcolumn
\subsection{Act 4: The Reveal}  

With all the pieces in place, the investigators must confront \textbf{Beatrice Langley}. She is visibly shaken when accused but clings to her innocence, insisting that she had \textbf{nothing to do with Mercer’s death}. However, as the investigators present their findings, cracks begin to show in her story.  

\begin{CommentBox}{Evidence That Breaks Her Resolve}  
	\begin{itemize}  
		\item \textbf{Traces of poison:} A broken glass vial, found near the pneumatic tube exit, contained the same poison that killed Mercer. Traces of the toxin linger on Beatrice’s clothing.  
		\item \textbf{Witness testimonies:} Multiple staff members recall Beatrice acting erratically—arriving shaken, disappearing after intermission, and returning only once the club was in an uproar.  
		\item \textbf{The torn letter:} Fragments of a document, partially burned in the dressing room stove, match the scrap found clutched in Mercer’s hand—evidence of a final desperate message. \textbf{Witnesses will testify} that Beatrice added fuel to the stove a short time before the murder scene was discovered.
		\item \textbf{The missing pocket watch:} Dropped in the servers’ area after she fled through the pneumatic tube; its location exposes her escape route.  
		\item \textbf{Inconsistencies in her alibi:} She initially claimed she was in her dressing room before and after her performance, but no one can confirm seeing her at the critical moment.  
	\end{itemize}  
\end{CommentBox}  

\noindent  
Faced with undeniable proof, Beatrice’s composure crumbles. If the investigators press her with a firm but measured approach, she may confess outright, revealing the truth about Mercer’s blackmail and the desperate decision that led to his death.  

However, if they push too aggressively or fail to secure a clear confession, Beatrice panics. She makes a break for the nearest exit—whether attempting to vanish into the crowd, lock herself in her dressing room, or even slip through the pneumatic tubes one last time. This could lead to a tense chase or a final dramatic confrontation as the investigators must decide whether to apprehend her themselves or alert the authorities before she disappears into the night.  

\begin{GmTips}  
	If you want to add tension, Beatrice’s flight can turn into a frantic pursuit through the back halls of the Brass Orchid, with obstacles such as locked doors, security guards, or even club patrons unwittingly getting in the way. A climactic moment could see her cornered on a balcony, deciding whether to surrender or make a desperate escape attempt.  
\end{GmTips}  