\begin{WyrdScenarioHeading}[The Nameless Moon]{The Nameless Moon}
	\label{scenario:the-nameless-moon}
	\index{The Nameless Moon}
    \index{Scenario!The Nameless Moon}
	
	A new moon has appeared above the City of Endless Dusk—one that has no name, casts no shadow, and was not there yesterday.

	When the phantom moon rose, the city's balance shattered. Beasts once thought extinct clawed their way through silverlit alleys. Sorcerers marked by old pacts foamed and transformed. Dreams twisted into prophecy. The players are summoned not to investigate, but to \textbf{eradicate}: a cult calling itself the \emph{Lunar Remnant} has reawakened an ancient ritual at the ruins of the skybound observatory, and unless the anchor holding the false moon is destroyed, it will become real—and permanent.

	\subsection*{Premise} 
	A silver cult has anchored a phantom moon over the city, unleashing monsters, unstable magic, and madness in its glow. The players must storm the observatory-fortress where the ritual continues, destroy the moon’s anchor, and survive long enough to escape the collapsing tower.

	\subsection*{Beneath the Moonlight}
	The ritual being cast at the observatory was first designed by the \textbf{Celestine Host}—a long-banished order of astrologer-priests who believed the gods were merely stars that could be named, chained, and bled.

	The ritual was never completed—until now. The \emph{Lunar Remnant} have recovered a fragment of the Celestine schema and rewritten it for violence. If it succeeds, the Nameless Moon will overwrite the real one… and those it touches may never be human again.
\end{WyrdScenarioHeading}


\begin{GmTips}
	This scenario moves fast. Begin with chaos, push into the assault, and end with an unstable magical finale. If the players hesitate, escalate the threat. The goal isn’t subtle tension—it’s pulpy, supernatural escalation with steel and fire.
\end{GmTips}

%\newcolumn

\subsection{Act I: Moonlight and Mayhem}

The game begins with the city already breaking down. The Nameless Moon has risen, casting a pale silver glow across the rooftops. Creatures once dormant are now awakened—scaled hounds with too many limbs, mist-born scavengers, and lunar-warped jackals. Citizens barricade themselves in doorways while scholars shout prophecies from balconies.

The players are summoned by \textbf{Nahema al-Zahir} (p. \pageref{npc:nahema-al-zahir}) who provides them with:

\begin{itemize}
  \item A map of the ruined \textbf{Sky-Fall Observatory}, now occupied by the Lunar Remnant cult.
  \item A description of the \textbf{Celestial Anchor}—a massive astrolabe relic that must be destroyed to break the moon's tether.
  \item A warning: time is short. The longer the moon remains, the more permanent its influence becomes.
\end{itemize}

The cult has sealed the roads. The players must either:

\begin{itemize}
  \item Fight their way through corrupted city streets swarming with moon-sick creatures.
  \item Traverse rooftops and sewer tunnels under threat of collapsing structures and unstable magic.
\end{itemize}

\subsubsection{Encounters}

As the players make their way through the silver-washed streets of the city, they’ll encounter creatures already warped by the phantom moon. These monsters serve as early threats, showcasing the unnatural chaos unleashed by the ritual. Each appears in a different kind of scene—allowing for variety in tone and terrain.

\paragraph{Lunar Spawn.}
These warped, jackal-like beasts move in packs and phase through debris to reach their prey. They strike during an ambush in a collapsed alleyway, where the players are forced to fight while navigating unstable footing and crumbling balconies. Their echo-howl may draw reinforcements if the players don’t finish them fast.

\paragraph{Silver-Bleed Myrmidon.}
A corrupted brute in jagged armour, encountered near a barricaded market street. It’s smashing through carts, statues, and fleeing civilians when the players spot it. Unlike the agile spawn, this thing is pure force. The fight becomes a brutal test of endurance—especially if the party tries to protect civilians.

\paragraph{Lunar Acolyte.}
The players encounter one of the cult’s lesser spellcasters while navigating a rooftop path toward the observatory. The Acolyte is mid-ritual, etching silver runes into a reflective disk. If interrupted, he summons unstable lunar energy that creates low gravity and floating debris. Use this fight to highlight the magical side of the threat—and hint that stronger cultists await.

\textbf{By the end of Act I}, the players should reach the base of the observatory tower—scarred by battle, wreathed in silver fire, and humming with cosmic resonance.


\begin{CommentBox}{Encounter Tips}
    Each encounter in Act I should push the players forward. Don’t overbalance toward lethal—this is the warm-up. Let them feel powerful, but not safe. Use shifting terrain, moonlight distortion, and cultic symbols to reinforce the supernatural atmosphere.
\end{CommentBox}



\subimport{./}{lunar-spawn.tex}
\subimport{./}{silver-bleed-myrmidon.tex}
\subimport{./}{lunar-acolyte.tex}



\subsection{Act II: The Tower of the Lunar Remnant}

The players arrive at the base of the Sky-Fall Observatory—an ancient star-temple turned fortress, now crowned in silver flame. The once-abandoned tower has been fortified by the cult, its entryway sealed by jagged silver sigils and guarded by warped zealots. Above, the Nameless Moon pulses like a heartbeat, looming directly overhead.

The goal is clear: reach the summit and destroy the \textbf{Celestial Anchor}, the arcane relic that tethers the phantom moon to this plane.

\textbf{The tower is a vertical gauntlet—each level stranger than the last.}

\subsubsection{Tower Structure}

\begin{itemize}
  \item \textbf{Ground Level – Entry Hold.}  
  Broken statuary, barricades of bone and scrap. Guarded by two \textbf{Silver-Bleed Myrmidons} and one \textbf{Lunar Acolyte} attempting a ritual. Players may storm the gate or climb through collapsed walls.

  \item \textbf{First Level – Hall of Echoing Light.}  
  Covered in polished mirrors etched with star-charts. Moonlight bends inside, creating illusions. Players must navigate a field of false images and celestial traps (e.g., beams that shift gravity).

  \item \textbf{Second Level – Chamber of Chains.}  
  A pit chamber lined with suspended silver chains that resonate with chanting. Here the cult holds victims being transformed by lunar rituals. If rescued, some may aid the players. The Lunar Acolyte leading the ritual may unleash dangerous spells like \textbf{Silver Burn} mid-combat.

  \item \textbf{Third Level – The Astrarium.}  \emph{(Boss Fight for Act II)}
  A grand, domed chamber where the stars once danced in alignment. Now cracked and malfunctioning, it creates unstable gravity pockets and unpredictable magical surges. This is the lair of the \textbf{Anchor Guardian}.

  \item \textbf{Summit – Anchor Platform.}  \emph{(Finale in Act III)}
  An open platform exposed to the sky, where the \textbf{Celestial Anchor} pulses with light. Cultists channel its energy skyward while the Nameless Moon grows sharper above. Destroying it will end the threat—but not without consequences.
\end{itemize}

\subsubsection{Tone and Momentum}
This act should feel like a desperate assault. Throw hazards at the players—mutated cultists, environmental instability, and magical warping. Let them feel powerful but tested. The deeper they go, the more distorted reality becomes.

\begin{CommentBox}{What the Anchor Is}
    The \textbf{Celestial Anchor} is an arcane astrolabe suspended in a levitating column of mirrored glass and moonstone. It draws energy from the false moon and feeds it back into the ritual circle below. It is not sentient, but it is protected—by magic, glyphs, and the \textbf{Anchor Guardian}.
\end{CommentBox}

\textbf{End of Act II:}
As the players reach the summit, they see the anchor glowing with unstable light. The ritual is nearly complete. One final defender remains between them and the artifact—something built from faith, silver, and starlight.


\subimport{./}{anchor-guardian.tex}



\subsection{Act III: Break the Moon}

The summit of the Sky-Fall Observatory is exposed to the heavens. Pillars of broken marble surround a raised platform where the \textbf{Celestial Anchor} spins slowly in place, a massive arcane astrolabe suspended between moonstone pylons. Silver light floods the chamber. The Nameless Moon hangs enormous overhead—sharper, closer, almost humming.

Around the platform, a final ring of cultists chant in rotating intervals, feeding energy into the Anchor. Their eyes glow. The ritual is nearly complete. If they are not stopped—and the Anchor not destroyed—the Nameless Moon will become a permanent fixture in the sky, remaking the world under its influence.

\subsubsection{Player Objectives}

\begin{itemize}
  \item \textbf{Disrupt or Defeat the Cultists.} Use force, stealth, or counter-magic to break the ritual circle. If left alone, the Anchor will become immune to physical damage within 3 rounds.
  
  \item \textbf{Reach the Anchor.} The Anchor is suspended on a raised disk surrounded by broken pathways, spinning fragments of platform, and occasional gravity surges. Getting to it may require Athletics, Leap, or a magical solution.

  \item \textbf{Destroy the Anchor.} It cannot be disabled by force alone—it requires either:
  \begin{itemize}
    \item A magical overload (e.g. \emph{Silver Burn}, \emph{Echo Binding}, or magical stress sacrifice),
    \item Physical destruction from a powerful artifact or relic weapon,
    \item Sabotage from within—requiring one player to enter the inner rings and risk exposure to raw moonlight.
  \end{itemize}
\end{itemize}

\subsubsection{The Anchor's Defenses}
\begin{itemize}
  \item Once the cultists fall, the Anchor begins to pulse erratically—casting waves of low gravity and magical interference (DL 2–3 to resist movement, suppression of spellcasting, etc.).
  \item The Anchor may “defend itself” by reanimating the bodies of fallen cultists or using a final surge of celestial energy as a last defense.
\end{itemize}

\subsubsection{Optional Twist}
If the players hesitate—or take too long—the Nameless Moon \textbf{partially manifests}. One player may see an eye open in the moon’s surface. The Anchor will begin to pull spells, relics, or even characters into orbit unless it is destroyed immediately.

\begin{CommentBox}{Final Moments}
    Destroying the Anchor ends the ritual. The Nameless Moon cracks, collapses into dust, or vanishes with a scream of displaced air.

    But nothing ends cleanly. The Anchor’s detonation may:
    \begin{itemize}
        \item Send a shockwave of moonlight across the city, permanently altering magic.
        \item Leave a curse on the one who struck the final blow.
        \item Tear open a rift to where the moon came from.
    \end{itemize}

    Choose one based on tone, or roll randomly to let fate decide. This is pulp sword \& sorcery: even victory should taste like dust and starlight.
\end{CommentBox}

