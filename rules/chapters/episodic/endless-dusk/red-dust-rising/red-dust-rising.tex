% !TeX root = ../../../../wyrd.tex

\begin{WyrdScenarioHeading}[Red Dust Rising]{Red Dust Rising}
	\label{scenario:red-dust-rising}
	\index{Red Dust Rising}
    \index{Scenario!Red Dust Rising}

	A desert storm has uncovered part of an ancient war machine buried beneath the Sable Plateau—an arcane siege engine from the age of the Bronze Dynasty. The site is unstable. Internal systems are starting to reawaken. If someone powers it up or breaks the wrong seal, it could march again.

	The players are hired to investigate the breach and shut the construct down before anyone else tries to claim it.

	\subsection*{Premise}
	A partially buried Bronze-era war machine has been exposed. The players must enter the ruin, assess the damage, and either secure or destroy the command core before it reactivates.

	\subsection*{The Machine Beneath the Sand}
	The construct was a mobile fortress—powered by alchemical engines, protected by wards, and guided by a crystalline command pillar. It was sealed after the dynasty’s fall. Now, with the storm-breach exposed, low-power systems are starting to stir.

	\subsection*{The Hook}
	Three days ago, a scout returned from the eastern dunes with a sand-cracked map and burns on his hands. He died an hour later, saying only: \emph{“It still hums.”}

	Now the players are hired by The Saphire Concoord to reach the ruin and shut it down. Payment is generous. Bonus if the core is removed or destroyed. The players are given:
	\begin{itemize}
	    \item A map to the breach site.
	    \item A sketch of the command pillar believed to house the engine’s control systems.
	    \item A supply token redeemable for basic climbing, light, or demolition gear.
	\end{itemize}

    The players have time to buy supplies, gather information, and prepare for the journey. They depart before dawn. Ahead lies stone, metal, and something old enough to be angry.
\end{WyrdScenarioHeading}


\subsection{Act I: The Breach}

The players arrive at the base of a red-stone mesa on the edge of the Sable Plateau. The wind has scoured the upper layers clean, exposing the ribbed plating of something ancient—metal and stone fused into a dome-like curve. A section has cracked open, revealing a narrow breach leading inside.

Scorch marks ring the entrance. The sand is glassy near the edges. Old bronze wards are half-dissolved, and the air smells like blood and ash.

\subsubsection*{Objectives}
\begin{itemize}
    \item Explore the exposed breach and gain access to the machine interior.
    \item Avoid or neutralise unstable systems near the entrance.
    \item Establish that the machine’s power systems are partially online.
\end{itemize}

\subsubsection*{Environmental Hazards}
\begin{itemize}
    \item \textbf{Alchemical Fumes:} A ruptured pipe near the breach releases bursts of heated vapour. Passing through without caution requires a DR 2 \textbf{Survival (Desert)} or \textbf{Diagnosis} roll to avoid 1 Fatigue.
    
    \item \textbf{Warded Paneling:} The inner doorway is blocked by partially intact arcane seals. Touching them without care triggers a static discharge (DR 2 \textbf{Thaumatology} or take 1 Wound).

    \item \textbf{Collapsed Support Beam:} The narrow entry corridor ends in a fallen slab of stone-and-metal plating. DR 2 \textbf{Climbing} or \textbf{Engineering} to bypass or shift safely. Otherwise, someone gets pinned (1 Fatigue and requires help).
\end{itemize}


\subsubsection*{First Encounter: Ash-Crawler}

As the players move through the collapsed corridor, stepping into the flickering half-light of the breached chamber, they disturb a panel of debris—causing a nearby maintenance hatch to slide open with a hiss of stale air and grinding metal.

From the shaft scuttles a small, spiderlike construct made of oxidised bronze and cracked plating. Its legs clatter across the floor like dry bones, and its eye-runes pulse erratically. Designed long ago to clear battlefield rubble, the construct now treats all movement as a threat to be excised.

\subimport{./}{ash-crawler.tex}


\subsubsection*{Progression}

Once inside, the players find a functional stairwell or lift-shaft leading deeper into the machine. Faint vibrations echo upward. Somewhere below, gears are turning.

At the end of this act, the players are inside the machine, past the initial hazard layer, and preparing to move into the unstable core zones.

\begin{CommentBox}{Tone and Pressure}
    This act should feel tense but contained. The players are alone, the environment is dangerous, and the construct is waking up slowly. Use creaking metal, flickering lights, and distant thuds to keep the pressure rising.
\end{CommentBox}

\subsection{Act II: Into the Inner Gears}

Past the breach, the machine descends into a layered interior of rusted catwalks, segmented chambers, and massive gear-wells. Pipes hum with alchemical pressure. Arcane seals flicker to life as the players move deeper. The construct is stirring.

The air smells of old oil and burnt copper. In places, the walls breathe heat. The players are not alone.

\subsubsection*{Objectives}
\begin{itemize}
    \item Navigate the interior access passages and reach the command core chamber.
    \item Bypass hazards, traps, or partial reactivation events.
    \item Survive a second major encounter with a defensive automaton.
\end{itemize}





\subsubsection*{Navigational Obstacles}
\begin{itemize}
    \item \textbf{Collapsed Bridge:} A critical gantry has fallen. Players must climb or leap across. DR 2 \textbf{Climbing} or \textbf{Jumping}. Failure means a 2-storey drop (1 Wound unless mitigated).

    \item \textbf{Pressure Burst:} A vented pipe hisses alchemical steam. DR 2 \textbf{Survival (Desert)} or \textbf{Engineering} to time safe passage. Failure causes 1 Fatigue from heat exposure.

    \item \textbf{Magnetic Pulse Field:} A trap triggers in a hallway lined with bronze plates. Metal weapons and gear drag toward the walls. DR 2 \textbf{Will} to move normally. Dropping metal gear avoids penalty.
\end{itemize}

\subsubsection*{Major Encounter: Geargrinder Sentinel}

Near the midpoint of the machine, the players cross a platform flanked by rotating gears and a rusted security pillar. As they pass, a large automaton unfolds from the wall—half-statue, half-wrecker.

It does not issue warnings. It attacks immediately.

\subimport{./}{geargrinder-sentinel.tex}

\subsubsection*{Progression}

Once defeated, the players locate a stairwell leading to the command core chamber. The stairs are partially collapsed, but navigable. Arcane light flickers below. They can hear something breathing through the walls—or the machine itself. The final chamber waits.



\subsection{Act III: The Control Core}

The players descend into the heart of the machine. The walls are curved with bronze and crystal inlay, pulsing faintly. Arc-light flickers through the seams. The floor vibrates with a deep hum.

At the centre of a sunken control chamber stands a crystalline pillar the size of a tree trunk. It emits a rhythmic pulse of light and low sound—like a heartbeat. Cables extend from it in every direction, and a ring of bronze sigils encircles the platform beneath it.

The command system is not dormant. It is waiting.

\subsubsection*{Objectives}
\begin{itemize}
    \item Confront the core and decide how to neutralise it.
    \item Survive the guardian bound to the chamber.
    \item Determine if the players will disable, destroy, or interact with the machine.
\end{itemize}

\subsubsection*{The Anchor Circle}

To reach the core, one player must cross the anchor ring—a narrow metal path inscribed with shifting glyphs. Anyone stepping inside must pass DR 2 \textbf{Thaumatology} or take 1 Fatigue from arcane interference.

Breaking the circle (e.g., scratching or chipping the ring) suppresses this effect but triggers a backlash: the guardian activates immediately.

\subsubsection*{Boss Encounter: Corebound Guardian}

The machine has a final failsafe: a construct of forged bronze and flickering runes bound to the control pillar. It does not move until the circle is broken or a player touches the interface.

\subimport{./}{corebound-guardian.tex}

\subsubsection*{After the Fight}

Once the Corebound Guardian is destroyed, the control chamber falls eerily quiet. The glyphs dim. The anchor ring cracks. For a moment, the machine seems dormant.

But as the players approach the command core, a low rumble vibrates through the floor. The central crystal flashes once—deep red—and the walls shift slightly. Something deeper in the construct begins to wake.

\subsubsection*{Interaction with the Core}

The players may attempt one of the following:

\begin{itemize}
    \item \textbf{Disable the Core:} Requires DR 3 \textbf{Engineering} or \textbf{Thaumatology}. On success, the control crystal dims but does not shut off entirely. There’s still energy flowing from below.
    
    \item \textbf{Damage the Core:} With force or a salvaged energy cell, the crystal can be fractured—causing it to leak arcane light and trigger an overload deeper in the machine. This causes structural shifts and may unlock lower levels.

    \item \textbf{Interfere with the Core:} A player may attempt to issue a command with DR 3 \textbf{Will} and DR 3 \textbf{Thaumatology}. If successful, they receive a half-coherent burst of ancient commands—some of which are not meant for mortals. They gain insight into a secondary chamber: the engine control vault.
\end{itemize}



\subsubsection*{Setting Up Act IV}

The Guardian’s body cracks and slumps to the floor. The light in the crystal core gutters like a dying flame—but it does not go out.

Instead, it begins to pulse. Slowly. Deliberately. Not like a machine—but like something alive.

Strange heat rises from the stone beneath the players' feet. Dust vibrates in thin lines across the floor. And then a sound—not a voice, not a word, but a \emph{presence}—pushes into the chamber. It feels old. Heavy. Watching.

The core flares once with red-gold light. Then the floor shudders and splits along hidden seams. A ring of stone grinds open to reveal a stairwell descending into darkness.

A dry wind escapes from below, thick with the smell of ash and copper. It carries the faint sound of chanting, though no voices are present.

\textbf{Whatever the players shut down here, it was only the surface. Something older stirs below. The final act begins.}





\subsection{Act IV: The Heart of the Machine}

The stairs descend into darkness—carved stone lined with ancient copper and bone-dry runes. The air grows hotter. The light dims to red. Sparks flicker from cracks in the wall like fireflies.

At the bottom: a vast chamber hollowed into the mesa itself. The walls pulse with veins of alchemical ore. Bronze pistons rest dormant in cradles of stone. At the centre lies the true heart of the war machine: a sunken altar, half-mechanical, half-flesh, wrapped in chains of blackened gold.

Around it, four ruined figures kneel in eternal prayer. Their bodies are mummified but twitch with each pulse of energy. These are the \textbf{Ashbound Vessels}—last remnants of the dynasty that fed the engine their will.

\subsubsection*{Objectives}
\begin{itemize}
    \item Face the Ashbound Vessels and disrupt their connection to the engine heart.
    \item Decide whether to destroy, bind, or claim the ancient core.
    \item Escape the ruin as it begins to collapse—or awaken.
\end{itemize}

\subsubsection*{Environmental Hazards}
\begin{itemize}
    \item \textbf{Molten Channels:} Alchemical fluid flows in open cuts across the floor. Stepping in causes 1 Wound. A successful DR 2 \textbf{Jumping} or \textbf{Acrobatics (if allowed)} avoids damage. Players can use debris or magic to cross safely.

    \item \textbf{Pulse Waves:} Every few minutes, the heart releases a wave of heat and force. All characters must pass DR 2 \textbf{Will} or lose their next minor action as they stagger or choke.

    \item \textbf{Psychic Pressure:} Any magical character who attempts to draw power here must pass DR 2 \textbf{Thaumatology} or take 1 Fatigue from backlash.
\end{itemize}

\subsubsection*{Final Encounter: Ashbound Vessels}

The four kneeling figures rise when the players approach the heart. Each one bears remnants of royal garb fused to twisted war-plate. Their eyes glow with alchemical fire. They do not speak. They defend the altar until destroyed.


\begin{CommentBox}{Alternate Ending Hooks}
    Depending on the players’ choices, the finale can end in several ways:
    \begin{itemize}
        \item \textbf{Destruction:} Shattering the altar causes a chain reaction. The heart implodes in fire and dust. Players must escape collapsing tunnels.
        
        \item \textbf{Suppression:} A complex ritual (DR 3 \textbf{Thaumatology}) may seal the heart for a generation. Failure causes backlash (1 Wound to all participants).
        
        \item \textbf{Possession:} A player who touches the heart directly may try to control it (DR 4 \textbf{Will}). Success grants power—but marks them. Others will come seeking it.
    \end{itemize}
\end{CommentBox}


\begin{NPC}[%
    name=Ashbound Vessel,%
    description=Dynastic Guardian of the Core%
  ]{Ashbound Vessel}
  
    \emph{A mummified remnant of the Bronze Dynasty, wrapped in gold-laced war robes and bound to the heart by ritual flame. It strikes with unnatural precision and burns with seething memory.}
  
    \subsubsection*{Encounter Details:}
    The Ashbound Vessel rises from its kneeling stance as the players enter the chamber. It fights silently, heedless of pain, and never retreats. When one is destroyed, the next awakens with greater urgency.

    \begin{SkillsBox}
        \Skilled & Brawling,  \\
        \Novice & Tactics \\
    \end{SkillsBox}

    \begin{TraitsBox}
      \item[Burning Wrath] — When a Vessel is defeated, the remaining Vessels increase their Brawling skill by +1.
      \item[Unbroken Vigil] — Gains +2 to Defense while adjacent to the Heart of the Machine.
    \end{TraitsBox}
  
    \DamageBox[totalfatigue=1,totalmild=2,totalmoderate=2,totalsevere=1]%

\end{NPC}


\subsubsection*{Ending the Scenario}

Once the heart is dealt with, the ruin begins to fall quiet—or collapse. The players return to the surface through choking dust and flickering red light.

\begin{itemize}
    \item If the heart is destroyed, the machine dies with a final exhale of heat.
    \item If bound, its glow dims—but its presence lingers in the stone.
    \item If claimed, it beats slower—but louder.
\end{itemize}

\textbf{The war engine sleeps. For now.}


% formatting bottom of the section
\end{multicols}
\clearpage
\begin{multicols}{2}
    