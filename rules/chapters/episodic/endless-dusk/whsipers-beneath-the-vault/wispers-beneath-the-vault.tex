% !TeX root = ../../../../wyrd.tex


\begin{WyrdScenarioHeading}[Whispers Beneath the Vault]{Whispers Beneath the Vault}
	\label{scenario:whispers-beneath-the-vault}
	\index{Whispers Beneath the Vault}
    \index{Scenario!Whispers Beneath the Vault}

	A private reliquary beneath the merchant quarter has gone silent. Built as a sealed archive for dangerous relics, the vault is rumoured to house cursed texts, enchanted chains, and forbidden treasures. When its alarms failed, two guards entered. Neither returned.

	Now, something whispers through the stone above. Locks twist open of their own accord. Lamps burn with no flame. And worse still, a brass-bound mirror—once sealed by five rites—has shattered.

	The players are hired by \textbf{Nahema al-Zahir} — Grand Envoy of the Sapphire Concord, to enter the vault, restore containment, and identify the source of the disturbance.

	\subsection*{Premise}
	A merchant vault beneath the city has been breached. Relics stir. Spirits whisper. A bound Djinn may be free. The players must descend, explore, and decide whether to restore the seals, destroy the contents—or make a bargain with something that remembers its freedom.

	\subsection*{The Djinn}
	The vault was constructed atop a binding circle that once held the Djinn \textbf{Haqar al-Sirr}, a spirit of sand, glass, and memory. The circle has been fractured, but not broken. Haqar is still bound—but barely. If restored, he may return to slumber. If released, he may grant power... or raze the city in gratitude.
\end{WyrdScenarioHeading}


\subsection*{The Briefing from Nahema}

The players are summoned to a shadowed chamber in the Concord’s lower offices. Nahema al-Zahir waits behind a desk of bone-inlaid teak, reading from a folded parchment sealed with five broken glyphs.

She does not rise.

\begin{quote}
    “Vault Seventeen was never meant to be opened again. It held nothing worth recovering—only things that couldn’t be destroyed. Now the wards are failing. The locks are breaking. And the whispering has started again.”
\end{quote}

\subsubsection*{What the Players Are Told}

\begin{itemize}
    \item The vault is located beneath the merchant quarter, five levels below the surface, hidden behind a false storeroom managed by the \textbf{Golden Charter of Locks and Ledgers}.
    
    \item Three days ago, a reliquary alarm went dark. Since then, one inspector and two guards have entered. None returned. Magical scrying has failed. Something is actively warding the space from outside intrusion.
    
    \item The vault was built to contain multiple unstable relics—but the most dangerous was the mirror that housed a bound Djinn: \textbf{Haqar al-Sirr}, sealed in glass by five rites after the Amber Revolt.
    
    \item The mirror has reportedly shattered. If Haqar has escaped—or is awakening—the vault may become unrecoverable.
    
    \item The Concord will pay well for confirmation, re-sealing, or destruction of the threat. Nahema does not want the Djinn freed or traded—especially not to rivals.
\end{itemize}

\subsubsection*{What They Are Given}

\begin{itemize}
    \item A sketched map of Vault 17’s entry corridors and containment layout.
    \item A shard of warded bronze that will glow in proximity to unstable relics.
    \item A Concord token for sealed passage through the city underworks.
\end{itemize}

\textbf{Nahema’s Warning:} “If you find the Djinn’s voice in your head, do not answer it. And if you do—lie. He will know anyway, but you might buy yourself a moment.”


\subsection{Act I: Descent into Vault 17}

The entrance is hidden behind a false storeroom beneath the Charter’s counting house. At the appointed hour, the players are let through by a pale clerk who refuses to descend the stair and locks the hatch behind them.

A stone passage leads downward through narrowing air. It grows warm. The light fades. At the base stands a heavy obsidian door inscribed with five glyphs. One glyph is blackened and cracked through the middle.

\subsubsection*{Objectives}
\begin{itemize}
    \item Breach the outer door and enter the vault’s sealed zone.
    \item Discover what became of the inspection team.
    \item Encounter the first corrupted guardian animated by Djinn energy.
\end{itemize}

\subsubsection*{Initial Environment}

The upper vault is built like a reliquary-temple: stone walls, arched chambers, and cold-burning sconces. Thick dust lines the halls, but a few recent signs break the stillness:

\begin{itemize}
    \item A shattered brass lamp embedded deep into a sandstone wall, as if hurled from within.
    \item Smeared blood across a ritual sealing circle, still faintly warm to the touch.
    \item A pair of burned leather boots beside an opened reliquary cabinet. No body remains.
\end{itemize}

\subsubsection*{First Encounter: Vault Wraith}

Beyond the relic chamber, the players are confronted by the remains of one of the missing guards—now animated by fractured Djinn essence. The body jerks unnaturally. Its eyes are black mirrors. It repeats, in a hoarse whisper:  
\textit{“This door stays sealed.”}

\begin{NPC}[%
    name=Vault Wraith,%
    description=Djinn-Touched Remnant%
  ]{Vault Wraith}
  
    \emph{The warped remnant of a vault guardian, sustained by the leaking power of Haqar al-Sirr. It retains only a single purpose—and will destroy anything that threatens it.}
  
    \begin{SkillsBox}
        \Expert & Brawling \\
        \Skilled & Stealth, Intimidation \\
        \Novice & Will, Thaumatology \\
    \end{SkillsBox}

    \begin{TraitsBox}
      \item[Fractured Flesh] — Takes only 1 Stress from any one physical attack unless they are magical or fire-based.
    \end{TraitsBox}
  
    \DamageBox[totalfatigue=3,totalmild=3,totalmoderate=2,totalsevere=1]%

\end{NPC}

When the Vault Wraith is defeated, its collapse disrupts the nearby wardline. Arcane runes along the chamber walls begin to flicker. They glow a dull red and emit a rising hum.

This triggers a flare of dormant glyphwork along the walls and ceiling. A wave of sudden heat and force floods the room:

\begin{quote}
    All characters within melee range must pass DR 2 \textbf{Athletics} or be knocked backward. Anyone who fails suffers 1 point of Stress from the impact.
\end{quote}

The flare dies as quickly as it appeared, leaving only scorched stone—and a faint smell of salt and burned paper.

\subsubsection*{Progression}

Once the Wraith is defeated, the players find the threshold behind it partially scorched—its seals broken not from outside, but from within. The stone beneath their feet pulses faintly with dry heat.

If a player succeeds on a DR 2 \textbf{Thaumatology} or \textbf{Notice} check, they detect residual arcane threads leading deeper—drawn toward a now-inactive seal. If no one succeeds, the vault itself shifts:

\begin{quote}
    A dry, whispering voice speaks through the walls: \emph{“Deeper. This was never the end.”} 
\end{quote}

Stone shifts. A seam opens where none should exist.


\textbf{The lower vault is awake.}


\subsection{Act II: Relics of Ruin}

Below the upper chambers lies the reliquary proper—a long, hexagonal corridor branching into sealed alcoves and collapsing vaults. This level once housed the most dangerous relics—now many chains lie broken, and arcane seals flicker in and out of focus.

The stone itself feels dry and hot to the touch. Whispers stir in the dust.

\subsubsection*{Objectives}
\begin{itemize}
    \item Navigate the relic halls and assess magical breaches.
    \item Find glyphs and verbal cues for the containment ritual in Act III.
    \item Survive or bypass arcane hazards, then face the vault guardians.
\end{itemize}

\subsubsection*{Vault Hazards and Clues}

\paragraph{1. Glyphfire Trap}
\textit{A cracked brass urn lies tipped in one alcove, alchemical ink pooled and dried across the floor. Above the arch, a faintly glowing glyph pulses red.}

\begin{description}
    \item[Trigger:] Entering the alcove without a DR 2 \textbf{Observation} or \textbf{Thaumatology} check.
    \item[Effect:] All in the alcove must pass DR 2 \textbf{Dodge} or suffer 1 Fatigue from a burst of glyphfire from the ceiling.
\end{description}

\begin{CommentBox}{Ritual Clue 1: The Glyph of Restraint}
    Players who examine the glyph (DR 2 \textbf{Thaumatology}) may copy the symbol into a journal or spellbook. This is the first anchor glyph required to complete the ritual in Act III.
\end{CommentBox}

\paragraph{2. Unmoored Relic – Crown of Discord}
\textit{A cracked iron circlet hangs by silver wire. It hums faintly, and anyone within Close range feels a tightening behind the eyes.}

\begin{description}
    \item[Trigger:] Approaching within Close range.
    \item[Effect:] DR 3 \textbf{Will}. On failure, the character experiences a vivid false memory of betrayal by another PC. May result in brief confrontation.
\end{description}

\begin{CommentBox}{Roleplay Tool Only}
    This relic is not part of the core ritual but introduces unease. If players study the glyphs etched along the base, a DR 2 \textbf{Occultism} check reveals the symbol of \textit{Zuhahir}, the name Haqar responds to.
\end{CommentBox}

\paragraph{3. Whisper-Split Hall}
\textit{A long chamber with tall mirrored walls—now fractured and fogged with ash. A central pedestal lies broken, surrounded by glass dust and scorched runes.}

\begin{description}
    \item[Effect:] Characters hear their own voices echoing back out of order. A DR 2 \textbf{Occultism} or \textbf{Arcane Lore} check reveals this room was used for Djinn invocation—the pedestal base still bears the full glyph \textbf{Zuhahir}.
\end{description}

\begin{CommentBox}{Ritual Clue 2: The Name of Binding}
    Players must speak or write the glyph \textit{Zuhahir} during Act III. This is the name Haqar was bound by—and is the second anchor of the sealing rite.
\end{CommentBox}

\subsubsection*{Magic Interference: Djinn Influence}

From this point forward, any spell cast in the vault causes warping.

\begin{description}
    \item[Effect:] The caster hears a whispered response from Haqar. Their spell flickers, but still functions. If the caster speaks \textit{Zuhahir} aloud, the pedestal briefly flashes with heat—this reaction is noted in Act III.
\end{description}

\subsubsection*{Guardian Encounter: The Shardbound Sentinels}

As the players descend toward the final vault, they pass between four tall mirrored shards. A low hum builds. Dust rises. Then the guardians step free—mirrored forms of dust and refracted light.

\textbf{These constructs do not speak. They strike spellcasters and relic-bearers first.}

\subsubsection*{Terrain Feature: Mirror Pathway}

\begin{itemize}
    \item \textbf{Reflected Attacks:} Ranged spells have a 50/50 chance of bouncing unless the caster succeeds DR 2 \textbf{Thaumatology}.
    \item \textbf{Illusion Cloak:} Standing between two mirrors gives +1 DR to all ranged attacks.
    \item \textbf{Shatter Pulse:} Breaking a mirror triggers a pulse. DR 2 \textbf{Dodge} or take 1 Fatigue.
\end{itemize}

\begin{NPC}[%
    name=Shardbound Sentinel,%
    description=Vault-Forged Guardian%
  ]{Shardbound Sentinel}
  
    \emph{Forged of fractured light and dust, these mirrored constructs strike in perfect rhythm.}
  
    \begin{SkillsBox}
        \Expert & Broadsword \\
        \Skilled & Tactics \\
        \Novice & Will \\
    \end{SkillsBox}

    \begin{TraitsBox}
      \item[Mirrorborn] — +2 to Defence when flanked by active mirrored shards.
      \item[Blade of Light] — Broadsword attacks deal +1 Stress against targets wearing metal armour.
    \end{TraitsBox}
  
    \DamageBox[totalfatigue=2,totalmild=2,totalmoderate=2,totalsevere=2]%

\end{NPC}

\subsubsection*{Vault Rewards}

\begin{description}
    \item[Shard-Fragment Pendant] — +2 to Will once per scene when resisting illusion or mental coercion. Flickers in the presence of lies.
    \item[Mirror-Wrought Buckler] — +2 to Dodge against the first magical attack each scene. May reflect a missed spell once per session.
    \item[Fractured Sentinel Blade] — +2 to Broadsword when duelling a single opponent. Ignores 1 point of magical resistance once per round.
\end{description}

\subsubsection*{End of Act II}

Behind the cracked pedestal is a melted wall still warm to the touch. Etched into the stone is the glyph \textbf{Zuhahir}.

\textbf{Trigger:}  
Touching the glyph while carrying the Shard-Fragment Pendant, or speaking the name aloud, causes the wall to open, revealing a heat-slick stair descending deeper.

\begin{CommentBox}{GM Note – Unlocking Act III}
    Players must either:
    \begin{itemize}
        \item Speak the name “Zuhahir” aloud in this act, or
        \item Carry the pendant and touch the glyph,
    \end{itemize}
    to proceed into Act III. If they do neither, allow a final Occultism or Observation roll to “feel” the glyph hum and pulse, prompting a response.
\end{CommentBox}

\textbf{The Djinn waits.}



\subsection{Act III: Beneath the Seal}

The stair spirals downward into a molten-cut shaft of obsidian and brass. The air is sweltering, heavy with memory and heat. Whispers press against the players' thoughts—not as words, but as impressions: old fire, sealed pain, and the echo of broken vows.

At the bottom lies a wide, dome-shaped chamber. The floor is etched in old sealing lines, partially disrupted by ash and dust. Five braziers stand cold around a broken pedestal. A single voice speaks—low, vast, and bitter.

\begin{quote}
    \textit{"Who remembers the bargain? Who dares speak it again?"}
\end{quote}

\subsubsection*{Objectives}
\begin{itemize}
    \item Confront the Djinn’s echo.
    \item Choose whether to restore the seal or release Haqar al-Sirr.
    \item Use clues from earlier acts to succeed or suffer the consequences.
\end{itemize}

\subsubsection*{The Echo of Haqar al-Sirr}

Haqar cannot yet take form—but his voice surrounds the players, and his power warps the space. Lights dim. Weapons hiss with heat. Magic is dangerous here.

\textbf{Haqar demands:}  
\begin{itemize}
    \item One character name the glyph from Act II (Zuhahir).
    \item One character stand in the brazier ring to complete the ritual.
    \item A second character speak the command phrase to rebind the seal—or defy it and free him.
\end{itemize}

\subsubsection*{The Seal Ritual}

\textbf{Required actions:}

\begin{enumerate}
    \item \textbf{Place the Glyph:} A character must trace the glyph "Zuhahir" in the center circle. DR 2 \textbf{Arcane Lore} or \textbf{Occultism}. On failure, the glyph burns away and must be redrawn using enchanted ink or blood.
    
    \item \textbf{Invoke the Name:} A character must speak "Zuhahir" aloud. This triggers a reaction—Haqar recoils momentarily.

    \item \textbf{Stabilize the Circle:} DR 3 \textbf{Thaumatology} or \textbf{Occultism} to align the braziers and complete the seal. On failure, the ritual fails—see below.
\end{enumerate}

\begin{CommentBox}{Optional: Extra Success}
    If the player uses the \textbf{Shard-Fragment Pendant} while invoking the name, reduce the DR by 1 for the stabilizing check.
\end{CommentBox}

\subsubsection*{If the Ritual Succeeds}

\begin{itemize}
    \item The heat vanishes. The chamber darkens. A final whisper: \textit{"You’ve remembered… this time."}
    \item The glyph burns gold, and the brazier light turns blue.
    \item The Concord will reward the players with gold and rare favour.
\end{itemize}

\subsubsection*{If the Ritual Fails or Is Rejected}

\begin{itemize}
    \item Haqar’s power flares. A burst of heat melts part of the chamber.
    \item The Djinn does not fully awaken—but his presence lingers.
    \item One character becomes partially marked—granting +1 to fire magic, but drawing Concord suspicion in future sessions.
\end{itemize}

\subsubsection*{If the Players Free Haqar}

\begin{itemize}
    \item The brazier fires ignite as if alive.
    \item A figure of glass and fire steps into the circle—Haqar al-Sirr reborn.
    \item He grants one request (within reason), then vanishes into the desert.
    \item The vault collapses behind them—forever lost.
\end{itemize}

\subsection*{End of the Scenario}

Whether Haqar is bound, freed, or left to whisper in ruins, the players have sealed their place in the city’s deeper lore. The Saphire Concord will ask questions. So will others.

The vault is closed. But something remembers their names.

\begin{CommentBox}{Follow-Up Threads}
    \begin{itemize}
        \item A magical brand may form on a PC’s palm—marking them as "One Who Heard".
        \item A future enemy may recognize the glyph and call it "The Fire Word".
        \item The broken seal could reactivate later in the campaign—or mutate into something stranger.
    \end{itemize}
\end{CommentBox}

\begin{NPC}[%
    name=Haqar al-Sirr,%
    description=The Djinn of Shattered Memory%
]{Haqar al-Sirr}

    \emph{Born from the burning winds of the eastern wastes and sealed after the Amber Revolt, Haqar al-Sirr is a spirit of fractured time, cursed glass, and eternal memory. His voice speaks in layered tongues. His gaze reflects all who stand before him—past, future, and forgotten.}

    \subsubsection*{Manifestation (If Freed):}
    A towering figure of molten brass and shattered mirror shards floats above a smoking circle. Flames flicker across his body without consuming him. His voice breaks glass and echoes into the thoughts of all who hear him. He does not move. The world moves around him.

    \vspace*{0.5\baselineskip}
    \begin{SkillsBox}
        \Expert & Occultism, Thaumatology \\
        \Skilled & Intimidation, Observation \\
        \Novice & Tactics, Fast-Talk, Will \\
    \end{SkillsBox}

    \begin{TraitsBox}
        \item[Djinn Sovereignty] — Haqar automatically succeeds at any magical contest unless opposed by another being of equal power. Mortal resistances apply at +3 DR.
        \item[Fire-Walked Memory] — Once per scene, force all nearby creatures to relive a traumatic memory. DR 3 \textbf{Will} or suffer 2 Fatigue and be staggered for 1 round.
        \item[Burning Reflection] — Anyone who strikes Haqar in melee takes 1 Fatigue from searing feedback unless they pass DR 2 \textbf{Will}.
        \item[Unshaped Flame] — Cannot be physically restrained or trapped. Ignores mundane armour and passes through sealed doors as smoke.
    \end{TraitsBox}

    \DamageBox[totalfatigue=5,totalmild=4,totalmoderate=3,totalsevere=2]

\end{NPC}

\begin{CommentBox}{Using Haqar al-Sirr in Play}

    Haqar al-Sirr is not intended to be a standard combat encounter. His stats represent a fully awakened Djinn—a supernatural entity whose power exceeds mortal boundaries. If the players confront him directly, they are likely to fail unless they:
    
    \begin{itemize}
        \item Possess unique relics designed to bind or banish him.
        \item Complete a complex containment or sealing ritual.
        \item Exploit specific weaknesses uncovered during the scenario.
    \end{itemize}
    
    You may choose to introduce Haqar in one of three ways:
    
    \begin{itemize}
        \item \textbf{Bound Form (Voice Only):} The Djinn remains trapped, speaking through fire, mirror, or dream. Use this for tension and temptation. He can still exert minor influence or offer deals.
        
        \item \textbf{Partial Manifestation:} Haqar emerges briefly in spirit form—just enough to fight, threaten, or test the players. Use this as a mid-point twist or climactic scene before resealing him.
    
        \item \textbf{Full Awakening:} Only use if the players explicitly release him. This should trigger major consequences, possibly ending the scenario or setting up a larger arc.
    \end{itemize}
    
    Avoid a “fair fight.” Instead, make this encounter about decisions, consequences, and negotiation. The fight, if it happens, should feel like resisting a flood—not dueling a man.
    
\end{CommentBox}

