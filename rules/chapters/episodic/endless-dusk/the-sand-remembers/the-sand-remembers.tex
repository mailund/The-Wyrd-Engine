% !TeX root = ../../../../wyrd.tex


\begin{WyrdScenarioHeading}[The Sands Remember]{The Sands Remember}
    \label{scenario:the-sands-remember}
    \index{The Sands Remember}
    \index{Scenario!The Sands Remember}
    
    The Sapphire Concord has uncovered an ancient tomb in the shifting eastern dunes—one that hasn’t seen moonlight in over a thousand years. A storm unearthed the entrance, revealing a stone arch engraved with pre-empire sigils and warnings long forgotten. The players are hired to escort a Concord scholar, \textbf{Faleen Marr}, and ensure her safe passage into and out of the tomb.

    But the sands remember what was buried. As the team descends deeper, time itself begins to blur—visions flicker across torchlight, footsteps echo from other eras, and the dead seem to speak truths that never were. Faleen seeks something more than knowledge, and the players may have to choose between completing the job or preventing something far worse from awakening beneath the dunes.

    \subsection*{Premise}
    When an ancient tomb is uncovered in the desert east of the city, the Sapphire Concord sends the players to assist in its exploration. But the mission is more than academic: strange phenomena emerge as the party ventures deeper, and their employer, a Concord scholar with personal motives, begins invoking forbidden rites. The players must decide how far to follow her into the past—and how much they’re willing to leave behind.

    \subsection*{Beneath the Sand}
    \textbf{Faleen Marr} believes the tomb houses the remains—and possibly the preserved memories—of the last Priest-Queen of the First City. She intends to perform a rite to “absorb” these memories and rebuild the lost magical tradition known as the \textbf{Echo Tongue}. Unfortunately, the tomb was sealed for a reason. If her ritual succeeds, she may not awaken as herself… and time within the tomb may unravel entirely.
\end{WyrdScenarioHeading}

\subsection{Act I: A Name Lost in Dust}

At the Game Master’s discretion, this act may begin with a formal summons: the players receive a parchment bearing the sigil of the \textbf{Sapphire Concord}, requesting their presence at the House of Blue Sand—a quiet estate tucked just beyond the Temple Quarter, known to employ relic-hunters, mercenaries, and discreet scholars.

There, they are greeted by an envoy or junior archivist who ushers them into a sun-dappled study filled with old maps, half-dissolved scrolls, and the faint scent of cinnamon ink. A sealed missive lies waiting. The wax bears the personal sigil of \textbf{Nahema al-Zahir}, Grand Envoy of the Concord.

The letter is brief and urgent: a powerful windstorm has unearthed a structure deep in the dunes east of the city—an \emph{unmapped tomb} bearing glyphs older than the empire. The players are to assist one of Nahema’s field agents, \textbf{Faleen Marr}, in surveying and securing the site. Discretion is vital. There may be relics of historical—or magical—value, and the Concord does not want rival guilds or desert cults learning of the find.

\vspace{0.5\baselineskip}
\subsubsection{Objectives}
\begin{itemize}
    \item Receive the job details and contract (formal or informal).
    \item Prepare for an expedition into the deep desert.
    \item Investigate Faleen Marr’s reputation if desired — revealing that she’s brilliant, but has been censured before for esoteric theories.
    \item Gather supplies, contacts, or possibly secure passage with a smuggler, caravan, or mystical guide.
\end{itemize}

\vspace{0.5\baselineskip}
After any desired downtime, the players begin the journey east. The caravan road dwindles into wind-swept stone. A night camp offers strange dreams—one player sees the tomb from above, shaped like a sand-cracked sigil… and hears a woman’s voice whisper, “You were mine once.”

When they arrive, a tent city has already begun to collapse under the returning sandstorm. \textbf{Faleen Marr} greets them with tension in her jaw, grateful for the help but clearly hiding something. The tomb’s entrance—a partially revealed stone arch—seems unnaturally preserved, its glyphs glowing faintly in the setting sun.

\subsection{Act II: The Fractured Path}

The descent into the tomb begins in eerie silence. The outer halls are dry and dustless, sealed with strange perfection. Glyphs line the walls in a script no one present recognises—unless a player has a magic or lore skill that allows interpretation. Some glyphs match those found on the sealed letter from Nahema… a detail she did not mention.

\textbf{Faleen Marr} grows more intense the deeper they go. She begins sketching glyphs feverishly and murmuring fragments in a tongue the players cannot identify. She insists she’s merely "remembering her training," but those trained in magic or insight may suspect otherwise.

\vspace{0.5\baselineskip}
\subsubsection{Tomb Challenges and Events}
\begin{itemize}
    \item \textbf{Memory Echoes:} Players may experience “slips” in time—brief flashes where they see themselves dressed as ancient attendants, priests, or tomb-guards.
    
    \item \textbf{Guardian Trials:} The tomb is protected by puzzles, traps, or lingering spirits. One chamber requires players to “answer” a riddle not in words, but in shared memory—failing that, they must confront a spectral echo of themselves.

    \item \textbf{Unstable Passageways:} Space twists subtly in some corridors. Players may pass the same glyphs more than once, or find a chamber has changed since they left it.
\end{itemize}

\newcolumn
\subimport{./}{faleen-marr}
\newcolumn



\subsubsection{Faleen’s Magic Deepens}

At some point during exploration, Faleen uses a hidden ritual component to activate a spell known as \textbf{Echo Binding}—drawing faint golden threads between herself and the walls. Any spellcasters feel a ripple through the tomb. From now on, reality itself grows thinner.

\subsubsection{Foreshadowing}
Players may find murals that depict the death of a queen—only in the last mural, the queen bears a striking resemblance to Faleen. One fresco appears to show the players themselves, kneeling before the tomb, etched in a style older than history.

\textbf{At the end of Act II,} the group reaches the central sanctum’s sealed door. Faleen announces that she is “ready.” Her tone has changed. Her eyes do not blink.


\subsection{Act III: Voices of the Forgotten}

As the players push beyond the central sanctum’s threshold, the world no longer behaves predictably. The air thickens. Light bends. Sound arrives out of order. The players have passed into a part of the tomb where time has not simply paused—it has layered itself, moment upon moment, waiting to be recalled.

The architecture no longer matches the murals. Rooms echo with voices that sound familiar, though no one has spoken. The walls show scenes from the past—of rituals, executions, and oaths—only the faces depicted keep changing… sometimes reflecting the players themselves.

\textbf{Faleen Marr} is now fully immersed in the ritual. She no longer refers to her notes. She begins speaking in a language no one taught her. Her voice takes on two tones. She believes she is uncovering history—but she is also becoming history.

\vspace{0.5\baselineskip}
\textbf{Key Events in Act III:}

\begin{itemize}
    \item \textbf{Temporal Fractures:} Players experience short memory slips, déjà vu, or confront “alternate versions” of themselves—ghostlike echoes enacting scenes from a forgotten life.
    
    \item \textbf{Mosaic Trial:} A ceremonial chamber challenges one player to step into a glowing mosaic depicting an ancient rite. They must “remember” a role and act it out. Success stabilizes the path forward. Failure invites the Queen’s attention—and suspicion.
    
    \item \textbf{Echo Encounters:} Spectral echoes of former servants, soldiers, or rebels may appear. They ask cryptic questions or reenact events—but speaking to them may alter the tomb’s reality.
    
    \item \textbf{Faleen’s Shift:} Faleen begins addressing the players as if they are people from the Priest-Queen’s court. She refers to herself in the third person—and sometimes the Queen’s name replaces hers in conversation.
\end{itemize}

\subsubsection{Optional Element: A Bargain}

The Restless Priest-Queen, still fragmented, reaches out through a mural or reflection and offers one of the players a choice:

\begin{quote}
    \emph{“I remember you. Your name… tastes of regret. Will you carry my voice forward? Or will you silence me again?”}
\end{quote}

Accepting her offer may grant power, insight, or an advantage in the final chamber—but also draws her further into the world. Rejecting it may earn her wrath.

\subsubsection{End of Act III}

The players finally reach the tomb’s heart: a circular chamber with a black mirrored floor and a throne of pale sandstone. Faleen stands before it, her hand raised. Behind her, the echo of the Priest-Queen takes shape. The ritual’s final phase begins.

\textbf{Act IV will determine what walks back into the world.}

\begin{CommentBox}{Accepting the Queen’s Bargain}

    If a player engages meaningfully with the Restless Priest-Queen—through dialogue, ritual, or willingly participating in her memory—they may receive a gift, insight, or magical boon. These should feel ancient, personal, and a little dangerous.
    
    \textbf{Possible rewards include:}
    \begin{itemize}
      \item \textbf{A Trait:} \emph{Echo-Bound} — Once per session, recall a memory not your own to gain a clue, bypass a mental challenge, or resist mind-affecting magic.
      
      \item \textbf{A Spell:} The player may learn \textbf{Echo Binding} or \textbf{Name of Dust} without needing to purchase it at creation (cost still applies for advancement).
      
      \item \textbf{A Vision:} The player receives a cryptic dream or flash-forward of a future event in the campaign—possibly one they have the power to change.
      
      \item \textbf{A Curse or Twist:} The Queen speaks through them in rare moments. NPCs may mistake them for her. Reality occasionally bends around their presence.
    \end{itemize}
    
    You are not obligated to grant anything—some players may reject the Queen entirely. But if they embrace her legacy, ensure her presence lingers in future stories.
    
\end{CommentBox}




\newcolumn
\subimport{./}{restless-priest-queen}
\newcolumn



\subsection{Act IV: The Choice Beneath the Seal}

The final chamber is vast, circular, and silent—its floor a mirror of obsidian glass, its ceiling lost in shadow. At its center stands a raised dais bearing a throne carved from pale sandstone. Here, time stutters.

\textbf{Faleen Marr} kneels before the throne, murmuring forgotten words in a voice that shifts with each breath. One moment she is herself. The next, she speaks as the Queen. Her ritual is nearly complete.

Behind her, an echo of the \textbf{Restless Priest-Queen} coalesces—not fully formed, but shimmering with intent. The chamber is charged with potential. If the ritual concludes uninterrupted, the Queen will either possess Faleen or merge with her completely.

\subsubsection{Player Options in This Scene}
\begin{itemize}
  \item \textbf{Interrupt the Ritual:} Players may use physical, magical, or social means to stop Faleen. This may involve damage, counter-rituals, persuasion, or destroying ritual components. The Queen will resist—possibly violently.
  
  \item \textbf{Negotiate with the Queen:} If the players have earned her trust, they may attempt to broker a pact or limit her influence. She may offer insight, power, or truth in exchange for a voice in the world again.

  \item \textbf{Let the Ritual Complete:} Allowing the Queen to merge with Faleen results in a new being—ancient and modern, curious and dangerous. This outcome is not necessarily catastrophic, but will permanently change the world (or campaign).

  \item \textbf{Channel the Queen into Someone Else:} A desperate gambit—one player may volunteer to take the Queen’s essence into themselves, suppressing or absorbing her influence. This should carry both reward and burden.
\end{itemize}

\subsubsection{Complications and Tension}
\begin{itemize}
  \item \textbf{Time Fractures Intensify:} Players may see alternate versions of this moment play out in the mirrors—where they fight each other, fail to act, or die. These are not visions—they are \emph{echoes}, and they bleed into the present.
  
  \item \textbf{Faleen’s Final Resistance:} If confronted, Faleen may lash out with Echo Binding or try to complete the ritual through sheer force of will. She believes this is the only way to preserve the truth of history.
  
  \item \textbf{Environmental Instability:} Each passing moment without resolution causes magical surges—tomb walls shift, floor patterns flicker between centuries, and relics melt into sand and reform as different objects.
\end{itemize}

\subsubsection{The Outcome Should Reflect Player Choice}

This final act is not just a confrontation—it is a question: \emph{Does the past deserve to return?} The Queen is not purely evil, and Faleen is not purely misguided. Let the players decide who speaks for the forgotten.

\begin{CommentBox}{What Happens Next?}
    If the Queen returns, she may walk the city under a false name—gathering followers or quietly rewriting memory. If she is stopped, her whispers may linger in one player’s dreams. If Faleen survives, her reputation may rise… or the Concord may disavow her.

    Use the consequences of this act to influence future scenarios, especially those involving magic, relics, or the boundaries between past and present.
\end{CommentBox}
