\documentclass[twopage,twocolumn,nodeprecatedcode,bg=print]{dndbook}

\usepackage[english]{babel}
\usepackage[utf8]{inputenc}
\usepackage[singlelinecheck=false]{caption}
\usepackage{listings}
\usepackage{shortvrb}
\usepackage{stfloats}
\usepackage{tikz}
\usepackage{xcolor}
\usepackage{xspace}
\usepackage{fontawesome}
\usepackage{pgffor}

\usepackage{imakeidx}
\makeindex

\usepackage{todo}

\captionsetup[table]{labelformat=empty,font={sf,sc,bf,},skip=0pt}

%% Fudge dice notation %%%%%%%%%%%%%%%%%
% Defining the three out comes
\newcommand{\FudgeDieP}{\faPlusSquare}
\newcommand{\FudgeDieM}{\faMinusSquare}
\newcommand{\FudgeDieB}{\faSquare}

% Defining shorthand for single die outcomes
\newcommand{\FudgeDie}[1]{%
    \ifx#1+\FudgeDieP
    \else\ifx#1-\FudgeDieM
    \else\FudgeDieB
    \fi\fi
}

% Translating a sequence of die rolls
\newcommand{\FudgeRes}[1]{%
    \edef\tempstr{#1} % Store input string in \tempstr
    \expandafter\FudgeLoop\tempstr\relax % Expand it into the loop
}
\def\FudgeLoop#1{%
    \ifx#1\relax % Stop when reaching \relax
    \else
        \FudgeDie{#1} % Process current character
        \expandafter\FudgeLoop % Recursive call
    \fi
}

%% Skill levels %%%%%%%%%%%%%%%%%
\newcommand{\Untrained}{\textbf{Untrained (0)}\xspace}
\newcommand{\Novice}{\textbf{Novice (+1)}\xspace}
\newcommand{\Skilled}{\textbf{Skilled (+2)}\xspace}
\newcommand{\Expert}{\textbf{Expert (+3)}\xspace}

%% Define commands for difficulty levels %%%%%%%%%%%%%%%%%
\newcommand{\Trivial}{\textbf{Trivial (-4)}\xspace}
\newcommand{\Simple}{\textbf{Simple (-3)}\xspace}
\newcommand{\Easy}{\textbf{Easy (-2)}\xspace}
\newcommand{\Basic}{\textbf{Basic (-1)}\xspace}
\newcommand{\Challenging}{\textbf{Challenging (0)}\xspace}
\newcommand{\Difficult}{\textbf{Difficult (+1)}\xspace}
\newcommand{\Formidable}{\textbf{Formidable (+2)}\xspace}
\newcommand{\Arduous}{\textbf{Arduous (+3)}\xspace}
\newcommand{\Extreme}{\textbf{Extreme (+4)}\xspace}
\newcommand{\Impossible}{\textbf{Impossible (+5)}\xspace}

\lstset{%
  basicstyle=\ttfamily,
  language=[LaTeX]{TeX},
  breaklines=true,
}

\title{The Wyrd Engine}
\author{Thomas Mailund}

\begin{document}
\DndSetThemeColor[DmgSlateGray]

\frontmatter
\maketitle
\tableofcontents

\mainmatter%



\chapter{The Wyrd Engine}



\DndDropCapLine{T}{he} Wyrd Engine is designed for fast-paced, story-driven play, blending the narrative freedom of Fate with a more structured approach to character abilities. The system emphasises quick character creation and streamlined mechanics, making it an excellent choice for one-shots and episodic campaigns. Game Masters should be able to generate all player characters for a session in less than an hour, and players should be able to pick up a pre-made character and start playing within minutes, allowing for flexible, drop-in play that suits rotating groups or short, focused sessions.

With accessibility in mind, \wyrd is built to be intuitive for newcomers to tabletop roleplaying games. By reducing mechanical complexity and focusing on descriptive actions, it ensures that even those with no prior experience can easily engage with the game. The system provides a strong foundation for storytelling while avoiding cumbersome rules, making it ideal for groups that want to dive straight into adventure without an extended learning curve.

\section{Types of Play}

Roleplaying games can be structured in different ways, each offering a unique experience. \wyrd is primarily designed for \emph{one-shots} and \emph{episodic play}, but it can also support longer campaigns with some adjustments.

\subsection{One-Shots}
A one-shot is a self-contained session that tells a complete story in a single sitting. These are excellent for introducing new players, testing out new settings, or running short, focused narratives without long-term commitment.

\subsubsection{Pros:}
\begin{itemize}
    \item Easy to set up and play with minimal preparation.
    \item Great for newcomers and drop-in players.
    \item Allows for high-stakes storytelling without long-term consequences.
\end{itemize}

\subsubsection{Cons:}
\begin{itemize}
    \item Limited time for character development.
    \item Less room for complex, unfolding plots.
\end{itemize}

\subsection{Episodic Play}
Episodic games consist of multiple short adventures featuring recurring characters. Each session is largely self-contained, but there may be ongoing story threads that connect them.

\subsubsection{Pros:}
\begin{itemize}
    \item Balances flexibility with continuity.
    \item Easy to accommodate changing player rosters.
    \item Encourages character growth while keeping stories manageable.
\end{itemize}

\subsubsection{Cons:}
\begin{itemize}
    \item May lack the deep, overarching narrative of long campaigns.
    \item Requires careful pacing to make each session feel complete.
\end{itemize}

\subsection{Campaign Play}
A campaign is a long-running game with an ongoing story, often spanning multiple sessions with the same characters and overarching narrative.

\subsubsection{Pros:}
\begin{itemize}
    \item Allows for deep character development and long-term storytelling.
    \item Provides a sense of progression and investment.
\end{itemize}

\subsubsection{Cons:}
\begin{itemize}
    \item Requires long-term player commitment.
    \item Can be difficult to maintain momentum if players miss sessions.
\end{itemize}

\wyrd is optimised for one-shots and episodic games, ensuring quick character creation and fast-paced play. However, it can support campaigns with minor modifications, such as introducing progression mechanics or expanding character options over time.

\section{Philosophy and Design Goals}
The Wyrd Engine is built upon the following key design principles:

\subsection{Narrative-Driven Mechanics}
While many systems provide detailed simulationist mechanics, The Wyrd Engine prioritises narrative flow. Rules are designed to reinforce storytelling rather than constrain it, ensuring that mechanics facilitate player agency and character development rather than slow down the action.

\subsection{Modular and Setting-Agnostic}
The Wyrd Engine is intended to be adaptable to multiple settings, from Victorian steampunk mysteries to cosmic horror and high fantasy. Core mechanics remain consistent, while setting-specific options allow groups to tailor the experience to their preferred genre.

\subsection{Accessibility and Ease of Play}
Complexity often serves as a barrier to entry for new players. Two staples of roleplaying games—\emph{narrative play}, where players act out scenes, and \emph{detailed rule sets}, rooted in strategy games—can be stumbling blocks. These two elements are paradoxically at odds: if improvisation is difficult, rules help resolve interactions, but overly complex systems slow down play. \textbf{The Wyrd Engine} leans toward narrative play, with most outcomes determined through roleplaying and the Game Master's discretion. However, its simple skills and traits system provides a structured resolution method when needed.

%\subsection{Character Progression}
%%% TODO: If there comes a section on character progression, update this section
%\todo{There is currently no progression, so rewrite this}
%Characters in The Wyrd Engine develop through a flexible advancement system that ensures steady growth while maintaining balance. Skill caps and structured trait progression prevent power creep, allowing for a long-term campaign structure where characters evolve meaningfully without becoming overpowered.

\subsection{Collaborative Storytelling}
Roleplaying is a shared experience, and The Wyrd Engine encourages player collaboration. Mechanics are designed to give all players opportunities to contribute meaningfully to the story, ensuring that every character has a role to play in the unfolding narrative.

\section{What The Wyrd Engine Is Not}
While the system borrows elements from both narrative and tactical games, it is not intended to be a rigid simulation of reality. It does not use attributes, equipment-heavy mechanics, or detailed statistical modelling. Instead, it focuses on storytelling flexibility while maintaining just enough mechanical structure to create meaningful choices in gameplay.

By keeping these goals in mind, The Wyrd Engine offers a roleplaying experience that is both structured and freeing, supporting deep character development and immersive storytelling without unnecessary mechanical complexity.



\part{Game Mechanics}

\chapter{The Wyrd Engine Core Mechanics}
\label{chap:core}

\DndDropCapLine{T}{he} Wyrd Engine is a lightweight, narrative-driven tabletop roleplaying system designed for quick character creation, streamlined play, and minimal bookkeeping. It aims to provide a simple yet flexible framework that new players can easily pick up while still offering enough depth to engage experienced groups. The system leans into storytelling and improvisation, ensuring that the mechanics never overshadow the unfolding drama of the game.

Unlike more complex RPG systems that emphasise character progression, detailed mechanics, and long-term development, the Wyrd Engine is built for episodic or one-shot adventures where characters are meant to be jumped into and played immediately. This makes it ideal for groups with varying levels of experience, casual game nights, convention settings, or groups that enjoy shifting between different settings and tones without committing to long-term character advancement. By focusing on scene-based resolution, simple skills and traits, and intuitive conflict resolution, the Wyrd Engine keeps the story moving forward while maintaining a satisfying level of challenge and tension.

While the system lacks deep specialisation mechanics, its flexibility allows players to create compelling, unique characters through traits, skills, and equipment that influence their play style. Success in the Wyrd Engine isn’t dictated by meticulous number-crunching but rather by player ingenuity, teamwork, and the creative use of their abilities. Every character is designed to be compelling and memorable right from the start, ensuring they have the tools to make an impact within the narrative. The result is a game that emphasises momentum, character-driven storytelling, and high-action scenarios without getting bogged down in excessive rules.

\input{content/mechanics/core/conflict-resolution}
\input{content/mechanics/core/skills}
\input{content/mechanics/core/traits}
\input{content/mechanics/core/gear}
\input{content/mechanics/core/difficulty-levels}
\input{content/mechanics/core/combat}
\input{content/mechanics/core/character-creation}
\input{content/mechanics/core/npcs}


\chapter{Adapting The Wyrd Engine}

\DndDropCapLine{E}{xtending} the game rules to fit your own settings and temperament, also known as \textbf{homebrewing}\index{homebrewing}, is part and parcel of the roleplaying experience, and \emph{The Wyrd Engine} is designed with this in mind.

\chapter{Non-player Characters}

\DndDropCapLine{N}{on}-player characters, also known as \textbf{NPCs}, are characters controlled by the GM that the players interact with. 
\chapter{Combat}
\index{Combat}
\label{chap:combat}

\DndDropCapLine{T}{he} core combat system, as described in the previous chapter, will suffice for any setting where combat is not a large part of the play. There is next to no combat in Agatha Christie's novels, so we don't need detailed combat mechanics in a setting modelled around such types of crime mysteries. They would only get in the way.

However, the role of combat in a game can vary significantly depending on the setting, the importance of combat in a given scenario, and the style of action you wish to create. Some settings favour \textbf{quick, brutal encounters}, where a single well-placed shot from a sniper or the swift blade of an assassin can end a fight in an instant. In contrast, other games may emphasise \textbf{heroic, drawn-out battles}, where warriors clash against hordes of monsters, trading blows in a struggle for survival.

The \textbf{tone and pacing of combat} should reflect the themes of your game. In a gritty, realistic setting, injuries may be devastating, making every decision in combat critical. A high-action cinematic game, on the other hand, may allow characters to withstand multiple attacks, diving through gunfire or dueling atop a burning airship without immediate risk of death.

For those who prefer \textbf{tactical complexity}, combat may involve detailed positioning, cover mechanics, and resource management, rewarding careful planning and teamwork. Alternatively, a more \textbf{freeform approach} might abstract combat into a series of dramatic exchanges, focusing on storytelling rather than strict mechanics.

No matter the approach, \wyrd provides a flexible combat system that can be adjusted to suit your narrative and playstyle. That is the topic of this chapter.


\section{Dealing damage}
\section{Recovery}

%% FIXME: most of what you can vary is the number of stress and wound boxes, how severe the penalties are for wounds, and how quickly you heal.
\chapter{Magic \& High Tech}

\DndDropCapLine{M}{agic}, and sufficiently advanced technology, can do anything you want. If only non-player characters (or monsters or gods or whathaveyou\ldots) have access to magic or high technology --- henceforth referred to as magic --- then a Game Master can often just decide by decree what magic can do. The powerlevel and capabilities will be whatever makes for a good story. But as soon as player characters need to interact with magic in any structured way, and \emph{especially} if they have access to magic themselves, then we need rules for what magic can do.






\part{One-Shot Scenarios}

\chapter{Crafting One-shots}\label{chap:crafting-one-shots}



\part{Episodic Settings}

\chapter{Crafting Episodic Settings}\label{chap:crafting-episodic-campaignes}
\chapter{The Grand Casebook}\label{chap:grand-casebook}

\begin{WyrdSettingHeading}
    \WyrdCapLine{L}{ondon}, 1896. A city of gaslit streets, towering factories, and secrets lurking in the shadows. This is an era of progress, where steam and steel reshape the world—but beneath the veneer of industry and refinement, the old mysteries remain. The line between science and the supernatural is thinner than most would dare to believe.

    You are part of The Grand Society of Inquiry, a clandestine organisation of detectives, scholars, and unconventional thinkers dedicated to unravelling the mysteries the world would rather forget. The police may handle mundane crimes, but when the case is impossible, when the authorities turn a blind eye, or when the answers defy reason, that is where you come in.

    The aristocracy hides more than it reveals. The city’s underworld knows whispers of truths the elite wish to bury. Strange happenings unfold in laboratories, occult circles, and long-forgotten ruins. It is your job to investigate, to bring truth to light—whether the world is ready for it or not.

    You will encounter murderers whose motives defy logic, inventions beyond their time, secret societies vying for power, and horrors that exist just beyond the veil of reason. Some mysteries should never be solved, but you have chosen to chase the truth regardless.

    London may not thank you for what you uncover. The truth is rarely comforting. But if not you, then who?

    So, tell me: What mystery has found its way to your doorstep tonight?
\end{WyrdSettingHeading}

\section{The Setting}

London in 1896 is a city of contradictions. At its heart lies a tension between progress and tradition, the rational and the arcane. Airships drift over soot-covered rooftops, automata assist in the factories, and steam-powered cabs rattle through the cobbled streets. Yet for all these marvels of industry, old fears still lurk in the fog. Ancient horrors persist in forgotten crypts, and whispers of the occult echo in gentlemen’s clubs and back alley gatherings.

This is a world where gaslight barely holds back the darkness, where rational minds struggle to explain the inexplicable. The Grand Casebook embraces the interplay between Victorian-era crime fiction, steampunk ingenuity, and the gothic supernatural.

\subsection{The Grand Society of Inquiry}

Founded in the wake of the Crimean War, The Grand Society of Inquiry was established by a coalition of scholars, detectives, and adventurers who recognised that certain mysteries lay beyond the reach of conventional authorities. Though their official purpose is to investigate "unusual" occurrences, they are as much a secret society as an investigative body. Their members come from all walks of life—former police officers, rogue academics, disgraced aristocrats, and those who have glimpsed the supernatural and can never return to ignorance.

The Society operates in secrecy, liaising with those who have knowledge of the unseen world—whether they be alchemists, mesmerists, or reformed criminals. Their headquarters, a sprawling archive hidden beneath a London bookshop, contains a wealth of esoteric knowledge that only a select few are permitted to access.

\subsection{The Powers That Be}

While the Society pursues truth, others work to obscure it. Various factions hold sway over London, each with their own stake in its mysteries:

\begin{itemize}
    \item \textbf{Scotland Yard:} The official enforcers of law and order, most officers dismiss the supernatural, though a handful of seasoned inspectors have learned otherwise. The Yard tolerates the Society only when their interests align.
    \item \textbf{The Ministry of Esoteric Affairs:} A shadowy government branch that monitors supernatural activity. Their agents operate with impunity, and their goals often clash with those of the Society.
    \item \textbf{The Order of the Silver Dawn:} An occultist cabal that seeks power through ritual and ancient knowledge. Some claim their origins stretch back to the alchemists of the Elizabethan court.
    \item \textbf{The Industrial Magnates:} The great industrialists of London have their own secrets, from illicit experiments to unspeakable dealings with forces beyond human comprehension.
    \item \textbf{The Underworld Syndicates:} Smugglers and thieves have always known the truth—London's alleys and docks are haunted by more than mere criminals.
\end{itemize}

\subsection{Types of Play}

The Grand Casebook is structured as an episodic mystery-driven setting, where each session presents a new case to unravel. While overarching plots may weave through multiple cases, each game is designed to be a self-contained investigation. The types of mysteries players may face include:

\begin{itemize}
    \item \textbf{Classic Crime:} Murders, thefts, and conspiracies with unexpected twists.
    \item \textbf{Scientific Anomalies:} Unstable inventions, rogue automata, and the consequences of reckless experimentation.
    \item \textbf{Supernatural Encounters:} Hauntings, curses, and beings that should not exist.
    \item \textbf{Political Intrigue:} Power struggles within the aristocracy, blackmail, and espionage.
    \item \textbf{Exploratory Adventures:} Venturing into forgotten catacombs, abandoned asylums, or hidden laboratories.
\end{itemize}

\subsection{Character Roles}

Players take on the roles of Society members, each bringing unique skills to the investigative team. Some possible roles include:

\begin{itemize}
    \item \textbf{The Detective:} A seasoned investigator skilled in deduction and intuition.
    \item \textbf{The Scientist:} A brilliant mind on the cutting edge of technological advancements.
    \item \textbf{The Occultist:} A scholar of the esoteric, familiar with arcane lore.
    \item \textbf{The Rogue:} A streetwise operative connected to the city’s underbelly.
    \item \textbf{The Aristocrat:} A well-connected socialite whose influence opens doors.
    \item \textbf{The Soldier:} A combat-trained veteran, ready to handle more physical threats.
\end{itemize}

\subsection{Rule Adaptations for This Setting}

The Grand Casebook modifies standard play to suit its unique blend of investigation, steampunk technology, and gothic horror. Some adjustments include:

\begin{itemize}
    \item \textbf{Stress and Wounds:} Psychological stress plays a more significant role, with lingering mental consequences affecting future investigations. You can leave out stresses and wounds entirely for most mystery adventures and simply act out any confrontation.
    \item \textbf{Tools of the Trade:} Players may access specialised investigative tools, such as clockwork analysers, ectoplasmic detectors, or enchanted relics.
    \item \textbf{Mystery Structure:} Cases follow a structured flow, focusing on gathering clues, making deductions, and confronting the truth.
    \item \textbf{Supernatural Threats:} Unnatural foes require specific knowledge or preparations to overcome, emphasising research as much as combat.
\end{itemize}


%% TODO: more world building

\section{Adventures}

The following adventures are aimed at 3-5 players and should take 2-4 hours to play. 

\subsection{The Call to Adventure}

At the heart of every investigation lies The Grand Society of Inquiry, an esteemed and enigmatic organisation dedicated to the relentless pursuit of truth. Operating from the opulent halls of the Grand Hall, the society boasts a network of detectives, scholars, and specialists, each possessing a unique skill set vital to solving the most perplexing cases.

When a new case emerges, summons are discreetly dispatched to those deemed most suited for the task at hand. These messages—delivered via courier, pneumatic tube, or even through more esoteric means—call upon select members to assemble and uncover the mystery that awaits. No two groups are ever quite the same, for the \textbf{Grand Analytical Engine}, a vast and intricate steam-powered construct housed in the depths of the Grand Hall, determines the composition of each investigative team.

\begin{WyrdComment}{Framing The Call to Adventure}
	The setup for starting adventures is typical for episodic games where the players can vary from session to session. Having an explanation for why the characters vary from case to case means that no further in-game explanation is needed.
\end{WyrdComment}

%% TODO: Something about the shared structure to the adventures here (the form of mystery adventures)

%% TODO: get this ito the world description

%The Grand Analytical Engine
%
%This marvel of engineering, a hybrid of Babbage’s Analytical Engine and the finest advancements in mechanical computation, processes a staggering wealth of information. Data is fed into its whirring mechanisms by archivists and clerks, cross-referencing past cases, skills, and affiliations. The result: a meticulously curated team, assembled not by human intuition, but by the cold, logical precision of brass gears and punched cards. Whether by fate or by cold calculation, those summoned are invariably drawn into intrigue, danger, and the pursuit of justice.


\input{content/episodic/grand-casebook/murder-at-the-brass-orchid}
\input{content/episodic/grand-casebook/clockmakers-deception}
\input{content/episodic/grand-casebook/the-silent-courier}


%% TODO: Add three more adventures








\printindex

\end{document}
