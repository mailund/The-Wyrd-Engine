\documentclass[twopage,twocolumn,openany,nodeprecatedcode]{dndbook}

\usepackage[english]{babel}
\usepackage[utf8]{inputenc}
\usepackage[singlelinecheck=false]{caption}
\usepackage{listings}
\usepackage{shortvrb}
\usepackage{stfloats}
\usepackage{tikz}
\usepackage{xcolor}
\usepackage{todo}

\newcommand{\FudgeDie}[1]{
    \tikz\draw[thick] (0,0) rectangle (0.3,0.3) node[pos=.5] {\small\textbf{\textcolor{black}{#1}}};
    \hspace{-8pt}
}
\newcommand{\FudgeRes}[4]{%
    \FudgeDie{#1}\FudgeDie{#2}\FudgeDie{#3}\FudgeDie{#4}
}
\captionsetup[table]{labelformat=empty,font={sf,sc,bf,},skip=0pt}

\lstset{%
  basicstyle=\ttfamily,
  language=[LaTeX]{TeX},
  breaklines=true,
}

\title{The Wyrd Engine}

\begin{document}

\frontmatter
\maketitle
\tableofcontents

\mainmatter%




\WyrdCapLine{T}{he} Wyrd Engine is designed for fast-paced, story-driven play, blending the narrative freedom of Fate with a more structured approach to character abilities. The system emphasises quick character creation and streamlined mechanics, making it an excellent choice for one-shots and episodic campaigns. Game Masters should be able to generate all player characters for a session in less than an hour, and players should be able to pick up a pre-made character and start playing within minutes, allowing for flexible, drop-in play that suits rotating groups or short, focused sessions.

With accessibility in mind, \wyrd is built to be intuitive for newcomers to tabletop roleplaying games. By reducing the mechanical complexity and focusing on descriptive actions, it ensures that even those with no prior experience can easily engage with the game. The system provides a strong foundation for storytelling while avoiding cumbersome rules, making it ideal for groups that want to dive straight into adventure without an extended learning curve.


\section{Types of Play}

Roleplaying games can be structured in different ways, each offering a unique experience. \wyrd is primarily designed for \emph{one-shots} and \emph{episodic play}, but it can also support longer campaigns with some adjustments.

\subsection{One-Shots}
A one-shot is a self-contained session that tells a complete story in a single sitting. These are excellent for introducing new players, testing out new settings, or running short, focused narratives without long-term commitment.

\subsubsection{Pros:}
\begin{itemize}
    \item Easy to set up and play with minimal preparation.
    \item Great for newcomers and drop-in players.
    \item Allows for high-stakes storytelling without long-term consequences.
\end{itemize}

\subsubsection{Cons:}
\begin{itemize}
    \item Limited time for character development.
    \item Less room for complex, unfolding plots.
\end{itemize}

\subsection{Episodic Play}
Episodic games consist of multiple short adventures featuring recurring characters. Each session is largely self-contained, but there may be ongoing story threads that connect them.

\subsubsection{Pros:}
\begin{itemize}
    \item Balances flexibility with continuity.
    \item Easy to accommodate changing player rosters.
    \item Encourages character growth while keeping stories manageable.
\end{itemize}

\subsubsection{Cons:}
\begin{itemize}
    \item May lack the deep, overarching narrative of long campaigns.
    \item Requires careful pacing to make each session feel complete.
\end{itemize}

\subsection{Campaign Play}
A campaign is a long-running game with an ongoing story, often spanning multiple sessions with the same characters and overarching narrative.

\subsubsection{Pros:}
\begin{itemize}
    \item Allows for deep character development and long-term storytelling.
    \item Provides a sense of progression and investment.
\end{itemize}

\subsubsection{Cons:}
\begin{itemize}
    \item Requires long-term player commitment.
    \item Can be difficult to maintain momentum if players miss sessions.
\end{itemize}

\wyrd is optimised for one-shots and episodic games, ensuring quick character creation and fast-paced play. However, it can support campaigns with minor modifications, such as introducing progression mechanics or expanding character options over time.

\section{Design Goals}
\wyrd is built upon the following key design principles:

\subsection{Narrative-Driven Mechanics}
While many systems provide detailed simulationist mechanics, \wyrd prioritises narrative flow. Rules are designed to reinforce storytelling rather than constrain it, ensuring that mechanics facilitate player agency and character development rather than slow down the action.

\subsection{Modular and Setting-Agnostic}
\wyrd is intended to be adaptable to multiple settings, from Victorian steampunk mysteries to cosmic horror and high fantasy. Core mechanics remain consistent, while setting-specific options allow groups to tailor the experience to their preferred genre.

\subsection{Accessibility and Ease of Play}
Complexity often serves as a barrier to entry for new players. Two staples of roleplaying games—\emph{narrative play}, where players act out scenes, and \emph{detailed rule sets}, rooted in strategy games—can be stumbling blocks. These two elements are paradoxically at odds: if improvisation is difficult, rules help resolve interactions, but overly complex systems slow down play. \wyrd leans toward narrative play, with most outcomes determined through roleplaying and the Game Master's discretion. However, its simple skills and traits system provides a structured resolution method when needed.

\subsection{Character Progression}

While \wyrd is primarily designed for one-shots and episodic play, it does allow for character progression. Players can choose to develop their characters over multiple sessions, gaining new skills and traits as they progress. This is done through a flexible advancement system that rewards players for their choices and actions, allowing them to shape their characters' growth in meaningful ways. The advancenment system is designed with narrative growth in mind, allowing characters to evolve in ways that reflect their experiences and choices rather than simply increasing numerical values. This approach encourages players to invest in their characters and the story, creating a more immersive experience.


\subsection{Collaborative Storytelling}
Roleplaying is a shared experience, and \wyrd encourages player collaboration. Mechanics are designed to give all players opportunities to contribute meaningfully to the story, ensuring that every character has a role to play in the unfolding narrative.

\section[What The Wyrd Engine Is Not]{What The Wyrd\\Engine Is Not}

While the system borrows elements from both narrative and tactical games, it is not intended to be a rigid simulation of reality. It does not use attributes, equipment-heavy mechanics, or detailed statistical modelling. Instead, \wyrd focuses on storytelling flexibility while maintaining just enough mechanical structure to create meaningful choices in gameplay.

By keeping these goals in mind, \wyrd offers a roleplaying experience that is both structured and freeing, supporting deep character development and immersive storytelling without unnecessary mechanical complexity.

\section{Influences}

\wyrd draws inspiration from a variety of sources, most notably GURPS and Fate Core. Both are robust and well-regarded systems, but in our experience, they can be overly complex for novice and casual players.

From GURPS, we’ve adopted a point-based approach and a focus on broad skill coverage, while deliberately avoiding highly specialised skill trees and the often overwhelming number of available options. From Fate Core, we’ve embraced its narrative-driven design philosophy, but we have chosen to omit the \emph{Aspects} system. While Aspects are highly flexible and engaging for experienced players, they can present a significant barrier to entry, as they require a great deal of improvisation and creative input. Although players often grow to enjoy using Aspects over time, we’ve found they can slow early engagement with the game.

If you're familiar with GURPS or Fate Core, many elements of \wyrd will feel familiar. If not, we believe you’ll still find the system approachable, intuitive, and enjoyable to play.


\WyrdFooterImage{img/pageart/machine-bottom}


\part{Game Mechanics}

\chapter{The Wyrd Engine Core Mechanics}
\label{chap:core}

\DndDropCapLine{T}{he} Wyrd Engine is a lightweight, narrative-driven tabletop roleplaying system designed for quick character creation, streamlined play, and minimal bookkeeping. It aims to provide a simple yet flexible framework that new players can easily pick up while still offering enough depth to engage experienced groups. The system leans into storytelling and improvisation, ensuring that the mechanics never overshadow the unfolding drama of the game.

Unlike more complex RPG systems that emphasise character progression, detailed mechanics, and long-term development, the Wyrd Engine is built for episodic or one-shot adventures where characters are meant to be jumped into and played immediately. This makes it ideal for groups with varying levels of experience, casual game nights, convention settings, or groups that enjoy shifting between different settings and tones without committing to long-term character advancement. By focusing on scene-based resolution, simple skills and traits, and intuitive conflict resolution, the Wyrd Engine keeps the story moving forward while maintaining a satisfying level of challenge and tension.

While the system lacks deep specialisation mechanics, its flexibility allows players to create compelling, unique characters through traits, skills, and equipment that influence their play style. Success in the Wyrd Engine isn’t dictated by meticulous number-crunching but rather by player ingenuity, teamwork, and the creative use of their abilities. Every character is designed to be compelling and memorable right from the start, ensuring they have the tools to make an impact within the narrative. The result is a game that emphasises momentum, character-driven storytelling, and high-action scenarios without getting bogged down in excessive rules.

\input{content/mechanics/core/conflict-resolution}
\input{content/mechanics/core/skills}
\input{content/mechanics/core/traits}
\input{content/mechanics/core/gear}
\input{content/mechanics/core/difficulty-levels}
\input{content/mechanics/core/combat}
\input{content/mechanics/core/character-creation}
\input{content/mechanics/core/npcs}


\chapter{Adding your own extensions}

\DndDropCapLine{E}{xtending} the game rules to fit your own settings and temperament, also known as \textbf{homebrewing}\index{homebrewing}, is part and parcel of the roleplaying experience, and \emph{The Wyrd Engine} is designed with this in mind.

\end{multicols}
\newpage
\section{Key NPCs}

In the following pages, you will find a selection of key NPCs designed to serve as recurring figures. Each character includes a brief description, a glimpse into their background, and a set of traits that can be used to enrich their presence, deepen interactions, and support the unfolding mystery in your game.

\begin{description}
    \item[Mr Alton Merriweather (page \pageref{npc:alton-merriweather})] --- The Chief Steward of the Grand Hall of Inquiry, Mr. Merriweather is a master of decorum and logistics. He oversees the estate’s day-to-day operations with clockwork precision, ensuring that investigators are well supplied and guests properly screened.

    \item[Inspector Quentin Hale (page \pageref{npc:inspector-hale})] --- A rising figure in the Metropolitan Police, Inspector Hale is known for his unwavering belief in procedure and a deep mistrust of private investigators. He often finds himself at odds with the Grand Society of Inquiry, viewing them as a disruptive influence on lawful investigation.
    
    \item[Kip “Knuckles” Mallory (page \pageref{npc:kip-mallory})] --- A streetwise information broker with a network of contacts throughout the city, Knuckles trades in secrets, half-truths, and debts too dirty for polite society. He is quick with a grin and quicker to vanish when the heat is on.

    \item[Dr Octavius Wren (page \pageref{npc:octavius-wren})] --- A brilliant but eccentric scientist, convinced that automata are gaining sentience. Publicly the leader of \emph{The Aetheric Liberty Assembly}, advocating for the rights of sentient machines, but secretly runs \emph{The Automata Liberation Army}—a radical group that seeks to free automata from oppression. He is a master of aetheric technology and has a knack for creating bizarre inventions.

\end{description}

\clearpage
    \subsection{{\small Chief Steward of the Grand Hall}\\ Mr Alton Merriweather}
    \label{npc:alton-merriweather}

        \emph{Unflappable, efficient, and eternally composed, Mr. Merriweather has served the Grand Society for over four decades—and he has never once been surprised.}
        \vspace{.5\baselineskip}
      
    \columnratio{0.375,0.375,0.25}
    \begin{paracol}{3}
        \subsubsection*{Background:}
        Mr Alton Merriweather has served as the chief butler and steward of the Grand Hall of Inquiry since the Society's early days. A master of decorum and logistics, he oversees the estate’s day-to-day operations with clockwork precision. Few know that he was once a field agent himself—though those who glimpse the faint scars beneath his cuffs might suspect a deeper past.
      
        Mr Merriweather maintains the perfect balance of discretion and authority. He ensures that investigators are well supplied, guests properly screened, and that no detail in the Grand Hall ever falls into disorder. While he speaks in clipped, courteous tones, there is steel behind his gaze and loyalty in every action.
      
        \switchcolumn
        \subsubsection*{Using in Play:}
        Mr Merriweather is an anchor NPC—reliable, ever-present, and a point of continuity between investigations. He can:
        \begin{itemize}
          \item Deliver mission briefings or dossiers from the Society’s analysts.
          \item Provide subtle guidance or nudge players toward overlooked details.
          \item Secure equipment, lodgings, or discreet transport.
          \item Offer cryptic remarks hinting at the Society’s deeper secrets.
        \end{itemize}
        He is not meant to overshadow the players, but rather to support them—like the butler in a mystery novel who knows more than he lets on. In times of need, he may reveal surprising resourcefulness, especially if the Grand Hall is ever under threat.
      
        \switchcolumn      
        \subsubsection{Skills}
            \noindent\Expert: Etiquette \\
            \noindent\Skilled: Insight, Logistics \\
            \noindent\Novice: Stealth, Medicine, Presence \\
        \subsubsection{Traits}
          \textbf{Unseen, Unshaken} — Once per session, appear at just the right moment—regardless of obstacles or distance.
      
    \end{paracol}
    \vspace{.5\baselineskip}
    \hrule
    \vspace{.5\baselineskip}

    \subsection{{\small By-the-Book Investigator}\\ Inspector Quentin Hale}
    \label{npc:inspector-hale}
    
    \emph{A stern and rising figure in the Metropolitan Police, Inspector Hale is known for his unwavering belief in procedure and a deep mistrust of private investigators.}
    \vspace{.5\baselineskip}
    
    \columnratio{0.375,0.375,0.25}
    \begin{paracol}{3}
        \subsubsection*{Background:}
        Inspector Quentin Hale is a career man with aspirations of high office. Intelligent, meticulous, and unyielding, he considers the Grand Society of Inquiry a disruptive influence on lawful investigation. While not antagonistic out of malice, his dedication to procedure and political advancement frequently puts him at odds with the Society’s methods. Despite this, he may occasionally seek their help when a case falls outside conventional explanation—grudgingly, of course.
        
        \switchcolumn
        \subsubsection*{Using in Play:}
        Inspector Hale works best as a recurring foil or rival—an NPC who applies pressure, raises stakes, and reminds players that their investigations exist within a broader system of law and politics. He may:
        \begin{itemize}
          \item Attempt to take over a case or block access to key evidence.
          \item Arrest a scapegoat if the players delay or antagonize him.
          \item Undermine the Grand Society’s reputation with the authorities.
          \item Call on the players in private when a case becomes “irregular.”
        \end{itemize}
        Use Hale to inject conflict, force clever diplomacy, or complicate scenes where the players operate in the open.
        
        \switchcolumn      
        \subsubsection{Skills}
            \noindent\Expert: Reasoning \\
            \noindent\Skilled: Discipline, Command \\
            \noindent\Novice: Awareness, Presence, Investigation \\
        \subsubsection{Traits}
            \textbf{Procedure is Power} — Gains a bonus when solving problems by following official protocols to the letter.\\
            \noindent\textbf{Authoritative Glare} — Can reroll when using rank or command presence to compel obedience.\\
    \end{paracol}

    \clearpage
    \subsection{{\small Whisper Broker}\\ Kip “Knuckles” Mallory}
    \label{npc:kip-mallory}
    
    \emph{Quick with a grin and quicker to vanish, Knuckles trades in secrets, half-truths, and debts too dirty for polite society.}
    \vspace{.5\baselineskip}
    
    \columnratio{0.375,0.375,0.25}
    \begin{paracol}{3}
        \subsubsection*{Background:}
        Kip Mallory, known on the streets as “Knuckles,” is an ex-pickpocket turned information broker. With a network of urchins, cabbies, and dockhands, he collects the underbelly’s whispers about crimes, scandals, and disappearances. Though rough around the edges, he is clever, pragmatic, and loyal to those who pay fairly and ask the right way.
        
        \switchcolumn
        \subsubsection*{Using in Play:}
        Knuckles is ideal for providing street-level intel—revealing cryptic leads and information the authorities overlook. He can:
        \begin{itemize}
          \item Drop hints about recent events or persons of interest.
          \item Offer minor favors in exchange for coin or a promise of future assistance.
          \item Connect players with the criminal underworld or serve as a bridge to dubious allies.
          \item Betray the party if their reputation becomes too dangerous.
        \end{itemize}
        Use him to add local color, steer investigations, and introduce tension from the shadows.
        
        \switchcolumn      
        \subsubsection{Skills}
            \noindent\Expert: Streetwise \\
            \noindent\Skilled: Deception, Awareness \\
            \noindent\Novice: Stealth, Presence, Mobility \\
        \subsubsection{Traits}
            \textbf{Too Quick to Catch} — Can reroll when evading capture or disappearing into a crowd.\\
            \noindent\textbf{Favour for a Favour} — Once per session, declare a helpful contact or resource—but you’ll owe Knuckles for it later.
    \end{paracol}
    \vspace{.5\baselineskip}
    \hrule
    \vspace{.5\baselineskip}

    \subsection{{\small Grand Artificer of the Aetheric Liberty Assembly}\\ Dr Octavius Wren}
\label{npc:octavius-wren}

    \emph{Visionary, rebel, and scholar of the forbidden spark. Dr Wren dreams not of progress, but of liberation through invention.}
    \vspace{.5\baselineskip}
  
    \columnratio{0.375,0.375,0.25}
    \begin{paracol}{3}
    \subsubsection*{Background:}
    Once a lauded professor at the Royal College of Natural Philosophy, Dr Octavius Wren vanished from public life after his controversial treatises on sentient automata and free energy were suppressed by the Crown. Years later, he re-emerged as the charismatic leader of the Aetheric Liberty Assembly—a coalition of inventors, exiles, and rogue thinkers who believe true freedom lies in decentralised aetheric technology.

    Secretly, he has been building the Automata Liberation Army from the more radical members of the Assembly. He believes that automata are gaining sentience and that they deserve the same rights as humans, and he and the Liberation Army are willing to go to any lengths to achieve this goal.

    \switchcolumn
    \subsubsection*{Using in Play:}
    Dr Wren can serve as:
    \begin{itemize}
      \item A philosophical foil to players who favour order over innovation.
      \item A source of illicit information, rare devices, or urgent warnings.
      \item A wildcard ally when a common threat emerges—though always on his own terms.
      \item The architect of a grand aetheric event that spirals out of control.
    \end{itemize}
    Though he speaks of liberty, Wren may sacrifice much in the name of progress—including people.

    Wren sees the Grand Society not as enemies, but as blind custodians of a decaying system. His speeches blend poetic fervour with mechanical insight, and his presence inspires fierce loyalty among his followers. Though soft-spoken and refined, there’s an unmistakable intensity in his eyes—a man who has glimpsed a world remade.

    \switchcolumn
    \subsubsection{Skills}
        \noindent\Expert: Engineering \\
        \noindent\Skilled: Rhetoric, Lore \\
        \noindent\Novice: Deception, Insight, Resources \\
    \subsubsection{Traits}
        \textbf{Voice of the Future} — When addressing a crowd or debating ideology, Wren gains advantage and can shift the mood of a scene. \\

    \end{paracol}
    % \vspace{.5\baselineskip}
    % \hrule
    % \vspace{.5\baselineskip}
% !TeX root = ../../../wyrd.tex


\WyrdCapLine{T}{he} core combat system outlined previously is sufficient for settings where combat isn’t a significant focus. In Agatha Christie-style mysteries, detailed combat rules would only clutter gameplay.

However, the importance and style of combat vary greatly between settings. Some games favour \textbf{quick, brutal encounters}, where a precise sniper shot or assassin’s blade swiftly ends a confrontation. Others emphasize \textbf{heroic, extended battles}, featuring characters bravely facing overwhelming odds.

The \textbf{tone and pacing of combat} should align with your game’s themes. A gritty setting might make injuries devastating and every choice critical, while a cinematic action game might allow daring heroics, letting characters survive improbable scenarios.

Players seeking \textbf{tactical complexity} may enjoy detailed positioning, cover, and resource management, rewarding careful planning. Alternatively, a more \textbf{freeform style} abstracts combat into dramatic narrative exchanges.

Furthermore, combat can significantly shape character development and storytelling. The outcomes of battles—both victories and defeats—can profoundly influence character arcs, relationships, and the broader narrative. A character who narrowly survives a deadly encounter might grapple with newfound fears or vulnerabilities, adding emotional depth to your story.

Additionally, combat encounters present opportunities for memorable narrative moments. A tense standoff, a heroic last stand, or a daring escape can become pivotal scenes that players recall long after the game ends. Thoughtfully designed combat can thus enrich the overall storytelling experience, providing dramatic stakes and moments of intense emotional engagement.

No matter your preferred style, \wyrd provides adaptable combat mechanics to suit your story and gameplay. That flexibility is the focus of this chapter.



\section{Combat Statistics}

Combat outcomes in \wyrd depend primarily on the skill difference between attackers and defenders, though dice rolls introduce some variability. Table \pagereftext{tbl:damage-probability} shows probabilities of inflicting damage based on skill disparities, expected damage per round, and average rounds needed to inflict 7+ damage (taking out a core character). The graphic on this page visually illustrates these probabilities.

\Graph[Damage per Attacker-Defender levels]{stats/damage_distribution.png}

Skill differences dominate combat outcomes by design. Each round favours the defender slightly (since ties do not deal damage). Within multiple rounds, the character with initiative attacks first, giving them a slight edge as well. Small differences in skill levels (1-2 levels) can have a large effect. A difference where the attacker has one level higher than the defender will not substantially shorten a combat --- it is expected to cut the rounds by half, from 11.1 to 5.5 --- but the probability of a character with +1 in attack and 0 in defence defeating a character with 0 in both attack and defence is 87.6\% compared to only 53.0\% if the two were evenly matched.

This emphasis on skills over randomness ensures predictable yet engaging gameplay, reinforcing the strategic importance of positioning and skill management. Any combat bonuses, for either attack or defence, can swing the battle. The long expected combat for equally skilled characters is also intentional. It prevents unfortunate characters from being eliminated in a single blown, reducing the randomness of combat. It does, however, mean that combat can be drawn out if the only combat actions are attacks and defending. But it generally shouldn't be.



\end{multicols}
\clearpage
\begin{DndTable}[header=Damage probability by relative skill level (Attack - Defence)]{crrrrrrrr}\label{tbl:damage-probability}
    \textbf{Attack - Defence} & \textbf{0 stress} & \textbf{1 stress} & \textbf{2 stress} & \textbf{3 stress} & \textbf{4 stress} & \textbf{5 stress} & \textbf{6 stress} & \textbf{7+ stress} \\
    -4 &  97.6\% &   1.7\% &   0.5\% &   0.1\% &   - &   - &   - &   - \\
    -3 &  93.6\% &   4.1\% &   1.7\% &   0.5\% &   0.1\% &   - &   - &   - \\
    -2 &  85.9\% &   7.7\% &   4.1\% &   1.7\% &   0.5\% &   0.1\% &   - &   - \\
    -1 &  73.9\% &  11.9\% &   7.7\% &   4.1\% &   1.7\% &   0.5\% &   0.1\% &   - \\
     0 &  58.4\% &  15.5\% &  11.9\% &   7.7\% &   4.1\% &   1.7\% &   0.5\% &   0.1\% \\
    +1 &  41.6\% &  16.9\% &  15.5\% &  11.9\% &   7.7\% &   4.1\% &   1.7\% &   0.6\% \\
    +2 &  26.1\% &  15.5\% &  16.9\% &  15.5\% &  11.9\% &   7.7\% &   4.1\% &   2.3\% \\
    +3 &  14.1\% &  11.9\% &  15.5\% &  16.9\% &  15.5\% &  11.9\% &   7.7\% &   6.4\% \\
    +4 &   6.4\% &   7.7\% &  11.9\% &  15.5\% &  16.9\% &  15.5\% &  11.9\% &  14.1\% \\
    +5 &   2.4\% &   4.1\% &   7.7\% &  11.9\% &  15.5\% &  16.9\% &  15.5\% &  26.0\% \\
    +6 &   0.7\% &   1.7\% &   4.1\% &   7.7\% &  11.9\% &  15.5\% &  16.9\% &  41.5\% \\
\end{DndTable}

\begin{DndTable}[header=Expected Damage in One Round]{lrrrrrrrrrrr}
    \textbf{Attacker - Defender} & \textbf{-4} & \textbf{-3} & \textbf{-2} & \textbf{-1} & \textbf{0} & \textbf{+1} & \textbf{+2} & \textbf{+3} & \textbf{+4} & \textbf{+5} & \textbf{+6} \\
    \textbf{Expected Damage}     & 0.0365      & 0.108       & 0.260       & 0.530        & 0.950       & 1.53        & 2.26         & 3.07         & 3.92         & 4.74         & 5.46      \\
\end{DndTable}


\begin{DndTable}[header=Expected Rounds to Accumulate 7+ Damage]{lrrrrrrrrrrr}
    \textbf{Attacker - Defender} & \textbf{-4} & \textbf{-3} & \textbf{-2} & \textbf{-1} & \textbf{0} & \textbf{+1} & \textbf{+2} & \textbf{+3} & \textbf{+4} & \textbf{+5} & \textbf{+6} \\
    \textbf{Expected Rounds}            & 210.5      & 89.3      & 40.3      & 19.3      & 11.1     & 5.5      & 3.8      & 2.8      & 2.3      & 1.9      & 1.6      \\
\end{DndTable}

The expected damage is the average damage that a player can expect to inflict in one round of combat, assuming that the player has the initiative and attacks first. The expected rounds to accumulate 7+ damage is the average number of rounds that it would take for a player to inflict 7+ damage on an opponent, assuming that the player has the initiative and attacks first.

\begin{DndTable}[header=Probability of player with initiative winning]{lrrrrrrrrr}
    &  \textbf{P2(0,0)} & \textbf{P2(0,1)} & \textbf{P2(0,2)} & \textbf{P2(1,0)} & \textbf{P2(1,1)} & \textbf{P2(1,2)} & \textbf{P2(2,0)} & \textbf{P2(2,1)} & \textbf{P2(2,2)}  \\
    \textbf{P1(0,0):} &  53.0\% &   9.3\% &   0.1\% &  17.6\% &   1.2\% &   0.0\% &   4.8\% &   0.2\% &   0.0\% \\
    \textbf{P1(0,1):} &  92.3\% &  51.2\% &   2.7\% &  53.0\% &   9.3\% &   0.1\% &  17.6\% &   1.2\% &   0.0\% \\
    \textbf{P1(0,2):} &  99.9\% &  97.5\% &  50.4\% &  92.3\% &  51.2\% &   2.7\% &  53.0\% &   9.3\% &   0.1\% \\
    \textbf{P1(1,0):} &  87.6\% &  53.0\% &   9.3\% &  56.2\% &  17.6\% &   1.2\% &  27.0\% &   4.8\% &   0.2\% \\
    \textbf{P1(1,1):} &  99.2\% &  92.3\% &  51.2\% &  87.6\% &  53.0\% &   9.3\% &  56.2\% &  17.6\% &   1.2\% \\
    \textbf{P1(1,2):} & 100.0\% &  99.9\% &  97.5\% &  99.2\% &  92.3\% &  51.2\% &  87.6\% &  53.0\% &   9.3\% \\
    \textbf{P1(2,0):} &  97.7\% &  87.6\% &  53.0\% &  85.0\% &  56.2\% &  17.6\% &  60.9\% &  27.0\% &   4.8\% \\
    \textbf{P1(2,1):} &  99.9\% &  99.2\% &  92.3\% &  97.7\% &  87.6\% &  53.0\% &  85.0\% &  56.2\% &  17.6\% \\
    \textbf{P1(2,2):} & 100.0\% & 100.0\% &  99.9\% &  99.9\% &  99.2\% &  92.3\% &  97.7\% &  87.6\% &  53.0\% \\
\end{DndTable}

Notation \textbf{Pn(A,D)} should be read as player \emph{n} has attack skills \emph{A} and defence skills \emph{D}. Player 1 has the initiative and attacks first. Evenly matched, the player that attacks first has a slight advantage. The probability that the second player wins is one minus the probability that the first player wins.

For all tables, we have not taken into account the effect of wound penalties or the use of combat maneuvers.

The tables are not intended to be used as a reference during play, but rather to give you an idea of the expected outcomes of combat. This can help the GM design combat encounters that are challenging but not impossible for the players.

\vspace{\baselineskip}
\hrule
\begin{multicols}{2}







\section{Making Combat Interesting}

Combat shouldn’t merely be a predictable dice-rolling exercise. \wyrd balances the active opposition mechanics used elsewhere, for both determining when an attack is successful and how much damage is inflicted, with a few additional mechanics to keep combat engaging. And these mechanics are well known as well: \textbf{using traits and gear} to obtain offensive or defensive bonuses, and using combat manoeuvres as \textbf{boosts} to gain additional advantages.

But before we consider applying these mechanics in combat, let us consider the alternative which is to let characters slug it out with no modifiers. This is a valid option, but it can lead to combat being a simple exercise in rolling dice and adding numbers, and unless the two characters are evenly matched, the outcome is strongly skewed in one direction or the other.

We will use the example of \emph{Anna the Assassin} and \emph{Brian the Barbarian} to illustrate this. Both characters have a \textbf{+1 Fight} skill, which they use for both attacking and defending. Initially, they are evenly matched, so the outcome of their combat is almost entirely dependent on the dice rolls, with only a slight advantage to the player with the initiative, in this case Anna.

\begin{Example}{Combat without modifiers}
    \emph{Anna the Assassin} jumps on top of her table at the \emph{Rusty Dagger Tavern}, blades gleaming in the flickering lantern light. Across the room, \emph{Brian the Barbarian} rises with a growl, knocking over his ale as he draws his enormous axe. 

    Anna has the initiative and attacks first. She rolls a \FudgeRes{+0--} = \textbf{-1} and Brian rolls a \FudgeRes{++--} = \textbf{0}. They both add their \textbf{Fight +1} but they cancel out. Since Anna's attack is below Brian's defence, she does not inflict damage. 

    \vspace{0.5\baselineskip}
    \begin{tcolorbox}[
        damageboxbase,
        title=Damage Boxes
    ]
    \begin{tabular}{@{}l l@{ } l@{ } l@{ } l@{ }}
        \textbf{Anna the Assassin} & \FatigueBoxes[0][3] & \MildWounds[0][1] & \ModerateWounds[0][1] & \SevereWounds[0][1] \\
        \textbf{Brian the Barbarian} & \FatigueBoxes[0][3] & \MildWounds[0][1] & \ModerateWounds[0][1] & \SevereWounds[0][1]
    \end{tabular}
    \end{tcolorbox}

    Brian retaliates with his own attack, rolling a \FudgeRes{++00} = \textbf{+2} against Anna's defence of \FudgeRes{+00-} = \textbf{0}. This time the attack is successful, and Brian inflicts \textbf{2 damage} on Anna.
   
    \begin{tcolorbox}[
        damageboxbase,
        title=Damage Boxes
    ]
    \begin{tabular}{@{}l l@{ } l@{ } l@{ } l@{ }}
        \textbf{Anna the Assassin} & \FatigueBoxes[2][3] & \MildWounds[0][1] & \ModerateWounds[0][1] & \SevereWounds[0][1] \\
        \textbf{Brian the Barbarian} & \FatigueBoxes[0][3] & \MildWounds[0][1] & \ModerateWounds[0][1] & \SevereWounds[0][1]
    \end{tabular}
    \end{tcolorbox}

    Now it is Anna's turn again. She rolls a \FudgeRes{++00} = \textbf{+2} against Brian's defence of \FudgeRes{++0-} = \textbf{+1}. This time, Anna's attack causes \textbf{1 damage} to Brian.

    \begin{tcolorbox}[
        damageboxbase,
        title=Damage Boxes
    ]
    \begin{tabular}{@{}l l@{ } l@{ } l@{ } l@{ }}
        \textbf{Anna the Assassin} & \FatigueBoxes[2][3] & \MildWounds[0][1] & \ModerateWounds[0][1] &\SevereWounds[0][1] \\
        \textbf{Brian the Barbarian} & \FatigueBoxes[1][3] & \MildWounds[0][1] & \ModerateWounds[0][1] &\SevereWounds[0][1]
    \end{tabular}
    \end{tcolorbox}

    Now Brian swings his axe again, rolling a \FudgeRes{+00-} = \textbf{0} against Anna's defence of \FudgeRes{+++0} = \textbf{+3}. The attack is smaller than the defence, so Brian does not inflict any damage.

    \begin{tcolorbox}[
        damageboxbase,
        title=Damage Boxes
    ]
    \begin{tabular}{@{}l l@{ } l@{ } l@{ } l@{ }}
        \textbf{Anna the Assassin} & \FatigueBoxes[2][3] & \MildWounds[0][1] & \ModerateWounds[0][1] & \SevereWounds[0][1] \\
        \textbf{Brian the Barbarian} & \FatigueBoxes[1][3] & \MildWounds[0][1] & \ModerateWounds[0][1] & \SevereWounds[0][1]
    \end{tabular}
    \end{tcolorbox}

\end{Example}

We could go on here, and there is close to a 50\% chance for both of the opponents to win, so some uncertainty in the outcome, but it is not very exciting to play out a battle this way.

We can vary the situation slightly using just traits. Anna the Assassin has a \textbf{Blade of the Night} trait that gives her a +2 bonus to attack rolls in the dark.

\begin{Example}{Exploiting Traits}
    \emph{Anna the Assassin} followed \emph{Brian the Barbarian} as he left the \emph{Rusty Dagger Tavern}, waiting for the right moment to strike. As Brian stepped into the dark alley to releave himself, Anna leapt from the shadows.

    The GM judges that the alley is dark enough for Anna to use her \textbf{Blade of the Night} trait, giving her a +2 bonus to attack rolls.

    She rolls a \FudgeRes{++00} = \textbf{+2} and adds her trait \textbf{+2}. Brian's defence is \FudgeRes{++0-} = \textbf{+1}. The difference is \textbf{+3}, so Anna inflicts \textbf{3 damage} on Brian.

    \vspace{0.5\baselineskip}
    \begin{tcolorbox}[
        damageboxbase,
        title=Damage Boxes
    ]
    \begin{tabular}{@{}l l@{ } l@{ } l@{ } l@{ }}
        \textbf{Anna the Assassin} & \FatigueBoxes[0][3] & \MildWounds[0][1] & \ModerateWounds[0][1] &\SevereWounds[0][1] \\
        \textbf{Brian the Barbarian} & \FatigueBoxes[3][3] & \MildWounds[0][1] & \ModerateWounds[0][1] & \SevereWounds[0][1]
    \end{tabular}
    \end{tcolorbox}

    Brian, now aware of Anna's presence, retaliates with a roar. He rolls a \FudgeRes{++0-} = \textbf{+1} against Anna's defence of \FudgeRes{+00-} = \textbf{0}. Anna's trait is only applicable for attacks, so she cannot add it here. The difference is \textbf{+1}, so Brian inflicts \textbf{1 damage} on Anna.

    \begin{tcolorbox}[
        damageboxbase,
        title=Damage Boxes
    ]
    \begin{tabular}{@{}l l@{ } l@{ } l@{ } l@{ }}
        \textbf{Anna the Assassin} & \FatigueBoxes[1][3] & \MildWounds[0][1] &\ModerateWounds[0][1] &\SevereWounds[0][1] \\
        \textbf{Brian the Barbarian} & \FatigueBoxes[3][3] & \MildWounds[0][1] &\ModerateWounds[0][1] &\SevereWounds[0][1]
    \end{tabular}
    \end{tcolorbox}

    Anna attacks again, rolling a \FudgeRes{++0-} = \textbf{+1} and adds \textbf{+2} for an attack of \textbf{+3} against Brian's defence of \FudgeRes{++0-} = \textbf{+1}. The difference is \textbf{+2}.

    \begin{tcolorbox}[
        damageboxbase,
        title=Damage Boxes
    ]
    \begin{tabular}{@{}l l@{ } l@{ } l@{ } l@{ }}
        \textbf{Anna the Assassin} & \FatigueBoxes[1][3] & \MildWounds[0][1] & \ModerateWounds[0][1] &\SevereWounds[0][1] \\
        \textbf{Brian the Barbarian} & \FatigueBoxes[3][3] &\MildWounds[1][1] &\ModerateWounds[1][1] &\SevereWounds[0][1]
    \end{tabular}
    \end{tcolorbox}

    At this point, Brian conceeds the fight.
\end{Example}

It is not that adding traits to make the battle more uneven also makes it more interesting --- if anything, it makes it less interesting since the chance of the outclassed character winning is so low. But at least such a combat encounter is over quickly, and the players can move on to the next scene. The point is not that skill or trait bonuses adds excitement to combat, however, but the use of traits and gear can make choosing the battlefield, the time and place, a strategicly important decision, which \emph{can} add excitement to combat.

\subsection{Changing the Battlefield}

Once a combat encounter is underway, the players might not be able to change the conditions to activate a trait, but sometimes they can --- if Anna and Brian were fighting in the tavern and Anna had the chance to throw the room into darkness, for example. If the players \emph{can} change the conditions they are fighting in, then that becomes a tactical goal. Increasing the attack or defence stats by one or two levels can be a significant advantage, and the players should be encouraged to use their traits and gear to gain that advantage.

\begin{Example}{Changing the Battlefield}
    \emph{Anna the Assassin} and \emph{Brian the Barbarian} find themselves locked in combat inside the \emph{Rusty Dagger Tavern}. The room is lit by swaying oil-lamps, and Anna's \textbf{Blade of the Night} trait—granting +2 to attacks in the dark—is currently useless.

    Anna decides to act. On her turn, instead of attacking, she uses an action to snuff out the main lantern by flipping a table into it. The GM calls for an \textbf{Athletics} \DL{2} check. Anna rolls \FudgeRes{+0+-} = \textbf{+1} and adds it to her \textbf{Athletics +2} skill. The lantern crashes to the floor, plunging the room into shadow.

    Brian roars in frustration and swings blindly, rolling a \FudgeRes{+00-} = \textbf{0}, but Anna defends with \FudgeRes{+++0} = \textbf{+3}, easily dodging in the darkness.

    Now it's Anna’s turn. With the room dark, her \textbf{Blade of the Night} activates. She attacks, rolling \FudgeRes{++0-} = \textbf{+1}, adds +2 from the trait, for a total of \textbf{+3}. Brian defends with \FudgeRes{+0--} = \textbf{-1}, giving Anna a difference of \textbf{+4}.

    \vspace{0.5\baselineskip}
    \begin{tcolorbox}[
        damageboxbase,
        title=Damage Boxes
    ]
    \begin{tabular}{@{}l l@{ } l@{ } l@{ } l@{ }}
        \textbf{Anna the Assassin} & \FatigueBoxes[0][3] & \MildWounds[0][1] & \ModerateWounds[0][1] & \SevereWounds[0][1] \\
        \textbf{Brian the Barbarian} & \FatigueBoxes[3][3] & \MildWounds[1][1] & \ModerateWounds[0][1] & \SevereWounds[0][1]
    \end{tabular}
    \end{tcolorbox}

    Realising he's completely outmatched in the dark, Brian stumbles toward the door, seeking light—or surrender.
\end{Example}


\subsection{Combat Maneuvers}

If the players cannot invoke their existing traits (or the traits of their gear), then they can still use combat maneuvers to gain bonuses to their attacks or defences.

Combat maneuvers are special actions that can be used to gain a temporary advantage in combat. If you are changing the battlefield to gain a bonus from a trait, you already posses the trait, but you need to change the situation to gain the bonus. Traits are narrow in scope, and not all situations will enable you to exploit them, even after taking actions to change the battlefield. Combat maneuvers are always available, however. At any time, you can spend an action to perform a combat maneuver, which will give you a bonus to your next attack or defence, but unlike traits, combat maneuver bonuses are transient and lost as soon as you use them, or as soon as an attempt to increase them fails.

In any round, instead of attacking, a character can
\begin{itemize}
    \item Do an \textbf{attack} combat maneuver to gain a \textbf{+2 bonus} to their next attack.
    \item Do a \textbf{defend} combat maneuver to gain a \textbf{+2 bonus} to their next defence.
\end{itemize}

Bonuses accumulated until they are used, or until the character fails a combat maneuver, in which case the entire accumulated bonus is lost. The two bonuses accumulate independently, and a failed maneuver does not affect the other bonus.

Doing a combat maneuver works like normal opposition rolls. A character should always be allowed to use theh skill they use for attacking or defending against a difficulty level of \textbf{2}, with \textbf{ties reducing the bonus to +1}, but the GM should also allow inventive players to use other skills if they can justify it. In that case, the GM should judge whether the opposition roll is passive or active and set appropriate difficulty levels for passive rolls. In the case of ties, the GM should judge whether the tie is a success or a failure, and what the consequences are, i.e., whether a tie reduces the bonus to +1 or whether it is a failure that doesn't remove the accumulated bonus.

\begin{Example}{Attack Combat Maneuvers}
    \emph{Anna the Assassin} is sneaking up on \emph{Brian the Barbarian}. She intends to jump him, which would be an attack, but her player figures that if she sneaks up close and stabs him in the back, she should get an attack bonus. The GM aggrees, but requires an active opposition roll, Anna's \textbf{Stealth} against Brian's \textbf{Notice}. Upon success, she will get a +2 bonus to her stab attack, but on failure or a tie Brian would get to attack with initiative.

    Anna rolls a \FudgeRes{++00} = \textbf{+2} and adds her \textbf{Stealth +2} against Brian's \FudgeRes{++0-} + \textbf{Notice +1}. The total is \textbf{+4} against \textbf{+2}, so the roll is a success, so she gains a +2 bonus to her next attack, an attack she immidiately makes.

    Anna attacks, rolling a \FudgeRes{++0-} = \textbf{+1} and adds her \textbf{Fight +1} and the \textbf{+2 bonus} from the combat maneuver for a total of \textbf{+4}. Brian defends with \FudgeRes{+000} = \textbf{+1} plus his \textbf{Fight +1} for a total of \textbf{+2}, giving Anna a difference of \textbf{+2}.
\end{Example}

This example shows that you can use a normal opposition roll to gain a combat maneuver bonus. Strictly speaking, the combat hadn't started yet, but preparing for battle is a valid combat maneuver, and the GM should allow it.

\begin{Example}{Defence Combat Maneuvers}
    \emph{Brian the Barbarian}, screaming from being stabbed in the back, throws himself behind a dumbster, trying to take cover. 

    This is a \textbf{defend} combat maneuver --- he is doing the action instead of attacking -- and Brian will use his \textbf{Athletics +2} against a \textbf{+2} difficulty level. He rolls \FudgeRes{++00} = \textbf{+2} and adds his \textbf{Athletics +2} for a total of \textbf{+4}, so he succeeds and gets a +2 bonus to his next defence.
    
    Anna attacks and rolls a \FudgeRes{+++-} = \textbf{+2} and adds her \textbf{Fight +1} for a total of \textbf{+3} (she no longer has the bonus she used for her stealth attack). Brian rolls \FudgeRes{++0-} + \textbf{Fight +1} plus the \textbf{+2} defence bonus for a total of \textbf{+4}. Being in cover behind the dumpster saved him from the attack.
\end{Example}

Even with combat manoeuvres, there is still a risk that a fight may drag on, with one character steadily building up attack bonuses while the other accumulates defensive ones—leaving their relative positions unchanged. This is mitigated somewhat by the chance of losing a bonus when a manoeuvre fails.

The base difficulty of \DL{2} means a character with a relevant skill of \textbf{+1} will only succeed about a third of the time (38.7\%), while a character with \textbf{+2} will succeed just under two-thirds of the time (61.7\%). Success is far from guaranteed, and the risk of failure—and losing the bonus—is significant. As a result, the tactic of simply stacking bonuses is not a reliable long-term strategy.

However, the ability to use non-combat skills to perform manoeuvres allows characters to play to their strengths—if they can be creative and the GM permits it. This opens up new tactical options for players, which is the true purpose of combat manoeuvres. They are not just a way to gain bonuses to attack or defence, but a tool for players to engage the system creatively and leverage a broader range of skills to gain the upper hand in combat.

When multiple characters are involved in combat, maneuvers also add a layer of tactical complexity. If two characters are attacking a third, the defender is effectively prevented from building up defence bonuses. The defence bonus they have will be expended on the first attack, so the attackers can decide to have one build up attack bonuses while the other attacks, and the defender cannot build a defence bonus against the boosted attack that will eventually come.

In larger battles, deciding who fights who, and how to use combat maneuvers, can be a tactical decision. If the players are fighting a group of enemies, they can choose to attack one at a time, or they can split up and attack multiple enemies at once. Their choices will determine how they can build up their own bonuses and what choices their opponents can make for their own combat maneuvers.

\section{Weapons and Armour}

Gear traits can enhance combat just like any other opposition rolls, and it’s natural to model weapons and armour as such traits. Fists are less effective than knives, which are in turn less effective than swords. Similarly, leather armour offers less protection than chainmail, which is weaker than full plate.

The level of detail you apply depends on the setting and how often combat arises in your game. In a setting where combat is rare, you might avoid complex rules altogether. But if combat is a central part of the game, then weapon and armour choice can become an important part of both character identity and tactical planning.

Below are examples of how gear traits can be used to model the effectiveness of different weapons and armour across different settings. These examples are not exhaustive but should serve as a helpful baseline. 

\begin{CommentBox}{A note of caution}
    Adding bonuses to weapons and armour can easily lead to an arms race. If every opponent and player continually escalates their gear bonuses, you may end up with excessive bookkeeping but no meaningful change to the gameplay. To avoid this, ensure players face enemies both less and more well-equipped than themselves. Gaining a powerful weapon to overcome a challenge can make for a compelling story—but simply scaling weapons in parallel with enemies leads to stagnation.
\end{CommentBox}

\subsection{Weapons}

Weapons can be modelled as gear traits that provide a bonus to attack rolls. Light weapons may grant +1, while heavier or more advanced weapons may grant +2 or more. However, excessive stacking of bonuses should be avoided—encourage variety in use and tactical application instead.

\subsubsection*{Fantasy}

\begin{itemize}
  \item \textbf{Unarmed / Improvised Weapon (0)} – Fists, chairs, tankards.
  \item \textbf{Dagger / Club (+1)} – Small, quick weapons that are easy to conceal or use in close quarters.
  \item \textbf{Sword / Axe / Spear (+2)} – Standard martial weapons with a reliable combat bonus.
  \item \textbf{Greatsword / Polearm (+3)} – Two-handed or powerful weapons with greater reach or impact.
  \item \textbf{Legendary Weapon (+4)} – Rare magical or mythic weapons with narrative weight. These should be plot-relevant.
\end{itemize}

\subsubsection*{Modern}

\begin{itemize}
  \item \textbf{Fist / Stun Baton (0)} – Non-lethal or improvised.
  \item \textbf{Knife / Pistol (+1)} – Standard sidearms or melee tools.
  \item \textbf{Shotgun / Assault Rifle (+2)} – Tactical weapons for combat scenarios.
  \item \textbf{Sniper Rifle / Heavy Weapon (+3)} – Long-range or high-calibre weapons; often slower or bulkier.
  \item \textbf{Prototype or Military-Grade Weapon (+4)} – Restricted or experimental tech, used sparingly.
\end{itemize}

\subsubsection*{Sci-Fi}

\begin{itemize}
  \item \textbf{Plasma Dagger / Energy Whip (+1)} – Futuristic melee weapons.
  \item \textbf{Laser Rifle / Gauss Gun (+2)} – Common energy weapons with precise or powerful shots.
  \item \textbf{Plasma Cannon / Anti-Matter Lance (+3)} – Devastating weapons, difficult to wield or maintain.
  \item \textbf{Relic of the Ancients (+4)} – Rare and potent alien or ancient technology, central to plot arcs.
\end{itemize}

\subsection{Armour}

Armour provides a bonus to defence rolls, reducing the chance of taking damage. Unlike weapons, armour often comes with trade-offs—such as reduced mobility, attention-drawing bulk, or limited availability in certain settings.

\subsubsection*{Fantasy}

\begin{itemize}
  \item \textbf{None / Clothing (0)} – Offers no real protection.
  \item \textbf{Leather Armour (+1)} – Light, flexible, and common among rogues or rangers.
  \item \textbf{Chainmail / Scale Armour (+2)} – Heavier protection at the cost of agility.
  \item \textbf{Plate Armour (+3)} – Full-body protection, often worn by elite knights.
  \item \textbf{Enchanted Armour (+4)} – Rare magical items that may confer additional narrative effects.
\end{itemize}

\subsubsection*{Modern}

\begin{itemize}
  \item \textbf{None / Casual Wear (0)} – No protective value.
  \item \textbf{Kevlar Vest (+1)} – Light ballistic protection against small arms.
  \item \textbf{Tactical Body Armour (+2)} – Offers improved coverage and resistance.
  \item \textbf{Bomb Suit / Riot Gear (+3)} – Maximum protection, but heavy and cumbersome.
  \item \textbf{Prototype Armour (+4)} – Advanced gear from research labs or special forces.
\end{itemize}

\subsubsection*{Sci-Fi}

\begin{itemize}
  \item \textbf{Nano-Weave Undersuit (+1)} – Flexible and stylish, useful for infiltration or agents.
  \item \textbf{Combat Exosuit (+2)} – Reinforced armour with HUD and power support.
  \item \textbf{Powered Armour (+3)} – Heavy-duty suits with strength amplification and shielding.
  \item \textbf{Void Armour (+4)} – Ancient or alien tech that defies conventional damage.
\end{itemize}



\section{Fighting Styles}

Not all combatants fight the same way. Some rely on brute strength, others on speed, cunning, or honed discipline. In \wyrd, you can represent different forms of combat using \textbf{fighting styles}—distinct techniques, schools, or traditions that combine specific skills, weapons, and tactics into recognisable approaches to battle.

Fighting styles can be purely narrative, or they can provide mechanical bonuses when used strategically. A style may work well against some opponents but poorly against others, introducing a natural system of strengths and weaknesses—like rock-paper-scissors, but more flexible and open to creative interpretation.

Fighting styles can be expressed using \textbf{traits}, or defined narratively by the GM and players. Some styles may grant a bonus in certain situations (e.g., against heavy armour, while surrounded, or in darkness), while others are designed to counter particular styles or skills.

\subsection*{Combining Skills and Weapons}

In a flexible system like \wyrd, fighting styles can be built by combining different skills with specific types of gear. Some examples:

\begin{itemize}
  \item A duelist might use \textbf{Rapport} with a rapier, turning insults and flourishes into distractions that act as boosts.
  \item A berserker could rely on \textbf{Physique} and heavy weapons to overwhelm foes, gaining bonuses when ignoring defence or attacking multiple opponents.
  \item A street brawler might combine \textbf{Deceive} with improvised weapons to create unexpected openings or feints.
  \item A monk could use \textbf{Will} to resist pain and channel inner focus into precise strikes.
\end{itemize}

The GM should encourage players to define how their fighting style works and reward creative combinations that match the character’s concept. A style should inform tactics and scene flavour, not just provide flat bonuses.

\subsection*{Style Counters and Technique Matchups}

To create a richer tactical space, you may define style interactions—some fighting styles are naturally strong or weak against others. For example:

\begin{itemize}
  \item \textbf{Iron Wall Style} (shield and spear, defensive posture) is effective against aggressive melee attackers but struggles against agile ranged foes.
  \item \textbf{Whispering Fang} (dagger and cloak, deception-based) excels at breaking enemy focus but is vulnerable to disciplined or intuitive fighters.
  \item \textbf{Stone Fist Boxing} (brute-force strikes) overpowers finesse-based styles but lacks adaptability against tricksters or feints.
  \item \textbf{Storm Serpent Form} (fluid motion, staff work) can counter slower styles, but is disrupted by grapplers or sudden aggressive charges.
\end{itemize}

These interactions do not need precise mechanics. Instead, treat them as situational modifiers, boosts, or justification for compelling outcomes in contested rolls. If one style clearly counters another in the fiction, grant the player a temporary boost or invoke a free aspect reflecting the advantage.

\subsection*{Style as Trait}

You may formalise a fighting style as a trait, such as:

\begin{itemize}
  \item \textbf{Trained in the Windblade School} — Gain +2 to create an advantage when using twin blades in open spaces.
  \item \textbf{Master of Red Lotus Fist} — Once per scene, ignore one point of damage when fighting unarmed.
  \item \textbf{Practitioner of the Twelve Strikes} — Gain a boost when successfully predicting and countering a known style.
\end{itemize}

As with other gear and character traits, these bonuses should be conditional and narratively grounded. A style becomes more meaningful when it shapes how a character approaches combat, not just what numbers they use.

\subsection*{Creating Your Own Styles}

Encourage players to invent styles suited to the setting. In a fantasy world, schools of swordplay may rival one another like noble houses. In modern settings, street-fighting techniques might evolve from urban subcultures. In sci-fi, martial forms might be adapted to zero-gravity or cybernetic bodies.

The goal is not to add complexity, but depth. A good fighting style helps define a character, enriches combat scenes, and offers opportunities for drama, rivalry, and growth.


\section{Designing Combat Encounters}

A good combat scene is more than a series of dice rolls. It should feel dynamic, cinematic, and full of opportunities for player creativity. In \wyrd, combat works best when it serves the story, engages the players' imagination, and gives everyone a chance to use their unique abilities. If every fight ends up as two characters exchanging blows until one runs out of boxes, something important is missing.

This section offers guidance on how to build more compelling encounters—ones that are not only balanced and mechanically interesting but also rich with narrative possibilities.

\subsection*{Leverage Traits and Narrative Hooks}

The simplest way to make combat more engaging is to ensure that the players’ traits are relevant. Each trait represents a part of the character’s identity or background. Design encounters where players can bring these traits into play:

\begin{itemize}
  \item A stormy rooftop chase where a trait like \textbf{Born on the Streets} might apply.
  \item A duel before a crowd where \textbf{Performer at Heart} can earn boosts through showmanship.
  \item A darkened tomb where a character with \textbf{Eyes Adjusted to the Dark} gains a crucial edge.
\end{itemize}

Encourage players to look for narrative justification to invoke their traits, and create situations where the fiction invites those connections. Even a simple skirmish can become memorable if it feels personal.

\subsection*{Terrain as a Tactical Resource}

Combat becomes more than trading attacks when the environment offers opportunities—and dangers.

Design the battlefield with features that can be used to gain advantage, such as:

\begin{itemize}
  \item \textbf{Cover}: Crates, statues, or vehicles that provide defensive bonuses.
  \item \textbf{Hazards}: Fires, cliffs, swinging chains, or unstable walkways that add tension.
  \item \textbf{Interactive objects}: Chandeliers, levers, crumbling walls, or magical artefacts.
  \item \textbf{Elevation or bottlenecks}: Platforms, narrow bridges, or spiral staircases that favour certain tactics.
\end{itemize}

Include aspects or situational advantages the players can discover or create—like “Loose Floorboards” or “Broken Balcony”—to encourage experimentation. Let clever use of the terrain grant boosts, free invokes, or even shift the course of battle.

\subsection*{Opponents With Personality}

Enemies should do more than just roll to hit. Make each foe feel unique by giving them:

\begin{itemize}
  \item \textbf{A defining trait or tactic}: e.g. “Shields of the Moon Guard” may always defend in formation.
  \item \textbf{A specific goal}: Instead of fighting to the death, maybe the villain is trying to escape, complete a ritual, or delay the players.
  \item \textbf{A weakness to discover}: An enemy may be immune to standard attacks but vulnerable to clever tactics or specific effects.
  \item \textbf{A dramatic flair}: Use monologues, emotional stakes, or surprise reinforcements to raise tension.
\end{itemize}

Opponents should also be capable of using the environment and creating their own advantages. A good enemy might throw a lantern to ignite the room, or use a grappling hook to flee across a rooftop.

\subsection*{Goals Beyond “Defeat All Enemies”}

If every combat ends when the last opponent falls, fights can feel repetitive. Introduce alternative or secondary objectives:

\begin{itemize}
  \item \textbf{Survive for a number of rounds} until backup arrives.
  \item \textbf{Protect a location or NPC} from waves of enemies.
  \item \textbf{Reach a lever, seal, or portal} while under fire.
  \item \textbf{Delay the enemy ritual} long enough for an ally to complete their task.
  \item \textbf{Retrieve an item} from the battlefield and escape.
\end{itemize}

Victory conditions that shift mid-fight—such as an enemy revealing a second form or reinforcements arriving—can also create surprise and momentum.

\subsection*{Use Boosts and Temporary Aspects}

Encourage players and enemies to create \textbf{boosts} and \textbf{temporary aspects}. These fleeting advantages make the flow of combat feel more dynamic and tactical.

Examples:
\begin{itemize}
  \item \textbf{Disarmed!} — After a clever create advantage action.
  \item \textbf{Pinned Behind Cover} — Created with a well-placed shot.
  \item \textbf{Thrown Off Balance} — A boost from a successful feint or trip.
\end{itemize}

By rewarding clever play with tangible benefits—even short-lived ones—you make the moment-to-moment action of combat more engaging.

\subsection*{Let the Players Shape the Fight}

Combat should never feel like the GM is simply executing a script. Let players influence the battlefield, shift the stakes, and change the conditions. Encourage actions like:

\begin{itemize}
  \item \textbf{Creating distractions} to split enemy forces.
  \item \textbf{Changing the environment}, such as plunging a room into darkness or collapsing a walkway.
  \item \textbf{Calling on allies} mid-fight through a trait or resource.
  \item \textbf{Escalating the situation}, e.g. drawing more guards, triggering alarms, or starting fires.
\end{itemize}

A combat scene becomes exciting when everyone at the table contributes ideas, builds on each other’s moves, and feels like they’re shaping the outcome together.

\subsection*{Escalation and Pacing}

Even well-designed fights can become stale if they drag on too long. Keep things moving by:

\begin{itemize}
  \item Tracking the fight’s \textbf{emotional stakes}—what changes if the players win or lose?
  \item Introducing \textbf{timed complications}, such as a door that must be unlocked while fighting.
  \item Raising the tension with \textbf{mid-combat twists}: reinforcements, betrayal, an unexpected monster.
  \item Letting enemies \textbf{retreat or surrender} if the tide turns.
\end{itemize}

Think of each combat as a narrative beat, not just a mechanical challenge. If the outcome no longer matters or the momentum is lost, consider wrapping up the scene with a concession or a dramatic finish.

\subsection*{Combat as a Conversation}

Finally, remember that combat in \wyrd is not a war game—it’s a storytelling conversation. The dice add suspense, but the story is what gives the fight meaning. The best combat encounters aren't just about who hits harder, but who risks something, who grows, and what changes because of it.


\chapter{Magic \& High Tech}

\DndDropCapLine{M}{agic}, and sufficiently advanced technology, can do anything you want. If only non-player characters (or monsters or gods or whathaveyou\ldots) have access to magic or high technology --- henceforth referred to as magic --- then a Game Master can often just decide by decree what magic can do. The powerlevel and capabilities will be whatever makes for a good story. But as soon as player characters need to interact with magic in any structured way, and \emph{especially} if they have access to magic themselves, then we need rules for what magic can do.






\part{One-Shot Scenarios}

A one-shot is a complete story told in a single session. Whether it's used for convention play, introducing new players, or exploring side stories, one-shots offer a focused, low-commitment narrative experience. Unlike ongoing campaigns or episodic series, they demand tight storytelling, strong hooks, and clear structure. Done well, they leave a lasting impact—and just enough mystery to haunt players after the final scene fades.

\section[What Makes One-Shots Unique]{What Makes\\ One-Shots Unique}

\subsection*{Contained Storytelling}
A one-shot begins and ends within a single session (typically 2–4 hours). There may be loose ends, but the central conflict must resolve within that time.

\subsection*{Limited Time, Focused Impact}
There is no room for sprawling subplots or excessive setup. Instead, the narrative must deliver immediate intrigue and fast escalation.

\subsection*{Character Simplicity}
Characters often have fewer complications or long-term arcs. Strong archetypes and clear motivations help players engage quickly.

\subsection*{Tone and Pacing}
One-shots often lean into strong tonal choices—horror, comedy, tragedy, or pulp adventure—because they don’t need to support tonal variation over time.

\section[Strengths and Limitations]{Strengths and\\ Limitations}

\subsection*{Strengths}
\begin{itemize}
    \item Easy to run with rotating or new players.
    \item Ideal for playtesting ideas, systems, or settings.
    \item Lower commitment encourages experimentation.
    \item Great for introducing your setting without overwhelming detail.
    \item Allows for high-impact, high-risk storytelling.
\end{itemize}

\subsection*{Limitations}
\begin{itemize}
    \item Limited time for character development or emotional depth.
    \item Harder to incorporate slow-build mysteries or subtle foreshadowing.
    \item Players unfamiliar with the system may need more support.
    \item Can feel rushed if poorly paced.
\end{itemize}

\section[Designing for One-Shots]{Designing for\\ One-Shots}

\subsection*{1. Strong Opening Hook}
Begin in the middle of the action or mystery. Skip slow setup—players should be invested within the first ten minutes.

\subsection*{2. Simple, Compelling Premise}
Keep the premise tight and clear. Examples:
\begin{itemize}
    \item A murder at midnight during a storm.
    \item An ancient vault opens for one night only.
    \item A ritual has begun—someone must stop it, or finish it.
\end{itemize}

\subsection*{3. Manageable Scope}
Limit the number of major NPCs, scenes, or factions. Three to five major beats is a good rule of thumb.

\subsection*{4. Clear Stakes and Urgency}
Give the players a reason to act now—time limits, escalating threats, or personal consequences.

\subsection*{5. Flexible Endings}
Prepare for a few possible outcomes, but don’t over-prepare. Be ready to adapt the resolution to the players’ choices.

\subsection*{6. Evocative Setting with Minimal Exposition}
Use bold, sensory descriptions. Establish tone with a few well-chosen details rather than lore dumps.

\section{Tools for Success}

\subsection*{Pre-Generated Characters}
Provide pre-built characters with short backstories, defined goals, and ties to the central conflict. This speeds up onboarding and ensures every PC has a stake in the story.

\subsection*{Scene Structure}
Use a modular outline:
\begin{itemize}
    \item \textbf{Scene I – The Hook:} Drop players into a mystery or conflict.
    \item \textbf{Scene II – Investigation or Complication:} Uncover clues or escalate the threat.
    \item \textbf{Scene III – Revelation or Confrontation:} Force a decision, battle, or twist.
    \item \textbf{Scene IV – Fallout or Resolution:} End on a strong note—closure, tragedy, victory, or a haunting question.
\end{itemize}

\subsection*{Running Tips}
\begin{itemize}
    \item Keep things moving—cut slow scenes quickly.
    \item Let players shape the tone and pace where possible.
    \item Use flashbacks or mid-game revelations to tie characters into the story.
    \item Embrace bold player decisions—they’re often the most memorable part of the session.
\end{itemize}


\section{Conclusion}

One-shots are like ghost stories told around a fire—brief, powerful, and unforgettable when done right. They reward clarity, creativity, and bold decisions. Whether you’re running a single night of suspense or laying the groundwork for something bigger, crafting a one-shot is an art worth mastering.

\part{Episodic Settings}

\chapter{The Grand Casebook}\label{chap:grand-casebook}

\begin{DndReadAloud}{}
\DndDropCapLine{L}{ondon}, 1896. A city of gaslit streets, towering factories, and secrets lurking in the shadows. This is an era of progress, where steam and steel reshape the world—but beneath the veneer of industry and refinement, the old mysteries remain. The line between science and the supernatural is thinner than most would dare to believe.

You are part of The Grand Society of Inquiry, a clandestine organisation of detectives, scholars, and unconventional thinkers dedicated to unravelling the mysteries the world would rather forget. The police may handle mundane crimes, but when the case is impossible, when the authorities turn a blind eye, or when the answers defy reason, that is where you come in.

The aristocracy hides more than it reveals. The city’s underworld knows whispers of truths the elite wish to bury. Strange happenings unfold in laboratories, occult circles, and long-forgotten ruins. It is your job to investigate, to bring truth to light—whether the world is ready for it or not.

You will encounter murderers whose motives defy logic, inventions beyond their time, secret societies vying for power, and horrors that exist just beyond the veil of reason. Some mysteries should never be solved, but you have chosen to chase the truth regardless.

London may not thank you for what you uncover. The truth is rarely comforting. But if not you, then who?

So, tell me: What mystery has found its way to your doorstep tonight?
\end{DndReadAloud}

\section{The Setting}

London in 1896 is a city of contradictions. At its heart lies a tension between progress and tradition, the rational and the arcane. Airships drift over soot-covered rooftops, automata assist in the factories, and steam-powered cabs rattle through the cobbled streets. Yet for all these marvels of industry, old fears still lurk in the fog. Ancient horrors persist in forgotten crypts, and whispers of the occult echo in gentlemen’s clubs and back alley gatherings.

This is a world where gaslight barely holds back the darkness, where rational minds struggle to explain the inexplicable. The Grand Casebook embraces the interplay between Victorian-era crime fiction, steampunk ingenuity, and the gothic supernatural.

\subsection{The Grand Society of Inquiry}

Founded in the wake of the Crimean War, The Grand Society of Inquiry was established by a coalition of scholars, detectives, and adventurers who recognised that certain mysteries lay beyond the reach of conventional authorities. Though their official purpose is to investigate "unusual" occurrences, they are as much a secret society as an investigative body. Their members come from all walks of life—former police officers, rogue academics, disgraced aristocrats, and those who have glimpsed the supernatural and can never return to ignorance.

The Society operates in secrecy, liaising with those who have knowledge of the unseen world—whether they be alchemists, mesmerists, or reformed criminals. Their headquarters, a sprawling archive hidden beneath a London bookshop, contains a wealth of esoteric knowledge that only a select few are permitted to access.

\subsection{The Powers That Be}

While the Society pursues truth, others work to obscure it. Various factions hold sway over London, each with their own stake in its mysteries:

\begin{itemize}
    \item \textbf{Scotland Yard:} The official enforcers of law and order, most officers dismiss the supernatural, though a handful of seasoned inspectors have learned otherwise. The Yard tolerates the Society only when their interests align.
    \item \textbf{The Ministry of Esoteric Affairs:} A shadowy government branch that monitors supernatural activity. Their agents operate with impunity, and their goals often clash with those of the Society.
    \item \textbf{The Order of the Silver Dawn:} An occultist cabal that seeks power through ritual and ancient knowledge. Some claim their origins stretch back to the alchemists of the Elizabethan court.
    \item \textbf{The Industrial Magnates:} The great industrialists of London have their own secrets, from illicit experiments to unspeakable dealings with forces beyond human comprehension.
    \item \textbf{The Underworld Syndicates:} Smugglers and thieves have always known the truth—London's alleys and docks are haunted by more than mere criminals.
\end{itemize}

\section{Types of Play}

The Grand Casebook is structured as an episodic mystery-driven setting, where each session presents a new case to unravel. While overarching plots may weave through multiple cases, each game is designed to be a self-contained investigation. The types of mysteries players may face include:

\begin{itemize}
    \item \textbf{Classic Crime:} Murders, thefts, and conspiracies with unexpected twists.
    \item \textbf{Scientific Anomalies:} Unstable inventions, rogue automata, and the consequences of reckless experimentation.
    \item \textbf{Supernatural Encounters:} Hauntings, curses, and beings that should not exist.
    \item \textbf{Political Intrigue:} Power struggles within the aristocracy, blackmail, and espionage.
    \item \textbf{Exploratory Adventures:} Venturing into forgotten catacombs, abandoned asylums, or hidden laboratories.
\end{itemize}

\subsection{Character Roles}

Players take on the roles of Society members, each bringing unique skills to the investigative team. Some possible roles include:

\begin{itemize}
    \item \textbf{The Detective:} A seasoned investigator skilled in deduction and intuition.
    \item \textbf{The Scientist:} A brilliant mind on the cutting edge of technological advancements.
    \item \textbf{The Occultist:} A scholar of the esoteric, familiar with arcane lore.
    \item \textbf{The Rogue:} A streetwise operative connected to the city’s underbelly.
    \item \textbf{The Aristocrat:} A well-connected socialite whose influence opens doors.
    \item \textbf{The Soldier:} A combat-trained veteran, ready to handle more physical threats.
\end{itemize}

\subsection{Rule Adaptations for This Setting}

The Grand Casebook modifies standard play to suit its unique blend of investigation, steampunk technology, and gothic horror. Some adjustments include:

\begin{itemize}
    \item \textbf{Stress and Wounds:} Psychological stress plays a more significant role, with lingering mental consequences affecting future investigations. You can leave out stresses and wounds entirely for most mystery adventures and simply act out any confrontation.
    \item \textbf{Tools of the Trade:} Players may access specialised investigative tools, such as clockwork analysers, ectoplasmic detectors, or enchanted relics.
    \item \textbf{Mystery Structure:} Cases follow a structured flow, focusing on gathering clues, making deductions, and confronting the truth.
    \item \textbf{Supernatural Threats:} Unnatural foes require specific knowledge or preparations to overcome, emphasising research as much as combat.
\end{itemize}

\section{Adventures}

The following adventures are aimed at 3-5 players and should take 2-4 hours to play. 

\input{content/episodic/grand-casebook/murder-at-the-brass-orchid}


\subsection{The Silent Courier}

The investigators are drawn into the case when the body of Henry Graves is discovered in the early hours of the morning; his pockets turned inside out except for the strange, untouched letter. The local police dismiss it as a robbery gone wrong, but those with a keen eye know better.

The players must follow the trail of clues left behind, track down those involved in the message’s delivery, and decipher the meaning of the letter. But they are not the only ones searching for the truth—dangerous individuals are watching their every move, determined to keep the past buried.

\subsubsection{Premise} 
A messenger is found dead in a foggy alley, clutching a letter sealed in an unknown cypher. The contents of the letter are clearly valuable—valuable enough to kill for. Who was the intended recipient, and what secret was worth a man’s life?

\subsubsection{What Really Happened} 
The messenger, Henry Graves, was delivering a coded message between two rival factions of a secret society. The letter contained evidence of a betrayal within their ranks. However, a third party, fearing exposure, intercepted the courier and silenced him before he could complete his task. The letter remains intact, but its sender and intended recipient remain a mystery—one the investigators must unravel before the killers strike again.











%\chapter{Text Boxes}
%
%The module has three environments for setting text apart so that it is drawn to the reader's attention. |DndReadAloud| is used for text that a game master would read aloud.
%
%\begin{DndReadAloud}
%  As you approach this module you get a sense that the blood and tears of many generations went into its making. A warm feeling welcomes you as you type your first words.
%\end{DndReadAloud}
%
%\section{As an Aside}
%The other two environments are the |DndComment| and the |DndSidebar|. The |DndComment| is breakable and can safely be used inline in the text.
%
%\begin{DndComment}{This Is a Comment Box!}
%  A |DndComment| is a box for minimal highlighting of text. It lacks the ornamentation of |DndSidebar|, but it can handle being broken over a column.
%\end{DndComment}
%
%The |DndSidebar| is not breakable and is best used floated toward a page corner as it is below.
%
%\begin{DndSidebar}[float=!b]{Behold the DndSidebar!}
%  The |DndSidebar| is used as a sidebar. It does not break over columns and is best used with a figure environment to float it to one corner of the page where the surrounding text can then flow around it.
%\end{DndSidebar}
%
%\section{Tables}
%The |DndTable| colors the even rows and is set to the width of a line by default.
%
%\begin{DndTable}[header=Nice Table]{XX}
%    Table head  & Table head \\
%    Some value  & Some value \\
%    Some value  & Some value \\
%    Some value  & Some value
%\end{DndTable}
%

\end{document}
