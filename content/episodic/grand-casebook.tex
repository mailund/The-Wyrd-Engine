\chapter{The Grand Casebook}\label{chap:grand-casebook}

\begin{WyrdSettingHeading}
    \DndDropCapLine{L}{ondon}, 1896. A city of gaslit streets, towering factories, and secrets lurking in the shadows. This is an era of progress, where steam and steel reshape the world—but beneath the veneer of industry and refinement, the old mysteries remain. The line between science and the supernatural is thinner than most would dare to believe.

    You are part of The Grand Society of Inquiry, a clandestine organisation of detectives, scholars, and unconventional thinkers dedicated to unravelling the mysteries the world would rather forget. The police may handle mundane crimes, but when the case is impossible, when the authorities turn a blind eye, or when the answers defy reason, that is where you come in.

    The aristocracy hides more than it reveals. The city’s underworld knows whispers of truths the elite wish to bury. Strange happenings unfold in laboratories, occult circles, and long-forgotten ruins. It is your job to investigate, to bring truth to light—whether the world is ready for it or not.

    You will encounter murderers whose motives defy logic, inventions beyond their time, secret societies vying for power, and horrors that exist just beyond the veil of reason. Some mysteries should never be solved, but you have chosen to chase the truth regardless.

    London may not thank you for what you uncover. The truth is rarely comforting. But if not you, then who?

    So, tell me: What mystery has found its way to your doorstep tonight?
\end{WyrdSettingHeading}

\section{The Setting}

London in 1896 is a city of contradictions. At its heart lies a tension between progress and tradition, the rational and the arcane. Airships drift over soot-covered rooftops, automata assist in the factories, and steam-powered cabs rattle through the cobbled streets. Yet for all these marvels of industry, old fears still lurk in the fog. Ancient horrors persist in forgotten crypts, and whispers of the occult echo in gentlemen’s clubs and back alley gatherings.

This is a world where gaslight barely holds back the darkness, where rational minds struggle to explain the inexplicable. The Grand Casebook embraces the interplay between Victorian-era crime fiction, steampunk ingenuity, and the gothic supernatural.

\subsection{The Grand Society of Inquiry}

Founded in the wake of the Crimean War, The Grand Society of Inquiry was established by a coalition of scholars, detectives, and adventurers who recognised that certain mysteries lay beyond the reach of conventional authorities. Though their official purpose is to investigate "unusual" occurrences, they are as much a secret society as an investigative body. Their members come from all walks of life—former police officers, rogue academics, disgraced aristocrats, and those who have glimpsed the supernatural and can never return to ignorance.

The Society operates in secrecy, liaising with those who have knowledge of the unseen world—whether they be alchemists, mesmerists, or reformed criminals. Their headquarters, a sprawling archive hidden beneath a London bookshop, contains a wealth of esoteric knowledge that only a select few are permitted to access.

\subsection{The Powers That Be}

While the Society pursues truth, others work to obscure it. Various factions hold sway over London, each with their own stake in its mysteries:

\begin{itemize}
    \item \textbf{Scotland Yard:} The official enforcers of law and order, most officers dismiss the supernatural, though a handful of seasoned inspectors have learned otherwise. The Yard tolerates the Society only when their interests align.
    \item \textbf{The Ministry of Esoteric Affairs:} A shadowy government branch that monitors supernatural activity. Their agents operate with impunity, and their goals often clash with those of the Society.
    \item \textbf{The Order of the Silver Dawn:} An occultist cabal that seeks power through ritual and ancient knowledge. Some claim their origins stretch back to the alchemists of the Elizabethan court.
    \item \textbf{The Industrial Magnates:} The great industrialists of London have their own secrets, from illicit experiments to unspeakable dealings with forces beyond human comprehension.
    \item \textbf{The Underworld Syndicates:} Smugglers and thieves have always known the truth—London's alleys and docks are haunted by more than mere criminals.
\end{itemize}

\subsection{Types of Play}

The Grand Casebook is structured as an episodic mystery-driven setting, where each session presents a new case to unravel. While overarching plots may weave through multiple cases, each game is designed to be a self-contained investigation. The types of mysteries players may face include:

\begin{itemize}
    \item \textbf{Classic Crime:} Murders, thefts, and conspiracies with unexpected twists.
    \item \textbf{Scientific Anomalies:} Unstable inventions, rogue automata, and the consequences of reckless experimentation.
    \item \textbf{Supernatural Encounters:} Hauntings, curses, and beings that should not exist.
    \item \textbf{Political Intrigue:} Power struggles within the aristocracy, blackmail, and espionage.
    \item \textbf{Exploratory Adventures:} Venturing into forgotten catacombs, abandoned asylums, or hidden laboratories.
\end{itemize}

\subsection{Character Roles}

Players take on the roles of Society members, each bringing unique skills to the investigative team. Some possible roles include:

\begin{itemize}
    \item \textbf{The Detective:} A seasoned investigator skilled in deduction and intuition.
    \item \textbf{The Scientist:} A brilliant mind on the cutting edge of technological advancements.
    \item \textbf{The Occultist:} A scholar of the esoteric, familiar with arcane lore.
    \item \textbf{The Rogue:} A streetwise operative connected to the city’s underbelly.
    \item \textbf{The Aristocrat:} A well-connected socialite whose influence opens doors.
    \item \textbf{The Soldier:} A combat-trained veteran, ready to handle more physical threats.
\end{itemize}

\subsection{Rule Adaptations for This Setting}

The Grand Casebook modifies standard play to suit its unique blend of investigation, steampunk technology, and gothic horror. Some adjustments include:

\begin{itemize}
    \item \textbf{Stress and Wounds:} Psychological stress plays a more significant role, with lingering mental consequences affecting future investigations. You can leave out stresses and wounds entirely for most mystery adventures and simply act out any confrontation.
    \item \textbf{Tools of the Trade:} Players may access specialised investigative tools, such as clockwork analysers, ectoplasmic detectors, or enchanted relics.
    \item \textbf{Mystery Structure:} Cases follow a structured flow, focusing on gathering clues, making deductions, and confronting the truth.
    \item \textbf{Supernatural Threats:} Unnatural foes require specific knowledge or preparations to overcome, emphasising research as much as combat.
\end{itemize}


%% TODO: more world building

\section{Adventures}

The following adventures are aimed at 3-5 players and should take 2-4 hours to play. 

\subsection{The Call to Adventure}

At the heart of every investigation lies The Grand Society of Inquiry, an esteemed and enigmatic organisation dedicated to the relentless pursuit of truth. Operating from the opulent halls of the Grand Hall, the society boasts a network of detectives, scholars, and specialists, each possessing a unique skill set vital to solving the most perplexing cases.

When a new case emerges, summons are discreetly dispatched to those deemed most suited for the task at hand. These messages—delivered via courier, pneumatic tube, or even through more esoteric means—call upon select members to assemble and uncover the mystery that awaits. No two groups are ever quite the same, for the \textbf{Grand Analytical Engine}, a vast and intricate steam-powered construct housed in the depths of the Grand Hall, determines the composition of each investigative team.

\begin{WyrdComment}{Framing The Call to Adventure}
	The setup for starting adventures is typical for episodic games where the players can vary from session to session. Having an explanation for why the characters vary from case to case means that no further in-game explanation is needed.
\end{WyrdComment}

%% TODO: Something about the shared structure to the adventures here (the form of mystery adventures)

%% TODO: get this ito the world description

%The Grand Analytical Engine
%
%This marvel of engineering, a hybrid of Babbage’s Analytical Engine and the finest advancements in mechanical computation, processes a staggering wealth of information. Data is fed into its whirring mechanisms by archivists and clerks, cross-referencing past cases, skills, and affiliations. The result: a meticulously curated team, assembled not by human intuition, but by the cold, logical precision of brass gears and punched cards. Whether by fate or by cold calculation, those summoned are invariably drawn into intrigue, danger, and the pursuit of justice.



\begin{WyrdScenarioHeading}{Murder at the Brass Orchid}
	The investigators are called to \textbf{The Brass Orchid}. The establishment is filled with wealthy patrons, performers, and staff—each with their own secrets to hide. The club’s reputation is at stake, and the clock is ticking before the police arrive to sweep things under the rug.

	The players must piece together the events of the evening, question patrons and staff, analyse the crime scene, and determine who had the means, motive, and opportunity to commit the crime. However, the deeper they dig, the more they realise that this murder is just the tip of the iceberg.

	\subsection*{Premise} 
	A high-society soirée at the exclusive cabaret, The Brass Orchid, is cut short when a well-connected financier is found dead in a locked room. The party was attended by the city's elite, but none saw the murder happen—or so they claim. The investigators must navigate a world of secrets, deception, and hidden rivalries to uncover the truth.

	\subsection*{What Really Happened} 
	\textbf{Beatrice Langley}, a hostess at The Brass Orchid, killed the financier, \textbf{Edward Mercer}, to protect herself from blackmail. Mercer had uncovered details about Beatrice’s past life and was threatening to expose her unless she paid a steep price. Desperate and out of options, she poisoned his drink and used the club’s pneumatic tube system to dispose of the evidence. However, a miscalculation led to certain clues being left behind.
\end{WyrdScenarioHeading}

\begin{WyrdGmTips}
	The suggested passive opposition rolls in the following are only that, suggestions. Feel free to adjust the difficulty based on the investigators' actions, skills, and the pace of the game. Remember that the goal is to keep the story moving forward, not to bog it down with unnecessary obstacles.
\end{WyrdGmTips}


\begin{WyrdFullNPC}[%
		name=Beatrice Langley,%
		description=The Orchid’s Most Enchanting Hostess,%
		float=!t%
	]{Beatrice Langley}
	
    \emph{A captivating hostess at the Brass Orchid, hiding a desperate past behind a charming smile.}
    
    \subsubsection*{Background:}
    Beatrice Langley built a reputation as one of the Brass Orchid’s most sought-after hostesses, but her true past is far less glamorous. Once entangled in dangerous affairs, she sought refuge in the club’s gilded halls, only to have her secrets catch up with her. When Edward Mercer threatened to expose her, she took the only way out she saw—murder.
    
    \subsubsection*{Skills:}
    \begin{itemize}
       \item \Expert: Deception
        \item \Skilled: Stealth, Persuasion
        \item \Novice: Awareness, Empathy, Etiquette
    \end{itemize}
    
    \subsubsection*{Traits:}
    \begin{itemize}
        \item \textbf{Charming Manipulator} – Gains a bonus when deceiving or misleading someone with her charms.
        \item \textbf{A Past Worth Killing For} – Once per session, may create an advantage related to her hidden past.
        \item \textbf{Desperate Measures} – Can reroll when acting under extreme pressure or life-threatening circumstances.
    \end{itemize}
\end{WyrdFullNPC}
\begin{WyrdFullNPC}[%
		name=Edward Mercer,%
		description=The Murder Victim,%
		float=!t%
	]{Edward Mercer}
	
    \emph{A cunning blackmailer who underestimated the desperation of those he ensnared.}
    
    \subsubsection*{Background:}
    Edward Mercer was well-known in London’s underworld for his talent for unearthing dirty secrets and using them to his advantage. He approached his victims with a cold, calculated patience, squeezing them for all they were worth. His last target, however, proved more dangerous than he anticipated—Beatrice Langley, a woman with too much to lose. He was found dead in his private lounge at the Brass Orchid, the victim of a locked-room murder.
\end{WyrdFullNPC}


\subsection{Act 0: Into the Fray}

At the Game Master's discretion, the summons to the \textbf{Grand Hall} may be role-played, allowing players to experience firsthand how the \textbf{Grand Society of Inquiry} assigns cases. A Society Official, with an air of quiet authority, presents the latest mystery: a locked-room murder at the prestigious \textbf{Brass Orchid}. The club’s owner, \textbf{Madame Yvette Duval}, contacted the Society in desperation, realising that only the most capable investigators could unravel the enigma before her reputation—and her clientele—are irreparably damaged.

The Brass Orchid remains under lockdown, but such restrictions cannot last indefinitely. Its wealthy and influential patrons will not tolerate confinement for long unless the authorities become involved. The pressure mounts: the investigators must reach the crime scene swiftly before key witnesses scatter and vital evidence is lost to time and subterfuge.

\begin{WyrdGmTips}
	In episodic settings, the \textbf{Call to Adventure} often renders such introductory scenes optional. However, in the first few sessions, as players familiarise themselves with the world, engaging in a scene outside the primary investigation can add depth and immersion. Receiving a case assignment is a natural opportunity to establish tone, introduce key NPCs, and reinforce the Society’s role in orchestrating these investigations.
\end{WyrdGmTips}

\subsection{Act 1: The Crime Scene}
\begin{WyrdFullNPC}[%
		name=Madame Yvette Duval,%
		description=The Brass Orchid’s Matron,%
		float=!t%
	]{Madame Yvette Duval}
	
    \emph{The esteemed owner of the Brass Orchid, a woman who knows the price of every secret whispered in her establishment.}
    
    \subsubsection*{Background:}
     Madame Duval built the Brass Orchid into London’s most exclusive cabaret and gambling house, catering to the city’s wealthiest and most influential figures. While she maintains a persona of graceful hospitality, she has survived in a cutthroat industry, using her intelligence and influence to navigate political and criminal circles alike.
    
    \subsubsection*{Skills:}
    \begin{itemize}
       \item \Expert: Persuasion
        \item \Skilled: Deception, Resources
        \item \Novice: Awareness, Etiquette, Insight
    \end{itemize}
    
    \subsubsection*{Traits:}
    \begin{itemize}
        \item \textbf{Silver-Tongued Schemer} – Gains a bonus when negotiating delicate matters or extracting information.
        \item \textbf{Web of Favors} – Once per session, call in a powerful favour from a well-connected patron.
        \item \textbf{A Whisper Can Kill} – Can reroll when leveraging blackmail or manipulating a dangerous individual.
    \end{itemize}
\end{WyrdFullNPC}



The investigators arrive at \textbf{The Brass Orchid}, where the air is thick with tension. The club’s usual vibrancy is subdued, with hushed murmurs among staff and patrons alike. A staff member, \textbf{Delilah "Della" Moreau}, was the first to notice something was amiss. Mercer, a regular patron, had not emerged from his private lounge as he typically would during the intermission. Concerned, she knocked on his door. When he did not respond, she fetched \textbf{Madame Yvette Duval}, the only person with an extra key to the private lounges. 

Upon unlocking the door, they were met with a grisly sight—Mercer’s lifeless body slumped in his chair, his drink half-finished, a scrap of paper clutched tightly in his hand. The room, untouched since their discovery, remains eerily undisturbed. Knowing that the scandal could ruin the Brass Orchid, Madame Duval took swift action. Rather than contacting the authorities, she turned to the \textbf{Grand Society of Inquiry}, summoning the investigators to handle the matter discreetly.

From the moment the body was found, no one has been allowed to enter the lounge—yet the investigators will soon find that secrets have a way of slipping through even the tightest of locks.


\subsubsection{Examining the Crime Scene}

Upon entering Mercer's private lounge, the investigators find the room frozen in time. A single lamp provides dim lighting, casting long shadows across the plush furniture. A card game sits abandoned at the table, with half-smoked cigars in an ashtray. The air is thick with the scent of liquor, tobacco, and a faint, lingering trace of something bitter—something off.

The body of Edward Mercer remains slumped in his chair, untouched since discovery. His expression is frozen in surprise, his grip unnaturally tight around a crumpled scrap of paper. The investigators are free to explore the scene, but careful examination will be required to extract meaningful details.


\begin{WyrdExplanation}{}
	\subsubsection*{Primary Clues}
	\begin{itemize}
		\item A \textbf{half-finished drink laced with poison}, still resting on the table near Mercer’s body. A faint almond scent lingers, barely noticeable beneath the overpowering aroma of brandy.
		\item The victim’s \textbf{missing pocket watch}, unaccounted for at the crime scene but later discovered in an unexpected location.
		\item A \textbf{scrap of torn paper}, crumpled tightly in Mercer’s hand, as though grasped in his final moments—either in desperation or as a final act of defiance.
		\item The \textbf{pneumatic tube system}, a hidden network connecting various parts of the club, shows signs of recent tampering.
	\end{itemize}

	\subsubsection*{What the clues reveal}
	\begin{itemize}
		\item \textbf{The poisoned drink} confirms the cause of death. The faint almond scent suggests cyanide or a similar fast-acting toxin but without an obvious delivery method.
		\item \textbf{The missing pocket watch}, later found in the servers' area, is not inherently suspicious—but its location is. It suggests that someone, likely a staff member, moved through that area after Mercer’s death. \textbf{Henry "Rigs" Rigby}, the bartender, recovered it but might need some persuasion to reveal the circumstances.
		\item \textbf{The scrap of torn paper} remains tightly clutched in Mercer’s hand. The jagged edge suggests it was ripped from a larger document. Whether Mercer seized it in a moment of panic or it was forcibly torn from him before he collapsed is unclear, but its contents might point to the motive.
		\item \textbf{The tampered pneumatic tube system} is the key to the locked-room mystery. It provides a discreet means of entry and escape, but only staff or someone intimately familiar with the club would know how to exploit it.
	\end{itemize}
\end{WyrdExplanation}

\begin{WyrdComment}{Investigating the Clues}
	\subsubsection*{The poisoned drink}
	Sitting on the table near Mercer's body, the glass contains a dark amber liquid, partially consumed. A faint almond scent lingers beneath the brandy's aroma.  
	\begin{itemize}
		\item \textbf{How to discover:} Simple observation will reveal the drink, but recognising the almond scent requires a \Basic \textbf{Notice} or \textbf{Investigate} check. Recognising this as the telltale scent of cyanide requires a success at the \Difficult level.
		\item \textbf{Further examination:} A character with medical knowledge may confirm cyanide poisoning, but testing the drink will require resources outside the club.
		\item \textbf{NPC reactions:} Madame Duval insists no one could have tampered with drinks \textbf{without her bartenders noticing}, subtly diverting suspicion.
	\end{itemize}
	
	\subsubsection*{The missing pocket watch}
	Mercer’s prized gold pocket watch is conspicuously absent from his body. 
	\begin{itemize}
		\item \textbf{How to discover:} Searching Mercer’s belongings will reveal its absence, but noticing the absence of something requires that you expect its presence. Any of the staff will know that Mercer always shows off his pocket watch, so if \textbf{Madam Duval is present} when the investigators examine the body, she will notice. Otherwise, have other NPCs drop hints about a watch later in the investigation.
	\end{itemize}
	
	\subsubsection*{The scrap of torn paper}
	Clutched tightly in Mercer’s lifeless hand, the small scrap appears hastily ripped from a larger document.
	\begin{itemize}
		\item \textbf{How to discover:} Anyone inspecting the body will notice the paper.
		\item \textbf{Further examination:} It is possible to pry the paper from Mercer's hands, but it must be done carefully to not tear it further.
		\item \textbf{What it reveals:} The scrap contains part of a name and a few words, possibly relating to Mercer's blackmail scheme.
		\item \textbf{NPC reactions:} Beatrice Langley, if questioned, will become visibly uncomfortable but will attempt to play innocent unless pressed. A \Basic \textbf{Empathy} roll will reveal her discomfort. At \Formidable or higher, the investigators will recognise her emotions as fear.
	\end{itemize}
	
	\subsubsection*{The tampered pneumatic tube system}
	A discreet brass panel built into the wall leads to the club’s internal message system, normally used for sending notes and receipts between staff areas. However, someone has recently pried it open, and subtle modifications suggest it was used for more than just correspondence.

	\begin{itemize}
		\item \textbf{How to discover:} Searching the walls near Mercer's table reveals (with a \Difficult \textbf{Investigate} or \textbf{Notice}) that the panel is slightly ajar, its edges scratched as if it was hastily forced open. Staff might mention the system in passing if prompted.
		\item \textbf{Further examination:} A \Challenging \textbf{Crafts} check confirms that the panel has been modified. At \Difficult \textbf{Craft} the examination reveals that the usual constraints, meant to restrict messages to small notes, have been bypassed—suggesting something larger was sent through. Additionally, the airflow mechanism appears to have been overridden, allowing the tube to function more like a one-way transport chute rather than a message system.
		\item \textbf{What it reveals:} The system connects to the servers' area, and traces of fine fabric fibres or a stray hair inside the tube hint that it was used to transport something—or someone. A close look reveals faint scuff marks on the panel's interior, possibly left by someone squeezing through.
		\item \textbf{NPC reactions:} Most staff will dismiss the idea that a person could fit inside, but a seasoned investigator might realise that someone \textbf{small or desperate} could have used the system as an escape route.
	\end{itemize}
\end{WyrdComment}

After examining the crime scene, the investigators are free to explore the Brass Orchid in search of clues. The investigation takes place in Act 2, where they will question staff and patrons, analyse testimonies, and piece together the events of the evening. The order in which they explore the locations in the next act is up to them.

\begin{WyrdGmTips}
	The mystery is designed to be straightforward, making it ideal for new players still learning the rules and getting comfortable with investigative roleplay. However, you can easily heighten the challenge by introducing conflicting testimonies from staff and patrons, forcing them to untangle half-truths, personal biases, and hidden agendas as they piece together what really happened that night.
\end{WyrdGmTips}

\subsection{Act 2: The Investigation} 

Players must navigate the web of lies surrounding the Brass Orchid’s elite clientele and staff. Key locations include:
\begin{WyrdExplanation}[Key Locations]
	\begin{itemize}
		\item \textbf{The performers’ dressing rooms}, where whispers of illicit affairs and secret dealings emerge.
		\item \textbf{The club’s bar}, where a bartender, \textbf{Henry ``Rigs'' Rigby}, may know more than he lets on.
		\item \textbf{The back office}, where financial records hint at Mercer’s recent blackmail attempts.
	\end{itemize}
\end{WyrdExplanation}

\begin{WyrdGmTips}
	Encourage players to interact with the environment beyond skill rolls—describe how their characters examine the clues, interpret body language, and make logical leaps. If they become stuck, use an NPC to nudge them toward a promising line of inquiry rather than outright giving answers.
\end{WyrdGmTips}

\noindent
A chase scene or social confrontation may occur if a suspect attempts to flee or cover up crucial evidence. The club’s owner, \textbf{Madame Yvette Duval}, will insist on discretion, urging players to avoid drawing attention.

\begin{WyrdFullNPC}[%
		name=The Staff of the Brass Orchid,%
		description=The people who keep the Orchid running,%
		float=!t%
	]{The Staff of the Brass Orchid}
	
    \emph{A well-oiled machine of waiters, bartenders, entertainers, and security staff, all working under the careful watch of Madame Duval. Each has their own secrets—and some know more about the murder than they let on.}
    
    \begin{WyrdGmTips}[color=bgtan]
    These are all Mook NPCs but can be fleshed out more (as Henry "Rigs" Rigby) if necessary. Most likely, you will not need their stats, though, but can use them simply to add flavour and drop clues and hints.
    \end{WyrdGmTips}
    
    \subsubsection*{Henry "Rigs" Rigby}
    Bartender – Henry “Rigs” Rigby has served drinks and collected secrets at the Brass Orchid for years. A man with a sharp eye and a sharper tongue, he knows how to keep patrons talking—especially when they’ve had a few too many.
    \textbf{What He Knows:} Rigs found Edward Mercer’s missing watch tucked behind the bar, likely dropped by someone in a hurry. He can also confirm that Beatrice Langley was seen speaking with Mercer earlier in the evening.

    
	\subsubsection*{Delilah "Della" Moreau}
	Head Hostess --- Poised and perceptive, Della keeps track of the Orchid’s clientele and ensures the staff stay in line.  
	\textbf{What She Knows:} Della saw Beatrice Langley looking distraught before her performance. She had a brief but intense exchange with Mercer in a quiet corner of the club, after which she rushed backstage, visibly shaken.

	\subsubsection*{Theo Finch}
	Croupier – A professional gambler with a silver tongue, Theo oversees the Orchid’s high-stakes tables.
	\textbf{What He Knows:} Theo recalls that Mercer was in high spirits that evening, boasting about his luck finally turning. He also saw him flash a folded note to Beatrice at the bar—something that made her go pale.

	\subsubsection*{Lucian "Lucky" Calloway}
	Security Chief – A former prizefighter turned bouncer, Lucky ensures that troublemakers are swiftly escorted out.
	\textbf{What He Knows:} Lucky was managing a rowdy patron at the time of the murder and didn't see much, but he did notice Beatrice leaving Mercer’s booth looking like she'd seen a ghost. He suspects there was more to their history than either let on.

	\subsubsection*{Marguerite "Maggie" Lavoie} 
	Cabaret Singer – The star performer at the Orchid, known for her breathtaking voice and her ability to read a room.
	\textbf{What She Knows:} Maggie saw Beatrice backstage, trembling before going onstage. She also overheard her muttering about someone “never leaving well enough alone” before she downed a glass of champagne and stormed off.
\end{WyrdFullNPC}

\begin{WyrdFullNPC}[%
		name=The Patrons of the Brass Orchid,%
		description=The elite clientele of the Orchid,%
		float=!t%
	]{The Patrons of the Brass Orchid}
	
    \emph{An exclusive mix of aristocrats, socialites, and shadowy figures seeking entertainment, influence, or illicit dealings. Many were present on the night of the murder—some more observant than others.}
        
	\subsubsection*{Lord Alistair Pembroke}
	Wealthy Industrialist – A steel magnate with an imposing presence, known for both his ruthless business tactics and his excessive gambling.  
	\textbf{What He Knows:} Pembroke had his own business to attend to at the tables, but he noticed Mercer acting smug and confident, calling for more drinks and toasting to "fortune smiling at last." 

	\subsubsection*{Genevieve Ashdown}
	Scandalous Socialite – A striking woman with a penchant for dangerous liaisons and whispered rumours. She thrives on court intrigue and nightlife gossip.
	\textbf{What She Knows:} Genevieve witnessed Beatrice and Mercer talking earlier in the night. She saw Beatrice grab Mercer’s wrist in desperation, pleading in hushed tones before Mercer simply laughed and pulled away.

	\subsubsection*{Dr. Elias Forsythe}
	Respected Physician – A surgeon with a growing reputation, attending the Orchid to enjoy his vices away from prying eyes.
	\textbf{What He Knows:} Dr. Forsythe noticed Beatrice downing a drink with shaking hands just before leaving for Mercer’s booth. He was too focused on his own affairs to linger, but he distinctly recalls her expression—not anger, but sheer dread.
\end{WyrdFullNPC}


\subsubsection{The Performers’ Dressing Rooms}
A backstage sanctuary for the Brass Orchid’s entertainers, the dressing rooms are filled with the scent of perfume, powder, and secrets. Between costume changes and whispered conversations, this space holds clues about hidden relationships, illicit affairs, and last-minute confrontations. If anyone saw Beatrice Langley before the murder, it would have been here.

Speaking with the club’s performers, the investigators learn that \textbf{Mercer and Beatrice} have been spending an unusual amount of time together lately. The prevailing gossip suggests an affair, though many find this unlikely—such a scandal would not go unnoticed, and \textbf{Madame Duval} would never tolerate it. Still, secrets have a way of slipping through even the most watchful eyes…


If the investigators take the time to search the dressing rooms carefully, they can uncover additional clues that paint a clearer picture of Beatrice’s state of mind before the murder:  

\begin{WyrdComment}{Clues to Discover}
	\begin{itemize}
		\item \textbf{Beatrice’s Travel Bag:} A half-packed bag in her dressing room suggests she was preparing to leave in haste. Its hurried state implies she either abandoned the plan or ran out of time.
		\item \textbf{A Torn Letter:} A small stove used to heat the performers' dressing room contains scraps of partially burned paper that can be spotted with a \Challenging \textbf{Notice} roll.
		A \Difficult \textbf{Notice} or \textbf{Crafting} reveals that the paper matches the torn note found in Mercer’s hand. If pieced together, it may hint at the nature of their final confrontation.
		\item \textbf{Testimonies from Performers:} Some performers recall Beatrice arriving shaken before her performance, while others remember her slipping away after intermission. None, however, can say where she went.
	\end{itemize}
\end{WyrdComment}


\subsubsection{The Club’s Bar}  
A bustling hub of conversation and vice, the club’s bar is where fortunes are won and lost, secrets change hands, and alliances are forged over a well-poured drink. The air is thick with the mingling scents of brandy, cigars, and ambition. At the centre of it all stands \textbf{Henry "Rigs" Rigby}, a bartender with an ear for whispers and a knack for knowing when to keep his mouth shut. He’s seen it all—but getting him to share what he knows will require a delicate touch or a not-so-subtle push.

As the investigators enter, they catch a glimpse of \textbf{Rigs hurriedly slipping something into his pocket}. Keen-eyed characters may notice a \textbf{hint of gold} flashing before it disappears (a \textbf{Notice} roll at \Difficult will confirm this). It’s \textbf{Mercer’s pocket watch}, and Rigs isn’t keen on explaining how he came by it. At first, he’ll feign ignorance, but a successful \textbf{Interrogate}, \textbf{Intimidate}, or \textbf{Rapport} roll at \Formidable will loosen his tongue—grudgingly.

\begin{WyrdComment}{Clues to Discover}  
	\begin{itemize}  
		\item \textbf{Mercer’s Missing Pocket Watch:} Rigs found it in the servers’ area after the murder, where Beatrice likely dropped it in her rush to escape. He will only admit this if pressured.  
		\item \textbf{Unsettled Debts:} A bar ledger records Mercer’s outstanding tabs—far higher than usual. However, in the past few weeks, he had been paying off large amounts, suggesting a new source of income.  
		\item \textbf{Patron Gossip:} Some recall Mercer speaking privately with Beatrice earlier that night, while others overheard him boasting about a “big payday” that was going to change everything.  
		\item \textbf{The Pneumatic Tube Exit:} The bar’s pneumatic system, normally used to deliver drinks to private lounges, has a discreet access point beneath the counter. Investigators examining it will find \textbf{signs of forced entry}—a clear indication of tampering.  
	\end{itemize}  
\end{WyrdComment}  

\begin{WyrdExplanation}[How Rigs Found the Watch]
	\textbf{Henry "Rigs" Rigby} insists he had nothing to do with Mercer’s murder—he just found himself in the wrong place at the wrong time. After the body was discovered and the club went into lockdown, he was doing his usual post-shift rounds behind the bar when something \textbf{caught his eye}. A glint of gold beneath the service counter near the \textbf{servers’ area}.  

	Curiosity got the better of him. He bent down, fished it out, and immediately wished he hadn’t. It was \textbf{Mercer’s pocket watch}. Rigs didn’t know how it had ended up there, but he did know one thing: he wanted no part of it. If word got out, he’d be dragged into the mess, and he had no interest in becoming a suspect. So, he \textbf{shoved it into his pocket} and carried on pouring drinks, hoping no one would notice.  

	Under pressure, Rigs will admit that he found the watch \textbf{near the entrance to the back hall}, close to where servers pick up drinks for the private lounges. This strongly suggests that \textbf{someone—likely staff or someone who knew their way around—moved through that area after Mercer’s death}. More importantly, the way it was found indicates that it had been \textbf{dropped in a hurry}, likely by someone escaping through the \textbf{pneumatic tube system}.  

	If investigators push him further, Rigs will recall something odd: \textbf{he heard a soft thud from the back hall minutes before he found the watch}. At the time, he thought nothing of it, assuming it was a staff member shifting crates. But now, in hindsight, it might have been the sound of someone landing after climbing out of the pneumatic system.  
\end{WyrdExplanation}  

\begin{WyrdExplanation}[The Pneumatic System]
	Beneath the bar, tucked behind a row of gleaming brass pipes and aged mahogany panelling, lies a \textbf{discreet access point} to the club’s \textbf{pneumatic tube system}. Normally, these tubes are used to send drink orders, notes, and discreet payments between the private lounges and the bar, but this particular panel has been \textbf{forcibly pried open}. The latch usually kept flush with the wall, is now bent slightly out of shape as if someone had wrenched it open in haste.  

	Upon closer inspection (\Challenging \textbf{Investigate}), \textbf{faint scratches} on the brass lining suggest that something—or someone—was pulled through recently. A \textbf{thin layer of dust} clings to the inner rim of the tube, disturbed in streaks where fingers or fabric may have brushed against it. Investigators with a \textbf{keen eye} may notice a \textbf{small shard of glass} caught between the tubing's metal framework, its edges glistening under the low bar light. If examined, it matches the \textbf{broken vial} found at the exit point, the lingering scent of bitter almonds confirming its deadly purpose.  

	The tube itself is narrow, \textbf{just large enough for a slender person to squeeze through}. A metal \textbf{service ladder} is affixed to the interior, meant for maintenance workers to access the system when needed. However, one of the lower rungs has been bent, possibly from the weight of someone climbing through in a hurry. Looking deeper inside, investigators can see where the \textbf{tube splits}, with one passage continuing toward the back hall and another leading \textbf{upward}, toward the private lounges—including Mercer’s.  	
\end{WyrdExplanation}


Any investigator willing to \textbf{crawl inside} will find it claustrophobic, the \textbf{walls cool and slick} from years of condensation. The air carries a \textbf{faint metallic tang}, mingled with the stale scent of old receipts and spilt brandy. A \Difficult \textbf{Notice} roll will reveal that a few \textbf{scraps of paper} cling to the corners of the passage, suggesting messages were hurriedly sent or torn up mid-transit. If they push forward, they may notice \textbf{a single dark thread caught on a rivet}—a clue that someone in dark clothing passed through recently.  

This passage is the key to unravelling \textbf{how the killer escaped the locked room}, but whether the investigators are willing to \textbf{follow the same route} remains to be seen…  

\begin{WyrdComment}{What Can Be Found in the Pneumatic System}  
	\begin{itemize}
		\item \textbf{Signs of tampering:} A bent latch, scratches, and disturbed dust suggest recent use.
		\item \textbf{A broken glass vial shard:} Found inside the tube, confirming poison use.
		\item \textbf{A service ladder with a bent rung:} Indicates someone climbed through in haste.
		\item \textbf{A split passage:} One leading toward the back hall, the other to the private lounges.
		\item \textbf{Traces of the killer’s passage:} A dark thread caught on a rivet, scattered paper scraps.
	\end{itemize}
\end{WyrdComment}

\subsubsection{The Back Office}  
Tucked away behind a locked door, the back office is where the club’s finances are managed, and sensitive dealings are recorded. The ledgers here reveal an interesting financial pattern. Mercer had accrued a \textbf{significant gambling debt} at the Brass Orchid over the past year—yet, in the past few weeks, he had begun paying it off in unusually large sums. Where did the money come from?  

\begin{WyrdComment}{Clues to Discover}  
	\begin{itemize}  
		\item \textbf{Financial Records:} The ledgers show that Mercer has made \textbf{several large payments} on his debt, suggesting he had recently come into a substantial amount of money. If the investigators follow this trail, they will discover that the timing aligns suspiciously with the time when \textbf{Beatrice} started spending substantially more time with him.  
	\end{itemize}  
\end{WyrdComment}  






\subsection{Act 3: The Reveal}  

With all the pieces in place, the investigators must confront \textbf{Beatrice Langley}. She is visibly shaken when accused but clings to her innocence, insisting that she had \textbf{nothing to do with Mercer’s death}. However, as the investigators present their findings, cracks begin to show in her story.  

\begin{WyrdComment}{Evidence That Breaks Her Resolve}  
	\begin{itemize}  
		\item \textbf{Traces of poison:} A broken glass vial, found near the pneumatic tube exit, contained the same poison that killed Mercer. Traces of the toxin linger on Beatrice’s clothing.  
		\item \textbf{Witness testimonies:} Multiple staff members recall Beatrice acting erratically—arriving shaken, disappearing after intermission, and returning only once the club was in an uproar.  
		\item \textbf{The torn letter:} Fragments of a document, partially burned in the dressing room stove, match the scrap found clutched in Mercer’s hand—evidence of a final desperate message. \textbf{Witnesses will testify} that Beatrice added fuel to the stove a short time before the murder scene was discovered.
		\item \textbf{The missing pocket watch:} Dropped in the servers’ area after she fled through the pneumatic tube; its location exposes her escape route.  
		\item \textbf{Inconsistencies in her alibi:} She initially claimed she was in her dressing room before and after her performance, but no one can confirm seeing her at the critical moment.  
	\end{itemize}  
\end{WyrdComment}  

\noindent  
Faced with undeniable proof, Beatrice’s composure crumbles. If the investigators press her with a firm but measured approach, she may confess outright, revealing the truth about Mercer’s blackmail and the desperate decision that led to his death.  

However, if they push too aggressively or fail to secure a clear confession, Beatrice panics. She makes a break for the nearest exit—whether attempting to vanish into the crowd, lock herself in her dressing room, or even slip through the pneumatic tubes one last time. This could lead to a tense chase or a final dramatic confrontation as the investigators must decide whether to apprehend her themselves or alert the authorities before she disappears into the night.  

\begin{WyrdGmTips}  
	If you want to add tension, Beatrice’s flight can turn into a frantic pursuit through the back halls of the Brass Orchid, with obstacles such as locked doors, security guards, or even club patrons unwittingly getting in the way. A climactic moment could see her cornered on a balcony, deciding whether to surrender or make a desperate escape attempt.  
\end{WyrdGmTips}  

\subsection{Resolutions} 
Depending on how the investigators handle the case, different outcomes may occur:
\begin{itemize}
	\item \textbf{Justice Served}: Beatrice is arrested or confesses, ensuring the truth is revealed.
	\item \textbf{A Deal in the Shadows}: The investigators allow Beatrice to flee, leveraging her knowledge for future gain.
	\item \textbf{The Wrong Culprit}: A scapegoat is framed, or the authorities arrest someone else entirely.
	\item \textbf{A Mystery Unsolved}: The players fail to piece everything together, leaving The Brass Orchid haunted by unanswered questions.
\end{itemize}

Regardless of the resolution, this case's events ripple across London’s elite, setting the stage for future intrigues.

\begin{WyrdScenarioHeading}{The Clockmaker’s Deception}
    A shocking murder has thrown London’s scientific and industrial circles into disarray. The esteemed inventor, \textbf{Dr Sebastian Thorne}, stands accused of killing a rival engineer, \textbf{Arthur Bellamy}, who was found dead in Thorne’s workshop. The evidence against him seems irrefutable—Bellamy’s body was discovered with blunt force trauma, and the only witness claims that one of Thorne’s own clockwork creations struck the fatal blow.

    But something about the case doesn’t add up. The mechanical automaton, a prototype designed to assist in fine-detail engineering, should be incapable of such an act. Was this an unfortunate accident, or has someone manipulated the scene to frame Thorne? The investigators must untangle the mystery before the city condemns a man who may be innocent—or worse, before a hidden truth shakes the foundations of science itself.

    \subsection*{Premise} 
    A renowned inventor is accused of murder when his latest clockwork creation is found standing over a dead body. The case seems open and shut, but a deeper conspiracy lurks beneath the surface. Was the machine truly responsible, or is someone using technology as a convenient scapegoat?

    \subsection*{What Really Happened} 
    Arthur Bellamy had uncovered a secret—one that threatened powerful interests within London’s scientific community. He arranged a meeting with Thorne under the guise of a professional discussion, intending to share his findings. However, before he could reveal the full truth, an unknown party silenced him.

    The real killer staged the scene, positioning Thorne’s automaton as the culprit. By tampering with the machine’s mechanisms and manipulating witnesses, they ensured that suspicion would fall on Thorne. Now, as the city rushes to condemn him, the investigators must uncover the true murderer, reveal the secret Bellamy died for, and navigate the dangerous underworld of industrial espionage.
\end{WyrdScenarioHeading}

\begin{WyrdGmTips}
    As with the previous scenario, you can act out the summoning to \textbf{The Grand Society of Inquiry} as a way to introduce the investigators to the case. If the set of player characters in this scenario differs from the player characters in the previous one, this would give you an excellent way of introducing the new characters to the players.
\end{WyrdGmTips}

\begin{WyrdGmTips}
    This case provides an excellent opportunity to explore themes of scientific advancement, ethical dilemmas, and the fear of technology gone rogue. The case may also lead into larger conspiracies within London's industrial elite, depending on how deep the investigators choose to dig.
\end{WyrdGmTips}

\subsection{Act 1: The Accusation}  
The investigators are summoned to the scene of the crime—the locked workshop of Dr Thorne. The city’s authorities have already decided his guilt, but the inconsistencies in the case suggest a deeper truth.

\begin{WyrdExplanation}[Key Elements of Act 1]
    \begin{itemize}
        \item \textbf{Examining the Crime Scene:} Bellamy was struck down in Thorne’s workshop. The automaton is positioned near the body, but no command sequence should have allowed it to act violently.
        \item \textbf{The Automaton:} A marvel of engineering, yet it lacks any known capacity for independent action. Its gears and actuators show signs of tampering.
        \item \textbf{Thorne’s Testimony:} The accused swears he is innocent, claiming he was in another room when the murder occurred.
        \item \textbf{The Witness:} A factory worker insists he saw the automaton move on its own to deliver the fatal strike. But is he telling the full truth?
    \end{itemize}
\end{WyrdExplanation}

\noindent
With the evidence stacked against Thorne, the investigators must uncover what really happened in the workshop that night.

\subsection{Act 2: The Hidden Conflict}  
As the investigation deepens, the players discover that Bellamy’s death was not a simple case of mechanical failure—it was a carefully orchestrated act of sabotage.

\begin{WyrdExplanation}[Key Elements of Act 2]
    \begin{itemize}
        \item \textbf{Bellamy’s Discovery:} The victim had uncovered something significant—plans, a prototype, or a hidden truth that made him a target.
        \item \textbf{The Secret Rivalry:} The industrial elite of London are at war behind closed doors. Bellamy and Thorne were both entangled in a larger battle over technological supremacy.
        \item \textbf{The Sabotaged Automaton:} Someone tampered with the machine’s internal mechanisms. If the players investigate closely, they may find evidence of deliberate reprogramming or mechanical interference.
        \item \textbf{A Race Against Time:} The longer the investigators take, the more pressure mounts to convict Thorne. Influential figures want the case closed quickly, and the truth buried.
    \end{itemize}
\end{WyrdExplanation}

\noindent
By the end of Act 2, the investigators should have a suspect—but proving their guilt will require uncovering their true motive.

\subsection{Act 3: The Mastermind Revealed}  
With all the pieces in place, the investigators must expose the true murderer before Thorne is sentenced.

\begin{WyrdExplanation}[Key Elements of Act 3]
    \begin{itemize}
        \item \textbf{The True Killer:} A rival inventor? A corrupt businessman? Or someone from Thorne’s own inner circle?
        \item \textbf{The Motive:} Bellamy’s research, a dangerous secret, or industrial sabotage? What truth was worth killing for?
        \item \textbf{The Final Confrontation:} The players must gather the final proof, present their case, or prevent another murder before the truth is lost forever.
    \end{itemize}
\end{WyrdExplanation}

\subsection{Resolution: Justice or Cover-Up?}  
The players’ choices will determine the final outcome:

\begin{itemize}
    \item \textbf{If Thorne is cleared:} He is freed, but powerful enemies remain.
    \item \textbf{If the killer is exposed:} The consequences will depend on their connections—justice may not always be served.
    \item \textbf{If the truth is buried:} The industrial elite breathe a sigh of relief, but the players leave knowing they only scratched the surface of something far larger.
\end{itemize}

One thing is certain: the march of progress is unstoppable, but the cost of invention is often paid in blood.
\begin{WyrdScenarioHeading}{The Silent Courier}
    The investigators are drawn into the case when the body of \textbf{Henry Graves} is discovered in the early hours of the morning; his pockets turned inside out except for the strange, untouched letter. The local police dismiss it as a robbery gone wrong, but those with a keen eye know better.

    The players must follow the trail of clues left behind, track down those involved in the message's delivery, and decipher the meaning of the letter. But they are not the only ones searching for the truth—dangerous individuals are watching their every move, determined to keep the past buried.

    \subsection*{Premise} 
    A messenger is found dead in a foggy alley, clutching a letter sealed in an unknown cypher. The contents of the letter are clearly valuable—valuable enough to kill for. Who was the intended recipient, and what secret was worth a man's life?

    \subsection*{What Really Happened} 
    The messenger, Henry Graves, was delivering a coded message between two rival factions of a secret society. The letter contained evidence of a betrayal within their ranks. However, a third party, fearing exposure, intercepted the courier and silenced him before he could complete his task. The letter remains intact, but its sender and intended recipient remain a mystery—one the investigators must unravel before the killers strike again.
\end{WyrdScenarioHeading}



\subsection{Act 1: The Body and the Letter}  
The investigators arrive at the crime scene—a foggy alley where Henry Graves was found dead. The police have ruled it a botched robbery, but subtle inconsistencies suggest otherwise.  

\begin{WyrdExplanation}[Key Elements of Act 1]
    \begin{itemize}
        \item \textbf{Examining the Crime Scene:} Players can search for physical evidence—how was Graves killed? What does the positioning of his body suggest?
        \item \textbf{The Letter:} The only item left untouched in his possession, written in an unfamiliar cipher. Why was it spared when everything else was taken?
        \item \textbf{Witnesses and Leads:} The investigators may find someone who heard or saw something—a vagrant, a night watchman, or a fellow courier. Their accounts might be fragmented, but they hint at someone following Graves before his death.
        \item \textbf{The Silent Pursuers:} A subtle but key element—players may not realize it yet, but they are being watched. The moment they take an interest in the case, their names are added to the list of people who know too much.
    \end{itemize}
\end{WyrdExplanation}

\noindent
Once the investigators realize this was no ordinary mugging, the mystery broadens. Who was Henry Graves delivering the letter to, and what was so important that it was worth his life?

\subsection{Act 2: The Trail of Secrets}  
Following leads from Act 1, the investigators begin piecing together Graves' movements before his death. His route suggests he was in contact with powerful individuals who rarely leave behind traces.  

\begin{WyrdExplanation}[Key Elements of Act 2]
    \begin{itemize}
        \item \textbf{Tracking the Letter’s Origin:} Discovering who wrote the letter is just as crucial as finding its recipient. The players must investigate Graves' recent commissions and any known associates.
        \item \textbf{The Rival Factions:} As the investigation deepens, it becomes clear that the letter is tied to a schism within a secretive society. Who is working against whom, and what information was in the letter?
        \item \textbf{Attempts to Stop the Investigation:} By this point, the players will have drawn attention. Shadowy figures may approach them with offers, threats, or outright attempts on their lives.
        \item \textbf{A Key Betrayal:} An NPC the investigators have relied on may be compromised, leading to a moment where the players question who they can trust.
    \end{itemize}
\end{WyrdExplanation}

\noindent
At the end of Act 2, the players should be closing in on the recipient of the letter. However, the conspiracy is still one step ahead, and the final piece of the puzzle remains missing—the full contents of the letter.

\subsection{Act 3: The Truth Unveiled}  
The final act sees the investigators face their most dangerous challenge yet. The true nature of the letter is revealed, and they must decide what to do with it.

\begin{WyrdExplanation}[Key Elements of Act 3]
    \begin{itemize}
        \item \textbf{The Letter’s Recipient:} At last, the players find the person who was meant to receive the letter. But will they be an ally, or do they have their own agenda?
        \item \textbf{The Real Enemy:} The true mastermind behind the murder emerges—was it a rogue faction leader, a powerful noble, or someone much closer than the players realized?
        \item \textbf{The Final Confrontation:} Whether it’s a chase, a duel of words, or a desperate escape, the players must navigate the resolution carefully. The wrong choice could cost them their lives.
        \item \textbf{The Fate of the Letter:} The letter contains damning evidence—exposing corruption, revealing a dangerous truth, or holding the key to an even larger mystery. What the players choose to do with it will shape the story’s aftermath.
    \end{itemize}
\end{WyrdExplanation}

\subsection{Resolution: The Consequences of Truth}  
The outcome of the scenario depends on how the investigators handle the final confrontation and the letter itself:

\begin{itemize}
    \item \textbf{If the letter is destroyed:} The conspiracy continues, but the players may have made powerful enemies or secret allies.
    \item \textbf{If the letter is revealed:} The truth spreads, but at what cost? Some factions may fall, others may rise, and new threats may emerge.
    \item \textbf{If the letter is delivered to its intended recipient:} The consequences will depend on who the recipient truly is and whether they were acting in good faith.
\end{itemize}

No matter the resolution, one thing is certain: \textbf{The Silent Courier} was only the beginning.


%% TODO: Add three more adventures
