\subsection{Act 1: The Crime Scene}
\begin{WyrdFullNPC}[%
		name=Madame Yvette Duval,%
		description=The Brass Orchid’s Matron,%
		float=!t%
	]{Madame Yvette Duval}
	
    \emph{The esteemed owner of the Brass Orchid, a woman who knows the price of every secret whispered in her establishment.}
    
    \subsubsection*{Background:}
     Madame Duval built the Brass Orchid into London’s most exclusive cabaret and gambling house, catering to the city’s wealthiest and most influential figures. While she maintains a persona of graceful hospitality, she has survived in a cutthroat industry, using her intelligence and influence to navigate political and criminal circles alike.
    
    \subsubsection*{Skills:}
    \begin{itemize}
       \item \Expert: Persuasion
        \item \Skilled: Deception, Resources
        \item \Novice: Awareness, Etiquette, Insight
    \end{itemize}
    
    \subsubsection*{Traits:}
    \begin{itemize}
        \item \textbf{Silver-Tongued Schemer} – Gains a bonus when negotiating delicate matters or extracting information.
        \item \textbf{Web of Favors} – Once per session, call in a powerful favour from a well-connected patron.
        \item \textbf{A Whisper Can Kill} – Can reroll when leveraging blackmail or manipulating a dangerous individual.
    \end{itemize}
\end{WyrdFullNPC}



The investigators arrive at \textbf{The Brass Orchid}, where the air is thick with tension. The club’s usual vibrancy is subdued, with hushed murmurs among staff and patrons alike. A staff member, \textbf{Delilah "Della" Moreau}, was the first to notice something was amiss. Mercer, a regular patron, had not emerged from his private lounge as he typically would during the intermission. Concerned, she knocked on his door. When he did not respond, she fetched \textbf{Madame Yvette Duval}, the only person with an extra key to the private lounges. 

Upon unlocking the door, they were met with a grisly sight—Mercer’s lifeless body slumped in his chair, his drink half-finished, a scrap of paper clutched tightly in his hand. The room, untouched since their discovery, remains eerily undisturbed. Knowing that the scandal could ruin the Brass Orchid, Madame Duval took swift action. Rather than contacting the authorities, she turned to the \textbf{Grand Society of Inquiry}, summoning the investigators to handle the matter discreetly.

From the moment the body was found, no one has been allowed to enter the lounge—yet the investigators will soon find that secrets have a way of slipping through even the tightest of locks.


\subsubsection{Examining the Crime Scene}

Upon entering Mercer's private lounge, the investigators find the room frozen in time. A single lamp provides dim lighting, casting long shadows across the plush furniture. A card game sits abandoned at the table, with half-smoked cigars in an ashtray. The air is thick with the scent of liquor, tobacco, and a faint, lingering trace of something bitter—something off.

The body of Edward Mercer remains slumped in his chair, untouched since discovery. His expression is frozen in surprise, his grip unnaturally tight around a crumpled scrap of paper. The investigators are free to explore the scene, but careful examination will be required to extract meaningful details.


\begin{WyrdExplanation}{}
	\subsubsection*{Primary Clues}
	\begin{itemize}
		\item A \textbf{half-finished drink laced with poison}, still resting on the table near Mercer’s body. A faint almond scent lingers, barely noticeable beneath the overpowering aroma of brandy.
		\item The victim’s \textbf{missing pocket watch}, unaccounted for at the crime scene but later discovered in an unexpected location.
		\item A \textbf{scrap of torn paper}, crumpled tightly in Mercer’s hand, as though grasped in his final moments—either in desperation or as a final act of defiance.
		\item The \textbf{pneumatic tube system}, a hidden network connecting various parts of the club, shows signs of recent tampering.
	\end{itemize}

	\subsubsection*{What the clues reveal}
	\begin{itemize}
		\item \textbf{The poisoned drink} confirms the cause of death. The faint almond scent suggests cyanide or a similar fast-acting toxin but without an obvious delivery method.
		\item \textbf{The missing pocket watch}, later found in the servers' area, is not inherently suspicious—but its location is. It suggests that someone, likely a staff member, moved through that area after Mercer’s death. \textbf{Henry "Rigs" Rigby}, the bartender, recovered it but might need some persuasion to reveal the circumstances.
		\item \textbf{The scrap of torn paper} remains tightly clutched in Mercer’s hand. The jagged edge suggests it was ripped from a larger document. Whether Mercer seized it in a moment of panic or it was forcibly torn from him before he collapsed is unclear, but its contents might point to the motive.
		\item \textbf{The tampered pneumatic tube system} is the key to the locked-room mystery. It provides a discreet means of entry and escape, but only staff or someone intimately familiar with the club would know how to exploit it.
	\end{itemize}
\end{WyrdExplanation}

\begin{WyrdComment}{Investigating the Clues}
	\subsubsection*{The poisoned drink}
	Sitting on the table near Mercer's body, the glass contains a dark amber liquid, partially consumed. A faint almond scent lingers beneath the brandy's aroma.  
	\begin{itemize}
		\item \textbf{How to discover:} Simple observation will reveal the drink, but recognising the almond scent requires a \Basic \textbf{Notice} or \textbf{Investigate} check. Recognising this as the telltale scent of cyanide requires a success at the \Difficult level.
		\item \textbf{Further examination:} A character with medical knowledge may confirm cyanide poisoning, but testing the drink will require resources outside the club.
		\item \textbf{NPC reactions:} Madame Duval insists no one could have tampered with drinks \textbf{without her bartenders noticing}, subtly diverting suspicion.
	\end{itemize}
	
	\subsubsection*{The missing pocket watch}
	Mercer’s prized gold pocket watch is conspicuously absent from his body. 
	\begin{itemize}
		\item \textbf{How to discover:} Searching Mercer’s belongings will reveal its absence, but noticing the absence of something requires that you expect its presence. Any of the staff will know that Mercer always shows off his pocket watch, so if \textbf{Madam Duval is present} when the investigators examine the body, she will notice. Otherwise, have other NPCs drop hints about a watch later in the investigation.
	\end{itemize}
	
	\subsubsection*{The scrap of torn paper}
	Clutched tightly in Mercer’s lifeless hand, the small scrap appears hastily ripped from a larger document.
	\begin{itemize}
		\item \textbf{How to discover:} Anyone inspecting the body will notice the paper.
		\item \textbf{Further examination:} It is possible to pry the paper from Mercer's hands, but it must be done carefully to not tear it further.
		\item \textbf{What it reveals:} The scrap contains part of a name and a few words, possibly relating to Mercer's blackmail scheme.
		\item \textbf{NPC reactions:} Beatrice Langley, if questioned, will become visibly uncomfortable but will attempt to play innocent unless pressed. A \Basic \textbf{Empathy} roll will reveal her discomfort. At \Formidable or higher, the investigators will recognise her emotions as fear.
	\end{itemize}
	
	\subsubsection*{The tampered pneumatic tube system}
	A discreet brass panel built into the wall leads to the club’s internal message system, normally used for sending notes and receipts between staff areas. However, someone has recently pried it open, and subtle modifications suggest it was used for more than just correspondence.

	\begin{itemize}
		\item \textbf{How to discover:} Searching the walls near Mercer's table reveals (with a \Difficult \textbf{Investigate} or \textbf{Notice}) that the panel is slightly ajar, its edges scratched as if it was hastily forced open. Staff might mention the system in passing if prompted.
		\item \textbf{Further examination:} A \Challenging \textbf{Crafts} check confirms that the panel has been modified. At \Difficult \textbf{Craft} the examination reveals that the usual constraints, meant to restrict messages to small notes, have been bypassed—suggesting something larger was sent through. Additionally, the airflow mechanism appears to have been overridden, allowing the tube to function more like a one-way transport chute rather than a message system.
		\item \textbf{What it reveals:} The system connects to the servers' area, and traces of fine fabric fibres or a stray hair inside the tube hint that it was used to transport something—or someone. A close look reveals faint scuff marks on the panel's interior, possibly left by someone squeezing through.
		\item \textbf{NPC reactions:} Most staff will dismiss the idea that a person could fit inside, but a seasoned investigator might realise that someone \textbf{small or desperate} could have used the system as an escape route.
	\end{itemize}
\end{WyrdComment}

After examining the crime scene, the investigators are free to explore the Brass Orchid in search of clues. The investigation takes place in Act 2, where they will question staff and patrons, analyse testimonies, and piece together the events of the evening. The order in which they explore the locations in the next act is up to them.

\begin{WyrdGmTips}
	The mystery is designed to be straightforward, making it ideal for new players still learning the rules and getting comfortable with investigative roleplay. However, you can easily heighten the challenge by introducing conflicting testimonies from staff and patrons, forcing them to untangle half-truths, personal biases, and hidden agendas as they piece together what really happened that night.
\end{WyrdGmTips}
