\chapter{Combat}
\index{Combat}
\label{chap:combat}

\DndDropCapLine{T}{he} core combat system, as described in the previous chapter, will suffice for any setting where combat is not a large part of the play. There is next to no combat in Agatha Christie's novels, so we don't need detailed combat mechanics in a setting modelled around such types of crime mysteries. They would only get in the way.

However, the role of combat in a game can vary significantly depending on the setting, the importance of combat in a given scenario, and the style of action you wish to create. Some settings favour \textbf{quick, brutal encounters}, where a single well-placed shot from a sniper or the swift blade of an assassin can end a fight in an instant. In contrast, other games may emphasise \textbf{heroic, drawn-out battles}, where warriors clash against hordes of monsters, trading blows in a struggle for survival.

The \textbf{tone and pacing of combat} should reflect the themes of your game. In a gritty, realistic setting, injuries may be devastating, making every decision in combat critical. A high-action cinematic game, on the other hand, may allow characters to withstand multiple attacks, diving through gunfire or dueling atop a burning airship without immediate risk of death.

For those who prefer \textbf{tactical complexity}, combat may involve detailed positioning, cover mechanics, and resource management, rewarding careful planning and teamwork. Alternatively, a more \textbf{freeform approach} might abstract combat into a series of dramatic exchanges, focusing on storytelling rather than strict mechanics.

No matter the approach, \wyrd provides a flexible combat system that can be adjusted to suit your narrative and playstyle. That is the topic of this chapter.


\section{Dealing damage}
\section{Recovery}

%% FIXME: most of what you can vary is the number of stress and wound boxes, how severe the penalties are for wounds, and how quickly you heal.