\documentclass[nodeprecatedcode,bg=print]{dndbook}
\usepackage{wyrd}
\raggedright
\pagestyle{empty}

\begin{document}
\large

\section*{Prologue: The Bell Tolls}
\begin{paracol}{2}
    Each player begins the scenario having received a letter—plain, yellowed, and sealed with cracked red wax. There is no return address. Only their name is written on the front, in handwriting they each recognise intimately: a parent, a sibling, a grandparent, a lost lover. Someone who disappeared without a trace, many years ago.

    \switchcolumn

    Inside the envelope is a prayer card from Saint Hieronymus Monastery, bearing the name \emph{Abbot Rian} and a half-faded devotional inscription. Alongside it, a simple note:

    \begin{Example}{}
    \textit{Come to the monastery. Something remains. It must not.}
    \end{Example}
\end{paracol}

\vspace{\baselineskip}
\section*{Act I: : Echoes Before the Silence}
\begin{paracol}{2}
    The journey begins in a rural village or isolated coaching inn—the last inhabited stop before the road to Saint Hieronymus disappears into mist and overgrowth. The air is chill and damp. Locals speak little of the monastery. Some seem to have forgotten it entirely; others go silent at its mention.

    \switchcolumn
    Each character arrives with a letter in hand. One by one, they realise they are not alone. Through handwriting, names, or the weight of shared grief, they begin to uncover what binds them: someone they loved once walked the monastery path—and never returned.
\end{paracol}
\subsection*{The Villagers of Ashwick}
\begin{paracol}{2}
    
    \subsubsection*{Mara Wren}
    \textbf{Innkeeper} – A hard-faced woman with silvering hair and a gaze like chipped granite. Runs the Crossed Keys, Ashwick’s only inn.\\
    \noindent\textbf{What She Knows:} Mara remembers when the monastery fell silent—but claims no one has gone up the hill in decades. Offers free lodging “for one night only” and bolts the doors at sundown.

    \vspace{0.5\baselineskip}
    \subsubsection*{Mother Tilda}
    \textbf{Wandering Herbalist} – Elderly, swathed in shawls, with a scent of lavender and peat. Carries folk charms and mutters old prayers.\\
    \noindent\textbf{What She Knows:} Tilda speaks in riddles. She warns of “a bell that tolls in dreams” and calls the players “echoes walking backwards.” Offers charms, but insists they’re useless “where names go to die.”

    \switchcolumn

    \vspace{0.5\baselineskip}
    \subsubsection*{Jonas Pike}
    \textbf{Stablehand} – Young, skittish, and tight-lipped unless bribed with coin or drink.\\
    \noindent\textbf{What He Knows:} Claims he saw a hooded figure watching from the tree line one foggy dawn. Swears he once heard a bell ring from the hills—though no one else did.

    \vspace{0.5\baselineskip}
    \subsubsection*{Father Anselm}
    \textbf{Village Priest} – Frail and soft-spoken, tending a small shrine older than the monastery.\\
    \noindent\textbf{What He Knows:} Cautions that the monks “reached too high and fell into silence.” Holds a faded registry of those who joined Saint Hieronymus. One name matches a player’s lost loved one.
\end{paracol}

\newpage
\section*{Act II: A Silence That Should Have Passed}


\end{document}
