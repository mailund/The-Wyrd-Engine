\documentclass[nodeprecatedcode,bg=print]{dndbook}
\usepackage{wyrd}
\raggedright
\pagestyle{empty}

\begin{document}
\large

\section*{Prologue: The Bell Tolls}
\begin{paracol}{2}
    Each player begins the scenario having received a letter—plain, yellowed, and sealed with cracked red wax. There is no return address. Only their name is written on the front, in handwriting they each recognise intimately: a parent, a sibling, a grandparent, a lost lover. Someone who disappeared without a trace, many years ago.

    \switchcolumn

    Inside the envelope is a prayer card from Saint Hieronymus Monastery, bearing the name \emph{Abbot Rian} and a half-faded devotional inscription. Alongside it, a simple note:

    \begin{Example}{}
    \textit{Come to the monastery. Something remains. It must not.}
    \end{Example}
\end{paracol}

\vspace{\baselineskip}
\section*{Act I: : Echoes Before the Silence}
\begin{paracol}{2}
    The journey begins in a rural village or isolated coaching inn—the last inhabited stop before the road to Saint Hieronymus disappears into mist and overgrowth. The air is chill and damp. Locals speak little of the monastery. Some seem to have forgotten it entirely; others go silent at its mention.

    \switchcolumn
    Each character arrives with a letter in hand. One by one, they realise they are not alone. Through handwriting, names, or the weight of shared grief, they begin to uncover what binds them: someone they loved once walked the monastery path—and never returned.
\end{paracol}
\subsection*{The Villagers of Ashwick}
\begin{paracol}{2}
    
    \subsubsection*{Mara Wren}
    \textbf{Innkeeper} – A hard-faced woman with silvering hair and a gaze like chipped granite. Runs the Crossed Keys, Ashwick’s only inn.\\
    \noindent\textbf{What She Knows:} Mara remembers when the monastery fell silent—but claims no one has gone up the hill in decades. Offers free lodging “for one night only” and bolts the doors at sundown.

    \vspace{0.5\baselineskip}
    \subsubsection*{Mother Tilda}
    \textbf{Wandering Herbalist} – Elderly, swathed in shawls, with a scent of lavender and peat. Carries folk charms and mutters old prayers.\\
    \noindent\textbf{What She Knows:} Tilda speaks in riddles. She warns of “a bell that tolls in dreams” and calls the players “echoes walking backwards.” Offers charms, but insists they’re useless “where names go to die.”

    \switchcolumn

    \vspace{0.5\baselineskip}
    \subsubsection*{Jonas Pike}
    \textbf{Stablehand} – Young, skittish, and tight-lipped unless bribed with coin or drink.\\
    \noindent\textbf{What He Knows:} Claims he saw a hooded figure watching from the tree line one foggy dawn. Swears he once heard a bell ring from the hills—though no one else did.

    \vspace{0.5\baselineskip}
    \subsubsection*{Father Anselm}
    \textbf{Village Priest} – Frail and soft-spoken, tending a small shrine older than the monastery.\\
    \noindent\textbf{What He Knows:} Cautions that the monks “reached too high and fell into silence.” Holds a faded registry of those who joined Saint Hieronymus. One name matches a player’s lost loved one.
\end{paracol}

\newpage
\section*{Act II: A Silence That Should Have Passed}

\begin{paracol}{2}
    \subsection{Haunting Incidents}

    Add one or two surreal moments as tension escalates:
    \begin{Example}{}
    \begin{itemize}
        \item A note in the player’s own handwriting appears inside a hymn book—dated years before.
        \item A bell tolls faintly from nowhere.
        \item A mirror reflects not their face, but someone they lost.
        \item One player speaks—but no one hears them for a full minute.
        \item A robed figure watches from the courtyard's edge, then vanishes.
    \end{itemize}
    \end{Example}

    These events should disturb without threatening. The monastery is not yet hostile—but its memory is stirring.

    \subsubsection{The Courtyard}
    Overgrown but untouched. Prayer stones lie scattered. Names on the wall of remembrance are scratched away.
    
    \begin{Example}{}
        \begin{itemize}
            \item A broken statue of Saint Hieronymus—its eyes are hollow, never carved. (Foreshadowing the \emph{Eyeless Abbot})
            \item An empty bell tower. A frayed rope dangles, the bell long gone. (Foreshadowing the \emph{Campana Silens})
        \end{itemize}
    \end{Example}

    \subsection{The Dormitories}
    Neat, undisturbed. Robes hang in place. Beds are made—some still warm.

    \begin{Example}{}
        \begin{itemize}
            \item A torn journal entry speaks of “the Bell” and “the Abbot’s final rite.”
            \item A name scratched into the wall matches one player’s lost loved one.
        \end{itemize}
    \end{Example}

    \switchcolumn
    \subsection{The Library}
    Dust-choked and silent. Books line the shelves. A few remain open on desks.

    \begin{Example}{}
        \begin{itemize}
            \item A history of the monastery details the monks’ pursuit of purity through silence. The final chapter is missing.
            \item A map shows the monastery layout, with a hidden passage marked in red from the Chapel to the crypts.
        \end{itemize}
    \end{Example}

    A \DL{+1} \textbf{Lore} check reveals the monks' interest in silence and healing. A \DL{+2} \textbf{Lore} or \textbf{Investigate} check reveals that a ritual involving the \emph{Campana Silens} led to their downfall. On a \DL{+3} success, the bell's twisted purpose is uncovered.

    \subsection{The Chapel}
    Dusty but intact. Candles remain. Pages of scripture are overwritten with indecipherable symbols.

    \begin{Example}{}
        \begin{itemize}
            \item A faded inscription on the wall: \emph{“To remember is to suffer.”}
            \item A tear-stained prayer card bearing the name of a player’s lost loved one.
        \end{itemize}
    \end{Example}

    A \DL{+1} \textbf{Investigate} check reveals the hidden passage to the crypts behind the altar.

    \begin{GmTips}
        Ensure the players find the passage. If they fail the check, provide narrative cues or bonuses. Consider:
        \begin{itemize}
            \item A spectral monk silently gesturing toward the wall
            \item Whispers or echoes leading in that direction
            \item A physical shift in the environment that draws their attention
        \end{itemize}
    \end{GmTips}

\end{paracol}

\newpage
\section*{Act III: What the\\ Stone Remembers}
\begin{paracol}{3}

    \begin{Example}{}
    \begin{itemize}\raggedright
        \item A monk who endlessly copies names into a ledger—names the players recognise as their own.
        \item A vision of the final rite, led by Abbot Rian, in which the monks are stripped of their voices and vanish.
        \item A soundless plea scratched into the walls: \emph{“Ring the bell. Let us be.”}
    \end{itemize}
    \end{Example}


\switchcolumn


    \begin{Example}{}
        \begin{itemize}
            \item A vision of the Abbot leading a silent rite—robes falling, mouths closing, voices erased.
            \item A name scratched into the stone: one of the players’ lost loved ones.
            \item A shadowy loved one reaching out: \emph{“When names were taken, the bell was sealed. To ring it is to suffer.”}
            \item A bell tolls in the distance. A whisper follows: \emph{“You can’t fight him without it.”}
        \end{itemize}
    \end{Example}

\switchcolumn
    \begin{Example}{Opening the Door}
        A player must speak the full name of their lost loved one aloud. This causes 1 Fatigue and opens the door with a low moan as if exhaling a forgotten breath.
    \end{Example}
\end{paracol}

\subsection{The Spectral Monks}
\columnratio{0.6}
\begin{paracol}{2}
    \raggedright

    \emph{They were once devout—seekers of peace through silence. Now they are only fragments: flickering memories and slivers of spirit, trapped in the stone. When the Abbot stirs, they move not as men, but as one mind.}

    \subsubsection*{Interaction:}
    Interacting with monks require a \textbf{Will} or \textbf{Empathy} check against \DL{0} or cost 1 \emph{Fatigue}. With a successful \textbf{Empathy} or \textbf{Focus} check at \DL{+2}, the monks may reveal fragmented memories.

    \subsubsection*{Combat:}
    The monks become hostile only when the Abbot commands them. In combat, they use \textbf{Focus + Silent Screams} to cause \emph{Fatigue}, but cannot inflict Wounds. They serve as a spiritual obstacle rather than a lethal threat, and will vanish once the \emph{Campana Silens} is rung.

    \switchcolumn\normalsize

    \begin{SkillsBox}
        \Skilled & Focus \\
        \Novice & Awareness \\
    \end{SkillsBox}

    \begin{TraitsBox}
        \item[Spectral Silence] — Cannot be harmed until the \emph{Campana Silens} is rung. Bound to the monastery's silence.
        \item[Silent Screams] — Use \textbf{Focus} to inflict \emph{Fatigue Stress} through psychic silence. Targets may roll \textbf{Will} to resist.
    \end{TraitsBox}

    \DamageBox[%
        totalfatigue=3,%
        totalmild=0,%
        totalmoderate=0,%
        totalsevere=0,%
    ]
\end{paracol}

\newpage
\section*{Act IV: The Thing That Waits}

\subsection{The Eyeless Abbot}
\columnratio{0.5}
\begin{paracol}{2}
    \emph{Once Abbot Rian, a devout spiritual leader who sought to transcend worldly identity through perfect silence. He succeeded far too well. What remains is not a man, but a vessel of stillness—an unspoken wound in the fabric of memory.}

    \subsubsection*{Background:}
    The Eyeless Abbot lingers at the heart of Saint Hieronymus Monastery, an immovable anchor of grief and spiritual ruin. During a failed rite meant to dissolve the self, he bound his own soul—and those of his brethren—into a silence so deep it became a presence. He does not see, does not speak, and cannot be harmed by mortal means. Only the toll of the \emph{Campana Silens} can pull him into reach of the living.

    \subsubsection*{Corporeal Form:}
    Once the bell is rung, the Abbot’s spectral shell collapses. What remains is a ravaged, half-human husk—robes torn and ritual-scars burned deep into pale, brittle flesh. His fused eyes cannot see, but he still perceives through the echoes of memory. He is no longer untouchable, but his presence remains overwhelming.

    \switchcolumn\normalsize

    \begin{SkillsBox}
        \Expert & Will \\
        \Skilled & Focus, Insight \\
        \Novice & Fight, Awareness \\
    \end{SkillsBox}

    \begin{TraitsBox}
    \item[Silence Hungers] — At the start of each round, all player characters suffer 1 \textbf{Fatigue} unless they succeed on a \DL{+2} \textbf{Will} or \textbf{Focus} check. This Trait can only inflict Fatigue, not Wounds.
    \item[Untouchable Form] — The Abbot is immune to all harm unless the \emph{Campana Silens} has been rung. This Trait is suppressed after the bell tolls.
    \item[Aura of Dread] — While in the Abbot’s presence, characters take a –1 penalty to all rolls. Suppressed once the bell is rung.
    \item[Erase the Self] — Once per scene, the Abbot may suppress one player’s Trait of choice. It remains unusable until the player invokes a memory of their lost loved one.
    \end{TraitsBox}

    \DamageBox[%
        totalfatigue=3,%
        totalmild=2,%
        totalmoderate=1,%
        totalsevere=0,%
    ]

\end{paracol}


\end{document}
