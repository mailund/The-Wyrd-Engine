% TEXINPUTS=../../dndbook:../../wyrdtex:

\documentclass[twocolumn,nodeprecatedcode,bg=print]{dndbook}
\usepackage{wyrd}

\begin{document}
\DndSetThemeColor[DmgSlateGray]
\chapter*{What Lingers}
\begin{WyrdSettingHeading}
    \WyrdCapLine{T}{here} are places the world forgets. Not through time or chance, but through silence—heavy, deliberate silence that settles like dust over memory, over names, over grief.

    \noindent
    Saint Hieronymus Monastery was once such a place. Tucked in the hills beyond the reach of rail or road, it served as a haven for those seeking peace, penance, or oblivion. Decades ago, it fell silent. No letters, no pilgrims, no word. The world moved on, and most forgot.

    But you did not forget. You couldn’t. Because someone you once loved—someone you still remember—was there when the silence began. And now, an envelope has arrived. No return address. Just your name, written in a familiar hand that time should have erased.

    Inside: a yellowed prayer card bearing the name \emph{Abbot Rian}, a brief message: \textit{"Come. Something remains. It must not."}

    The road to Saint Hieronymus is overgrown. The monastery stands still beneath a grey sky. No birds sing. The gates are open.

    Some silences are chosen. Others are imposed. This one—whatever it is—has waited for you.

    What lingers in the silence... and what will it take to bring it to light?
\end{WyrdSettingHeading}

\begin{CommentBox}{Scenario Overview}
    \textbf{Tone:} Gothic horror, spiritual decay, emotional weight
    
    \noindent
    \textbf{Setting:} Saint Hieronymus Monastery — a remote, long-abandoned spiritual retreat frozen in silence
    
    \noindent
    \textbf{Structure:}
    \begin{itemize}
        \item \textbf{Act I – Echoes Before the Silence:} The characters meet at a rural waystation; they discover their shared summons and personal connections to the monastery
        \item \textbf{Act II – A Silence That Should Have Passed:} Arrival at the monastery; unnatural silence; signs of lost lives
        \item \textbf{Act III – What the Stone Remembers:} Exploration and hallucinations; discovery of the \emph{Campana Silens}
        \item \textbf{Act IV – The Thing That Waits:} Confrontation with the Eyeless Abbot; ringing the Bell; final combat
    \end{itemize}
    
    \noindent
    \textbf{Recommended Players:} 2–4
    
    \noindent
    \textbf{Playtime:} 2–3 hours
    
    \noindent
    \textbf{Key Themes:} Memory, silence, spiritual erosion, the persistence of grief
    
    \noindent
    \textbf{Main Threat:} The Eyeless Abbot — a spectral monk whose silence consumes names, voices, and identity
    
    \noindent
    \textbf{Key Item:} \emph{Campana Silens} — a sacred funerary bell that must be rung to make the Abbot vulnerable
\end{CommentBox}
    
\emph{What Lingers} is a mystery of atmosphere and memory, intended to unfold slowly and build toward a climactic confrontation. Players arrive at the abandoned Saint Hieronymus Monastery, drawn by personal ties to those who vanished decades ago. The silence they encounter is not merely the absence of sound—it is a force, pervasive and oppressive, one that resists being broken.

The scenario is divided into four acts: 
\begin{itemize}
    \item \textbf{Act I} introduces the players and their shared connection to the monastery.
    \item \textbf{Act II} immerses them in the eerie stillness of the monastery, where they begin to experience the effects of the silence.
    \item \textbf{Act III} leads them into the crypts, where they confront echoes of the past and discover the \emph{Campana Silens}.
    \item \textbf{Act IV} culminates in a confrontation with the Eyeless Abbot, where they must ring the Bell to break his hold over their memories.
\end{itemize}
Each act tests different facets of play—emotional resilience, problem-solving, and tactical cooperation—culminating in a single boss battle where success hinges on recovering what was lost: identity, remembrance, and the power to speak one’s truth aloud.

\subsection*{The Premise}
Decades ago, Abbot Rian led the monks of Saint Hieronymus in a dangerous spiritual experiment. Obsessed with the idea of achieving divine purity through silence, he performed a ritual intended to sever worldly identity and bring the brothers into perfect spiritual stillness.

At the centre of the rite was the \emph{Campana Silens}, a funerary bell traditionally used to preserve memory and guide souls to rest. Rian reversed its purpose—twisting the sacred toll into a metaphysical anchor that would bind the monks to silence, preventing both speech and spiritual passage. The rite succeeded—terribly. The monks were erased. Only their robes remained.

Rian himself became the Eyeless Abbot, a spectral presence held together by silence and grief. He is no longer truly human. The monastery has remained untouched ever since, locked in a state of unnatural quiet. The letter that summoned the players may have been sent by a fading fragment of one of the monks—or by the Bell itself. The silence is fraying. Something within still remembers.

\emph{To lay the Abbot to rest, the players must find and ring the Campana Silens—risking collapse, confrontation, and the return of names long buried.}



\subsection*{The Letter}

Each player begins the scenario having received a letter—plain, yellowed, and sealed with cracked red wax. There is no return address. Only their name is written on the front, in handwriting they each recognise intimately: a parent, a sibling, a grandparent, a lost lover. Someone who disappeared without a trace, many years ago.

Inside the envelope is a prayer card from Saint Hieronymus Monastery, bearing the name \emph{Abbot Rian} and a half-faded devotional inscription. Alongside it, a simple note:

\begin{Example}
   \textit{Come to the monastery. Something remains. It must not.}
\end{Example}
\begin{CommentBox}{Who Sent the Letters?}
    The letters that summoned the players to Saint Hieronymus Monastery have no clear origin—and that is intentional. The handwriting belongs to loved ones long lost, yet the paper is pristine, the ink fresh. No mundane explanation fits.
    
    The truth is this: the letters were sent by a lingering spiritual fragment—a faint echo of one of the monks who resisted the rite. His name is lost, his voice long silenced, but some ember of will remained. Drawn to the players by threads of memory and grief, this presence reached across the veil through the only tool it had left: the Bell.
    
    The \emph{Campana Silens} remembers. Though corrupted, it still resonates with the identities it once preserved. Through it, the spirit crafted the letters—not as messages, but as echoes of personal connection, imitating the handwriting of the loved ones each character lost. The Bell called out, and those who still remembered… heard.
    
    \emph{GMs may choose to reveal this secret during the climax or leave it unsaid. The ambiguity strengthens the tone—some truths are better left incomplete.}
\end{CommentBox}
    
The letter offers no explanation, yet the handwriting is unmistakable—and impossibly preserved. Whether out of longing, guilt, curiosity, or fear, the characters set out for the hills where Saint Hieronymus stands in ruin. No traffic goes that way. The path is overgrown. The air feels thinner the closer they get. The monastery gates stand open, and the world behind them falls silent.

\begin{CommentBox}{Relevant Skills}
    This scenario emphasises emotional, investigative, and spiritual challenges over physical ones. Only one combat skill—\textbf{Fight}—is used, and only during the final confrontation. Most skill use centres around resisting silence (as a spiritual attack), uncovering forgotten truths, and interpreting haunting illusions.
    
    \vspace{0.5\baselineskip}
    
    \subsubsection*{Mental and Emotional}
    \begin{itemize}
        \item \textbf{Empathy} — Connecting with echoes, emotional projections, or grieving spirits.
        \item \textbf{Focus} — Holding concentration in hallucinations or performing spiritual attacks. \textbf{Focus} is used for spiritual combat in this scenario.
        \item \textbf{Insight} — Understanding visions, illusions, or lost memories.
        \item \textbf{Lore} — Interpreting the monastery’s history, rites, and the Bell’s purpose.
        \item \textbf{Will} — Resisting psychic pressure, spiritual erosion, and identity loss. \textbf{Will} is used to resist spiritual attacks in this scenario.
    \end{itemize}
    
    \subsubsection*{Social and Interpersonal}
    \begin{itemize}
        \item \textbf{Etiquette} — Recognising religious traditions or sacred customs
        \item \textbf{Persuasion} — Calming others or convincing hesitant allies
    \end{itemize}
    
    \subsubsection*{Perceptual and Investigative}
    \begin{itemize}
        \item \textbf{Awareness} — Noticing details, spiritual distortions, or vanishing sounds
        \item \textbf{Investigation} — Examining rooms, deciphering inscriptions, finding hidden relics
    \end{itemize}
    
    \subsubsection*{Physical}
    \begin{itemize}
        \item \textbf{Athletics} — Dodging debris, climbing, moving in collapsing areas (Act IV)
        \item \textbf{Fight} — Captures all types of ranged and melee combat. (Used only during the final act, once the Abbot becomes vulnerable)
    \end{itemize}
    
    \subsubsection*{Skill Highlights by Act}
    \begin{itemize}
        \item \textbf{Act I –} Insight, Empathy
        \item \textbf{Act II –} Awareness, Focus, Will, Lore
        \item \textbf{Act III –} Lore, Empathy, Insight, Will
        \item \textbf{Act IV –} Athletics, Fight, Focus, Will
    \end{itemize}    
\end{CommentBox}
    

\section*{Act I – Echoes Before the Silence}

Before the players ever glimpse the monastery, they are drawn together by a shared summons—and a shared absence. This act sets the emotional foundation: grief unspoken, memory unresolved. It offers space to form connections and feel the first tremors of unease.

\subsection*{Setting: The Last Waystation}

The journey begins in a rural village or isolated coaching inn—the last inhabited stop before the road to Saint Hieronymus disappears into mist and overgrowth. The air is chill and damp. Locals speak little of the monastery. Some seem to have forgotten it entirely; others go silent at its mention.

Each character arrives with a letter in hand. One by one, they realise they are not alone. Through handwriting, names, or the weight of shared grief, they begin to uncover what binds them: someone they loved once walked the monastery path—and never returned.

\subsubsection{Mechanical Focus}
This act is roleplay-heavy with minimal skill checks. Use \textbf{Insight}, \textbf{Empathy}, or \textbf{Lore} for interactions with villagers. Set the tone with subtle unease: a candle that flickers against the wind, a remembered voice, a shadow that lingers too long.

\subsubsection{Scenes and Options}

The focus of Act I is tone and emotional resonance. The players should feel the ache of loss and the mystery of shared summons. It also introduces the party and the world they’re stepping into.

\begin{Example}[Possible Scenes]
\begin{itemize}
    \item \textbf{Arrival and Suspicion:} Each character enters the village separately. The innkeeper or a wary local reacts oddly—perhaps to the names on their letters.
    
    \item \textbf{Recognition and Revelation:} A player recognises the handwriting on another’s letter. Shared grief becomes the first bridge between strangers.

    \item \textbf{A Shared Past:} Players discover their lost relatives once knew each other—or vanished together. A dusty ledger or forgotten trunk may reveal others who made this same journey, and never returned.

    \item \textbf{A Warning:} A village elder, priest, or traveller mutters a cryptic warning: \emph{“That place has been quiet too long. Whatever’s there... remembers.”}
\end{itemize}
\end{Example}

Use this act to foreshadow the monastery’s fate and the role of the \emph{Campana Silens}. The players should not yet grasp the full threat, but subtle references to a bell and silence will seed the tension that blooms later.

\begin{Example}[Hints and Clues]
    \begin{itemize}
        \item \textbf{The Bell:} A villager recalls a funerary bell that once rang from the monastery. No one has heard it in years.
        \item \textbf{The Last Toll:} An old man remembers the final time the bell rang. “Something changed after that. Like the air forgot how to carry sound.”
    \end{itemize}
\end{Example}

\begin{NPC}[%
    description=The last voices before the silence,%
]{The Villagers of Ashwick}
    \emph{Ashwick lies in the shadow of Saint Hieronymus. Mist-choked and half-forgotten, its people are wary of strangers and grow uneasy at any mention of the monastery. Those who speak do so with long pauses—as if listening for something before they answer.}

    \vspace{0.5\baselineskip}
    \subsubsection*{Mara Wren}
    \textbf{Innkeeper} – A hard-faced woman with silvering hair and a gaze like chipped granite. Runs the Crossed Keys, Ashwick’s only inn.\\
    \textbf{What She Knows:} Mara remembers when the monastery fell silent—but claims no one has gone up the hill in decades. Offers free lodging “for one night only” and bolts the doors at sundown.

    \vspace{0.5\baselineskip}
    \subsubsection*{Mother Tilda}
    \textbf{Wandering Herbalist} – Elderly, swathed in shawls, with a scent of lavender and peat. Carries folk charms and mutters old prayers.\\
    \textbf{What She Knows:} Tilda speaks in riddles. She warns of “a bell that tolls in dreams” and calls the players “echoes walking backwards.” Offers charms, but insists they’re useless “where names go to die.”

    \vspace{0.5\baselineskip}
    \subsubsection*{Jonas Pike}
    \textbf{Stablehand} – Young, skittish, and tight-lipped unless bribed with coin or drink.\\
    \textbf{What He Knows:} Claims he saw a hooded figure watching from the tree line one foggy dawn. Swears he once heard a bell ring from the hills—though no one else did.

    \vspace{0.5\baselineskip}
    \subsubsection*{Father Anselm}
    \textbf{Village Priest} – Frail and soft-spoken, tending a small shrine older than the monastery.\\
    \textbf{What He Knows:} Cautions that the monks “reached too high and fell into silence.” Holds a faded registry of those who joined Saint Hieronymus. One name matches a player’s lost loved one.
\end{NPC}

\subsubsection{Ending the Act}
The act ends when the players choose to follow the overgrown path together. The wind stills. The trees thicken. And through the mist, the dark silhouette of Saint Hieronymus rises—waiting.


\section*{Act II – A Silence That Should Have Passed}

The players arrive at Saint Hieronymus Monastery. Nestled deep in the hills, it stands as if held outside of time—weathered but unbroken, its gate ajar. The air is motionless. No birds call. No wind stirs. The silence is too complete.

From the moment they step within the grounds, sound begins to change. Footsteps muffle. Whispers vanish. Even breath feels swallowed. The monastery is not merely quiet—it is \emph{hungry}.

\subsection{Exploration and Atmosphere}

Players may explore key areas of the monastery. Each is more about tone than answers. They feel watched. The silence presses closer. Something stirs beneath memory.

\subsubsection{The Courtyard}
Overgrown but untouched. Prayer stones lie scattered. Names on the wall of remembrance are scratched away.

\begin{Example}
    \begin{itemize}
        \item A broken statue of Saint Hieronymus—its eyes are hollow, never carved. (Foreshadowing the \emph{Eyeless Abbot})
        \item An empty bell tower. A frayed rope dangles, the bell long gone. (Foreshadowing the \emph{Campana Silens})
    \end{itemize}
\end{Example}

\subsubsection{The Dormitories}
Neat, undisturbed. Robes hang in place. Beds are made—some still warm.

\begin{Example}
    \begin{itemize}
        \item A torn journal entry speaks of “the Bell” and “the Abbot’s final rite.”
        \item A name scratched into the wall matches one player’s lost loved one.
    \end{itemize}
\end{Example}

\subsubsection{The Library}
Dust-choked and silent. Books line the shelves. A few remain open on desks.

\begin{Example}
    \begin{itemize}
        \item A history of the monastery details the monks’ pursuit of purity through silence. The final chapter is missing.
        \item A map shows the monastery layout, with a hidden passage marked in red from the Chapel to the crypts.
    \end{itemize}
\end{Example}

A \Challenging \textbf{Lore} check reveals the monks' interest in silence and healing. A \DL{+2} \textbf{Lore} or \textbf{Investigate} check reveals that a ritual involving the \emph{Campana Silens} led to their downfall. On a \DL{+3} success, the bell's twisted purpose is uncovered.

\subsubsection{The Chapel}
Dusty but intact. Candles remain. Pages of scripture are overwritten with indecipherable symbols.

\begin{Example}
    \begin{itemize}
        \item A faded inscription on the wall: \emph{“To remember is to suffer.”}
        \item A tear-stained prayer card bearing the name of a player’s lost loved one.
    \end{itemize}
\end{Example}

A \Challenging \textbf{Investigate} check reveals the hidden passage to the crypts behind the altar.

\begin{GmTips}
    Ensure the players find the passage. If they fail the check, provide narrative cues or bonuses. Consider:
    \begin{itemize}
        \item A spectral monk silently gesturing toward the wall
        \item Whispers or echoes leading in that direction
        \item A physical shift in the environment that draws their attention
    \end{itemize}
\end{GmTips}

\vspace{\baselineskip}
The deeper they go, the more the silence weighs. Whispers intrude. Forgotten memories rise. Even names begin to slip away.

\subsection{Haunting Incidents}

Add one or two surreal moments as tension escalates:
\begin{Example}
\begin{itemize}
    \item A note in the player’s own handwriting appears inside a hymn book—dated years before.
    \item A bell tolls faintly from nowhere.
    \item A mirror reflects not their face, but someone they lost.
    \item One player speaks—but no one hears them for a full minute.
    \item A robed figure watches from the courtyard's edge, then vanishes.
\end{itemize}
\end{Example}

These events should disturb without threatening. The monastery is not yet hostile—but its memory is stirring.

\subsection{Tone and Progression}

This act focuses on emotional erosion, ambient dread, and the loss of certainty. No enemies appear yet. The horror is slow, quiet, and watching.


\vspace{\baselineskip}\noindent
\textsc{As the act ends}, the players should have discovered the passage to the crypts, heard fragments of forgotten rites, and encountered references to the \emph{Campana Silens}—a name that feels both sacred and dangerous.



\section*{Act III – What the Stone Remembers}

The monastery has not given up all its secrets. Beneath the chapel, the players uncover the entrance to the crypts—a place that feels older, colder, and heavier than the world above. The silence here is not just oppressive—it is consuming. Even thought feels muted. In these buried halls, the boundaries between memory and reality begin to unravel.

\subsection{Echoes and Illusions}

The crypts are haunted by the Spectral Monks—faded echoes of those who once lived, now reduced to emotional impressions imprinted upon stone and silence. They are not traditional ghosts, but fractured memories clinging to ritual and sorrow. Some kneel in prayer. Others write endlessly. A few seem to watch the players with wordless, unseen eyes.

Players may witness or interact with the following:
\begin{Example}
    \begin{itemize}
        \item A monk who endlessly copies names into a ledger—names the players recognise as their own.
        \item A vision of the final rite, led by Abbot Rian, in which the monks are stripped of their voices and vanish.
        \item A soundless plea scratched into the walls: \emph{“Ring the bell. Let us be.”}
    \end{itemize}
\end{Example}

\begin{NPC}[%
	description=Lingering echoes bound in silence,%
]{The Spectral Monks}

    \emph{They were once devout—seekers of peace through silence. Now they are only fragments: flickering memories and slivers of spirit, trapped in the stone. When the Abbot stirs, they move not as men, but as one mind.}

    \subsubsection*{Background:}
    Erased in the failed rite, the monks persist as ambient sorrow. During exploration, they are non-hostile and may be seen performing fragments of their old routines. They cannot speak, but may respond to a player invoking a lost loved one’s name with an \textbf{Empathy} roll at \DL{+2}.

    \subsubsection*{Combat:}
    The monks become hostile only when the Abbot commands them. In combat, they use \textbf{Focus + Silent Screams} to cause \emph{Fatigue}, but cannot inflict Wounds. They serve as a spiritual obstacle rather than a lethal threat, and will vanish once the \emph{Campana Silens} is rung.

    \vspace{0.5\baselineskip}

    \begin{SkillsBox}
        \Skilled & Focus \\
        \Novice & Awareness \\
    \end{SkillsBox}

    \begin{TraitsBox}
        \item[Spectral Silence] — Cannot be harmed until the \emph{Campana Silens} is rung. Bound to the monastery's silence.
        \item[Silent Screams] — Use \textbf{Focus} to inflict \emph{Fatigue Stress} through psychic silence. Targets may roll \textbf{Will} to resist.
    \end{TraitsBox}

    \DamageBox[%
        totalfatigue=3,%
        totalmild=0,%
        totalmoderate=0,%
        totalsevere=0,%
    ]

\end{NPC}

Interacting with the monks is emotionally exhausting. Each attempt may require a \textbf{Will} or \textbf{Empathy} check against \Basic or cost 1 \emph{Fatigue}. With a successful \textbf{Empathy} or \textbf{Focus} check at \DL{+2}, the monks may reveal fragmented memories, including glimpses of the ritual, lost loved ones, or the bell’s significance.

\begin{Example}[Visions and Echoes]
    \begin{itemize}
        \item A vision of the Abbot leading a silent rite—robes falling, mouths closing, voices erased.
        \item A name scratched into the stone: one of the players’ lost loved ones.
        \item A shadowy loved one reaching out: \emph{“When names were taken, the bell was sealed. To ring it is to suffer.”}
        \item A bell tolls in the distance. A whisper follows: \emph{“You can’t fight him without it.”}
    \end{itemize}
\end{Example}

These clues establish that the funeral bell \emph{Campana Silens} is key. They do not reveal its exact purpose of the bell—maintaining mystery while giving players a goal.

\subsection{The Silent Gate}

At the crypt’s far end stands a vast oak door, bound in black iron. It is unnaturally cold to the touch and engraved with the same words found on the players’ prayer cards: \emph{“To remember is to suffer.”} When approached, the Spectral Monks emerge from the walls, form a silent circle around the door and the players, and then vanish—leaving only the sound of breath in the dark. In passing, one of the monks whispers: \emph{“The door yields only to remembrance. The name must be spoken. The offering must be real.”}

The door cannot be opened by force. It is not only locked, but bound by a spiritual mechanism. It requires a deliberate act to pass through—one that acknowledges the weight of memory and breaks the spell of silence. \emph{The players should feel the weight of the moment. This is not just a door—it is a threshold between silence and remembrance.}

\begin{Example}[Opening the Door]
    A player must speak the full name of their lost loved one aloud. This causes 1 Fatigue and opens the door with a low moan as if exhaling a forgotten breath.
\end{Example}

Once opened, the door swings inward slowly, revealing the vast, hollow chamber beyond: the meditation hall, empty but for the echo of someone waiting.

\begin{GmTips}
    The solution to opening the door has been hinted at multiple places in the first three acts, with references to names of lost loved ones. If the players struggle, consider:
    \begin{itemize}
        \item A vision of the loved one, urging them to speak
        \item A player’s own voice echoing back at them, repeating the name
    \end{itemize}
\end{GmTips}




\section*{Act IV \textendash{} The Thing That Waits}

\begin{NPC}[description=Silence Incarnate]{The Eyeless Abbot}

    \emph{Once Abbot Rian, a devout spiritual leader who sought to transcend worldly identity through perfect silence. He succeeded far too well. What remains is not a man, but a vessel of stillness—an unspoken wound in the fabric of memory.}

    \subsubsection*{Background:}
    The Eyeless Abbot lingers at the heart of Saint Hieronymus Monastery, an immovable anchor of grief and spiritual ruin. During a failed rite meant to dissolve the self, he bound his own soul—and those of his brethren—into a silence so deep it became a presence. He does not see, does not speak, and cannot be harmed by mortal means. Only the toll of the \emph{Campana Silens} can pull him into reach of the living.

    \subsubsection*{Corporeal Form:}
    Once the bell is rung, the Abbot’s spectral shell collapses. What remains is a ravaged, half-human husk—robes torn and ritual-scars burned deep into pale, brittle flesh. His fused eyes cannot see, but he still perceives through the echoes of memory. He is no longer untouchable, but his presence remains overwhelming.

    \vspace{0.5\baselineskip}

    \begin{SkillsBox}
        \Expert & Will \\
        \Skilled & Focus, Insight \\
        \Novice & Fight, Awareness \\
    \end{SkillsBox}

    \begin{TraitsBox}
    \item[Silence Hungers] — At the start of each round, all player characters suffer 1 \textbf{Fatigue} unless they succeed on a \DL{+2} \textbf{Will} or \textbf{Focus} check. This Trait can only inflict Fatigue, not Wounds.
    \item[Untouchable Form] — The Abbot is immune to all harm unless the \emph{Campana Silens} has been rung. This Trait is suppressed after the bell tolls.
    \item[Aura of Dread] — While in the Abbot’s presence, characters take a –1 penalty to all rolls. Suppressed once the bell is rung.
    \item[Erase the Self] — Once per scene, the Abbot may suppress one player’s Trait of choice. It remains unusable until the player invokes a memory of their lost loved one.
    \end{TraitsBox}

    \DamageBox[%
        totalfatigue=3,%
        totalmild=2,%
        totalmoderate=1,%
        totalsevere=0,%
    ]

\end{NPC}


The final confrontation takes place in the monastery's meditation hall\textemdash{}a vast, domed space of cracked stone, with faded frescoes on high walls and an oppressive weight in the air. Dust hangs unnaturally still. There is no sound.

The Eyeless Abbot is already there. He does not speak. He does not need to. His presence – vast and wrong – dominates the space. The silence deepens with every breath.

Above and behind him, partially veiled in shadow, looms the \emph{Campana Silens} in its rusted iron frame. The way to it is treacherous: crumbling stairs, collapsing ledges, and a steep climb. It is visible, but distant.

The players cannot yet harm the Abbot. This is not a battle they can win – not until they ring the bell.

\subsection*{Phase One – A Losing Battle}

The fight begins with the Abbot and a wave of Spectral Monks. The Abbot is untouchable, phasing through attacks and unraveling traits. The monks drain Fatigue, preparing the players for defeat. During this phase:

\begin{Example}
    \begin{itemize}
    \item The Abbot is immune to all damage. His \textbf{Untouchable Form} suppresses any attempt to harm him.
    \item At the start of each round, \textbf{Silence Hungers} forces all players to make a \textbf{Will} or \textbf{Focus} check (\DL{+2}) or take 1 Fatigue.
    \item Spectral Monks (one per player) attack using spiritual pressure to deplete Fatigue.
    \item The \textbf{Aura of Dread} imposes a \textbf{-1 penalty} to all rolls while in the Abbot's presence.
    \end{itemize}
\end{Example}

\noindent
Players quickly realise they cannot win by force.

\subsection*{Reaching the Bell}

\begin{CommentBox}{Campana Silens}
    \emph{Campana Silens}, the Silent Bell, is a sacred funerary relic once used to preserve the names of the dead. Forged in iron and consecrated in ancient rites, it was designed to protect memory during the final passage—ensuring the deceased would not be forgotten.
    
    But Abbot Rian twisted its purpose. Instead of preserving memory, he used it to seal it away. The bell became an anchor for silence, stillness, and spiritual imprisonment. Now, it is the only thing capable of disrupting the unnatural peace he created.
    
    \vspace{0.5\baselineskip}\noindent
    \textbf{Ringing the Bell} shatters the silence, transforming the Eyeless Abbot to from his spirit back into flesh and bone.
    
    \vspace{0.5\baselineskip}
    \textbf{Mechanics:}
    \begin{itemize}
        \item \textbf{Action:} A player may ring the Bell as an action. Doing so causes an immediate shift in the environment (dust, sound, collapsing illusions).
        \item \textbf{Effect:} The Eyeless Abbot becomes vulnerable to \emph{Wounds} and the Spectral Monks dissolve.
        \item \textbf{Cost:} Each round after the Bell is rung, all characters gain 1 \emph{Fatigue} as the veil weakens and the burden of memory floods back.
    \end{itemize}
\end{CommentBox}

As they fight, players may notice the \emph{Campana Silens} above. Clues from earlier visions or carvings should now click into place. To reach the bell:

\begin{Example}
    \begin{itemize}
    \item A player must disengage and climb to the bell platform.
    \item Climbing takes four turns, each step requiring a successful \DL{+1} roll using \textbf{Athletics}.
    \item A \textbf{tied roll} means that the character slips and loses a turn, but does not fall to the ground.
    \item A \textbf{failed roll} results in a fall, causing 1 \emph{Stress} and losing the turn.
    \item The Abbot may try to stop them by sending Spectral Monks to intercept.
    \end{itemize}
\end{Example}


When the bell is rung, the room convulses. Illusions shatter. The Abbot howls. His form becomes real.

\begin{Example}[The Abbot's Physical Form]
    When the bell tolls, the silence shatters.  The Eyeless Abbot convulses, staggering backward as the weight of memory crashes down around him. The spectral veil rips away—and what remains is something horrifyingly human… and not. He is no longer a ghost. He is pain, preserved in bone. A wound in the shape of a man.
    
    His robes hang in tatters, scorched and stiff with age, clinging to a body twisted by decades of spiritual torment. His skin is pale stone, stretched paper-thin over a gaunt frame. Symbols from the ritual are seared into his flesh—some burned, others carved, many half-erased as though even the rite tried to forget him.
\end{Example}


\subsection*{Phase Two – Break the Silence}

The Abbot becomes corporeal. The Spectral Monks dissolve, and the silence breaks—but the bell takes its toll.
\begin{Example}
    \begin{itemize}
    \item The Abbot loses \textbf{Untouchable Form} and \textbf{Aura of Dread}.
    \item The \textbf{Campana Silens} causes all players to gain 1 Fatigue at the end of each round.
    \item The Abbot can no longer attack with \textbf{Focus} (for spiritual attack) but must use \textbf{Fight} (for physical attacks).
    \item He may use \textbf{Erase the Self} once, suppressing a player Trait until the end of combat.
    \end{itemize}
\end{Example}

Once the bell tolls, the Abbot becomes vulnerable, and as a physical opponent he should not be overly difficult for the players to defeat, even if they are fatigued. 

Once defeated, the Abbot collapses into dust. A final toll echoes, and the silence ends.

\balance % FIXME: placement of this guy is unstable


\subsection*{Victory and Aftermath}

If the Abbot is defeated:
    \begin{itemize}
    \item The silence lifts. The air returns. Names are remembered.
    \item The monastery begins to collapse, dissolve, or fall into ruin.
    \item The players may each hear their lost loved one speak a final word, then fade.
    \end{itemize}

If they fail:
    \begin{itemize}
    \item Their names are forgotten. Their voices fade.
    \item They remain in the monastery forever—silent, spectral echoes of themselves.
    \end{itemize}

\vspace{\baselineskip}\noindent
\emph{Some silences were never meant to be broken. But some names should never be forgotten.}

\WyrdFooterImage{img/gothic_mountains_cleaned}
\clearpage


%%% Characters
\begin{WyrdCharacterSheet}
    {Dr. Clara Ashcroft} 
    {“Some stories are not told. They are buried. And they bleed through the silence.”}
  
    A cultural anthropologist specialising in ritual practice and memory, Clara Ashcroft is quiet, meticulous, and emotionally withdrawn. Her life's work revolves around forgotten spiritual sites and fringe religious communities, but her most personal mystery lies within Saint Hieronymus. Years ago, her younger sister, Evelyn, joined the monastery’s contemplative program—and was never heard from again.
  
    \subsection{Background}
    Clara was a senior lecturer at a university in Edinburgh, well-regarded for her research into memory, ritual, and folk beliefs. When Evelyn disappeared, Clara left academia behind and began investigating isolated monasteries, vanished communes, and other places where silence clings. Her work has become solitary and obsessive—an academic pursuit turned personal.
  
    \subsection{Appearance}
    Clara wears a long, functional raincoat with a laptop bag or field satchel always over her shoulder. Her hair is loosely tied, and she favours muted colours. Her eyes are sharp, but tired. She's often mistaken for a journalist or archaeologist—but speaks like someone who’s seen too much.
  
    \subsection{Personality}
    Clara is thoughtful, distant, and deeply rational—but beneath her composed surface is a raw grief she’s never allowed herself to feel. She treats emotion like data, but the closer she comes to the truth, the harder that becomes. She avoids connection but is fiercely protective once it's made.
  
    \subsection{Connection to the Monastery}
    Evelyn Ashcroft disappeared at Saint Hieronymus while participating in a modern-day spiritual retreat. Clara always suspected something was wrong—but had no proof, no access. Now, the letter has given her the first lead in nearly twenty years.
  
    \subsection{Goals}
    Clara seeks to uncover what happened to her sister, and by doing so, expose the truth about the monastery’s silence. She also hopes, secretly, to find some version of forgiveness—for Evelyn, and for herself.
  
    \begin{WyrdStatsBlock}[profile=img/characters/clara_ashcroft]
        \begin{SkillsBox}
            \Expert & Lore \\
            \Skilled & Focus, Awareness \\
            \Novice & Empathy, Insight, Investigation
        \end{SkillsBox}

        \begin{TraitsBox}
            \item[Driven to Understand] — Gain a \textbf{+2} bonus when analysing ritual, folklore, or spiritual mechanisms.
            \item[Buried Guilt] — Once per session, may clear all Fatigue after confronting a painful memory. The memory must be shared aloud to the group. Confronting the painful memory takes one combat action.
            \item[Cold Logic, Quiet Grief] — Reroll a failed \textbf{Insight} or \textbf{Empathy} check related to grief or memory.
        \end{TraitsBox}
  
        \DamageBox[]

    \end{WyrdStatsBlock}
  \end{WyrdCharacterSheet}
  
  \begin{WyrdCharacterSheet}
    {Isaac Bellamy} 
    {“The war ended, but not for me. Some fights just change their shape.”}
  
    A former infantryman turned private security contractor, Isaac Bellamy is a man carved out of trauma and silence. After years spent in conflict zones, he returned to find his younger brother—his only family—had gone to Saint Hieronymus on a spiritual retreat. That was the last he ever heard from him.
  
    \subsection{Background}
    Isaac joined the military straight out of school and served multiple tours before transitioning into high-risk private security work. He’s seen what silence can mean in a firefight, and what it can hide in grief. His brother Elijah sought peace at the monastery after a breakdown—Isaac always feared it was a cult in disguise. When the place went quiet, Isaac went searching... but never found anything. Until now.
  
    \subsection{Appearance}
    Isaac wears a weathered field jacket, dark jeans, and durable hiking boots. His close-cropped hair and guarded stance mark him as someone used to danger. A locket with Elijah’s photo hangs around his neck, worn from years of handling.
  
    \subsection{Personality}
    Isaac is blunt, steady, and doesn’t waste words. He’s not easily rattled—but his silence isn’t peace, it’s armor. Loyalty means everything to him, and guilt weighs heavy on his shoulders. He’s used to being the one who makes the hard choices when no one else will.
  
    \subsection{Connection to the Monastery}
    Elijah Bellamy disappeared after joining a silent spiritual retreat hosted by Saint Hieronymus. Isaac never trusted it. When the monastery ceased all communication, he came looking. Now, years later, the letter has cracked the silence wide open.
  
    \subsection{Goals}
    Isaac wants the truth about what happened to his brother. He’s not afraid of ghosts, only of failing someone he loved—again. If he has to fight the dark to get answers, so be it.
  
    \begin{WyrdStatsBlock}[profile=img/characters/isaac_bellamy]

        \begin{SkillsBox}
            \Expert & Will \\
            \Skilled & Athletics, Fight \\
            \Novice & Awareness, Insight, Will \\
        \end{SkillsBox}
  
        \begin{TraitsBox}
            \item[Combat Tempered] — Gain a bonus when acting under pressure or facing fear head-on.
            \item[The Locket] — Once per session, reroll a failed check after invoking a memory of his brother.
            \item[Trained to Endure] — Reduce incoming Stress by 2 once per scene.
        \end{TraitsBox}
  
        \DamageBox

    \end{WyrdStatsBlock}
\end{WyrdCharacterSheet}
  

\begin{WyrdCharacterSheet}
    {Margot Delaney} 
    {“Faith asks us to believe. But no one tells you what to do when belief breaks.”}
  
    Margot Delaney is a former nurse and lifelong believer who once placed her hope in Saint Hieronymus Monastery. Years ago, her teenage daughter Eleanor was sent there to recover from a breakdown. When the monastery fell silent, so did the answers. Margot was told to let go. She never did.
  
    \subsection{Background}
    Margot worked in palliative care for over twenty years, walking others through the long process of grief. She once found strength in faith, but the unanswered questions around her daughter’s disappearance cracked something inside her. She left the church, left her job, and began quietly investigating on her own—looking for signs, for records, for anything the silence couldn’t bury.
  
    \subsection{Appearance}
    Margot wears a wool coat, boots, and practical layers—dressed for wind and wet ground. Her hair is pulled back neatly, with strands of grey at the temples. She carries a single worn photograph in her coat pocket, and her expression often lingers between sorrow and resolve.
  
    \subsection{Personality}
    Calm, warm, and unshakably determined. Margot is the kind of person others lean on—until they realise she’s carrying more weight than she shows. She speaks softly, but never vaguely. When she commits to something, she sees it through.
  
    \subsection{Connection to the Monastery}
    Margot sent her daughter Eleanor to Saint Hieronymus for spiritual healing after a mental health crisis. She visited twice. On the third attempt, she was told the monastery had closed its gates. No further contact was ever made.
  
    \subsection{Goals}
    Margot wants the truth—not just for Eleanor, but for herself. She seeks closure, but also understanding. She hopes to find proof that her daughter mattered—that her story didn’t simply vanish.
  
    \begin{WyrdStatsBlock}[profile=img/characters/margot_delaney]

        \begin{SkillsBox}
            \Expert & Empathy \\
            \Skilled & Will, Insight \\
            \Novice & Lore, Focus, Persuasion \\
        \end{SkillsBox}
  
        \begin{TraitsBox}
            \item[Grief Made Graceful] — Once per session, may grant a \textbf{+2} to an ally's roll by offering emotional support.
            \item[Photograph in the Pocket] — Gain a bonus when resisting psychic or emotional attacks tied to memory or identity.
            \item[Unanswered Faith] — Reroll a failed \textbf{Will} or \textbf{Empathy} check when confronting spiritual or moral conflict.
        \end{TraitsBox}
  
        \DamageBox

    \end{WyrdStatsBlock}
\end{WyrdCharacterSheet}
  
\begin{WyrdCharacterSheet}
    {Micah Rios} 
    {“It’s not the dead that scare me. It’s what they remember—and what they want us to forget.”}
  
    Micah Rios has always been able to feel the echo of things left unfinished. Whether it's ghosts, trauma, or silence itself, something in him is tuned to it. He never sought this gift—but when the letter arrived, written in the hand of an uncle who vanished before Micah was born, he knew exactly where he had to go.
  
    \subsection{Background}
    Micah grew up in foster care, always knowing he was “sensitive” in a way that others weren’t. He’s spent time working at grief support groups, volunteering in spiritual communities, and studying dream psychology. He doesn’t talk much about his family—but he keeps a journal full of sketches and messages from dreams he hasn’t told anyone about.
  
    \subsection{Appearance}
    Micah wears layered, practical clothes—a denim jacket over a hoodie, worn boots, and a small crystal pendant. He often pauses mid-thought, distracted by something no one else noticed. His calm voice hides a deep current of unease.
  
    \subsection{Personality}
    Empathetic and soft-spoken, Micah avoids conflict but isn’t afraid of difficult truths. He’s open-minded, intuitive, and carries himself like someone used to not being believed. Despite everything, he genuinely wants to help others find peace—because he’s never quite known it himself.
  
    \subsection{Connection to the Monastery}
    Micah’s mother never spoke of her brother, but he appeared in Micah’s dreams for years. When the letter arrived in that uncle’s handwriting, Micah knew it wasn’t just a coincidence. Something old is calling to him—and he may be the only one who can hear it clearly.
  
    \subsection{Goals}
    Micah wants to understand what’s haunting the monastery—not just for his uncle’s sake, but to better understand his own strange connection to death and memory. If he can help others find peace, maybe he can find some of his own.
  
    \begin{WyrdStatsBlock}[profile=img/characters/micah_rios]
        \begin{SkillsBox}
            \Expert & Empathy \\
            \Skilled & Insight, Lore \\
            \Novice & Will, Fight, Persuasion \\
        \end{SkillsBox}
  
        \begin{TraitsBox}
            \item[Spirit-Touched] — Once per session, receive a vision or sensation tied to an echo or memory.
            \item[Dream Journal] — Gain a bonus when interpreting symbols, hauntings, or unresolved grief.
            \item[Sensitive Aura] — May reroll failed Insight or Awareness when detecting unseen presences.
        \end{TraitsBox}
  
        \DamageBox

    \end{WyrdStatsBlock}
\end{WyrdCharacterSheet}
  
\begin{WyrdCharacterSheet}
    {Nina Maddox} 
    {“There’s always a reason. Even if you don’t like what it is.”}
  
    Nina Maddox doesn’t believe in ghosts. Or at least, she doesn’t admit to it. A freelance investigator with a background in private security, Nina specialises in cleaning up messes no one wants to admit exist. She’s not here because she believes in hauntings—she’s here because someone thought she should be.
  
    \subsection{Background}
    Nina served in law enforcement before moving into private security and missing persons work. She’s seen cults, hoaxes, and grieving families manipulated by con artists. When she got a letter from her estranged father—dead over a decade—she assumed it was a prank. Still, she couldn’t ignore the place it mentioned: Saint Hieronymus Monastery. She’d heard of it before, in the case files she was never supposed to read.
  
    \subsection{Appearance}
    Short-cropped hair, sharp eyes, and a lean, athletic build. Nina dresses in a leather jacket and cargo jeans, always carrying a duffel with tools, water, and a flashlight. She walks like someone who’s used to being underestimated—and making people regret it.
  
    \subsection{Personality}
    Blunt, sceptical, and resourceful. Nina doesn’t buy into rituals or spirits, but she respects grief—and she knows that sometimes, belief can be more dangerous than truth. She doesn’t scare easy, but what she finds at the monastery might just shake that.
  
    \subsection{Connection to the Monastery}
    Her father investigated Saint Hieronymus decades ago and never spoke of it again. His death was ruled a suicide. The letter bearing his handwriting simply said: \emph{“They never let me leave. Maybe you can.”}
  
    \subsection{Goals}
    Nina wants to find the source of the monastery’s silence and put it to rest—by force, if necessary. She doesn’t believe in ghosts, but she does believe in justice. And if there’s something still trapped there, it’s about to find out what she's made of.
  
    \begin{WyrdStatsBlock}[profile=img/characters/nina_maddox]

        \begin{SkillsBox}
            \Expert & Fight \\
            \Skilled & Awareness, Focus \\
            \Novice & Will, Investigation, Insight
        \end{SkillsBox}
  
        \begin{TraitsBox}
            \item[No-Nonsense Grit] — Gain a bonus when pushing through fear, injury, or disbelief.
            \item[Professional Instincts] — Reroll a failed Awareness or Investigation check tied to danger or traps.
            \item[Break the Pattern] — Once per session, ignore the effects of a supernatural influence for one round.
        \end{TraitsBox}
  
        \DamageBox

    \end{WyrdStatsBlock}
\end{WyrdCharacterSheet}
  
\end{document}
