\documentclass[nodeprecatedcode,bg=print]{dndbook}
\usepackage{wyrd}
\raggedright
\pagestyle{empty}

\begin{document}

\vspace*{\fill}
\begin{center}
    \includegraphics[width=.1\textwidth]{img/wyrd-logo}
\end{center}
\vspace*{\fill}
\begin{center}
    {\Huge\DndFontPart One-Page Wyrd}\\
    \vspace{1em}
\end{center}
\vspace*{\fill}

\begin{paracol}{2}
    \section*{Skill Rolls}

    Any time a character wish to overcome an obstacle, they
    \begin{itemize}
        \item Roll a \textbf{4dF}
        \item Add an appropriate \textbf{skill level}
        \item Add any relevalt \textbf{Traits} bonuses
        \item Add any relevant \textbf{Gear} bonuses
    \end{itemize}
    The resulting score is their \textbf{total score}.

    \section*{Passive Opposition}
    With \textbf{passive opposition}, the player's score has to equal or exceed a \textbf{difficulty level} set by the GM. At \textbf{ties}, the player will generally succeed, but the GM may impose a complication or twist.

    \section*{Active Opposition}
    With \textbf{active opposition}, the player rolls against an opposing character's score. The player wins if their score is higher than the opponent's. At \textbf{ties}, the defender will generally succeed, but the GM may impose a complication or twist.

    \section*{Improving Scores*}
    Players can improve their scores in three ways:
    \begin{itemize}
        \item \textbf{Teamwork} allows multiple characters to work together on a task. A character will attempt to overcome an obstacle as usual, using skill + trait. Each participlant must succeed on an appropriate roll and this will add a +1 bonus to the total score for each success.
        \item \textbf{Boosts} are temporary bonuses created by a character. Requires creative use of skills, traits or gears to add a +1 bonus to the total score. The GM may require a roll to create the boost.
    \end{itemize}

    \switchcolumn

    \section*{Combat}
    Combat is a series of \textbf{opposed rolls} between the attacker and the defender. The attacker rolls their \textbf{attack score} and the defender rolls their \textbf{defense score}. The difference between the two scores determines the outcome of the attack. For each point of difference, the attacker deals \textbf{1 stress} to the defender. If the defender's score is higher, they successfully defend against the attack.

    \section*{Stress}
    Each character has a \textbf{stress pool} that represents their ability to withstand damage. When a character takes stress, they reduce their stress pool by the amount of stress taken. Stress is split into \textbf{Fatigue} and \textbf{Wounds}. Fatigue is temporary and does not affect the character's abilities. Wounds take time to heal and will reduce relevant skill rolls by 1 for \textbf{Mild Wounds}, 2 for \textbf{Moderate Wounds}, and 3 for \textbf{Severe Wounds}. If a character's stress pool is reduced to 0, they are taken out.

    \DamageBox[]

    \section*{Boons*}
    If either player in a conflict beats the other by more than 3 points, they can receive a boon, giving them a +2 bonus for their next roll.

\end{paracol}

\vspace*{\fill}
\begin{center}
    \includegraphics[width=.1\textwidth]{img/wyrd-logo}
\end{center}
\vspace*{\fill}

{\small * Optional rules.}

\end{document}
